\documentclass[12pt]{article}
\usepackage[margin=1in]{geometry}
\usepackage{amsmath, amsfonts,amsthm,amssymb,epigraph,etoolbox,mathtools,setspace,enumitem}  
\usepackage{tikz}
\usetikzlibrary{datavisualization} 
\usepackage[makeroom]{cancel} 
\usepackage[linguistics]{forest}
\usetikzlibrary{patterns}
\newcommand{\N}{\mathbb{N}}
\newcommand{\Z}{\mathbb{Z}}
\newcommand{\R}{\mathbb{R}}
\newcommand{\Q}{\mathbb{Q}}
\newcommand{\Mod}[1]{\ (\mathrm{mod}\ #1)}
\newcommand{\Lim}[1]{\mathrm{lim}(#1)}
\newcommand{\Abs}[1]{\left\vert #1 \right\vert}
\newcommand{\Dom}[1]{\mathrm{dom}\left(#1\right)}
\newcommand{\Range}[1]{\mathrm{range}(#1)}

\newlist{legal}{enumerate}{10}
\setlist[legal]{label=(\alph*)}
\setenumerate[legal]{label=(\alph*)}

\DeclarePairedDelimiter\bra{\langle}{\rvert}
\DeclarePairedDelimiter\ket{\lvert}{\rangle}
\DeclarePairedDelimiterX\braket[2]{\langle}{\rangle}{#1\delimsize\vert #2}


\newenvironment{theorem}[2][Theorem]{\begin{trivlist} \item[\hskip \labelsep {\bfseries #1}\hskip \labelsep {\bfseries #2.}]}{\end{trivlist}}
\newenvironment{lemma}[2][Lemma]{\begin{trivlist} \item[\hskip \labelsep {\bfseries #1}\hskip \labelsep {\bfseries #2.}]}{\end{trivlist}}
\newenvironment{result}[2][Result]{\begin{trivlist} \item[\hskip \labelsep {\bfseries #1}\hskip \labelsep {\bfseries #2.}]}{\end{trivlist}}
\newenvironment{exercise}[2][Exercise]{\begin{trivlist} \item[\hskip \labelsep {\bfseries #1}\hskip \labelsep {\bfseries #2.}]}{\end{trivlist}}
\newenvironment{problem}[2][Problem]{\begin{trivlist} \item[\hskip \labelsep {\bfseries #1}\hskip \labelsep {\bfseries #2.}]}{\end{trivlist}}
\newenvironment{question}[2][Question]{\begin{trivlist} \item[\hskip \labelsep {\bfseries #1}\hskip \labelsep {\bfseries #2.}]}{\end{trivlist}}
\newenvironment{corollary}[2][Corollary]{\begin{trivlist} \item[\hskip \labelsep {\bfseries #1}\hskip \labelsep {\bfseries #2.}]}{\end{trivlist}}
\newenvironment{solution}[1][Solution]{\begin{trivlist} \item[\hskip \labelsep {\bfseries #1}]}{\end{trivlist}}

\setlength\epigraphwidth{8cm}
\setlength\epigraphrule{0pt}

\makeatletter
\patchcmd{\epigraph}{\@epitext{#1}}{\itshape\@epitext{#1}}{}{}
\makeatother

\begin{document}
  
 \title{Homomorphisms and isomorphisms}
   \author{Juan Patricio Carrizales Torres}
     \date{May 23, 2023}
       \maketitle

       The notion of an \textit{isomorphism} is that two groups have the same gorup-theoretic structure (any property that can be derived from the axioms of the group holds for both groups). Let $(G,*)$ and $(H,\cdot)$ be groups. A map $\phi: G\to H$ such that $\phi(x*y)=\phi(x)\cdot\phi(y)$ for all $x,y\in G$ is a homomorphism. For this map to be considered an ismorphism, it must be bijective. The symbol $\cong$ represent the equivalence isomorphic relation. Since $\cong$ is an equivalence relation in the set $\mathfrak{G}$ of all groups, it follows that there are equivalence clases that are isomorphic. This is important for the classification of groups using isomorphisms.\\

       Also, one can show isomorphic relationships between sets by looking at their group presentations. Suppose for a group $G$ with generators $\left\{r_{1},\dots,r_{m} \right\}$ and group $H$ with some subset $\left\{s_{1},\dots,s_{m}  \right\}$ that every relation in $G$ is fulfilled in $H$ when $r_{i}$ is changed to $s_{i}$. Then there is a unique homomorphism $\varphi:G\to H$ such that $r_{i}\to s_{i}$. Furthermore, if $\left\{ s_{1},\dots,s_{m} \right\}$ is the set of generators for $H$, then $\varphi$ is surjective. Also, if $|G|=|H|$, then $\varphi$ is injective and so an isomorphism.
\section{Excercises}

Let $G$ and $H$ be groups.
\begin{problem}{1}
 Let $\varphi:G\to H$ be a homomorphism.
\begin{enumerate}
  \item Prove that $\varphi(x^{n})=\varphi(x)^{n}$ for all $n\in \Z^{+}$.
\begin{proof}
  We proceed by induction. First, note that $\varphi(e\cdot x) = \varphi(e)\cdot \varphi(x) = \varphi(x)$ for any $x\in G$ and so $\varphi(e)=e'$ (Recall that there is a unique identity element in a group). Since $\varphi:G\to H$ is a homomorphism, it follows that $\varphi(x^{0}) = \varphi(e) = e' = \varphi(x)^{0}$ for any $x\in G$. Hence, the result is true for $n=0$. \\

  Now, assume that $\varphi(x^{k}) = \varphi(x)^{k}$ for any $x\in G$ and some nonnegative integer $k$. We show that $\varphi(x^{k+1}) = \varphi(x)^{k+1}$. Observe that
\begin{align*}
  \varphi(x^{k+1}) &= \varphi(x^{k}\cdot x^{1})\\
  &= \varphi(x^{k})\cdot \varphi(x) = \varphi(x)^{k}\cdot \varphi(x)\\
  &= \varphi(x)^{k+1}.
\end{align*}
By the Principle of Mathematical Induction, $\varphi(x^{n})=\varphi(x)^{n}$ for any $x\in G$ and any $k\in \Z^{+}$.
\end{proof}
  \item Do part (a) for $n=-1$ and deduce that $\varphi(x^{n})=\varphi(x)^{n}$ for all $n\in \Z$.
\begin{proof}
  Let $x\in G$. Then $\varphi(x\cdot x^{-1}) = \varphi(x)\cdot \varphi(x^{-1})$ and $\varphi(x\cdot x^{-1})=\varphi(e) = e'$. Since there is a unique inverse for any element of a group, it follows that $\varphi(x)\cdot \varphi(x^{-1})=e'$ implies that $\varphi(x^{-1})=\varphi(x)^{-1}$, namely, it is the unique inverse of $\varphi(x)$. Now, consider what happens when we plug the inverse of $x^{n}$ inside the function, namely, $\varphi(x^{-n})$ for some $n\in \N$. Note that,
\begin{align*}
  \varphi\left( x^{-n} \right) &= \varphi\left( \left(x^{-1}\right)^{n} \right) \\
  &= \varphi\left( x^{-1} \right) ^{n} = \left( \varphi(x)^{-1} \right)^{n} \\
  &= \varphi(x)^{-n}.
\end{align*}
Therefore, $\varphi(x^{n}) = \varphi(x)^{n}$ for all $n\in\Z$.
\end{proof}
\end{enumerate}
\end{problem}
\begin{problem}{2}
  If $\varphi: G\to H$ is an isomorphism, prove that $|\varphi(x)| = |x|$ for all $x\in G$. Deduce that any two isomorphic groups have the same number of elements of order $n$ for each $n\in \Z^{+}$. Is the result true if $\varphi$ is only assumed to be a homomorphism?
\begin{proof}
  First, we show that $|\varphi(x)|=|x|$. Consider any $x\in G$ with some order $n=|x|$, where $n\in \Z^{+}$. Then, $\varphi(x^{n}) = \varphi(e) = e' = \varphi(x)^{n}$ and so $\varphi(x)$ has a finite order. Let $k=|\varphi(x)|\in \Z^{+}$. Note that, 
\begin{align*}
  \varphi(x)^{k} &= e' = \varphi(e) = \varphi(x^{k})
\end{align*}
Then, $k\leq n$ $(\varphi(x)^{k} = \varphi(x)^{n})$ and $n\leq k$ $(x^{n} = x^{k})$, which implies that $n=k$.\\
For the previous argument, we only used the results that are derived from the homorphic property of the function and didn't need to assume that $\varphi$ was bijective. Thus, for any homomorphic function $\varphi: G\to H$, $|\varphi(x)|=|x|$ for all $x\in G$.\\

Now, we show that the homomorphism $\varphi:G\to H$ being a bijection is a sufficient condition for both $G$ and $H$ to have the same number of elements of order $n$.\\
If the homomorphism $\varphi:G\to H$ is a bijection, then there is a one-to-one correspondence $x\leftrightarrow \varphi(x)$, where $|x|=|\varphi(x)|$ for any $x\in G$. Hence, both isomorphic groups have the same number of elements of order $n$ for each $n\in \Z^{+}$. \\

Note that this only applies when we consider any $n$, including $\infty$. There can be homomorphic groups such that they have the same quantity of elements with order $n\in \Z^{+}$.
\end{proof}
\end{problem}

\begin{problem}{3}
  If $\varphi:G\to H$ is an isomorphism, prove that $G$ is abelian if and only if $H$ is abelian. If $\varphi G\to H$ is a homomorphism, what additional condiitons on $\varphi$ (if any) are sufficient to ensure that if $G$ is abelian, then so is $H$?
\begin{proof}
  Since $\varphi:G\to H$ is an isomorphism, there is a one-to-one correspondence $x\leftrightarrow y$, where $x\in G$ and $y\in H$. Basically, we can ``translate'' one element from a group to the other, and viceversa. We just have to show that the abelian property is maintained throug the isomorphism. Note that,
\begin{align*}
  x\cdot y = y\cdot x &\implies \varphi(x\cdot y) = \varphi(x) * \varphi(y) = \varphi(y) * \varphi(x) = \varphi (y\cdot x). \forall x,y,\in G\\
  \\
  y*x =x*y &\implies \varphi^{-1}(y*x) = \varphi^{-1}(y)\cdot \varphi^{-1}(x) = \varphi^{-1}(x)\cdot \varphi^{-1}(y) = \varphi^{-1}(x*y). \forall x,y,\in H,
\end{align*}
because the inverse of an isomorphism is homomorphic.\\
In the case of a homomorphism $\varphi: G\to H$ an additional condition to ensure that if $G$ is abelian, then so is $H$, is that $\varphi$ must be surjective. This is so because this implies that the image of $\varphi$ is $H$ and so the commutativity of $G$ can be trasmitted to all the elements of $H$ through the surjective homomorphism $\varphi$. 
\end{proof}
\end{problem}

\begin{problem}{I}
  The homomorphism $\varphi:G\to H$ is injective if and only if there is a one-to-one correspondence between the identities of both groups.
\begin{proof}
  The direct is clear, since a homomorphism maps the identity of $G$ ot the identity of $H$. So let's assume the converse. Then, $e\leftrightarrow e'$. Consider some $x,y \in G$ such that $\varphi(y) = \varphi(x)$. Then, $e' = \varphi(x)* \varphi(y)^{-1} = \varphi(x\cdot y^{-1})$, which implies that $x\cdot y^{-1}=e$ and so $x=y$. The homomorphism $\varphi$ is injective.
\end{proof}
\end{problem}
       \end{document}


