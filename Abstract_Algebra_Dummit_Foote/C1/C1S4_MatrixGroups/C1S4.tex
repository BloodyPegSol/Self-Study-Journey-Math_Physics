\documentclass[12pt]{article}
\usepackage[margin=1in]{geometry}
\usepackage{amsmath, amsfonts,amsthm,amssymb,epigraph,etoolbox,mathtools,setspace,enumitem}  
\usepackage{tikz}
\usetikzlibrary{datavisualization} 
\usepackage[makeroom]{cancel} 
\usepackage[linguistics]{forest}
\usetikzlibrary{patterns}
\newcommand{\N}{\mathbb{N}}
\newcommand{\Z}{\mathbb{Z}}
\newcommand{\R}{\mathbb{R}}
\newcommand{\Q}{\mathbb{Q}}
\newcommand{\Mod}[1]{\ (\mathrm{mod}\ #1)}
\newcommand{\Lim}[1]{\mathrm{lim}(#1)}
\newcommand{\Abs}[1]{\left\vert #1 \right\vert}
\newcommand{\Dom}[1]{\mathrm{dom}\left(#1\right)}
\newcommand{\Range}[1]{\mathrm{range}(#1)}

\newlist{legal}{enumerate}{10}
\setlist[legal]{label=(\alph*)}
\setenumerate[legal]{label=(\alph*)}

\DeclarePairedDelimiter\bra{\langle}{\rvert}
\DeclarePairedDelimiter\ket{\lvert}{\rangle}
\DeclarePairedDelimiterX\braket[2]{\langle}{\rangle}{#1\delimsize\vert #2}


\newenvironment{theorem}[2][Theorem]{\begin{trivlist} \item[\hskip \labelsep {\bfseries #1}\hskip \labelsep {\bfseries #2.}]}{\end{trivlist}}
\newenvironment{lemma}[2][Lemma]{\begin{trivlist} \item[\hskip \labelsep {\bfseries #1}\hskip \labelsep {\bfseries #2.}]}{\end{trivlist}}
\newenvironment{result}[2][Result]{\begin{trivlist} \item[\hskip \labelsep {\bfseries #1}\hskip \labelsep {\bfseries #2.}]}{\end{trivlist}}
\newenvironment{exercise}[2][Exercise]{\begin{trivlist} \item[\hskip \labelsep {\bfseries #1}\hskip \labelsep {\bfseries #2.}]}{\end{trivlist}}
\newenvironment{problem}[2][Problem]{\begin{trivlist} \item[\hskip \labelsep {\bfseries #1}\hskip \labelsep {\bfseries #2.}]}{\end{trivlist}}
\newenvironment{question}[2][Question]{\begin{trivlist} \item[\hskip \labelsep {\bfseries #1}\hskip \labelsep {\bfseries #2.}]}{\end{trivlist}}
\newenvironment{corollary}[2][Corollary]{\begin{trivlist} \item[\hskip \labelsep {\bfseries #1}\hskip \labelsep {\bfseries #2.}]}{\end{trivlist}}
\newenvironment{solution}[1][Solution]{\begin{trivlist} \item[\hskip \labelsep {\bfseries #1}]}{\end{trivlist}}

\setlength\epigraphwidth{8cm}
\setlength\epigraphrule{0pt}

\makeatletter
\patchcmd{\epigraph}{\@epitext{#1}}{\itshape\@epitext{#1}}{}{}
\makeatother

\begin{document}
  
 \title{Section 1.4: Matrix Groups}
   \author{Juan Patricio Carrizales Torres}
     \date{May 8, 2023}
       \maketitle

       Before describing the matrix group, we must define what a \textit{field} is. A field is a set $F$ with two binary operations $+$ and $\cdot$ such that both $(F,+)$ and $(F/\left\{ 0 \right\},\cdot)$ are abelian groups. Also, the distributive law holds, namely, for any $a,b,c\in F$
\begin{align*}
  a\cdot(b+c) = a\cdot b + a\cdot c.
\end{align*}

Then, the general linear group $GL_{n}(F)$ is the set of all $n\times n$ matrices with entries from the field $F$ and nonzero determinant, where the associative matrix multiplication is the binary operation. Two useful results regarding general linear groups are the following: 
\begin{enumerate}
  \item if $F$ is a finite field, then $|F|=p^{m}$ for some prime $p$ and integer $m$.
  \item if $|F|=q<\infty$, then $|GL_{n}(F)|=(q^{n}-1)(q^{n}-q)(q^{n}-q^{2})\dots(q^{n}-q^{n-1})$.
\end{enumerate}
	
\section{PROBLEMS}

Let $F$ be a field and let $n\in \Z^{+}$.
\begin{problem}{1}
  Prove that $|GL_{2}(F_{2})|=6$
\begin{proof}
  This general linear group $GL_{2}(F_{2})$ contains $2\times 2$ matrices 
\begin{align*}
\begin{pmatrix}
  b_{1} & b_{2}\\
  b_{3} & b_{4}
\end{pmatrix},
\end{align*}
where $b_{1},b_{2},b_{3},b_{4}\in F_{2}$ and $b_{3}\cdot b_{2} - b_{4}\cdot b_{1} \neq  0$ (nonzero determinant). Then, $b_{3}\cdot b_{2} \neq b_{4}\cdot b_{1}$ (Recall that $\cdot$ is the binary operation in $F_{2}$ such that ($F_{2}/\left\{ 0 \right\},\cdot$) is a group). Then, the statement $|GL_{2}(F_{2})|=6$ is equivalent to saying that there are 6 possible unique equations $b_{3}\cdot b_{2} \neq b_{4}\cdot b_{1}$ for elements $b_{1},b_{2},b_{3},b_{4}\in F_{2}$. Let's call the instance $b\cdot a$ a \textit{binary multiplication}.\\
Because multiplication is closed, it follows that it is equal to some element inside $F_{2}$ and so we must find all ways to accomodate \textit{binary multiplications} in the equation such that one side is $0$ and the other is $1$. 
Before doing that, we have to look at the 4 possible \textit{binary multiplications}. We know that $0$ is the \textit{additive identity} and that the other element $1$ is the \textit{multiplicative identity} and its own additive and multiplicative inverse. Then, it follows that 
\begin{align*}
  0\cdot 1 &= (1+1)\cdot 1 = 1\cdot 1+ 1\cdot 1\\
 &= 0 + 0 = 0 \\
 &= 0\cdot 0 = 0\cdot (1+1) \\
 &= 0\cdot 1 + 0 \cdot 1 = 0 + 0.
\end{align*}
and $1\cdot 1 = 1$. Then, all binary multiplications, except for $1\cdot 1$, are equal to $0$. \\
Now, let one side of the equation be $1$, which there is only one binary multiplication able to represent that, namely, $1\cdot1$. Then, we only have 3 binary multiplications out of the possible 4 that we can place at the other side such that two sides are not equal $(1\cdot 0, 0\cdot 1, 0\cdot 0)$. Hence, per side there are 3 possible non equal equations and so there are 6 possible equations such that the binary multiplications at each side are not equal. 
\end{proof}
\end{problem}

\begin{problem}{2}
  Write out all the elements of $GL_{2}(F_{2})$ and compute the order of each element.
\begin{solution}
  We have the following elements with their respective orders (n):
\begin{align*}
\begin{pmatrix}
  1 & 0\\
  1 & 1
\end{pmatrix}, &n = 2\\
\begin{pmatrix}
  1 & 1\\
  0 & 1
\end{pmatrix}, &n = 2\\
\begin{pmatrix}
  1 & 0\\
  0 & 1
\end{pmatrix}, &n = 1 \text{(identity matrix)}\\
\begin{pmatrix}
  1 & 1\\
  1 & 0
\end{pmatrix}, &n = 3\\
\begin{pmatrix}
  0 & 1\\
  1 & 1
\end{pmatrix}, &n = 3\\
\begin{pmatrix}
  0 & 1\\
  1 & 0
\end{pmatrix}, &n = 2\\
\end{align*}
\end{solution}
\end{problem}
\begin{problem}{3}
  Show that $GL_{2}(F_{2})$ is non-abelian.
\begin{proof}
  Note that
\begin{align*}
\begin{pmatrix}1 & 1\\ 0 & 1\end{pmatrix} \begin{pmatrix} 1 & 1 \\ 1 & 0\end{pmatrix} &= \begin{pmatrix} 0 & 1\\ 1 & 0\end{pmatrix}\\
&\neq \begin{pmatrix} 1 & 0\\ 1 & 1\end{pmatrix} = \begin{pmatrix} 1 & 1\\ 1 & 0\end{pmatrix}\begin{pmatrix} 1 & 1\\ 0 & 1\end{pmatrix}
\end{align*}
\end{proof}
\end{problem}

\begin{problem}{4}
  Show that if $n$ is not prime then $\Z/n\Z$ is not a field.
\begin{proof}
  Suppose that $n$ is not prime. Then, there is at least one integer $1<k<n$ that is a factor. Hence, $n=kq_{1}q_{2}\dots q_{m}$ and so $l=q_{1}q_{2}\dots q_{m}$ is an integer (factor) such that $1<l<n$ and $k\cdot l = n$. Therefore, $\overline{k},\overline{l}$ are two elements in $\Z/n\Z$ such that $\overline{k}\cdot\overline{l}=\overline{k\cdot l}=\overline{n} = \overline{0}$, the additive identity. Hence, $\Z/n\Z^{\times}$ is not closed under multiplication, which implies that it is not a group.
\end{proof}
\end{problem}
\begin{problem}{5}
  Show that $GL_{n}(F)$ is a finite group if and only if $F$ has a finite number of elements.
\begin{proof}
  First assume that $|F|=n$ for some $n\in\N$. Then, there are $n$ possible ways to accomodate the $n$ elements in an entry. Therefore, there are $n^{n\times n}$ different ways to accomodate the elements of $F$ in the entries of a $n\times n$ matrix. Then, $|GL_{n}(F)|\leq n^{n\times n}$ which is finite.\\
  Now, for the converse, assume that $F$ has an infinity of elements. Note that the set of diagonal matrices
\begin{align*}
  D = \{ A=\begin{pmatrix} d_{1} \\ & d_{2} \\ & & \ddots \\ & & & d_{n}\end{pmatrix} | \text{det}(A)\neq 0 \iff d_{1},d_{2},\dots,d_{n}\neq 0\}
\end{align*}
is a subgroup of $GL_{n}(F)$, namely the inverse of a diagonal matrix with nonzero determinant is a diagonal matrix with nonzero determinant, and the multiplication of two diagonal matrices with nonzero determinant results in a diagonal matrix with nonzero determinant. We show that one can construct an infinity ofdiagonal matrices with nonzero determinant. Note that, for some fixed $a\in F/\left\{ 0 \right\}$ and every $b\in F/\{0\}$,
\begin{align*} 
\begin{pmatrix}
  a \\ & a\\ & & \ddots \\ & & & b
\end{pmatrix}
\end{align*}
is a diagonal matrix with nonzero determinant. Hence, $D$ has an infinity of elements and so $GL_{n}(F)$ has an infinity of elements.
\end{proof}
\end{problem}
\begin{problem}{10}
  Let $G=\left\{ \begin{pmatrix} a& b\\ 0 & c\end{pmatrix}| a,b,c\in\R,a\neq 0, c\neq 0\right\}$.
\begin{enumerate}
  \item Compute the product of $\begin{pmatrix}a_{1} & b_{1}\\ 0 & c_{1}\end{pmatrix}$ and $\begin{pmatrix}a_{2} & b_{2} \\ 0 & c_{2}\end{pmatrix}$ to show that $G$ is closed under matrix multiplication.
\begin{solution}
  Note that 
\begin{align*}
  \begin{pmatrix}a_{1} & b_{1}\\ 0 & c_{1}\end{pmatrix} \begin{pmatrix} a_{2} & b_{2} \\ 0 & c_{2}\end{pmatrix} = \begin{pmatrix} a_{1}a_{2} & a_{1}b_{2} + b_{1}c_{2}\\ 0 & c_{1}c_{2}\end{pmatrix}.
\end{align*}
Because $a_{1},a_{2},c_{1},c_{2}\neq 0$, it follows that $a_{1}a_{2},c_{1}c_{2}\neq 0$ and so $G$ is closed under matrix multiplication.
\end{solution}
  \item Find the matrix inverse of $A=\begin{pmatrix}a & b\\ 0 & c\end{pmatrix}$ and deduce that $G$ is closed under inverses.
\begin{solution}
  Consider some element $B=\begin{pmatrix}a_{2} & b_{2}\\ 0 & c_{2} \end{pmatrix}$ of $G$. Since $A\in G$ it follows that $a_{1},c_{1}\neq 0$. According to the result of the matrix multiplication showed in (a), for $B$ to be an inverse of $A$ it must be true that $a_{1}a_{2} = c_{1}c_{2} = 1$ and $a_{1}b_{2}+b_{1}c_{2} = 0\iff a_{1}b_{2} = -b_{1}c_{2}$. \\

  Then, $a_{2} = a_{1}^{-1} \neq 0 , c_{2} = c_{1}^{-1}\neq 0$ and $b_{2} = (- b_{1})c_{1}^{-1}a_{1}^{-1}$ which are elements of the field $F$. Hence, the inverse of $A$ exists in $G$.  Thus, $G$ is closed under inverses.
\end{solution}
  \item Deduce that $G$ is a subgroup of $GL_{2}(\R)$.
\begin{solution}
  The set $G$ over $\R$ is closed under matrix multiplication, closed under inverses and there is the identity $\begin{pmatrix} 1 & 0\\ 0 & 1\end{pmatrix}$. Hence, $G$ is a group. Furthermore, note that for any $\begin{pmatrix} a & b\\ 0 & c\end{pmatrix}$ in $G$, $ac-0\neq 0$ (nonzero determinant). Thus, $G$ is a subgroup of $GL_{2}(\R)$.
\end{solution}
  \item Prove that the set of elements of $G$ whose two diagonal entries are equal (i.e., $a=c$) is also subgroup of $GL_{2}(\R)$.
\begin{proof}
  Let the set in question be represented by $B$. From (a) we know that the matrix multiplication of two elements in $B$ results in the matrix $\begin{pmatrix}a_{1}a_{2} & a_{1}b_{2}+b_{1}c_{2}\\ 0 & c_{1}c_{2}\end{pmatrix}$, where $a_{1}a_{2} = c_{1}c_{2}$ since $a_{1}=c_{1}$ and $a_{2}=c_{2}$. Hence, $B$ is closed under matrix multiplication.\\

  From (b), the inverse of any matrix in $B\subseteq G$ is represented by $\begin{pmatrix} a^{-1} & (-b)c^{-1}a^{-1}\\ 0 & c^{-1} \end{pmatrix}$, where $a^{-1} = c^{-1}$ since $a=c$ (in the group $F^{\times}$ the inverses are unique). Therefore, $B$ is closed under matrix multiplication.\\
 Finally, the identity matrix is an element of $B$. Hence, $B$ is a subgroup of $GL_{2}(\R)$.
\end{proof}
\end{enumerate}
\end{problem}

The next exercise introduces the \textit{Heisenberg group} over the field $F$ and develops some of its basic properties.
\begin{problem}{11}
  Let $H(F)=\left\{ \begin{pmatrix}1 & a & b\\ 0 & 1 & c\\ 0 & 0 & 1\end{pmatrix} | a,b,c\in F\right\}-$called the \textit{Heisenberg group} over $F$. Let $X=\begin{pmatrix}1 & a & b\\ 0 & 1 & c\\ 0 & 0 & 1\end{pmatrix}$ and $Y=\begin{pmatrix}1 & d & e \\ 0 & 1 & f\\ 0 & 0 & 1\end{pmatrix}$ be elements of $H(F)$.
\begin{enumerate}
  \item Compute the matrix product $XY$ and deduce that $H(F)$ is closed under matrix multiplication. Exhibit explicit matrices such that $XY\neq YX$ (so that $H(F)$ non-abelian).
\begin{proof}
  First we show that $H(F)$ is closed under matrix multiplication. Note that
\begin{align*}
  XY = \begin{pmatrix} 1 & d+a & e + fa+b \\ 0 & 1 & f + c \\ 0 & 0 & 1\end{pmatrix}.
\end{align*}
Since $F$ is closed under addition and multiplication ($0a=0$ for all $a\in F$) it follows that $d+a,e+fa+b,f+c\in F$. Thus, $H(F)$ is closed under matrix multiplication. Furthermore, note that $e+fa+b = b+cd+e$ (the left hand side comes from the entry $(1,3)$ of the matrix $YX$) is not true when $fa=0$ and $cd=1$, the additive and multiplicative identities, respectively. Hence, $H(F)$ is non-abelian.
\end{proof}
  \item Find an explicit formula for the matrix inverse $X^{-1}$ and deduce that $H(F)$ is closed under inverses.
\begin{proof}
   Note that for $Y=X^{-1}$ to be true a necessary and sufficient condition is that $d=-a$, $f=-c$ and $e=ca-b$. Note that $-a,-c,ca-b\in F$ and so $H(F)$ is closed under inverses. 
\end{proof}
  \item Prove the associative law for $H(f)$ and deduce that $H(F)$ is a group of order $|F|^{3}$. (Do not assume that matrix multiplication is associative.)
\begin{proof}
  Consider the matrices $X,Y, Z=\begin{pmatrix} 1 & g & h \\ 0 & 1 & i\\ 0 & 0 & 1\end{pmatrix}$. Then
\begin{align*}
  (XY)Z &= \begin{pmatrix} 1 & d+a & e+fa+b \\ 0 & 1 & f+c\\ 0 & 0 & 1\end{pmatrix} \begin{pmatrix} 1 & g & h\\ 0 & 1 & i \\ 0 & 0 & 1\end{pmatrix}\\
  &= \begin{pmatrix} 1 & g+d+a & h+(d+a)i+e+fa+b\\ 0 & 1 & i + f +c \\ 0 & 0 & 1\end{pmatrix}\\
  &= \begin{pmatrix} 1 & a & b\\ 0 & 1 & c\\ 0 & 0 & 1\end{pmatrix} \begin{pmatrix} 1 & d+g & e+di + h \\ 0 & 1 & f+i \\ 0 & 0 & 1\end{pmatrix}\\
  &= X(YZ).
\end{align*}
Hence, the matrix multiplication in $H(F)$ is associative. In order to calculate the order of $H(F)$, we consider how many matrices can be constructed using the given conditions as restrictions. Consider the matrix $X$. It is a ``squeleton'' for any matrix element of $H(G)$, where the only values that can be changed are $a,b,c$. Since $a,b,c\in F$, it follows, in the case that $F$ is finite,  that there are $|F|\cdot|F|\cdot|F|$ possible combinations of $a,b,c$ and so the order of $H(F)$ is $|F|^{3}$. Clearly, if $F$ is infinite then $H(F)$ is infinite.
\end{proof}
  \item Find the order of each element of the finite group $H(\Z/2\Z)$.
\begin{proof}

\end{proof}
  \item Prove that every nonidentity element of the group $H(\R)$ has infinite order.
\begin{proof}
  \begin{lemma}{e.1}
    Let $X = \begin{pmatrix} 1 & a & b\\ 0 & 1 & c\\ 0 & 0 & 1\end{pmatrix}\in H(F)$ for some field $F$. Then,  
\begin{align*}
  X^{n} = \begin{pmatrix} 1 & n\cdot a & n\cdot b + m \\ 0 & 1 & n\cdot c\\ 0 & 0 & 1\end{pmatrix},
\end{align*}
where $m\in F$ and $n\in \N$.
\begin{proof}
  We proceed by induction. By \textbf{(a)}, $X^{2}=\begin{pmatrix} 1 & 2a & 2b+ca \\ 0 & 1 & 2c \\ 0 & 0 & 1\end{pmatrix}$. Hence, the result is true for $n=2$. Now, assume that the result is true for $X^{k}$ for some $k\in \N$. We show that it follows that the result is true for $X^{k+1}$. We know that, 
\begin{align*}
  X^{k+1} &= X^{k}X = \begin{pmatrix} 1 & k\cdot a & k\cdot b + m \\ 0 & 1 & k\cdot c \\ 0 & 0 & 1\end{pmatrix} \begin{pmatrix}1 & a & b \\ 0 & 1 & c \\ 0 & 0 &1\end{pmatrix}\\
  &=  \begin{pmatrix} 1 & k\cdot a + a & k\cdot b + m + b +k\cdot ac \\ 0 & 1 & c+k\cdot c\\ 0 & 0 & 1\end{pmatrix}\\
  &= \begin{pmatrix} 1 & (k+1)a & (k+1)b + (m+k\cdot ac)\\ 0 & 1 & (k+1)c\\ 0 & 0 & 1\end{pmatrix}.
\end{align*}
Since $m+k\cdot ac\in F$ (closed under addition and multiplication) it follows that the result is true for $X^{k+1}$. By the Principle of Mathematical Induction, the result is true.
\end{proof}
\end{lemma}
Consider some nonidentity element $X\in H(F)$. Then, at least one of the variables $a,b,c$ is nonzero. Then, for some $n\in \N$, at least one of the elements $n\cdot a, n\cdot b, n\cdot c$ is nonzero. By \textbf{Lemma e.1}, $X^{n}$ is a nonidentity matrix for any $n\in \N$. Hence, every nonidentity element of $H(F)$ has infinite order.
\end{proof}
\end{enumerate}
\end{problem}
       \end{document}


