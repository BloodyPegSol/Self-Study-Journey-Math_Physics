\documentclass[12pt]{article}
\usepackage[margin=1in]{geometry}
\usepackage{amsmath, amsfonts,amsthm,amssymb,epigraph,etoolbox,mathtools,setspace,enumitem}  
\usepackage{tikz}
\usetikzlibrary{datavisualization} 
\usepackage[makeroom]{cancel} 
\usepackage[linguistics]{forest}
\usetikzlibrary{patterns}
\newcommand{\N}{\mathbb{N}}
\newcommand{\Z}{\mathbb{Z}}
\newcommand{\R}{\mathbb{R}}
\newcommand{\Q}{\mathbb{Q}}
\newcommand{\Mod}[1]{\ (\mathrm{mod}\ #1)}
\newcommand{\Lim}[1]{\mathrm{lim}(#1)}
\newcommand{\Abs}[1]{\left\vert #1 \right\vert}
\newcommand{\Dom}[1]{\mathrm{dom}\left(#1\right)}
\newcommand{\Range}[1]{\mathrm{range}(#1)}

\newlist{legal}{enumerate}{10}
\setlist[legal]{label=(\alph*)}
\setenumerate[legal]{label=(\alph*)}

\DeclarePairedDelimiter\bra{\langle}{\rvert}
\DeclarePairedDelimiter\ket{\lvert}{\rangle}
\DeclarePairedDelimiterX\braket[2]{\langle}{\rangle}{#1\delimsize\vert #2}


\newenvironment{theorem}[2][Theorem]{\begin{trivlist} \item[\hskip \labelsep {\bfseries #1}\hskip \labelsep {\bfseries #2.}]}{\end{trivlist}}
\newenvironment{lemma}[2][Lemma]{\begin{trivlist} \item[\hskip \labelsep {\bfseries #1}\hskip \labelsep {\bfseries #2.}]}{\end{trivlist}}
\newenvironment{result}[2][Result]{\begin{trivlist} \item[\hskip \labelsep {\bfseries #1}\hskip \labelsep {\bfseries #2.}]}{\end{trivlist}}
\newenvironment{exercise}[2][Exercise]{\begin{trivlist} \item[\hskip \labelsep {\bfseries #1}\hskip \labelsep {\bfseries #2.}]}{\end{trivlist}}
\newenvironment{problem}[2][Problem]{\begin{trivlist} \item[\hskip \labelsep {\bfseries #1}\hskip \labelsep {\bfseries #2.}]}{\end{trivlist}}
\newenvironment{question}[2][Question]{\begin{trivlist} \item[\hskip \labelsep {\bfseries #1}\hskip \labelsep {\bfseries #2.}]}{\end{trivlist}}
\newenvironment{corollary}[2][Corollary]{\begin{trivlist} \item[\hskip \labelsep {\bfseries #1}\hskip \labelsep {\bfseries #2.}]}{\end{trivlist}}
\newenvironment{solution}[1][Solution]{\begin{trivlist} \item[\hskip \labelsep {\bfseries #1}]}{\end{trivlist}}

\setlength\epigraphwidth{8cm}
\setlength\epigraphrule{0pt}

\makeatletter
\patchcmd{\epigraph}{\@epitext{#1}}{\itshape\@epitext{#1}}{}{}
\makeatother

\begin{document}
  
\title{Section 1.2: Dihedral Groups}
   \author{Juan Patricio Carrizales Torres}
     \date{Feb 13, 2023}
       \maketitle

       One group is known as a Dihedral Group. It's origin can be traced to the study of the symmetries of geometric entities known as $n$-gons. Hand-wavy explained, a symmetry is a rigid motion of all points in the n-gon such that the resulting copy can be placed over the original one, namely, points of some ``nature'' lay on top of points of the same nature. Then, the Dihedral group, also represented by $D_{2n}$, is the set of symmetries of the $n$-gon. Clearly, we must first represent them by appropiate mathematical objects. Thus, symmetries can be defined as functions $f:V\to V$, where $V$ is the set of vertices of the $n-gon$. This has to do with the fact that the position of the vertices determine the position of all points (particularly, due to the ``symmetric'' shape of $n$-gons, one must only know the distance from two points to characterize all points on the $n$-gon [for an elaborated argument about this fact check ``Dihedral Groups'' by Keith Conrad]). Thus, the identity $e$ function leaves vertices intact, the inverse of the symmetry ``reverses'' the rigid motion (brings vertices to the position they were at before the application of the symmetry), and the permutations are defined by the operation of composition of functions, which is associative. \\

      For $n$-gons, two basic symmetries are the $2\pi/n$ rotation $r$ applied to vertex some vertex labelled as $1$ and the reflection $s$ about the line of symmetry of $1$ and the center of the $n$-gon. This is not enough to show that they are the only ones, there could be more. However, one can show the limit $|D_{2n}|\leq 2n$ and construct $2n$ symmetries by permutations of $s$ and $r$. 

      To construct the $2n$ symmetries, We show the following useful properties of the Dihedral Group:
\begin{enumerate}
  \item $r^{n}\neq r^{m}$ such that $n\neq m$ and $n,m\in \left\{ 0,1,\dots,n-1 \right\}$.
\begin{proof}
  Consider the vertex labeled by $1$, which is at some position $i\in \left\{ 1,\dots,n \right\}$. Note that $r$ maps vertex $1$ from position $\bar{i}$ to position $\overline{i+1}$ and $r^{0}=e$ (we use the $n$ residue classes to account the cyclic nature of the permutation of symmetries on $n$-gons, where each vertex can be represented by the class and labeled by the representative number of that class). Thus, $r(r): \overline{i+1}\to\overline{i+2}$ and so, by induction, $r^{m}: \bar{i}\to \overline{i+m}$ for $m\in \left\{ 0,\dots, n-1\right\}$. Clearly, if $m,k<n$ and  $m\neq k$, then $\bar{m}\neq \bar{k}$ and so $r^{m}\neq r^{k}$. Furthermore, $r^{n}:\bar{i}\to\overline{i+n} = e:\bar{i}\to\bar{i}$. Therefore, $|r|=n$.  
\end{proof}
\item $|s|=2$
\begin{proof}
  We know that $s:\overline{1+i}\to\overline{1-i}$ for all $0\leq i \leq n-1$ and $s^{0} = e$. Note that $\overline{1-1} = \overline{0} = \overline{1+(n-1)}$ and so $s(s):\overline{1+(n-1)}\to\overline{-n+2}$. Because $\overline{-n+2} = \overline{2}$, it follows that $s^{2}$ brings the point initially at vertex $\overline{2}$ back to $\overline{2}$. Furthermore, by definition, the vertex at position $\overline{1}$ remains intact. Thus, since the position of two adjacent vertices determines the position of the other points on the $n$-gon, $s^{2}=e$.
\end{proof}
\item $s\neq r^{i}$ for any nonnegative integer $i$.
\begin{proof}
  We know that $s$ mantain one point ate vertex $\overline{1}$ and moves the position of the point at vertex $\overline{i+1}$. Also, any permutation of $r$ either changes the position of all points or keeps them intact $r^{e}$. Therefore, $s\neq r^{i}$ for any nonnegative $i$. 
\end{proof}
\item $sr^{i} \neq sr^{i}$ for $0\leq i\neq j<n$.
\begin{proof}
  Since $\overline{i}\neq \overline{j}$ for $0\leq i,j<n$, it follows that $r^{i}\neq r^{j}$. Therefore, they have at least on vertex with different points and so $sr^{i}\neq sr^{j}$. 
\end{proof}
\item $rs = r^{-1}s$
\begin{proof}
  Note that $s$ moves points $1$ and $2$ to vertices $\overline{1}$ and $\overline{n}$, respectively. Then $rs$ moves points $1$ and $2$ to vertices $\overline{2}$ and $\overline{1}$, respectively.\\

  Now, $r^{-1}$ brings point $1$ and $2$ to vertices $\overline{n}$ and $\overline{1}$, respectively. Then, $sr^{-1}$ moves points $1$ and $2$ to vertices $\overline{2}$ and $\overline{1}$. \\

  Thus, $rs = sr^{-1}$.
\end{proof}
\item $r^{i}s = sr^{-i}$ for any $i$.
\begin{proof}
  It suffices to show that $r^{i}s = s^{-i}$ for any nonegative integer $i$.Clearly, $r^{0}s = sr^{-0} = s$.  Now, we proceed by induction. We know that $rs = sr^{-1}$. Now, assume that $r^{k}s = sr^{-k}$ for any $1\leq k$. Then, 
\begin{align*}
  r^{k+1}s &= r(r^{k}s) = r(sr^{-k})\\
  &= (rs)r^{-k}= sr^{-1}r^{-k} = sr^{-(k+1)}.
\end{align*}
\end{proof}
\end{enumerate}

Note that $(6)$ is a relation that (6) is a commutative relation, which show a way to move powers of an element of a binary operation to the opposite side. Furthermore, any permutation can be reduced to $s^{k}r^{i}$, where $k\in\left\{ 0,1 \right\}$ and $i\in \left\{ 0,\dots,n-1 \right\}$, with relations (1), (2) and (6).\\

Furthermore, there are two important concepts. A \textbf{generator} is any set $S\subset G$ of some group $G$, such that every element in $G$ can be expressed as a finite product of elements and their inverses of $S$. This set comes with \textbf{Relations} between its elements, there can be many, however, the idea is to look for the lowest quantity of relations to reduce any product of $G$ to a finite product of the elements and their inverses in $S$. Thus, any group can have a representation 
\begin{align*}
  G= <S| R_{1},R_{2},\dots,R_{n}>.
\end{align*}
Its important to note that the choice of representation for a group is important, because, depending on the relations chosen, they can be interwined in such a way that a lot of collapsing occurs. BY collapsing, I believe, the authour means that at first it gives an impression of a limit of elements in the group just to end up realizing that a lot of them can be reduced, lowering the limit. \\

In the following excercises, $D_{2n}$ has the usual presentation $D_{2n}=<r,s|r^{n}=s^{2}=1,rs=sr^{-1}>$.
\begin{problem}{2}
  Use the generators and relations above to show that if $x$ is any element of $D_{2n}$, which is not a power of $r$, then $rx = xr^{-1}$.
\begin{proof}
  If $x$ is not a power of $r$, then $x=sr^{k}$ for some $k\in\Z$. Then, 
\begin{align*}
  rx &= r\left( sr^{k} \right) = \left( rs \right)r^{k}\\
  &= sr^{-1}r^{k} = sr^{k-1} = sr^{k}r^{-1} = xr^{-1}.
\end{align*}
\end{proof}
\end{problem}
\begin{problem}{3}
  Every element of $D_{2n}$, which is not a power of $r$ has order 2. Deduce that $D_{2n}$ is generated by the two elements $s$ and $sr$, both having order 2. 
\begin{proof}
  First, we show that every element of $D_{2n}$ that is not a power of $r$ has order $2$. Any such element can be reduced to the general expression $sr^{k}\neq e$ for $k\in\left\{ 0,1,\dots,n-1 \right\}$. Then, 
\begin{align*}
  (sr^{k})^{2} &= (sr^{k})(sr^{k}) = (sr^{k})(r^{-k}s)\\
  &= s(r^{k}r^{-k})s = s^{2} = e.
\end{align*}
Now, we proove that every element of $D_{2n}$ can be expressed as a finite product of the elements $s$ and $sr$, including their inverses. First, consider any product of $r$. Then, 
\begin{align*}
  r^{k} &= (s\cdot sr)^{k}
\end{align*}
and so
\begin{align*}
  sr^{k} &= s(s\cdot sr)^{k}.
\end{align*}
Using (4) one can conclude that all elements of $D_{2n}$ can be reduced to powers of $r$ and these multiplied by $s$. Thus, $S=\left\{ s,sr \right\}$ generates $D_{2n}$.
\end{proof}
\end{problem}
\begin{problem}{4}
  If $n=2k$ is even and $n\geq 4$, show that $z=r^{k}$ is an element of order 2 which commutes with all elements of $D_{2n}$. Show also that $z$ is the only nonidentity element of $D_{2n}$ which commutes with all elements of $D_{2n}$. 
\begin{proof}
  First we show the following Lemma, 
\begin{lemma}
  Considering elements of $D_{2n}$, any nonidentity power of $r^{k}$ $(1\leq k \leq n-1)$ commutes with some nonpower of $r$ if and only if $r^{k}= r^{-k}$.
\begin{proof}
  If $r^{k} = r^{-k}$, it follows that $|r^{k}|=2$. Furthermore, every nonpower of $r$ has order 2 and so any nonpower of $r$ commutes with $r^{k}$.\\
  For the converse, assume that $r^{k}$ $(1\leq k \leq n-1)$ commutes with any nonpower of $r$. Then, for any $0\leq i\leq n-1$,
\begin{align*}
  r^{k}(sr^{i}) &= (sr^{i})r^{k} \implies sr^{-k}r^{i} = (sr^{i})r^{k} \implies\\
  sr^{i-k} &= sr^{i+k} \implies r^{-k}r^{i} = r^{k}r^{i} \implies r^{-k} = r^{k}.
\end{align*}
\end{proof}
\end{lemma}
By the relations of the $D_{2n}$ group, we know that $|r|=n=2k$. Then, $|r^{k}|=2$ and so it is the only nonidentity power of $r$ with order 2 (Recall the bijection from $\N$ to the even integers). Then, it is the only nonidentity power of $r$ that is equal to its inverse.  Thus, by Lemma, $r^{k}$ is the only nonidentity power of $r$ such that every nonpower of $r$ commutes with. Therefore, $r^{k}$ is the only nonidentity element that commutes with all elements.   
\end{proof}
\end{problem}
       \end{document}


