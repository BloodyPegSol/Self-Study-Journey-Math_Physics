\documentclass[12pt]{article}
\usepackage[margin=1in]{geometry}
\usepackage{amsmath, amsfonts,amsthm,amssymb,epigraph,etoolbox,mathtools,setspace,enumitem}  
\usepackage{tikz}
\usetikzlibrary{datavisualization} 
\usepackage[makeroom]{cancel} 
\usepackage[linguistics]{forest}
\usetikzlibrary{patterns}
\newcommand{\N}{\mathbb{N}}
\newcommand{\Z}{\mathbb{Z}}
\newcommand{\R}{\mathbb{R}}
\newcommand{\Q}{\mathbb{Q}}
\newcommand{\Mod}[1]{\ (\mathrm{mod}\ #1)}
\newcommand{\Lim}[1]{\mathrm{lim}(#1)}
\newcommand{\Abs}[1]{\left\vert #1 \right\vert}
\newcommand{\Dom}[1]{\mathrm{dom}\left(#1\right)}
\newcommand{\Range}[1]{\mathrm{range}(#1)}

\newlist{legal}{enumerate}{10}
\setlist[legal]{label=(\alph*)}
\setenumerate[legal]{label=(\alph*)}

\DeclarePairedDelimiter\bra{\langle}{\rvert}
\DeclarePairedDelimiter\ket{\lvert}{\rangle}
\DeclarePairedDelimiterX\braket[2]{\langle}{\rangle}{#1\delimsize\vert #2}


\newenvironment{theorem}[2][Theorem]{\begin{trivlist} \item[\hskip \labelsep {\bfseries #1}\hskip \labelsep {\bfseries #2.}]}{\end{trivlist}}
\newenvironment{lemma}[2][Lemma]{\begin{trivlist} \item[\hskip \labelsep {\bfseries #1}\hskip \labelsep {\bfseries #2.}]}{\end{trivlist}}
\newenvironment{result}[2][Result]{\begin{trivlist} \item[\hskip \labelsep {\bfseries #1}\hskip \labelsep {\bfseries #2.}]}{\end{trivlist}}
\newenvironment{exercise}[2][Exercise]{\begin{trivlist} \item[\hskip \labelsep {\bfseries #1}\hskip \labelsep {\bfseries #2.}]}{\end{trivlist}}
\newenvironment{problem}[2][Problem]{\begin{trivlist} \item[\hskip \labelsep {\bfseries #1}\hskip \labelsep {\bfseries #2.}]}{\end{trivlist}}
\newenvironment{question}[2][Question]{\begin{trivlist} \item[\hskip \labelsep {\bfseries #1}\hskip \labelsep {\bfseries #2.}]}{\end{trivlist}}
\newenvironment{corollary}[2][Corollary]{\begin{trivlist} \item[\hskip \labelsep {\bfseries #1}\hskip \labelsep {\bfseries #2.}]}{\end{trivlist}}
\newenvironment{solution}[1][Solution]{\begin{trivlist} \item[\hskip \labelsep {\bfseries #1}]}{\end{trivlist}}

\setlength\epigraphwidth{8cm}
\setlength\epigraphrule{0pt}

\makeatletter
\patchcmd{\epigraph}{\@epitext{#1}}{\itshape\@epitext{#1}}{}{}
\makeatother

\begin{document}
  
\title{Section 1.7: Group Actions}
   \author{Juan Patricio Carrizales Torres}
     \date{Jul 10, 2023}
       \maketitle

       Intuitively speaking, a \textbf{Group Action} by some group $G$ on a set $A$, is some type of ``action'' that permutes the elements of $A$ in such a manner that group operations are maintained. More precisely, a \textbf{group action} by some group $G$ on a set $A$ is a map $\varphi: G\times A \to A$ such that for every $(g,a)\in G\times A$, 
\begin{enumerate}
  \item $\varphi((g,a)) = ga \in A$.
  \item $g_{1}\circ(g_{2}a) = (g_{1}\circ g_{2})a$.
  \item $1a = a$.
\end{enumerate}
Furthermore, if we let $\sigma_{g}:A\to A$ be a representation of the group action og the element $g$ on $A$, defined by $\sigma_{g}(a) = ga$, then 
\begin{enumerate}
  \item $\sigma_{g} \in S_{A}$, namely, a symmetric permutation of $A$.
  \item The map $G\to S_{A}$ defined by $g \to sigma_{g}$ is homomorphic.
\end{enumerate}
This map $G\to S_{A}$ is called \textit{permutation representation} associated to the given action. Where there is a bijective correspondence between the group action  $G\to A$ and its permutation representation $G\times A \to A$. This is very usefull, since it implies that both are the same thing but expressed differently.\\

Other important concepts are the ``faithful'' characteristic and the kernel. A permutation representation is \textbf{faithfull} if it is injective. A kernel of a group action is the set
\begin{align*}
  A = \left\{ b\in G :  gb = b, b\in A\right\} \supset \sigma_{1} = i_{A}.
\end{align*}
Note that the left cancellation law in groups $(a\circ b = a\circ c \implies b=c)$ implies that the group action of a group $G$ over itself defined by \textit{left multplication} is \textbf{faithful}.
       \end{document}


