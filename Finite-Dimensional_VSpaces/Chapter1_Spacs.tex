\documentclass[12pt]{article}
\usepackage[margin=1in]{geometry}
\usepackage{amsmath, amsfonts,amsthm,amssymb,epigraph,etoolbox,mathtools,setspace,enumitem}  
\usepackage{tikz}
\usetikzlibrary{datavisualization} 
\usepackage[makeroom]{cancel} 
\usepackage[linguistics]{forest}
\usetikzlibrary{patterns}
\newcommand{\N}{\mathbb{N}}
\newcommand{\Z}{\mathbb{Z}}
\newcommand{\R}{\mathbb{R}}
\newcommand{\Q}{\mathbb{Q}}
\newcommand{\Mod}[1]{\ (\mathrm{mod}\ #1)}
\newcommand{\Lim}[1]{\mathrm{lim}(#1)}
\newcommand{\Abs}[1]{\left\vert #1 \right\vert}
\newcommand{\Dom}[1]{\mathrm{dom}\left(#1\right)}
\newcommand{\Range}[1]{\mathrm{range}(#1)}

\newlist{legal}{enumerate}{10}
\setlist[legal]{label=(\alph*)}
\setenumerate[legal]{label=(\alph*)}

\DeclarePairedDelimiter\bra{\langle}{\rvert}
\DeclarePairedDelimiter\ket{\lvert}{\rangle}
\DeclarePairedDelimiterX\braket[2]{\langle}{\rangle}{#1\delimsize\vert #2}


\newenvironment{theorem}[2][Theorem]{\begin{trivlist} \item[\hskip \labelsep {\bfseries #1}\hskip \labelsep {\bfseries #2.}]}{\end{trivlist}}
\newenvironment{lemma}[2][Lemma]{\begin{trivlist} \item[\hskip \labelsep {\bfseries #1}\hskip \labelsep {\bfseries #2.}]}{\end{trivlist}}
\newenvironment{result}[2][Result]{\begin{trivlist} \item[\hskip \labelsep {\bfseries #1}\hskip \labelsep {\bfseries #2.}]}{\end{trivlist}}
\newenvironment{exercise}[2][Exercise]{\begin{trivlist} \item[\hskip \labelsep {\bfseries #1}\hskip \labelsep {\bfseries #2.}]}{\end{trivlist}}
\newenvironment{problem}[2][Problem]{\begin{trivlist} \item[\hskip \labelsep {\bfseries #1}\hskip \labelsep {\bfseries #2.}]}{\end{trivlist}}
\newenvironment{question}[2][Question]{\begin{trivlist} \item[\hskip \labelsep {\bfseries #1}\hskip \labelsep {\bfseries #2.}]}{\end{trivlist}}
\newenvironment{corollary}[2][Corollary]{\begin{trivlist} \item[\hskip \labelsep {\bfseries #1}\hskip \labelsep {\bfseries #2.}]}{\end{trivlist}}
\newenvironment{solution}[1][Solution]{\begin{trivlist} \item[\hskip \labelsep {\bfseries #1}]}{\end{trivlist}}

\setlength\epigraphwidth{8cm}
\setlength\epigraphrule{0pt}

\makeatletter
\patchcmd{\epigraph}{\@epitext{#1}}{\itshape\@epitext{#1}}{}{}
\makeatother

\begin{document}
  
\title{Chapter 1: Spaces}
\author{Juan Patricio Carrizales Torres}
\date{Sep 29, 2022}
\maketitle
\section{Fields}

In Linear Algebra, we will be working with numbers from any type of class/set. Hence, to simplify things and make them more general, we will introduce the idea of fields. A \textbf{field} is a set of objects (including numbers) called \textbf{scalars} with operations of addition and multiplication that fulfill the following rules (let $\alpha$ and $\beta$ be scalars):
\begin{enumerate}
  \item \textbf{Addition}
\begin{enumerate}
  \item commutatitivity, $\alpha+\beta = \beta + \alpha$.
  \item associativity, $\alpha + (\beta + \gamma) = (\alpha + \beta) + \gamma$.
  \item additive identity, there is a unique scalar $0$ such that for every scalar $\alpha$, $\alpha + 0 = \alpha$.
  \item additive inverse, for each scalar $\alpha$ there is a unique scalar $-\alpha$ such that $\alpha + (- \alpha) = 0$.
\end{enumerate}
\item \textbf{Multiplication}
\begin{enumerate}
  \item commutativity, $\alpha\beta = \beta\alpha$.
  \item associativity, $\gamma(\alpha\beta) = (\gamma\alpha)\beta$.
  \item multiplicative identity, there is a unique scalar $1$ for every scalar $\alpha$ such that $1\alpha = \alpha$.
  \item multiplicative inverse, for every nonzero scalar $\beta$, there is a unique $\beta^{-1}$ such that $\beta\beta^{-1}=1$.
\end{enumerate}
\item \textbf{Linearity}
\begin{enumerate}
  \item Multiplication is distributive over addition, $\alpha(\beta + \gamma) = \alpha\beta + \alpha\gamma$.
\end{enumerate}
\end{enumerate}
For instance, the class of real numbers and the class of complex numbers are fields.
\subsection{Excercises}
\begin{problem}{1}
  Almost all the laws of elementary arithmetic are consequences of the axioms defining a field. Prove, in particular, that if $\mathcal{F}$ is a field, and if $\alpha,\beta$ and $\gamma$ belong to $\mathcal{F}$, then the following relations hold.
\begin{enumerate}
  \item $0+\alpha = \alpha$
\begin{proof}
  Due to the commutativity property of addition, $\alpha = \alpha + 0 = 0 + \alpha$.
\end{proof}
  \item If $\alpha + \beta = \alpha + \gamma$, then $\beta = \gamma$.
\begin{proof}
  Due to the additive inverse, associativity and commutativity, $\alpha + \beta + (-\alpha) = \alpha + (\beta + (-\alpha)) = ( \alpha + (-\alpha))+\beta = \beta = \gamma$.
\end{proof}
  \item $\alpha + (\beta-\alpha) = \beta$.
\begin{proof}
  Just like in $(b)$, 
\begin{align*}
  \alpha + (\beta+(-\alpha)) &= \alpha + (-\alpha + \beta)\\
  &= (\alpha + (-\alpha)) +\beta = 0 + \beta\\
  &= \beta.
\end{align*}
\end{proof}
\item $\alpha\cdot 0 = 0\cdot \alpha = 0$. (In this case, the dot indicates multiplication).
\begin{proof}
  Note that
\begin{align*}
  0\cdot \alpha + (-0\cdot\alpha) &= 0 = (0+0)\alpha + (-0\cdot\alpha)\\
  &= 0\cdot\alpha + (0\cdot\alpha + (-0\cdot\alpha)) = 0\cdot\alpha\\
  &= \alpha\cdot 0
\end{align*}
\end{proof}
\item $(-1)\alpha = -\alpha$
\begin{proof}
  Observe that
\begin{align*}
  \alpha + (-\alpha) &= 0 = 0\alpha\\
  &= (1-1)\alpha = \alpha + (-1)\alpha.
\end{align*}
By $(b)$, $-\alpha = (-1)\alpha$.
\end{proof}
\item $(-\alpha)(-\beta) = \alpha\beta$.
\begin{proof}
  By $(e)$, $(-\alpha)(-\beta) = ((-1)\alpha)((-1)\beta)$. Then,
\begin{align*}
  ( (-1)\alpha)( (-1)\beta) &= (-1)(\alpha( (-1)\beta))\\
  &= (-1)( (\alpha(-1))\beta) = (-1)( ((-1)\alpha)\beta)\\
  &= ( (-1)( (-1)\alpha))\beta = ( ( (-1)(-1))\alpha)\beta\\
&= (1\alpha)\beta = \alpha\beta
\end{align*}
\end{proof}
\item $\alpha\beta = 0 \implies \alpha = 0 \text{ or }\beta=0$.
\begin{proof}
  Let $\alpha\beta = 0$. Note that either $\alpha = 0$ or $\alpha \neq 0$. In the first case, the result is true. In the case of the latter, there is some $\alpha^{-1}$ and so $\alpha^{-1}\alpha\beta = 1\beta = \alpha^{-1}0 = 0$. 
\end{proof}
\end{enumerate}
\end{problem}

\end{document}


