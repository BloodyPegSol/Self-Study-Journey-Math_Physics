\documentclass[12pt]{article}
\usepackage[margin=1in]{geometry}
\usepackage{amsmath, amsfonts,amsthm,amssymb,epigraph,etoolbox,mathtools,setspace,enumitem}  
\usepackage{tikz}
\usetikzlibrary{datavisualization} 
\usepackage[makeroom]{cancel} 
\usepackage[linguistics]{forest}
\usetikzlibrary{patterns}
\newcommand{\N}{\mathbb{N}}
\newcommand{\Z}{\mathbb{Z}}
\newcommand{\R}{\mathbb{R}}
\newcommand{\Q}{\mathbb{Q}}
\newcommand{\Mod}[1]{\ (\mathrm{mod}\ #1)}
\newcommand{\Lim}[1]{\mathrm{lim}(#1)}
\newcommand{\Abs}[1]{\left\vert #1 \right\vert}
\newcommand{\Dom}[1]{\mathrm{dom}\left(#1\right)}
\newcommand{\Range}[1]{\mathrm{range}(#1)}

\newlist{legal}{enumerate}{10}
\setlist[legal]{label=(\alph*)}
\setenumerate[legal]{label=(\alph*)}

\DeclarePairedDelimiter\bra{\langle}{\rvert}
\DeclarePairedDelimiter\ket{\lvert}{\rangle}
\DeclarePairedDelimiterX\braket[2]{\langle}{\rangle}{#1\delimsize\vert #2}


\newenvironment{theorem}[2][Theorem]{\begin{trivlist} \item[\hskip \labelsep {\bfseries #1}\hskip \labelsep {\bfseries #2.}]}{\end{trivlist}}
\newenvironment{lemma}[2][Lemma]{\begin{trivlist} \item[\hskip \labelsep {\bfseries #1}\hskip \labelsep {\bfseries #2.}]}{\end{trivlist}}
\newenvironment{result}[2][Result]{\begin{trivlist} \item[\hskip \labelsep {\bfseries #1}\hskip \labelsep {\bfseries #2.}]}{\end{trivlist}}
\newenvironment{exercise}[2][Exercise]{\begin{trivlist} \item[\hskip \labelsep {\bfseries #1}\hskip \labelsep {\bfseries #2.}]}{\end{trivlist}}
\newenvironment{problem}[2][Problem]{\begin{trivlist} \item[\hskip \labelsep {\bfseries #1}\hskip \labelsep {\bfseries #2.}]}{\end{trivlist}}
\newenvironment{question}[2][Question]{\begin{trivlist} \item[\hskip \labelsep {\bfseries #1}\hskip \labelsep {\bfseries #2.}]}{\end{trivlist}}
\newenvironment{corollary}[2][Corollary]{\begin{trivlist} \item[\hskip \labelsep {\bfseries #1}\hskip \labelsep {\bfseries #2.}]}{\end{trivlist}}
\newenvironment{solution}[1][Solution]{\begin{trivlist} \item[\hskip \labelsep {\bfseries #1}]}{\end{trivlist}}

\setlength\epigraphwidth{8cm}
\setlength\epigraphrule{0pt}

\makeatletter
\patchcmd{\epigraph}{\@epitext{#1}}{\itshape\@epitext{#1}}{}{}
\makeatother

\begin{document}
  
 \title{Chapter 1: Vector Spaces}
   \author{Juan Patricio Carrizales Torres}
     \date{Aug 17, 2022}
       \maketitle


       \begin{problem}{1.1}
	Let $\mathcal{V}$ be a vector space over $\mathbb{F}$. Show that if $\alpha,\beta \in \mathbb{F}$ and if $\mathbf{v}$ is a nonzero vector in $\mathcal{V}$, then $\alpha \mathbf{v} = \beta \mathbf{v} \implies \alpha = \beta$. [HINT: $\alpha - \beta \neq 0 \implies \mathbf{v} = (\alpha - \beta)^{-1}(\alpha - \beta)\mathbf{v}$.]
    \begin{proof}
      Suppose, to the contrary, that there are distinct $\alpha,\beta\in \mathbb{F}$ such that for some nonzero $\mathbf{v} \in \mathcal{V}$ we have $\alpha \mathbf{v} = \beta \mathbf{v}$. Then, $\alpha - \beta \neq 0$ and so $\mathbf{v} = (\alpha - \beta)^{-1}(\alpha - \beta)\mathbf{v}$. Hence,  
    \begin{equation*}
      \mathbf{v} = (\alpha-\beta)^{-1} \alpha \mathbf{v} - (\alpha-\beta)^{-1} \beta \mathbf{v}  = (\alpha-\beta)^{-1} (\alpha \mathbf{v} - \beta \mathbf{v}).
    \end{equation*}
    Since $\alpha \mathbf{v} = \beta \mathbf{v}$, it follows that $\alpha \mathbf{v} - \beta \mathbf{v} = \beta \mathbf{v} - \beta \mathbf{v} = \mathbf{0}$. This implies that $\mathbf{v} = (\alpha - \beta)^{-1} \mathbf{0} = \mathbf{0}$. This is a contradiction to our assumption that $\mathbf{v}$ was nonzero.\\

    Another way to prove this directly is by using the fact, for some $\alpha \in \mathbb{F}$ and nonzero vector $\mathbf{v}$, that $\alpha\mathbf{v} = \mathbf{0} \implies \alpha = 0$. A proof reads as follows:\\

    Let $\alpha,\beta \in \mathbb{F}$ and $\mathbf{v}\in \mathcal{V}$ be some nonzero vector such that $\alpha \mathbf{v} = \beta \mathbf{v}$. Then, $\alpha \mathbf{v} - \beta \mathbf{v} = \beta \mathbf{v} - \beta \mathbf{v} = \mathbf{0}$ and so $(\alpha - \beta) \mathbf{v} = \mathbf{0}$. Since $\mathbf{v}$ is nonzero, it follows that $\alpha - \beta = 0$ and so $\alpha = \beta$.
    \end{proof}
    \end{problem}
    \begin{problem}{1.2}
      Show that the space $\R^{3}$ endowed with the rule
    \begin{equation*}
      \mathbf{x}\; \square \; \mathbf{y} = \begin{bmatrix} \max(x_{1},y_{1}) \\ \max(x_{2},y_{2}) \\ \max(x_{3},y_{3}) \end{bmatrix}
    \end{equation*}
    for vector addition and the usual rule for scalar multiplication is not a vector space over $\R$. 
    \begin{proof}
      We show that this space has no unique additive identity. Consider some $\mathbf{x} = (x_{1},x_{2},x_{3})$. Then, both $\mathbf{y} = (x_{1}-1,x_{2}-1,x_{3}-1)$ and $\mathbf{z} = (x_{1}-2, x_{2}-2, x_{3}-2)$ are in $\R^{3}$ and they are distinct. Note that $\mathbf{x} \; \square \; \mathbf{y} = \mathbf{x}$ and $\mathbf{x} \; \square \; \mathbf{z} = \mathbf{x}$. \\

      In fact, one can easily show that there is no vector that is an additive inverse of every vector (the zero vector $\mathbf{0}$) since one can easily construct a vector with elements lower than the ones from any other vector.
    \end{proof}
    \end{problem}
    \begin{problem}{1.3}
      Let $\mathcal{C}\subset \R^{3}$ denote the set of vectors $\mathbf{a} = \begin{bmatrix} a_{1}\\ a_{2}\\ a_{3}\end{bmatrix}$ such that th polynomial $a_{1} + a_{2}t + a_{3}t^{2} \geq 0$ for every $t\in \R$. Show that it is closed under vector addition (i.e., $\mathbf{a},\mathbf{b} \in \mathcal{C} \implies \mathbf{a}+\mathbf{b} \in \mathcal{C}$), but that $\mathcal{C}$ is not a vector space over $\R$. [REMARK: A set $\mathcal{C}$ with the indicated two properties is called a $\mathbf{cone}$.]
    \begin{proof}
      We first show that $\mathcal{C}$ is closed under addition. Consider any $\mathbf{a}, \mathbf{b}\in \mathcal{C}$. Then, for every $t\in \R$ we have $a_{1} + a_{2}t + a_{3}t^{2} \geq 0$ and $b_{1}+b_{2}t+b_{3}t^{2}\geq 0$. Then, 
    \begin{align*}
      a_{1} + a_{2}t + a_{3}t^{2} + b_{1}+b_{2}t+b_{3}t^{2} &= (a_{1} + b_{1}) + (a_{2} + b_{2})t + (a_{3}+b_{3})t^{2} \geq 0
    \end{align*}
    for every $t\in \R$. Thus, $\mathbf{a} + \mathbf{b} = \begin{bmatrix} a_{1}+b_{1}\\ a_{2} + b_{2}\\ a_{3}+b_{3} \end{bmatrix} \in \mathcal{C}$. However, it is not close under scalar multiplication. Consider some nonzero $\mathbf{v} \in \mathcal{C}$ and let $\alpha = -1$. Since $v_{1} + v_{2}t + v_{3}t^{2} \geq 0$ for every $t\in \R$, it follows that $(-v_{1}) + (-v_{2})t + (-v_{3})t^{2}<0$ for every positive $t$. Hence, $(-1)\mathbf{v} \not\in \mathcal{C}$ and so it is not a vector space over $\R$.
    \end{proof}
    \end{problem}
    \begin{problem}{1.5}
      Let $\mathcal{F}$ denote the set of continuous real-valued functions $f(x)$ on the interval $0\leq x\leq 1$. Show that $\mathcal{F}$ is a vector space over $\R$ with respect to the natural rules of vector addition $\left( \left( f_{1} + f_{2} \right)(x) = f_{1}(x) + f_{2}(x) \right)$ and scalar multiplication $\left( (\alpha f)(x) = \alpha f(x) \right)$.
      \begin{proof}
      \begin{enumerate}
	\item \textbf{Closed under vector addition}

	  Consider two functions $f,g \in \mathcal{F}$. Let $x\in [0,1]$. Then, $f(x), g(x) \in \R$ and so $(f + g)(x) = f(x) + g(x) \in \R$ since $\R$ is closed under addition. Therefore, $f+g$ is a real-valued function on the interval $[0,1]$ and so $(f+g)\in \mathcal{F}$.
	\item \textbf{Closed under scalar multiplication}

	  Consider some function $f\in \mathcal{F}$ and real number $\alpha$. Let $x\in [0,1]$. Then, $f(x)\in \R$ and so $ (\alpha f)(x) = \alpha f(x) \in \R$ since $\R$ is closed under multiplication. Thus, $\alpha f$ is a real-valued function on the interval $[0,1]$ and so $\alpha f \in \mathcal{F}$.  
	\item \textbf{Vector addition is commutative}

	  Let $f,g\in \mathcal{F}$ and $x\in [0,1]$. Then, $(f+g)(x) = f(x) + g(x) = g(x) + f(x) = (g+f)(x)$ since addition in the set of real numbers is commutative.
	\item \textbf{Vector addition is associative}

	  Let $f,g,h \in \mathcal{F}$ and $x\in [0,1]$. Then, $\left( \left( f+g \right)+h \right)(x) = (f+g)(x) + h(x) = f(x)+g(x)+h(x) = f(x) + (g+h)(x) = \left( f +\left( g+h\right) \right)(x)$ since addition in $\R$ is associative (the order of addition does not matter).
	\item \textbf{Existence of additive identity}

	  Let $f:[0,1]\to \R$ be defined by $f(x) = 0$ for all $x\in [0,1]$. Then, $f$ is a continous real-valued function and so $f\in \mathcal{F}$. Consider any $g\in \mathcal{F}$ and let $a\in [0,1]$. Then, $(f + g)(a) = f(a) + g(a) = 0+ g(a) = g(a)$ since $0$ is the additive identity  of real numbers. Thus, $f$ is an additive identitive in $\mathcal{F}$. 
	\item \textbf{Existence of additive inverse}

	  Consider some $f\in \mathcal{F}$. Let $g:[0,1]\to \R$ be defined by $g(x) = -f(x)$ for all $x\in [0,1]$. Consider some $x\in [0,1]$ and so $(f+g)(x) = f(x) + g(x) = f(x) -f(x) = 0$. Hence, $g$ is the additive inverse of $f$.
	\item $f \in \mathcal{F} \implies (1)f = f$
	  
	  Let $f\in \mathcal{F}$. Consider any $x\in[0,1]$ and so $f(x) = (1)f(x)$. Thus, $f = (1)f$. 
      \item For any $\alpha,\beta \in \R$ and vector $f \in \mathcal{F}$, $\alpha(\beta f) = (\alpha\beta) f$

       Let $f\in \mathcal{F}$ and $\alpha,\beta \in \R$. Consider any $x\in[0,1]$ and so $\alpha(\beta f)(x) = \alpha(\beta f(x)) = (\alpha\beta)f(x)$ since multiplication in $\R$ is associative. Thus, $\alpha(\beta f) = (\alpha \beta)f$
      \item For any $\alpha,\beta \in \R$ and vector $f \in \mathcal{F}$, $(\alpha + \beta) f = \alpha f + \beta f$

	Let $f\in \mathcal{F}$ and $\alpha,\beta \in \R$. Consider any $x\in [0,1]$ and so $(\alpha + \beta)f(x) = \alpha f(x) + \beta f(x) = (\alpha f + \beta f)(x)$ since multiplication over addition is distributive for real numbers.
    \end{enumerate}
    \end{proof}
    \end{problem}
    \begin{lemma}{1}
      Let $\mathcal{S}$ be a nonempty subset of a vector space $\mathcal{M}$ over $\mathbb{F}$. Then, $\mathcal{S}$ is a vector space if and only if for every pair of vectors $\mathbf{v}, \mathbf{a} \in \mathcal{S}$ and $\alpha, \beta\in \mathbb{F}$, $\alpha \mathbf{v} + \beta \mathbf{a} \in \mathcal{S}$.
    \end{lemma}
    \begin{proof}
	Assume that $\mathcal{S}$ is a vector space and so it is closed under addition and scalar multiplication. Let $\mathbf{v},\mathbf{a} \in \mathcal{S}$ and $\alpha, \beta \in \mathbb{F}$, then $\alpha \mathbf{v}, \beta \mathbf{a} \in \mathcal{S}$ and so $\alpha \mathbf{v} + \beta \mathbf{a} \in \mathcal{S}$.\\
	Suppose, for every pair of vectors $\mathbf{v}, \mathbf{a} \in \mathcal{S}$ and $\alpha, \beta\in \mathbb{F}$, that $\alpha \mathbf{v} + \beta \mathbf{a} \in \mathcal{S}$. Let $\alpha = 0$ and $\beta\in \mathbb{F}$. Consider any vectors $\mathbf{v},\mathbf{a} \in \mathcal{S}$. Then, $\alpha \mathbf{v} = 0$ is the additive identity of $\mathcal{M}$ and so $\beta\mathbf{a} = \alpha \mathbf{v} + \beta\mathbf{a} \in \mathcal{S}$. Thus, $\mathcal{S}$ is closed under scalar multiplication.\\
	Consider some vectors $\mathbf{v},\mathbf{a}\in \mathcal{S}$ and let $\alpha = \beta = 1$. Then, $\mathbf{v} + \mathbf{a} = (1)\mathbf{v} + (1)\mathbf{a} = \alpha\mathbf{v} + \beta \mathbf{a} \in \mathcal{S}$ since $\mathbf{v}, \mathbf{a} \in \mathcal{M}$. Therefore, $\mathcal{S}$ is closed under addition and so it is a vector space.
    \end{proof}
    \begin{problem}{1.6}
      Let $F_{0}$ denote the set of continuous real-valued functions $f(x)$ on the interval $0\leq x \leq 1$ that met the auxiliary constraints $f(0) =0$ and $f(1) = 0$. Show that $F_{0}$ is a vector space over $\R$ with respect to the natural rules of vector addition and scalar multiplication that were introducd in \textbf{Excercise 1.5} and that $F_{0}$ is a subspace of the vector space $\mathcal{F}$ that was considered there.
    \end{problem}
    \begin{proof}
      By definition, $F_{0} \subseteq \mathcal{F}$. Let's prove that it is closd under addition and scalar multiplication. Consider some $f, g\in F_{0}$ and $\alpha,\beta\in \R$. Then, $\alpha f + \beta g$ is a real-valued function since $f,g\in \mathcal{F}$. Particularly, $(\alpha f + \beta g)(0) = \alpha f(0) + \beta g(0) = 0 + 0 = \alpha f(1) + \beta g(1) = (\alpha f+ \beta g)(1)$ and so, by condition, it is a vector in $F_{0}$. Therefore, $F_{0}$ is a subspace of $\mathcal{F}$. 
    \end{proof}
    \begin{problem}{1.7}
      Let $F_{1}$ denote the set of continuous real-valued functions $f(x)$ on the interval $0\leq x \leq 1$ that meet the auxiliary constraints $f(0)=0$ and $f(1)=1$. Show that $F_{1}$ is not a vector space over $\R$ with respect to the natural rules of vector addition and scalar multiplication that were introduced in \textbf{Exercise 1.5}.
    \end{problem}
    \begin{proof}
      We know that $F_{1}\subseteq \mathcal{F}$. Consider some $f\in F_{1}$. Then, $(2)f$ is a continuous real-valued function since $f\in \mathcal{F}$. However, note that $(2f)(1) = (2)f(1) = 2 \neq 1$ and so $(2)f\not\in F_{1}$. Hence, $F_{1}$ is not closed under scalar multiplication and so $F_{1}$ is not a subspace of $\mathcal{F}$. 
    \end{proof}
\end{document}


