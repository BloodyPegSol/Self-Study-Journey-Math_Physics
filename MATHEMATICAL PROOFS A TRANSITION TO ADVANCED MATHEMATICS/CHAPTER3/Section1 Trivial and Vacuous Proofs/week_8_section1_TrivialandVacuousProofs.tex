\documentclass[12pt]{article}
\usepackage[margin=1in]{geometry}
\usepackage{amsmath,amsthm,amssymb,epigraph,etoolbox,mathtools,setspace,enumitem} 
\usepackage{tikz}
\usetikzlibrary{datavisualization}
\usepackage[makeroom]{cancel} 
\usepackage[linguistics]{forest}
\usetikzlibrary{patterns}
\newcommand{\N}{\mathbb{N}}
\newcommand{\Z}{\mathbb{Z}}
\newcommand{\R}{\mathbb{R}}
\newcommand{\Q}{\mathbb{Q}}

\newenvironment{theorem}[2][Theorem]{\begin{trivlist}
		\item[\hskip \labelsep {\bfseries #1}\hskip \labelsep {\bfseries #2.}]}{\end{trivlist}}
\newenvironment{lemma}[2][Lemma]{\begin{trivlist}
		\item[\hskip \labelsep {\bfseries #1}\hskip \labelsep {\bfseries #2.}]}{\end{trivlist}}
\newenvironment{exercise}[2][Exercise]{\begin{trivlist}
		\item[\hskip \labelsep {\bfseries #1}\hskip \labelsep {\bfseries #2.}]}{\end{trivlist}}
\newenvironment{problem}[2][Problem]{\begin{trivlist}
		\item[\hskip \labelsep {\bfseries #1}\hskip \labelsep {\bfseries #2.}]}{\end{trivlist}}
\newenvironment{question}[2][Question]{\begin{trivlist}
		\item[\hskip \labelsep {\bfseries #1}\hskip \labelsep {\bfseries #2.}]}{\end{trivlist}}
\newenvironment{corollary}[2][Corollary]{\begin{trivlist}
		\item[\hskip \labelsep {\bfseries #1}\hskip \labelsep {\bfseries #2.}]}{\end{trivlist}}
\newenvironment{solution}[2][Solution]{\begin{trivlist}
		\item[\hskip \labelsep {\bfseries #1}\hskip \labelsep {\bfseries #2.}]}{\end{trivlist}}

\setlength\epigraphwidth{8cm}
\setlength\epigraphrule{0pt}

\makeatletter
\patchcmd{\epigraph}{\@epitext{#1}}{\itshape\@epitext{#1}}{}{}
\makeatother


\begin{document}
	
	\title{Week 8}
	\author{Juan Patricio Carrizales Torres \\
		Section 1: Trivial and Vacuous Proofs}
	\date{September 15, 2021}
	\maketitle
	
	\begin{problem}{1}
		Let $x\in \R$. Prove that if $0<x<1$, then $x^{2}-2x+2\neq 0$.
		\begin{proof}
			Note that
			\begin{equation*}
				x^{2}-2x+1=(x-1)^{2}\geq 0
			\end{equation*}
			Thus, $x^{2}-2x+2 = (x-1)^{2}+1\geq 1 >0$. Hence, $x^{2}-2x+2\neq 0$ is true for all $x\in \R$ and the implication is true trivially.  
		\end{proof}
	\end{problem}
 	
 	\begin{problem}{2}
 		Let $n\in \N$. Prove that if $|n-1|+|n+1|\leq 1$, then $|n^{2}-1|\leq 4$.
 		\begin{proof}
 			 Note that for $n \in \N$, $|n+1|\geq 2 > 1$. Thus, $|n-1|+|n+1|\leq 1$ is false for all $n\in \N$ and the implication follows vacuously.
 		\end{proof}
 	\end{problem}
 	
 	\begin{problem}{3}
 		Let $r\in \Q^{+}$. Prove that if $\frac{r^{2}+1}{r}\leq 1$, then $\frac{r^{2}+2}{r}\leq 2$.
 		\begin{proof}
 			Note that $\frac{r^{2}+1}{r} = r+\frac{1}{r}$. If $r \geq 1$, then $r+\frac{1}{r}>1$. On the other hand, if $0<r<1$, then $\frac{1}{r}>1$ and $r+\frac{1}{r}>1$. Hence, $\frac{r^{2}+1}{r}\leq 1$ is false for all $r\in \Q^{+}$ and the implication follows vacuously. 
 		\end{proof}	
 	\end{problem}
  
 	\begin{problem}{4}
 		Let $x \in \R$. Prove that if $x^{3}-5x-1 \geq 0$, then $(x-1)(x-3)\geq -2$.
 		\begin{solution}{}
 			Note that $(x-1)(x-3)=x^{2}-4x+3$. Also, 
 			\begin{equation*}
 				x^{2}-4x+4 = (x-2)^{2}\geq 0
 			\end{equation*}
 		Thus, $x^{2}-4x+3 = (x-2)^{2}-1\geq -1$. Therefore, $(x-1)(x-3)\geq -2$ is true for all $x \in \R$ and this implication is true trivially.
 		\end{solution}
 	\end{problem}
 
 	\begin{problem}{5}
 		Let $n \in \N$. Prove that if $n+\frac{1}{n}<2$, then $n^{2}+\frac{1}{n^{2}}<4$.
 		\begin{solution}{}
 			For $n=1$, $n+\frac{1}{n}=2$. On the other cases, $n\geq 2$. Thus, $n+\frac{1}{n}<2$ is false for all $n \in \N$ and this implication follows vacuously.
 		\end{solution}
 	\end{problem}
 
 	\begin{problem}{6}
 		Prove that if $a$, $b$ and $c$ are odd integers such that $a+b+c=0$, then $abc<0$. (You are permitted to use well-known properties of integers here.)
 		\begin{solution}{}
 			Let $a=2m+1$, $b=2k+1$ and $c=2n+1$ for some $m,k,n\in \Z$. Note that $a+b+c = 2(m+k+n)+3$. The sum of three odd integers is an odd integer. Since $0$ is an even integer, it follows that $a+b+c=0$ is false for all odd integers $a,b,c$. This implication is true vacuously.
 		\end{solution} 
 	\end{problem}
 
 	\begin{problem}{7}
 		Prove that if $x$, $y$ and $z$ are three real numbers such that $x^{2}+y^{2}+z^{2}<xy+xz+yz$, then $x+y+z>0$.
 		\begin{solution}{}
 			Since $(x-y)^{2}+(y-z)^{2}+(z-x)^{2} \geq 0$, it follows that $2x^{2}+2y^{2}+2z^{2} - 2xy - 2yz - 2xz\geq 0$. Therefore, $x^{2}+y^{2}+z^{2}\geq xy+xz+yz$ for all $x,y,z \in \R$. This implication is true vacuously.
 		\end{solution}
 	\end{problem}
\end{document}