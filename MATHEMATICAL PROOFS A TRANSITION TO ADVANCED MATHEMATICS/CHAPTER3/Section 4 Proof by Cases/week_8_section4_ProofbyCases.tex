\documentclass[12pt]{article}
\usepackage[margin=1in]{geometry}
\usepackage{amsmath,amsthm,amssymb,epigraph,etoolbox,mathtools,setspace,enumitem} 
\usepackage{tikz}
\usetikzlibrary{datavisualization}
\usepackage[makeroom]{cancel} 
\usepackage[linguistics]{forest}
\usetikzlibrary{patterns}
\newcommand{\N}{\mathbb{N}}
\newcommand{\Z}{\mathbb{Z}}
\newcommand{\R}{\mathbb{R}}
\newcommand{\Q}{\mathbb{Q}}

\newenvironment{theorem}[2][Theorem]{\begin{trivlist}
		\item[\hskip \labelsep {\bfseries #1}\hskip \labelsep {\bfseries #2.}]}{\end{trivlist}}
\newenvironment{lemma}[2][Lemma]{\begin{trivlist}
		\item[\hskip \labelsep {\bfseries #1}\hskip \labelsep {\bfseries #2.}]}{\end{trivlist}}
\newenvironment{exercise}[2][Exercise]{\begin{trivlist}
		\item[\hskip \labelsep {\bfseries #1}\hskip \labelsep {\bfseries #2.}]}{\end{trivlist}}
\newenvironment{problem}[2][Problem]{\begin{trivlist}
		\item[\hskip \labelsep {\bfseries #1}\hskip \labelsep {\bfseries #2.}]}{\end{trivlist}}
\newenvironment{question}[2][Question]{\begin{trivlist}
		\item[\hskip \labelsep {\bfseries #1}\hskip \labelsep {\bfseries #2.}]}{\end{trivlist}}
\newenvironment{corollary}[2][Corollary]{\begin{trivlist}
		\item[\hskip \labelsep {\bfseries #1}\hskip \labelsep {\bfseries #2.}]}{\end{trivlist}}
\newenvironment{solution}[2][Solution]{\begin{trivlist}
		\item[\hskip \labelsep {\bfseries #1}\hskip \labelsep {\bfseries #2.}]}{\end{trivlist}}

\setlength\epigraphwidth{8cm}
\setlength\epigraphrule{0pt}

\makeatletter
\patchcmd{\epigraph}{\@epitext{#1}}{\itshape\@epitext{#1}}{}{}
\makeatother


\begin{document}
	
	\title{Week 9}
	\author{Juan Patricio Carrizales Torres \\
		Section 4: Proof by Cases}
	\date{September 21, 2021}
	\maketitle
	
Let $S$ be the domain of some open sentence $P(x)$. It can be useful to observe that an arbitrary $x\in S$ may have one of two or more properties (i.e. belong to a particular subset of $S$). If one proves that $P(x)$ is true for each property that $x$ can have, then $P(x)$ is shown to be true. Then the proof of $P(x)$ can be divided into cases, and, if necessary, each case into subcases. This method is called Proof by Cases.\\
When the proof of two cases is similar, one can use \textbf{without loss of generality} (WLOG) to indicate that only the proof of one situation is needed. 

\begin{problem}{26}
	Prove that if $n\in \Z$, then $n^{2}-3n+9$ is odd.
	\begin{proof}
		We proceed by cases, according to whether $n$ is even or odd.\\
		\textit{Case 1.} Assume $n$ is even. Then $n=2k$ for some $k\in \Z$. Therefore,
		\begin{equation*}
			(2k)^{2}-3(2k)+9=4k^{2}-6k+8+1=2(2k^{2}-3k+4)+1
		\end{equation*}
	Since $2k^{2}-3k+4\in \Z$, $n^{2}-3n+9$ is odd.\\
	\textit{Case 2.} Assume $n$ is odd. Then $n=2k+1$ for some $k\in \Z$. Thus,
	\begin{equation*}
		(2k+1)^{2}-3(2k+1)+9=4k^{2}+4k+1-6k-3+9=4k^{2}-2k+1+6 = 2(2k^{2}-k+3)+1
	\end{equation*}
	Because $2k^{2}-k+3\in \Z$, $n^{2}-3n+9$ is odd.
	\end{proof}
\end{problem}

\begin{problem}{27}
	Prove that if $n\in \Z$, then $n^{3}-n$ is even.
	\begin{proof}
		For this proof we consider two cases.\\
		\textit{Case 1.} Assume $n$ is even. Then $n=2k$ for some $k\in \Z$. Hence,
		\begin{equation*}
			(2k)^{3}-2k=8k^{3}-2k=2(4k^{3}-k)
		\end{equation*} 
		Since $4k^{3}-k$ is an integer, $n^{3}-n$ is even.\\
		\textit{Case 2.} Let $n$ be odd. Then $n=2k+1$ for some $k\in \Z$. Therefore,
		\begin{align*}
			(2k+1)^{3}-2k-1 &= 8k^{3}+12k^{2}+6k+1-2k-1\\
			&= 8k^{3}+12k^{2}+4k\\
			&= 2(4k^{3}+6k^{2}+2k)
		\end{align*}
	Because $4k^{3}+6k^{2}+2k$ is an integer, it follows that $n^{3}-n$ is even.
	\end{proof} 
\end{problem}

\begin{problem}{28}
	Let $x,y\in \Z$. Prove that if $xy$ is odd, then $x$ and $y$ are odd.
	\begin{proof}
		Assume that $x$ or $y$ is even. Without loss of generality, let $x$ be even. Then $x=2a$ for some $a\in \Z$. Therefore,
			$2a(y) = 2(ay)$.\\
	Since $ay\in \Z$, it follows that $xy$ is even.
	\end{proof}
\end{problem}

\begin{problem}{29}
	Let $a,b \in \Z$. Prove that if $ab$ is odd, then $a^{2}+b^{2}$ is even.
	\begin{proof}
		Assume $ab$ is odd. By result 28, $a$ and $b$ are odd. Therefore $a=2m+1$ and $b=2n+1$ for some $m,n\in  \Z$. Hence,
		\begin{align*}
			(2m+1)^{2}+(2n+1)^{2}&=4m^{2}+4m+1+4n^{2}+4n+1\\
			&= 4m^{2}+4m+4n^{2}+4n+2\\
			&= 2(2m^{2}+2m+2n^{2}+2n+1)
		\end{align*}
	Since $2m^{2}+2m+2n^{2}+2n+1 \in \Z$, it follows that $a^{2}+b^{2}$ is even.
	\end{proof}
\end{problem}

\begin{problem}{30}
	Let $x,y\in \Z$. Prove that $x-y$ is even if and only if $x$ and $y$ are of the same parity.\\
	
		\textbf{Lemma} If $n\in \Z$, then $n$ and $-n$ are of the same parity.
		\begin{proof}
			We consider two cases.\\
			\textit{Case 1.} Assume $n$ is even. Then $n=2k$ for some $k\in \Z$. Therefore, $-n=-(2k)=2(-k)$.
			Since $-k$ is an integer, it follows that $n$ is even.\\
			\textit{Case 2.} Let $n$ be odd. Then $n=2k+1$ for some $k\in \Z$. Thus, $-n=-(2k+1)=-2k-1=-2k-2+1=2(-k-1)+1$. Since $-k-1$ is an integer, $-n$ is odd.
		\end{proof}
	\begin{proof}
		We now proceed to prove the result.\\
		Let $x$ and $y$ be of opposite parity. By lemma, $x+(-y)$ is a sum of two integers of opposite parity. Therefore, by theorem 16, $x-y$ is odd.\\
		For the converse, assume $x$ and $y$ are of the same parity. By lemma, $x+(-y)$ is a sum of two integers of the same parity. Therefore, by theorem 16, $x-y$ is even.
	\end{proof}
\end{problem}

\begin{problem}{31}
	Let $a,b\in \Z$. Prove that if $a+b$ and $ab$ are of the same parity, then $a$ and $b$ are even.
	\begin{proof}
		Assume that $a$ or $b$ are odd. We consider two cases.\\
		\textit{Case 1.} Without loss of generality, let $b$ be odd and $a$ be even. Then $b=2m+1$ and $a=2k$ for some $m,k \in \Z$. Therefore,
		\begin{align*}
			a+b&=2k+2m+1=2(k+m)+1\\
			ab &= (2k)(2m+1)=4km+2k=2(2km+k)
		\end{align*}
	Since both $2km+k$ and $k+m$ are integers, $a+b$ is odd and $ab$ is even. They are of opposite parity.\\
	
	\textit{Case 2.} Let $a$ and $b$ be odd. Then $a=2k+1$ and $b=2m+1$ for some $k,m \in \Z$. Thus,
	\begin{align*}
		a+b &= 2k+1+2m+1=2(k+m+1)\\
		ab &= (2k+1)(2m+1) = 4km+2k+2m+1=2(2km+k+m)+1
	\end{align*}
	Because both $k+m+1$ and $2km+k+m$ are integers, $a+b$ is even and $ab$ is odd. They are of opposite parity.
	\end{proof}
\end{problem}

\begin{problem}{32}
	$(\text{a})$ Let $x$ and $y$ be integers. Prove that $(x+y)^{2}$ is even if and only if $x$ and $y$ are of the same parity.
	\begin{proof}
		For this proof we will use the following two theorems:\\
		\textbf{Theorem 12} Let $x\in \Z$. Then $x^{2}$ is even if and only if $x$ is even.\\
		\textbf{Theorem 16} Let $x,y\in \Z$. Then $x$ and $y$ are of the same parity if and only if $x+y$ is even.\\
		
		Assume $(x+y)^{2}$ is even. By theorem 12, $x+y$ is even. Thus, by theorem 16, $x$ and $y$ are of the same parity.\\
		For the converse, let $x$ and $y$ be of the same parity. By theorem 16, $x+y$ is even and so, by theorem 12, $(x+y)^{2}$ is even.
	\end{proof}

	$(\text{b})$ Restate the result in $(\text{a})$ in terms of odd integers.
	\begin{solution}{}
		Let $x,y\in \Z$. Then $(x+y)^{2}$ is odd if and only if $x$ and $y$ are of opposite parity.
	\end{solution}
\end{problem}

\begin{problem}{33}
	Let $A=\{1,2,3\}$ and $B=\{2,3,4\}$ be subsets of $S=\{1,2,3,4\}$. Let $n\in S$. Prove that $2n^{2}-5n$ is either $(\text{a})$ positive and even or $(\text{b})$ negative and odd if and only if $n\notin A\cap B$.
	\begin{proof}
		Let $n\in S$ such that $n\in A\cap B$. Then $n\in \{2,3\}$. If $n=2$, then $2n^{2}-5n=-2$ is even and negative; while, if $n=3$, then $2n^{2}-5n=3$ is odd and positive.\\
		For the converse, let $n\in S$ such that $n\notin A\cap B$. Then $n\in \{1,4\}$. If $n=1$, then $2n^{2}-5n=-3$ is negative and odd. If $n=4$, then $2n^{2}-5n=12$ is positive and even.
	\end{proof} 
\end{problem}

\begin{problem}{34}
	Let $A=\{3,4\}$ be a subset of $S=\{1,2,\ldots,6\}$. Let $n\in S$. Prove that if $\frac{n^{2}(n+1)^{2}}{4}$ is even, then $n\in A$.
	\begin{proof}
		Let $n\in S$ such that $n\notin A$. Then $n=\{1,2,5,6\}$. If $n=1$, then $n^{2}(n+1)^{2}/4=1$ is odd. If $n=2$, then $n^{2}(n+1)^{2}/4=9$ is odd. If $n=5$, then $n^{2}(n+1)^{2}/4=225$ is odd. If $n=6$, then $n^{2}(n+1)^{2}/4=441$ is odd. 
	\end{proof}
\end{problem}

\begin{problem}{35}
	Prove for every nonnegative integer $n$ that $2^{n}+6^{n}$ is an even integer.
	\begin{proof}
		Let $n$ be a nonnegative integer. We consider two cases for this proof.\\ 
		\textit{Case 1.} Assume $n\in \Z$ such that $n = 0$. Therefore, $2^{0}+6^{0}=2$ is an even integer.\\
		\textit{Case 2.} Let $n > 0$. Then $n-1 \geq 0$. Note that
		\begin{align*}
			2^{n}+6^{n} &= 2(2^{n-1})+ 6(6^{n-1})\\
						&= 2(2^{n-1} + 3(6^{n-1}))
		\end{align*}
	Since $n-1 \geq 0$, it follows that $2^{n-1} + 3(6^{n-1}) \in \Z$. Thus, $2^{n}+6^{n}$ is an even integer.
	\end{proof}
\end{problem}

\end{document}