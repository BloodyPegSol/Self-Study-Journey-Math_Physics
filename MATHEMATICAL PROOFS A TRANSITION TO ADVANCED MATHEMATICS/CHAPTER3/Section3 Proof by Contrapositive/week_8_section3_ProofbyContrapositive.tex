\documentclass[12pt]{article}
\usepackage[margin=1in]{geometry}
\usepackage{amsmath,amsthm,amssymb,epigraph,etoolbox,mathtools,setspace,enumitem} 
\usepackage{tikz}
\usetikzlibrary{datavisualization}
\usepackage[makeroom]{cancel} 
\usepackage[linguistics]{forest}
\usetikzlibrary{patterns}
\newcommand{\N}{\mathbb{N}}
\newcommand{\Z}{\mathbb{Z}}
\newcommand{\R}{\mathbb{R}}
\newcommand{\Q}{\mathbb{Q}}

\newenvironment{theorem}[2][Theorem]{\begin{trivlist}
		\item[\hskip \labelsep {\bfseries #1}\hskip \labelsep {\bfseries #2.}]}{\end{trivlist}}
\newenvironment{lemma}[2][Lemma]{\begin{trivlist}
		\item[\hskip \labelsep {\bfseries #1}\hskip \labelsep {\bfseries #2.}]}{\end{trivlist}}
\newenvironment{exercise}[2][Exercise]{\begin{trivlist}
		\item[\hskip \labelsep {\bfseries #1}\hskip \labelsep {\bfseries #2.}]}{\end{trivlist}}
\newenvironment{problem}[2][Problem]{\begin{trivlist}
		\item[\hskip \labelsep {\bfseries #1}\hskip \labelsep {\bfseries #2.}]}{\end{trivlist}}
\newenvironment{question}[2][Question]{\begin{trivlist}
		\item[\hskip \labelsep {\bfseries #1}\hskip \labelsep {\bfseries #2.}]}{\end{trivlist}}
\newenvironment{corollary}[2][Corollary]{\begin{trivlist}
		\item[\hskip \labelsep {\bfseries #1}\hskip \labelsep {\bfseries #2.}]}{\end{trivlist}}
\newenvironment{solution}[2][Solution]{\begin{trivlist}
		\item[\hskip \labelsep {\bfseries #1}\hskip \labelsep {\bfseries #2.}]}{\end{trivlist}}

\setlength\epigraphwidth{8cm}
\setlength\epigraphrule{0pt}

\makeatletter
\patchcmd{\epigraph}{\@epitext{#1}}{\itshape\@epitext{#1}}{}{}
\makeatother


\begin{document}
	
	\title{Week 9}
	\author{Juan Patricio Carrizales Torres \\
		Section 3: Proof by Contrapositive}
	\date{September 18, 2021}
	\maketitle
	 
	Let $P(x)$ and $Q(x)$ be two open sentences over some domain $S$. Suppose one wants to prove that $P(x)\Rightarrow Q(x)$ is true for all $x\in S$, but it is not directly clear how or whether a direct proof can be applied. Then one can prove directly that the contrapositive of $P(x)\Rightarrow Q(x)$, namely $\sim Q(x) \Rightarrow \sim P(x)$, is true for each $x \in S$ since an implication and its contrapositive are logically equivalent. One assumes that $\sim Q(x)$ is true for an arbitrary $x\in S$ and proves that $\sim P(x)$ is true for this same $x$. This method is known as a Proof by Contrapositive.\\
	Note that if one proves the truth of $\forall x\in S, P(x)\Rightarrow Q(x)$, then the truth of $\forall x\in S, \sim Q(x)\Rightarrow \sim P(x)$ is also proven, and vice versa. Therefore it is understood that for all $x\in S$ for which $P(x)$ is true, $Q(x)$ is true. Also, for all $x \in S$ for which $Q(x)$ is false, $P(x)$ is false.
	\begin{problem}{16}
		Let $x\in \Z$. Prove that if $7x+5$ is odd, then $x$ is even.
		\begin{proof}
			Assume that $x$ is odd. Then $x=2k+1$ for some $k\in \Z$. Hence,
			\begin{equation*}
				7(2k+1)+5=14k+12=2(7k+6)
			\end{equation*}
		Since $7k+6$ is an integer, it follows that $7x+5$ is even.
		\end{proof}
	\end{problem}

	\begin{problem}{17}
		Let $n\in \Z$. Prove that if $15n$ is even, then $9n$ is even.
		\begin{proof}
			Let $15n$ be even. Then $15n = 2k$ for some $k\in \Z$. Note that
			\begin{equation*}
				9n = 15n-6n= 2k-6n = 2(k-3n)
			\end{equation*} 
		Because $k-3n$ is an integer, it follows that $9n$ is even.
		\end{proof}
	\end{problem}

	\begin{problem}{18}
		Let $x\in \Z$. Prove that $5x-11$ is even if and only if $x$ is odd.
		\begin{proof}
			Assume $x$ is even. Then $x=2m$ for some $m\in \Z$. Therefore,
			\begin{equation*}
				5(2m)-11 = 10m-12+1=2(5m-6)+1
			\end{equation*}
			Since $5m-6$ is an integer, $5x-11$ is odd.\\
			For the converse, let $x$ be odd. Then $x=2k+1$ for some $k\in \Z$. Hence,
			\begin{equation*}
				5(2k+1)-11 = 10k+5-11=10k-6=2(5k-3)
			\end{equation*}
		Since $5k-3$ is an integer, it follows that $5x-11$ is even.
		\end{proof}
	\end{problem}

	\begin{problem}{19}
		Let $x \in \Z$. Use a lemma to prove that if $7x+4$ is even, then $3x-11$ is odd.\\
		
			\textbf{Lemma} Let $x\in \Z$. If $7x+4$ is even, then $x$ is even.
			\begin{proof}
				Assume $x$ is odd. Then $x=2n+1$ for some $n\in \Z$. Therefore,
				\begin{equation*}
					7(2n+1)+4= 14n+7+4=14n+6+4+1=2(7n+5)+1
				\end{equation*}
			Since $7n+5$ is an integer, $7x+4$ is odd.\\
			\end{proof}
			We are now ready to prove the result in problem 19.
			\begin{proof}
				Let $7x+4$ be even. Then by lemma, $x=2k$ for some $k\in \Z$. Therefore,
				\begin{equation*}
					3(2k)-11=2(3k)-12+1=2(3k-6)+1 
				\end{equation*}
			Since $3k-6$ is an integer, it follows that $3x-11$ is odd.
			\end{proof}
	\end{problem}

	\begin{problem}{20}
		Let $x\in \Z$. Prove that $3x+1$ is even if and only if $5x-2$ is odd.
		\begin{proof}
			Assume $3x+1$ is even. Then $3x+1=2k$ for some $k\in \Z$. Note that
			\begin{equation*}
				5x-2 = (3x+1)+2x-4+1=2k+2x-4+1=2(k+x-2)+1
			\end{equation*}
		Since $k+x-2$ is an integer, $5x-2$ is odd.\\
		For the converse, assume $5x-2$ is odd. Then $5x-2=2k+1$ for some $k\in \Z$. Note that
		\begin{equation*}
			3x+1=(5x-2)-2x+3=2k+1-2x+3=2(k-x+2)
		\end{equation*}
	Since $k-x+2$ is an integer, it follows that $3x+1$ is even.
		\end{proof}
	\end{problem}

	\begin{problem}{21}
		Let $n\in \Z$. Prove that $(n+1)^{2}-1$ is even if and only if $n$ is even.
		\begin{proof}
			Assume $n$ is odd. Then $n=2k+1$ for some $k\in \Z$. Therefore
			\begin{equation*}
				(n+1)^{2}-1=(2k+2)^{2}-1=4k^{2}+8k+4-1=4k^{2}+8k+3=2(2k^{2}+4k+1)+1
			\end{equation*}
			Because $2k^{2}+4k+1$ is an integer, $(n+1)^{2}-1$ is odd.\\
			For the converse, assume $n$ is even. Then $n=2k$ for some $k\in \Z$. Thus,
			\begin{equation*}
				(2k+1)^{2}-1=4k^{2}+4k+1-1=2(2k^{2}+2k)
			\end{equation*}
		Since $2k^{2}+2k$ is an integer, $(n+1)^{2}-1$ is even.
		\end{proof}
	\end{problem}

	\begin{problem}{22}
		Let $S=\{2,3,4\}$ and let $n\in S$. Use a proof by contrapositive to prove that if $n^{2}(n-1)^{2}/4$ is even, then $n^{2}(n+1)^{2}/4$ is even.
		\begin{proof}
			Let $n\in S$ such that $n^{2}(n+1)^{2}/4$ is odd. Then $n=2$ and so $n^{2}(n-1)^{2}/4=1$ is odd.
		\end{proof}
	\end{problem}

	\begin{problem}{23}
		Let $A=\{0,1,2\}$ and $B=\{4,5,6\}$ be subsets of $S=\{0,1,\ldots,6\}$. Let $n\in S$. Prove that if $\frac{n(n-1)(n-2)}{6}$ is even, then $n \in A\cup B$.
		\begin{proof}
			Assume $n \notin A\cup B$, where $n\in S$. Then $n=3$ and so $\frac{n(n-1)(n-2)}{6}=1$ is odd.
		\end{proof}
	\end{problem}

	\begin{problem}{24}
		Let $n\in \Z$. Prove that $2n^{2}+n$ is odd if and only if $\cos \frac{n\pi}{2}$ is even.
		\begin{proof}
			Assume $\cos \frac{n\pi}{2}$ is odd. Then $\cos \frac{n\pi}{2}=\pm 1$ and $n$ is even. Therefore, $n=2k$ for some $k\in \Z$. Thus,
			\begin{equation*}
				2(2k)^{2}+2k=8k^{2}+2k=2(4k^{2}+k)
			\end{equation*}
		Since $4k^{2}+k$ is an integer, $2n^{2}+n$ is even.\\
		For the converse, let $\cos \frac{n\pi}{2}$ be even. Then $\cos \frac{n\pi}{2}=0$ and $n$ is odd. Therefore, $n=2k+1$ for some $k\in \Z$. Thus,
		\begin{equation*}
			2(2k+1)^{2}+2k+1=2(4k^{2}+4k+1)+2k+1=2(4k^{2}+5k+1)+1
		\end{equation*}
	Because $4k^{2}+5k+1$ is an integer, it follows that $2n^{2}+n$ is odd.
		\end{proof}
	\end{problem}

	\begin{problem}{25} 
		Let $\{A,B\}$ be a partition of the set of $S=\{1,2,\ldots,7\}$, where $A=\{1,4,5\}$ and $B=\{2,3,6,7\}$. Let $n\in S$. Prove that if $\frac{n^{2}+3n-4}{2}$ is even, then $n\in A$.
		\begin{proof}
			Let $n\in S$ such that $n\notin A$. Then $n \in B=\{2,3,6,7\}$. Note that $\frac{2^{2}+3(2)-4}{2}=3$, $\frac{3^{2}+3(3)-4}{2}=7$, $\frac{6^{2}+3(6)-4}{2}=25$ and $\frac{7^{2}+3(7)-4}{2}=33$ are all odd integers.
		\end{proof}
	\end{problem}
\end{document}