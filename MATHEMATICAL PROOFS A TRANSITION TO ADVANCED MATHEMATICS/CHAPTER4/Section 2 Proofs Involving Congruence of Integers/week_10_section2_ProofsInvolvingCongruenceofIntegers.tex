\documentclass[12pt]{article}
\usepackage[margin=1in]{geometry}
\usepackage{amsmath,amsthm,amssymb,epigraph,etoolbox,mathtools,setspace,enumitem}  
\usepackage{tikz}
\usetikzlibrary{datavisualization}
\usepackage[makeroom]{cancel} 
\usepackage[linguistics]{forest}
\usetikzlibrary{patterns}
\newcommand{\N}{\mathbb{N}}
\newcommand{\Z}{\mathbb{Z}}
\newcommand{\R}{\mathbb{R}}
\newcommand{\Q}{\mathbb{Q}}
\newcommand{\Mod}[1]{\ (\mathrm{mod}\ #1)}

\newenvironment{theorem}[2][Theorem]{\begin{trivlist}
		\item[\hskip \labelsep {\bfseries #1}\hskip \labelsep {\bfseries #2.}]}{\end{trivlist}}
\newenvironment{lemma}[2][Lemma]{\begin{trivlist}
		\item[\hskip \labelsep {\bfseries #1}\hskip \labelsep {\bfseries #2.}]}{\end{trivlist}}
\newenvironment{exercise}[2][Exercise]{\begin{trivlist}
		\item[\hskip \labelsep {\bfseries #1}\hskip \labelsep {\bfseries #2.}]}{\end{trivlist}}
\newenvironment{problem}[2][Problem]{\begin{trivlist}
		\item[\hskip \labelsep {\bfseries #1}\hskip \labelsep {\bfseries #2.}]}{\end{trivlist}}
\newenvironment{question}[2][Question]{\begin{trivlist}
		\item[\hskip \labelsep {\bfseries #1}\hskip \labelsep {\bfseries #2.}]}{\end{trivlist}}
\newenvironment{corollary}[2][Corollary]{\begin{trivlist}
		\item[\hskip \labelsep {\bfseries #1}\hskip \labelsep {\bfseries #2.}]}{\end{trivlist}}
\newenvironment{solution}[2][Solution]{\begin{trivlist}
		\item[\hskip \labelsep {\bfseries #1}\hskip \labelsep {\bfseries #2.}]}{\end{trivlist}}

\setlength\epigraphwidth{8cm}
\setlength\epigraphrule{0pt}

\makeatletter
\patchcmd{\epigraph}{\@epitext{#1}}{\itshape\@epitext{#1}}{}{}
\makeatother


\begin{document}
	
	\title{Week 11}
	\author{Juan Patricio Carrizales Torres \\
		Section 2: Proofs involving congruence of integers}
	\date{September 29, 2021}
	\maketitle
	
	For integers $a,b,n\in \Z$, where $n\geq 2$, we say that $a$ \textbf{is congruent to }$b$ \textbf{modulo of }$n$ ($a \equiv b (\text{mod }n)$) if and only if $n\mid (a-b)$. In other words, $a$ and $b$ must have the same remainder when divided by $n$. 
	
	Let $n\in \Z$ such that $n\geq 2$. For all integers $a$, there is exactly one $b \in \{0,1,2,\ldots, n-1\}$ for which the following holds
	\begin{equation*}
		a\equiv b (\text{mod }n) 
	\end{equation*} 
	This means that $a$ can have only one of $b \in \{0,1,2,\ldots, n-1\}$ as a remainder when divided by $n$.\\
	
	\textbf{Interesting results of congruence of integers}\\
	
	\textbf{Result 9} Let $a,b,k$ and $n$ be integers where $n\geq 2$. If $a \equiv b \Mod n$, then $ka \equiv kb \Mod n$\\
	\textbf{Result 10} Let $a,b,c,d,n\in \Z$ where $n\geq 2$. If $a\equiv b \Mod n$ and $c\equiv d \Mod n$, then $a+c \equiv b+d \Mod n$.\\
	\textbf{Result 11} Let $a,b,c,d,n \in \Z$ where $n\geq 2$. If $a\equiv b \Mod n$ and $c\equiv d \Mod n$, then $ac \equiv bd \Mod n$. \\
	
	\textbf{EXERCISES}\\
	
	\begin{problem}{14}
		Let $a,b,n\in \Z$, where $n\geq 2$. Prove that if $a \equiv b (\text{mod }n)$, then $a^{2}\equiv b^{2} (\text{mod }n)$.
		\begin{proof}
			Assume $a\equiv b (\text{mod }n)$. Then, $n\mid (a-b)$. Hence, $a-b=nx$ for some $x\in \Z$. Note that
			\begin{equation*}
				a^{2}-b^{2}=(a-b)(a+b)=n(x(a+b))
			\end{equation*}
		Since $x(a+b)\in \Z$, $n\mid (a^{2}-b^{2})$ and so $a^{2}\equiv b^{2} (\text{mod }n)$.
		\end{proof}
	\end{problem}

	\begin{problem}{15}
		Let $a,b,c,n \in \Z$, where $n\geq 2$. Prove that if $a \equiv b \Mod n$ and $a \equiv c \Mod n$, then $b\equiv c \Mod n$.
		\begin{proof}
			Assume $a \equiv b \Mod n$ and $a \equiv c \Mod n$. Then $n\mid(a-b)$ and $n\mid (a-c)$. Therefore, $a-b=xn$ and $a-c = yn$, where $x,y\in \Z$, and so $a=xn+b$. Therefore, 
			\begin{align*}
				(xn+b)-c &= yn\\
				b-c &= yn-xn = n(y-x)
			\end{align*}
			Since $y-x\in \Z$, $n\mid (b-c)$ and so $b \equiv c \Mod n$\\
			
			(Since $a \equiv b \Mod n$ and $a \equiv c \Mod n$, both $a$ and $b$ have the same residue when divided by $n$.)
		\end{proof}
	\end{problem}

	\begin{problem}{16}
		Let $a,b \in \Z$. Prove that if $a^{2}+2b^{2} \equiv 0 \Mod 3$, then either both $a$ and $b$ are congruent to 0 module of 3 or neither is congruent to 0 module of 3.\\ 
		\begin{proof}
			Assume that exactly one of $a$ and $b$ is congruent to 0 module of 3. We consider two cases.\\
			\textit{Case 1.} $a \equiv 0 \Mod 3$ and $b \not\equiv 0 \Mod 3$. By Result 4.6, $a^{2} \equiv 0 \Mod 3$ and $b^{2} \equiv 1 \Mod 3$. Then,  $2b^{2}\equiv 2 \Mod 3$. Thus, $a^{2} + 2b^{2} \equiv 2 \Mod 3$ and so $a^{2} + 2b^{2} \not\equiv 0 \Mod 3$.\\
			
			\textit{Case 2.} $b \equiv 0 \Mod 3$ and $a \not\equiv 0 \Mod 3$. By Result 4.6, $a^{2} \equiv 1 \Mod 3$ and $b^{2} \equiv 0 \Mod 3$. Then, $2b^{2} \equiv 0 \Mod 3$. Thus, $a^{2} + 2b^{2} \equiv 1 \Mod 3$ and so $a^{2} + 2b^{2} \not\equiv 0 \Mod 3$. 
		\end{proof}
	\end{problem}

	\begin{problem}{17}
		(a) Prove that if $a$ is an integer such that $a \equiv 1 \Mod 5$, then $a^{2}\equiv 1 \Mod 5$.
		\begin{solution}{a}
			Assume $a\equiv 1\Mod 5$. Then $5\mid (a-1)$ and so $a-1 = 5x$ for some $x\in \Z$. Note that
			\begin{align*}
				(a-1)(a+1) &= 5x(a+1)\\
				a^{2}-1 &= 5(x(a+1))
			\end{align*}
			Because $(x(a+1))\in \Z$, $5\mid (a^{2}-1)$ and so $a^{2} \equiv 1 \Mod 5$.
		\end{solution}
	\end{problem}

	\begin{problem}{18}
		Let $m,n\in \N$ such that  $m\geq 2$ and $m\mid n$. Prove tha if $a$ and $b$ are integers such that $a \equiv b \Mod n$, then $a \equiv b \Mod m$.
		\begin{proof}
			Let $a,b \in \Z$ such that $a \equiv b \Mod n$. Then, $n\mid (a-b)$. Therefore, $a-b=nk$ for some $k\in \Z$. Note that $m\mid n$ and so $n = mx$, where $x\in \Z$. Therefore, $a-b = (mx)k = m(xk)$. Since $xk\in \Z$, $m\mid (a-b)$ and so $a \equiv b \Mod m$.
		\end{proof}
	\end{problem}
 
	\begin{problem}{19}
		Let $a,b\in \Z$. Show that if $a\equiv 5 \Mod 6$ and $b\equiv 3 \Mod 4$, then $4a+6b \equiv 6 \Mod 8$.
		\begin{proof}
			Assume $a\equiv 5 \Mod 6$ and $b\equiv 3 \Mod 4$. Then $6\mid (a-5)$ and $4 \mid (b-3)$. Therefore, $a-5 = 6x$ and $b-3 = 4y$ for some integers $x$ and $y$. So $a = 6x+5$ and $b=4y+3$. Therefore,
			\begin{equation*}
				4(6x+5) + 6(4y+3) = 24x+20+24y+18 = 24x+24y+38 = 8(3x+3y+4)+6
			\end{equation*}
		Thus, $(4a + 6b)-6 = 8(3x+3y+4)$. Since $3x+3y+4\in \Z$, $8\mid((4a+6b)-6) $ and so $4a+6b \equiv 6 \Mod 8$.
		\end{proof}
	\end{problem}

	\begin{problem}{20}
		 Result 12 states: Let $n \in \Z$. If $n^{2}\not\equiv n \Mod 3$, then $n\not\equiv 0 \Mod 3$ and $n \not\equiv 1 \Mod 3$. State and prove the converse of this result.\\
		 
		 Let $n \in \Z$. If $n\not\equiv 0 \Mod 3$ and $n \not\equiv 1 \Mod 3$, then $n^{2}\not\equiv n \Mod 3$. 
		 \begin{proof}
		 	Assume $n\not\equiv 0 \Mod 3$ and $n \not\equiv 1 \Mod 3$. Then $n \equiv 2 \Mod 3$ and so $3\mid (n-2)$. Therefore $n-2=3x$ for some $x\in \Z$ and so $n=3x+2$. Note that
		 	\begin{equation*}
		 		n^{2}-n = (3x+2)^{2}-3x-2 = 9x^{2}+12x+4-3x-2 = 9x^{2}+9x+2 = 3(3x^{2}+3x)+2
		 	\end{equation*}
	 		Since $3x^{2}+3x \in \Z$, $3\nmid (n^{2}-2)$ and so $n^{2} \not\equiv n \Mod 3$.
		 \end{proof}
	 
	 	(b) State the conjunction of Result 12 and its converse using "if and only if".
	 \begin{solution}{b}
	 	Let $n\in \Z$. Then $n^{2} \not\equiv n \Mod 3$ if and only if $n \not\equiv 0 \Mod 3$ and $n \not\equiv 1 \Mod 3$.
	 \end{solution}
	\end{problem}

	\begin{problem}{21}
		Let $a\in \Z$. Prove that $a^{3} \equiv a \Mod 3$.
		\begin{proof}
			Assume $a\in \Z$. Then either $a=3q$, $a=3q+1$ or $a=3q+2$ for some $q\in \Z$. We consider these 3 cases.\\
			\textit{Case 1.} $a=3q$, where $q\in \Z$. Note that
			\begin{equation*}
				a^{3} - a = (3q)^{3}-3q=27q^{3}-3q=3(9q^{3}-q)
			\end{equation*}
			Since $9q^{3}-q\in \Z$, $3\mid (a^{3}-a)$ and so $a^{3} \equiv a \Mod 3$.\\
			\textit{Case 2.} $a=3q+1$, where $q\in \Z$. Note that
			\begin{equation*}
				a^{3}-a = (3q+1)^{3}-3q-1=27q^{3}+27q^{2}+9q+1-3q-1=3(9q^{3}+9q^{2}+2q)
			\end{equation*}
			Since $9q^{3}+9q^{2}+2q \in \Z$, $3\mid (a^{3}-a)$ and so $a^{3} \equiv a \Mod 3$.\\
			\textit{Case 3.} $a=3q+2$, where $q\in \Z$. Note that 
			\begin{equation*}
				a^{3}-a = (3q+2)^{3}-3q-2 = 27q^{3}+54q^{2}+36q+8 -3q -2  = 3(9q^{3}+18q^{2}+11q+2)
			\end{equation*}
			Since $9q^{3}+18q^{2}+11q+2 \in \Z$, $3\mid (a^{3}-a)$ and so $a^{3} \equiv a \Mod 3$.\\
		\end{proof}
	\end{problem}

\begin{problem}{24}
	Let $x$ and $y$ be even integers. Prove that $x^{2} \equiv y^{2} \Mod {16}$ if and only if either (1) $x \equiv 0 \Mod 4$ and $y \equiv 0 \Mod 4$ or (2) $x\equiv 2 \Mod 4$ and $y \equiv 2 \Mod 4$.
	\begin{proof}
		Let $x$ and $y$ be even integers. Then, each of $x$ and $y$ is either congruent to 0 or 2 modulo of 4. First, we assume that  either (1) $x \equiv 0 \Mod 4$ and $y \equiv 0 \Mod 4$ or (2) $x\equiv 2 \Mod 4$ and $y \equiv 2 \Mod 4$. We consider the following two cases.\\
		\textit{Case 1.} $x \equiv 0 \Mod 4$ and $y \equiv 0 \Mod 4$. Then, $4\mid x$ and $4\mid y$, and so $x = 4n$ and $y = 4m$ for some $n,m\in \Z$. Note that,
		\begin{equation*}
			x^{2} -y ^{2} = (x+y)(x-y) = (4n +4m)(4n-4m) = 4(n+m)4(n-m) = 16((n+m)(n-m)) 
		\end{equation*}
		Since $(n+m)(n-m)\in \Z$, $16 \mid (x^{2} -y ^{2})$ and so $x^{2} \equiv y^{2} \Mod {16}$.\\
		\textit{Case 2.} $x\equiv 2 \Mod 4$ and $y \equiv 2 \Mod 4$. Then, $4\mid (x-2)$ and $4\mid (y-2)$, and so $x-2 = 4n$ and $y-2 = 4m$ for some $n,m\in \Z$. Therefore, $x = 4n+2$ and $y = 4m +2$. Note that,
		\begin{align*}
			x^{2}-y^{2} &= (x+y)(x-y) = (4n+2+4m+2)(4n+2-4m-2) = (4n+4m+4)(4n-4m) \\
			&= 4(n+m+1)4(n-m) = 16(n+m+1)(n-m)
		\end{align*}
		Since $(n+m+1)(n-m) \in \Z$, $16\mid (x^{2}-y^{2})$ and so $x^{2} \equiv y^{2} \Mod {16}$.\\
		 
		For the converse, let $x$ and $y$ be even integers such that exactly one of them is congruent to 0 modulo of 4 and the other is congruent to 2 modulo of 4. Without loss of generality, assume $x \equiv 2 \Mod 4$ and $y\equiv 0 \Mod 4$. Then, $4\mid (x-2)$ and $4\mid y$, and so $x = 4n+2$ and $y =4m$ for some $n,m \in \Z$. Note that,
		\begin{equation*}
			x^{2}-y^{2} = (4n+2)^{2}-(4m)^{2} = 16n^{2} +16n +4-16m^{2} = 16(n^{2}+n-m^{2})+4
		\end{equation*}
	Since $n^{2}+n-m^{2}\in \Z$, $16\nmid (x^{2}-y^{2})$ and so $x^{2} \not\equiv y^{2} \Mod{16}$.
	\end{proof}
\end{problem} 

	
\end{document}