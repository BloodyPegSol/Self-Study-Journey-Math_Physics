\documentclass[12pt]{article}
\usepackage[margin=1in]{geometry}
\usepackage{amsmath,amsthm,amssymb,epigraph,etoolbox,mathtools,setspace,enumitem}  
\usepackage{tikz}
\usetikzlibrary{datavisualization}
\usepackage[makeroom]{cancel} 
\usepackage[linguistics]{forest}
\usetikzlibrary{patterns}
\newcommand{\N}{\mathbb{N}}
\newcommand{\Z}{\mathbb{Z}}
\newcommand{\R}{\mathbb{R}}
\newcommand{\Q}{\mathbb{Q}}
\newcommand{\Mod}[1]{\ (\mathrm{mod}\ #1)}

\newenvironment{theorem}[2][Theorem]{\begin{trivlist}
		\item[\hskip \labelsep {\bfseries #1}\hskip \labelsep {\bfseries #2.}]}{\end{trivlist}}
\newenvironment{lemma}[2][Lemma]{\begin{trivlist}
		\item[\hskip \labelsep {\bfseries #1}\hskip \labelsep {\bfseries #2.}]}{\end{trivlist}}
\newenvironment{exercise}[2][Exercise]{\begin{trivlist}
		\item[\hskip \labelsep {\bfseries #1}\hskip \labelsep {\bfseries #2.}]}{\end{trivlist}}
\newenvironment{problem}[2][Problem]{\begin{trivlist}
		\item[\hskip \labelsep {\bfseries #1}\hskip \labelsep {\bfseries #2.}]}{\end{trivlist}}
\newenvironment{question}[2][Question]{\begin{trivlist}
		\item[\hskip \labelsep {\bfseries #1}\hskip \labelsep {\bfseries #2.}]}{\end{trivlist}}
\newenvironment{corollary}[2][Corollary]{\begin{trivlist}
		\item[\hskip \labelsep {\bfseries #1}\hskip \labelsep {\bfseries #2.}]}{\end{trivlist}}
\newenvironment{solution}[2][Solution]{\begin{trivlist}
		\item[\hskip \labelsep {\bfseries #1}\hskip \labelsep {\bfseries #2.}]}{\end{trivlist}}

\setlength\epigraphwidth{8cm}
\setlength\epigraphrule{0pt}

\makeatletter
\patchcmd{\epigraph}{\@epitext{#1}}{\itshape\@epitext{#1}}{}{}
\makeatother


\begin{document}
	
	\title{Week 12}
	\author{Juan Patricio Carrizales Torres \\
		Section 4: Proofs involving sets}
	\date{October 09, 2021}
	\maketitle
	 
	If one wants to prove $A=B$, it is necessary to show both set inclusions $A\subseteq B$ and $B\subseteq A$, namely, some arbitrary $x\in A$ if and only if $x\in B$. In the case that either $A=\emptyset$ or $B=\emptyset$, their respective set inclusions follow vacuously.\\
	\textbf{NOTE.} Assume all sets discussed belong to some Universal set.
	
	\begin{problem}{40}
		Let $A$ and $B$ be sets. Prove that $A\cup B = (A-B)\cup(B-A)\cup(A\cap B)$.
		\begin{proof}
			 First we show that $A\cup B \subseteq (A-B)\cup(B-A)\cup(A\cap B)$. Let $x\in A\cup B$. Then $x\in A$ or $x\in B$. Without loss of generality, let $x\in A$. We consider two cases.\\
			 \textit{Case 1.} $x\in A$ and $x\in B$. Then $x\in A\cap B$ and so $x\in (A-B)\cup(B-A)\cup(A\cap B)$.\\
			 \textit{Case 2.} $x\in A$ and $x\not\in B$. Then $x\in A-B$ and so $x\in (A-B)\cup(B-A)\cup(A\cap B)$.\\
			  Hence, $A\cup B \subseteq (A-B)\cup(B-A)\cup(A\cap B)$.\\
			  
			 We then show that $(A-B)\cup(B-A)\cup(A\cap B) \subseteq A\cup B$. Let $x\in (A-B)\cup(B-A)\cup(A\cap B)$. Then, $x\in A-B$, $x\in B-A$ or $x\in A\cap B$. In all cases either $x\in A$ or $x\in B$. Therefore, $x\in A\cup B$ and so $(A-B)\cup(B-A)\cup(A\cap B) \subseteq A\cup B$.
		\end{proof}
	\end{problem}

	\begin{problem}{41}
		In result 21, it was proved for sets $A$ and $B$ that $A\cup B = A$ if and only if $B\subseteq A$. Provide another proof of this result by giving a direct proof of the implication "If $A\cup B = A$, then $B\subseteq A$" and a proof by contrapositive of its converse.
		\begin{proof}
			First assume that $A\cup B = A$. Then $A\cup B \subseteq A$ and $A \subseteq A\cup B$. Note that by definition, $B \subseteq A\cup B$. Since $B \subseteq A\cup B$ and $A\cup B \subseteq A$, it follows that $B\subseteq A$.\\
			
			For the converse, assume that $A\cup B \neq A$. Then either $A\cup B \not\subseteq A$ or $A\not\subseteq A\cup B$. Note that the former leads to a contradiction, thus we only consider that $A\cup B \not\subseteq A$. Then there is some $y\in A\cup B$ such that $y\not\in A$. Therefore, $y\in B$. Since $y\not\in A$ and $y\in B$, it follows that $B\not\subseteq A$. 
		\end{proof}
	\end{problem}

	\begin{problem}{42}
		Let $A$ and $B$ be sets. Prove that $A\cap B= A$ if and only if $A \subseteq B$.
		\begin{proof}
			Assume that $A \not\subseteq B$. Then, there is some $x \in A$ such that $x \not\in B$. Hence $x\not\in A\cap B$. Since $x\not\in A\cap B$ and $x\in A$, it follows that $A\not\subseteq A\cap B$ and so $A \neq A\cap B$.\\
			
			For the converse, assume that $A\cap B \neq A$. Then either $A\not\subseteq A\cap B$ or $A\cap B \not\subseteq A$. Note that the former leads to a contradiction, thus we only consider $A\not\subseteq A\cap B$. Then, there is some $y\in A$ such that $y\not\in A\cap B$. Hence, $y\not\in B$. Since $y\in A$ and $y\not\in B$, it follows that $A\not\subseteq B$.
		\end{proof}
	\end{problem}

	\begin{problem}{43}
		(a) Give and example of three sets $A$, $B$ and $C$ such that $A\cap B = A\cap C$ but $B\neq C$.
		\begin{solution}{a}
			Let $A=\{1\}$, $B=\{1,2\}$ and $C=\{1,3\}$. Then, $A\cap B = A\cap C = \{1\}$.
		\end{solution} 
		(b) Give an example of three sets $A$, $B$ and $C$ such that $A\cup B = A\cup C$ but $B\neq C$.
		\begin{solution}{b}
			Let $A=\{1,2,3\}$, $B=\{2\}$ and $C=\{3\}$. Then, $A\cup B=A\cup C = \{1,2,3\}$
		\end{solution}
		(c) Let $A$, $B$ and $C$ be sets. Prove that if $A\cap B = A\cap C$ and $A\cup B = A\cup C$, then $B=C$.
		\begin{proof}
			Assume $B\neq C$. Then, either $B\not\subseteq C$ or $C\not\subseteq B$. Without loss of generality, let $B\not\subseteq C$. This means that there is some $y\in B$ such that $y\not\in C$. We consider the following cases.\\
			\textit{Case 1.} $y\in A$. Since $y\in A$, $y \in B$ and $y\not\in C$, it follows that $y\in A\cap B$ and $y\not\in A\cap C$, and so $A\cap B \not\subseteq A\cap C$. Hence, $A\cap B \neq A\cap C$.\\
			\textit{Case 2.} $y\not\in A$. Since $y\not\in A$, $y \in B$ and $y\not\in C$, it follows that $y \in A\cup B$ and $y\not\in A\cup C$, and so $A\cup B \not\subseteq A\cup C$. Therefore, $A\cup B \neq A\cup C$.
		\end{proof}
	\end{problem}

	\begin{problem}{44}
		Prove that if $A$ and $B$ are sets such that $A\cup B \neq \emptyset$, then $A\neq \emptyset$ or $B\neq \emptyset$.
		\begin{proof}
			Let $A = \emptyset$ and $B = \emptyset$. Then, $A\cup B = \emptyset\cup \emptyset = \emptyset$.
		\end{proof}
	\end{problem}

	\begin{problem}{45}
		Let $A=\{n\in\Z: n\equiv 1 \Mod 2\}$ and $B=\{n\in \Z: n\equiv 3 \Mod 4\}$. Prove that $B\subseteq A$.
		\begin{proof}
			 Let some $x\in B$. Then, $x\in \Z$ and $x \equiv 3 \Mod 4$, and so $x = 4m + 3$ for some integer $m$. Note that $x = 4m + 3 = 4m + 2 +1 = 2(2m+1)+1$. Since $2m+1\in \Z$, it follows that $2\mid (x-1)$ and so $x \equiv 1 \Mod 2$. So $x \in A$. Therefore, $B\subseteq A$.
		\end{proof}
	\end{problem}

	\begin{problem}{46}
		Let $A$ and $B$ be sets. Prove that $A\cup B = A\cap B$ if and only if $A=B$.
		\begin{proof}
			First assume $A\neq B$. We show that $A\cup B \neq A\cap B$. Then, either $A\not\subseteq B$ or $B\not\subseteq A$, say the former. Thus, there is some $a\in A$ such that $a\not\in B$. Therefore, $a\in A\cup B$ and $a\not\in A\cap B$. Hence $A\cup B \neq A\cap B$.\\
			For the converse, suppose $A=B$. Therefore, $A\cup B = A\cap B = A = B$.
		\end{proof}
	\end{problem}

	\begin{problem}{47}
		Let $A=\{n\in \Z: n\equiv 2 \Mod 3\}$ and $B=\{n\in\Z : n\equiv 1 \Mod 2\}$.\\
		(a) Describe the elements of the set $A-B$.
		\begin{solution}{a}
			If $x\in A$, then $x\in \Z$ and $x \equiv 2 \Mod 3$, and so $x = 3m +2$ where $m\in \Z$. If $x\in B$, then $x\in \Z$ and $x\equiv 1 \Mod 2$, and so $x = 2q +1$ where $q\in \Z$ (the odd integers). Note that $x = 3m+2$ is odd when $m$ is odd, and it is even when $m$ is even. The set $A-B$ contains all integers $x$ such that $x\equiv 2 \Mod 3$ and $x$ is even. Thus, $A-B = \{n\in \Z: n = 3m+2 \text{, where }m \text{ is an even integer}\}$. Since $m$ is an even integer, it follows that $m = 2b$ for some integer $b$. Then, $3m+2 = 3(2b) +2 = 6b +2$. Therefore, $A-B = \{n\in \Z: n = 6b+2 \text{, where }b\in \Z\}$
		\end{solution}
		(b) Prove that if $n\in A\cap B$, then $n^{2} \equiv 1 \Mod {12}$.
		\begin{proof}
			  Assume $n\in A\cap B$. Then, $n\in \Z$, $n\equiv 2 \Mod 3$ and $n \equiv 1 \Mod 2$. This means that $n = 3q+2$ and $n$ is odd. Thus, $q$ is odd and so $q = 2m +1$ for some $m\in \Z$. Then, $n = 3(2m+1)+2 = 6m+3+2= 6m+5$ and so
			  \begin{align*}
			  	n^{2}-1 &= (6m+5)^{2}-1\\
			  	 &= 36m^{2}+60m+25-1\\
			  	 &= 12(3m^{2}+5m+2)
			  \end{align*}
		  	Since $3m^{2}+5m+2\in \Z$, it follows that $12\mid (n^{2}-1)$ and so $n^{2}\equiv 1 \Mod {12}$.
		\end{proof}
	\end{problem}
	
	\begin{problem}{48}
		Let $A=\{n\in \Z:2\mid n\}$ and  $B=\{n\in\Z:4\mid n\}$. Let $n\in \Z$. Prove that $n\in A-B$ if and only if $n=2k$ for some odd integer $k$.
		\begin{proof}
			First, let $n\in A-B$. Then, $n\in \Z$, $2\mid n$ and $4\nmid n$. Then, $n$ is even and so $n = 2k$ where $k\in \Z$. Since $4\nmid n$, it follows that $k$ must not be even. Thus, $k$ is odd.
			
			For the converse, assume $n=2k$ for some odd integer $k$. Then, $k = 2m+1$ where $m\in \Z$ and so $n=2(2m+1) = 4m+2$. Since $n = 2(2m+1)$, it follows that $2\mid n$ and so $n\in A$. Also, since $n = 4m+2$, it follows that $4\nmid n$ and so $n\not\in B$. Hence, $n\in A-B$.\\
		\end{proof}
	\end{problem}

	\begin{problem}{49}
		Prove for every two sets $A$ and $B$ that $A=(A-B)\cup(A\cap B)$.
		\begin{proof}
			First we prove that $A\subseteq (A-B)\cup(A\cap B)$. Let $x\in A$. We consider two cases.\\
			\textit{Case 1.} $x\in B$. Since $x\in A$ and $x\in B$, it follows that $x\in A\cap B$. Thus, $x \in (A-B)\cup(A\cap B)$.\\
			\textit{Case 2.} $x \not\in B$. Since $x\in A$ and $x \not\in B$, it follows that $x\in A-B$. Hence, $x\in (A-B)\cup(A\cap B)$.\\
			Therefore, $A\subseteq (A-B)\cup(A\cap B)$\\
			
			We then prove that $(A-B)\cup(A\cap B)\subseteq A$. Let $y\in (A-B)\cup(A\cap B)$. Then, either $y\in A-B$ or $y\in A\cap B$. Both cases imply that $y\in A$. Therefore, $(A-B)\cup(A\cap B)\subseteq A$.\\
			Hence, $A=(A-B)\cup(A\cap B)$.
		\end{proof}
	\end{problem} 

	\begin{problem}{50}
		Prove for every two sets $A$ and $B$ that $A-B$, $B-A$ and $A\cap B$ are pairwise disjoint.
		\begin{proof}
			Let $x\in A-B$. Then, $x\in A$ and $x\not\in B$. Therefore, $x\not\in B-A$ and $x\not\in A\cap B$. Hence, $(A-B) \cap (B-A) = \emptyset$ and $(A-B) \cap (A\cap B) = \emptyset$.\\
			
			Let $y \in B-A$. Then, $y\in B$ and $y\not\in A$. Therefore, $y\not\in A\cap B$ and so $(B-A)\cap (A\cap B) = \emptyset$.\\
			Thus, $A-B$, $B-A$ and $A\cap B$ are pairwise disjoint.
		\end{proof}
	\end{problem}
\end{document}