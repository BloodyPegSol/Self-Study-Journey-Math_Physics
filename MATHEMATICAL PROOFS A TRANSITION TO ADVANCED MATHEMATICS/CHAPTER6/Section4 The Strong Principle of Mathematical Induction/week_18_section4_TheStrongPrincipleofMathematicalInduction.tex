\documentclass[12pt]{article}
\usepackage[margin=1in]{geometry}
\usepackage{amsmath,amsthm,amssymb,epigraph,etoolbox,mathtools,setspace,enumitem}  
\usepackage{tikz}
\usetikzlibrary{datavisualization}
\usepackage[makeroom]{cancel} 
\usepackage[linguistics]{forest}
\usetikzlibrary{patterns}
\newcommand{\N}{\mathbb{N}}
\newcommand{\Z}{\mathbb{Z}}
\newcommand{\R}{\mathbb{R}}
\newcommand{\Q}{\mathbb{Q}}
\newcommand{\Mod}[1]{\ (\mathrm{mod}\ #1)}

\DeclarePairedDelimiter\bra{\langle}{\rvert}
\DeclarePairedDelimiter\ket{\lvert}{\rangle}
\DeclarePairedDelimiterX\braket[2]{\langle}{\rangle}{#1\delimsize\vert #2}


\newenvironment{theorem}[2][Theorem]{\begin{trivlist}
		\item[\hskip \labelsep {\bfseries #1}\hskip \labelsep {\bfseries #2.}]}{\end{trivlist}}
\newenvironment{lemma}[2][Lemma]{\begin{trivlist}
		\item[\hskip \labelsep {\bfseries #1}\hskip \labelsep {\bfseries #2.}]}{\end{trivlist}}
	\newenvironment{result}[2][Result]{\begin{trivlist}
			\item[\hskip \labelsep {\bfseries #1}\hskip \labelsep {\bfseries #2.}]}{\end{trivlist}}
\newenvironment{exercise}[2][Exercise]{\begin{trivlist}
		\item[\hskip \labelsep {\bfseries #1}\hskip \labelsep {\bfseries #2.}]}{\end{trivlist}}
\newenvironment{problem}[2][Problem]{\begin{trivlist}
		\item[\hskip \labelsep {\bfseries #1}\hskip \labelsep {\bfseries #2.}]}{\end{trivlist}}
\newenvironment{question}[2][Question]{\begin{trivlist}
		\item[\hskip \labelsep {\bfseries #1}\hskip \labelsep {\bfseries #2.}]}{\end{trivlist}}
\newenvironment{corollary}[2][Corollary]{\begin{trivlist}
		\item[\hskip \labelsep {\bfseries #1}\hskip \labelsep {\bfseries #2.}]}{\end{trivlist}}
\newenvironment{solution}[2][Solution]{\begin{trivlist}
		\item[\hskip \labelsep {\bfseries #1}\hskip \labelsep {\bfseries #2.}]}{\end{trivlist}}

\setlength\epigraphwidth{8cm}
\setlength\epigraphrule{0pt}

\makeatletter
\patchcmd{\epigraph}{\@epitext{#1}}{\itshape\@epitext{#1}}{}{}
\makeatother


\begin{document}
	
	\title{Week ??}
	\author{Juan Patricio Carrizales Torres \\
		Section 4: The Strong Principle of Mathematical Induction}
	\date{January 27, 2022}
	\maketitle 
	
	Another form of mathematical induction is commonly known as \textbf{The Strong Principle of Mathematical Induction}. This technique is quite similar to the Principle of Mathematical Induction (in fact, if you use either to prove a theorem, it is possible to prove the same result with the remaining technique). The general Theorem goes as
	\begin{theorem}{20}
		\textbf{The Strong Principle of Mathematical Induction} For a fixed integer $m$, let $S=\{a\in \Z:a\geq m\}$. For each $n\in S$, let $P(n)$ be a statement. If
		\begin{enumerate}
			\item $P(m)$ is true and
			\item the quantified statement 
			\begin{center}
			For any $k\in S$, if $P(i)$ for every integer $i$ with $m\leq i \leq k$, then $P(k+1)$. 
		\end{center}
			is true,
		\end{enumerate}
		then $\forall n\in S, P(n)$
	\end{theorem}
	Note that in the inductive hypothesis went from $\forall k\in S, P(k)\implies P(k+1)$ to $\forall k\in S, P(m)\wedge P(m+1)\wedge \ldots\wedge P(k)\implies P(k+1)$. Basically, with the \textbf{Strong Principle of Mathematical Induction} you have a more "inclusive" hypothesis, namely, you assume that $P(m)\wedge P(m+1)\wedge \ldots\wedge P(k)$ for any $k\in S$. 
	\begin{problem}{41}
		A sequence $\{a_{n}\}$ is defined recursively by $a_{1}=1$ and $a_{2} = 2a_{n-1}$ for $n\geq 2$. Conjecture a formula for $a_{n}$ and verify that your conjecture is correct.
		\begin{result}{}
			Let the sequence $\{a_{n}\}$ be defined recursively by $a_{1} = 1$ and $a_{2} = 2a_{n-1}$ for $n\geq 2$. Then $a_{n} = 2^{n-1}$, where $n\in \N$.
		\begin{proof}
			We proceed by induction. Since $a_{1} = 2^{0} = 1$, the result is true for $n=1$. Assume that $a_{k} = 2^{k-1}$ for some $k\in \N$. We show that $a_{k+1} = 2^{k} $. Note that,
			\begin{align*}
				a_{k+1} &= 2a_{k}\\
				&= 2(2^{k-1})\\
				&=  2^{k-1+1} = 2^{k}.
			\end{align*}
		By the Principle of Mathematical Induction, this result is true. 
		\end{proof}
		\end{result}
	\end{problem}
	
	\begin{problem}{42}
		A sequence $\{a_{n}\}$ is defined recursively by $a_{1} = 1, a_{2} = 2$ and $a_{n} = a_{n-1} +2a_{n-2}$ for $n\geq 3$. Conjecture a formula for $a_{n}$ and verify that your conjecture is correct.
		
		\begin{result}{}
			Let $\{a_{n}\}$ be a sequence defined recursively by $a_{1} = 1, a_{2} = 2$ and $a_{n} = a_{n-1} +2a_{n-2}$ for $n\geq 3$. Then, $a_{n} = 2^{n-1}$ for any positive integer $n$.
		\begin{proof}
				We proceed by strong induction. Since $a_{1} = 2^{1-1} = 1$, it follows that the result is true for $n=1$. Suppose that $a_{i} = 2^{i-1}$ for $1\leq i \leq k$, where $k\in \N$. We show that $a_{k+1} = 2^{k}$. Note that $a_{2} = a_{1+1} = 2^{1} = 2$ and so $a_{k+1} = 2^{k} $ is true for $k=1$ and $k\geq 2$. Since $k+1 \geq 3$, it follows that
				\begin{align*}
					a_{k+1} &= a_{k} + 2a_{k-1} = 2^{k-1} +2\cdot 2^{k-2}\\
					&= 2^{k-1} + 2^{k-1} = 2^{k}.
				\end{align*}
			By the Strong Principle of Mathematical Induction, this result is true.
		\end{proof}
		\end{result}
	\end{problem}

	\begin{problem}{43}
		A sequence $\{a_{n}\}$ is defined recursively by $a_{1} = 1, a_{2} = 4, a_{3} = 9$ and 
		\begin{equation*}
			a_{n} = a_{n-1} - a_{n-2} + a_{n-3} +2(2n-3)
		\end{equation*}
		for $n\geq 4$. Conjecture a formula for $a_{n}$ and prove that your conjecture is correct.
		\begin{result}{}
			Let $\{a_{n}\}$ be a sequence defined by $a_{1} = 1, a_{2} = 4, a_{3} = 9$ and
				\begin{equation*}
				a_{n} = a_{n-1} - a_{n-2} + a_{n-3} +2(2n-3)
			\end{equation*}
			for $n\geq 4$. Then, $a_{n} = n^{2}$ for all $n\in \N$.
		\end{result}
		\begin{proof}
			We proceed by the Strong Principle of Mathematical Induction. Because $a_{1} = 1 = 1^{2}$, it follows that $a_{n} = n^{2}$ when $n=1$. Assume that $a_{i} = i^{2}$ for $1\leq i \leq k$ for some $k\in \N$. We prove that $a_{k+1}= (k+1)^{2}$. Since $a_{1+1} = 4 = (1+1)^{2}$ and $a_{2+1} = 9 = (2+1)^{2}$, it follows that $a_{k+1}= (k+1)^{2}$ is true for $k=1,2$ and so $k\geq 3$. Because $k+1\geq 4$, it follows that
			\begin{align*}
				a_{k+1} &= a_{k} - a_{k-1} + a_{k-2} +2(2(k+1)-3)\\
				&= k^{2} - (k-1)^{2} + (k-2)^{2} + 4k-2\\
				&= k^{2} - k^{2} +2k-1+k^{2} -4k +4 +4k-2\\
				&= k^{2} +2k+1 = (k+1)^{2}
			\end{align*} 
		By the Principle of Mathematical Induction, this result is true.
		\end{proof}
	\end{problem}

	\begin{problem}{44}
		Consider the sequence $F_{1}, F_{2}, F_{3}, \ldots,$ where
		\begin{equation*}
			F_{1} = 1,F_{2} = 1,F_{3} = 2,F_{4} = 3,F_{5} = 5 \text{ and } F_{6} = 8.
		\end{equation*}
	The terms of this sequence are called \textbf{Fibonacci numbers}.
	\begin{enumerate}[label=(\alph*)]
		\item  Define the sequence of Fibonacci numbers by means of a recurrence relation.\\
		
		We can express the sequence of Fibonacci numbers by means of the recursively defined sequence $\{a_{n}\}$, where $a_{1} =1, a_{2} = 1$ and
		\begin{equation*}
			a_{n} = a_{n-1} + a_{n-2}
		\end{equation*}
	for $n\geq 3$.
		
		\item Prove that $2\mid F_{n}$ if and only if $3\mid n$.
		\begin{proof}
			Let $P(n):$ $3\mid n$, where $n\in \N$, if and only if $F_{n}$ is even. We proceed by the Strong Principle of Mathematical Induction. Since $3\nmid 1$ and $F_{1} = 1$, it follows that $P(1)$ is true. Assume that $P(i)$ is true for $1\leq i \leq k$, where $k\in \N$. We show that $P(k+1)$ is true. Because $3 \nmid (1+1)$ and $F_{1+1} = 1$, it follows that $P(k+1)$ is true for $k=1$ and so $k\geq 2$. Hence, $k+1\geq 3$. \\
			
			Let $3\mid (k+1)$. Then, $k+1 = 3c$ for some integer $c$ and so $k = 3c-1 = 3(c-1)+2$ and $k-1 = 3c-2= 3(c-1)+1$. Therefore, $k$ and $k+1$ are not divisible by 3, and so, by the inductive hypothesis, $F_{k+1} = F_{k} + F_{k-1}$ is the sum of two odd integers, which leads to an even integer. 
			
			We now show the converse. Suppose that $3\nmid k+1$. Then, either $k+1 = 3c +1$ or $k+1 = 3c+2$ for some integer $c$. It is easy to see that in both cases, either $3\mid k$ or $3 \mid (k-1)$, but not both. Hence, by our inductive assumption, $F_{k+1} = F_{k} + F_{k-1}$ is the sum of two integers of opposite parity which implies that $F_{k+1}$ is odd. 
						
			By the Strong Principle of Mathematical Induction, the implication $P(n)$ is true for all $n\in \N$.\\
		\end{proof}
	\end{enumerate}
	\end{problem}

	\begin{problem}{45}
		Use the Strong Principle of Mathematical Induction to prove that for each integer $n\geq 12$, there are nonnegative integers $a$ and $b$ such that $n=3a+7b$.
		\begin{proof}
			We proceed by using the Strong Principle of Mathematical Induction. Since $12 = 3(4) + 7(0)$, it follows that the result is true for $n=12$. Assume that $i = 3a + 7b$, where $a$ and $b$ are some arbitrary nonnegative integers, for $12 \leq i \leq k$. We show that the result is true for $k+1$. Because 
			\begin{align*}
			13 = 3(2) + 7(1) \text{ and}\\ 
			14 = 3(0) + 7(2),
		\end{align*}
			  it follows that the result is true for $k=13,14$ and so $k\geq 14$. Hence, $k+1\geq 15$. Note that
			  \begin{align*}
			  	k+1 = (k-2) + 3
			  \end{align*}
		  and, by the inductive hypothesis, we have
		  		\begin{align*}
		  			k+1 &= (3x + 7y) + 3\\
		  				&= 3(x+1) + 7(y). 
		  		\end{align*}
	  		where $x$ and $y$ are some arbitrary nonnegative integers.
	  		Since $x+1$ and $y$ are nonnegative integers, it follows that, by the Strong Principle of Mathematical Induction, that the for each integer $n\geq 12$, there are nonnegative integers $a$ and $b$ such that $n=3a+7b$.
		\end{proof}
	\end{problem}

	\begin{problem}{46}
		Use the Strong Principle of Mathematical Induction to prove the following. Let $S=\{i\in \Z: i\geq 2\}$ and let $P$ be a subset of $S$ with the properties that $2,3\in P$ and if $n\in S$, then either $n\in P$ or $n = ab$, where $a,b\in S$. Then every element of $S$ either belongs to $P$ or can be expressed as a product of elements of $P$. 
		[Note: You might recognize the set $P$ of primes. This is an important theorem in mathematics.]
		\begin{proof}
			We proceed by strong induction. Assume that $S=\{i\in \Z : i\geq 2\}$ and let $P$ be a subset of $S$ with the desired properties. Since $2\in P$, it follows that the result is true for $n=2$. Assume that either $i\in P$ or $i$ is equal to the product of elements of $P$ for $2\leq i \leq k$, where $k\geq 2$. We show that either $k+1\in P$ or $k+1$ is a product of elements of $P$. . \\
			 
			Since $k+1\in S$, it follows that either $k+1\in P$ or $k+1=ab$, where $a,b\in S$. In the former, the result is satisfied. In the case of the latter,  $k\not\in P$ and $k+1 = ab$ where $a,b\in S$. However, we know that $2\leq a \leq k$ and $2\leq b \leq k$, which implies, by our inductive hypothesis, that each integer of $a$ and $b$ is either in $P$ or is a product of elements of $P$. In all possible cases, the integer $k+1$ ends up being a product of elements of $P$.
			By the Strong Principle of Mathematical Induction, this result is true.
		\end{proof}
	\end{problem}

	\begin{problem}{47}
		Prove that there exists an odd integer $m$ such that every odd integer $n$ with $n\geq m$ can be expressed either as $3a+11b$ or as $5c+7d$ for nonnegative integers $a,b,c$ and $d$.
		\begin{proof}
			Let $m = 17$. Since $17 = 3(2)+11(1)$, it follows that the result is true for $n=17$. Assume for $17\leq i \leq k$ that if $i$ is odd, then $i$ can be expressed either as $3a+11b$ or as $5c+7d$ for nonnegative integers $a,b,c$ and $d$. We show that if $k+1$ is odd, then it can be expressed either as $3e+11f$ or as $5g+7h$ for nonnegative integers $e,f,g$ and $h$. Note that
			\begin{align*}
				19 &= 5(1)+7(2),\\
				21 &= 3(7) + 11(0),\\
				23 &= 3(4) +11(1),\\
				25 &= 3(1) + 11(2),\\
				27 &= 5(4) + 7(1),\\
				29 &= 3(6) + 11(1).
			\end{align*}
		Hence $k\geq 29$ and so $k+1 \geq 30$. Suppose that $k+1$ is odd. Then,
		\begin{align*}
			k+1 &= (k-11) +12.
		\end{align*}
	Since $17 \leq k-11 \leq k$ and $k-11$ is odd, it follows that either $k-11 = 3e + 11f$ or $k-11 = 5g + 7h$ for some nonnegative integers $e,f,g$ and $h$. Note that $12 = 3(4) = 5+7$. Hence, if $k-11 = 3e+11f$, then $k+1 = 3e+11f +3(4) = 3(e+4)+11f$. On the other hand, if $k-11 = 5g+7h$, then $k+1 = 5g+7h + (5+7) = 5(g+1) + 7(h+1)$. 
	
	By the Strong Priciple of Mathematical Induction, every odd integer $n$ with $n\geq 17$ can be expressed either as $3a+11b$ or as $5c+7d$ for nonnegative integers $a,b,c$ and $d$.
		\end{proof}
	\end{problem}
\end{document}