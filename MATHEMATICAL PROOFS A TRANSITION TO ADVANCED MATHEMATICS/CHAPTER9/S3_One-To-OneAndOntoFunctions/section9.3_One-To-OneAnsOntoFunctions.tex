\documentclass[12pt]{article}
\usepackage[margin=1in]{geometry}
\usepackage{amsmath, amsfonts,amsthm,amssymb,epigraph,etoolbox,mathtools,setspace,enumitem}  
\usepackage{tikz}
\usetikzlibrary{datavisualization} 
\usepackage[makeroom]{cancel} 
\usepackage[linguistics]{forest}
\usetikzlibrary{patterns}
\newcommand{\N}{\mathbb{N}}
\newcommand{\Z}{\mathbb{Z}}
\newcommand{\R}{\mathbb{R}}
\newcommand{\Q}{\mathbb{Q}}
\newcommand{\Mod}[1]{\ (\mathrm{mod}\ #1)}



\newlist{legal}{enumerate}{10}
\setlist[legal]{label=(\alph*)}

\DeclarePairedDelimiter\bra{\langle}{\rvert}
\DeclarePairedDelimiter\ket{\lvert}{\rangle}
\DeclarePairedDelimiterX\braket[2]{\langle}{\rangle}{#1\delimsize\vert #2}


\newenvironment{theorem}[2][Theorem]{\begin{trivlist} \item[\hskip \labelsep {\bfseries #1}\hskip \labelsep {\bfseries #2.}]}{\end{trivlist}}
\newenvironment{lemma}[2][Lemma]{\begin{trivlist} \item[\hskip \labelsep {\bfseries #1}\hskip \labelsep {\bfseries #2.}]}{\end{trivlist}}
\newenvironment{result}[2][Result]{\begin{trivlist} \item[\hskip \labelsep {\bfseries #1}\hskip \labelsep {\bfseries #2.}]}{\end{trivlist}}
\newenvironment{exercise}[2][Exercise]{\begin{trivlist} \item[\hskip \labelsep {\bfseries #1}\hskip \labelsep {\bfseries #2.}]}{\end{trivlist}}
\newenvironment{problem}[2][Problem]{\begin{trivlist} \item[\hskip \labelsep {\bfseries #1}\hskip \labelsep {\bfseries #2.}]}{\end{trivlist}}
\newenvironment{question}[2][Question]{\begin{trivlist} \item[\hskip \labelsep {\bfseries #1}\hskip \labelsep {\bfseries #2.}]}{\end{trivlist}}
\newenvironment{corollary}[2][Corollary]{\begin{trivlist} \item[\hskip \labelsep {\bfseries #1}\hskip \labelsep {\bfseries #2.}]}{\end{trivlist}}
\newenvironment{solution}[1][Solution]{\begin{trivlist} \item[\hskip \labelsep {\bfseries #1}]}{\end{trivlist}}

\setlength\epigraphwidth{8cm}
\setlength\epigraphrule{0pt}

\makeatletter
\patchcmd{\epigraph}{\@epitext{#1}}{\itshape\@epitext{#1}}{}{}
\makeatother

\begin{document}
  
 \title{Section 9.3: One-To-One and Onto Functions}
   \author{Juan Patricio Carrizales Torres}
     \date{Jul 23, 2022}
       \maketitle
  We have seen that a function from $A$ to $B$ is a relation that fulfills the following condition:      
    \begin{equation*}
      a=b \implies f(a)=f(b).
    \end{equation*}
  Furthermore, functions can posses to important properties. A function $f:A\to B$ is said to be \textbf{One-to-One} if every image is unique to its respective $x\in A$, namely,
    \begin{align*}
      f(a)=f(b) \implies a=b\\
      \equiv a\neq b \implies f(a)\neq f(b).
    \end{align*}
 Obviously, for this to be true, $B$ must contain at least the same number of elements as $A$, namely, $|A|\leq |B|$. On the other hand, the function $f$ is said to be \textbf{Onto} if every element in $B$ is the image of some element of $A$, namely,
    \begin{equation*}
      b\in B \implies \exists a\in A, f(a)=b.
    \end{equation*}
    Hence, $f(A)=B$. Clearly, $|B|\leq |A|$, otherwise, there would be not enough elements of $A$ to cover all elements of $B$. Then, if a function is both one-to-one and onto, then $|A|=|B|$.
    \begin{problem}{20}
      A function $f:\Z \to \Z$ is defined by $f(n)=2n+1$. Determine whether $f$ is injective, surjective. 
    \begin{solution}
      First we show that it is injective. Consider two $f(a)=f(b)$ for some $a,b\in \Z$. Then, $2a+1=2b+1$. Substracting 1 to both sides, we get $2a=2b$. Dividing by 2, we obtain $a=b$. However, it is not surjective. Consider any even integer $r$ and so there is no integer $n$ such that $f(n)=2n+1=r$.
    \end{solution}
    \end{problem}
    \begin{problem}{21}
	A function $f:\Z\to \Z$ is defined by $f(n)=n-3$. Determine whether $f$ is injective, surjective.
    \begin{solution}
      The function is both injective and surjective. Consider some some $f(a)=f(b)$ for $a,b\in \Z$. Then, $a-3=b-3$ and so $a=b$. Now, let $y\in \Z$. Note that $x=b+3$ is an integer. Then $f(x) = (y+3)-3=b$.
    \end{solution}
    \end{problem}
    \begin{problem}{23}
      Prove or disprove: For every nonempty set $A$, there exists an injective function $f:A\to \mathcal{P}(A)$.
      \begin{proof}
	Let $g:A\to \mathcal{P}(A)$ be deifned by $f(n)=\left\{ n \right\}$. We show that it is injective. Consider some element $f(a)=f(b)$ for $a,b\in A$, then $\left\{ a \right\}=\left\{ b \right\}$, which implies that $a=b$. Due to the individuality of each element of $A$, $f(n)$ is injective.	
    \end{proof}
    Note that it is impossible to define a function $f:A\to \mathcal{P}(A)$ that is surjective since $|A|<2^{|A|} = |\mathcal{P}(A)|$. 
    \end{problem}
    \begin{problem}{24}
      Determine whether the function $f:\R\to \R$ defined by $f(x)=x^{2}+4x+9$ is one-to-one, onto.
    \begin{solution}
      We show that it is not one-to-one. Consider some $f(x)=f(y)$ for $x,y\in \R$. Then, $x^{2}+4x+9 = y^{2}+4y+9$ and so $x^{2}+4x - (y^{2}+4y)=0$. Note that  
    \begin{align*}
      x^{2}+4x-(y^{2}+4y) &= (x^{2}-y^{2})+4(x-y)\\
      &= (y+x)(x-y)+4(x-y) = (x-y)(y+x+4) = 0.
    \end{align*}
    Hence, either $x-y=0$ or $y+x+4=0$. In the latter, $y=-(x+4)$. For instace, if $x=3$ and $y=1$, then $f(x)=f(y)$.\\
    Also, it is not surjective. Note that   
    \begin{align*}
    x^{2}+4x+9 &= \left(x^{2}+4x+4\right)-4+9\\
    &= \left(x+2\right)^{2}+5\geq 5.
    \end{align*}
    Thus, there is no $x\in \R$ such that $f(x)<4$.

    \end{solution}
    \end{problem}
    \begin{problem}{25}
      Is there a function $f:\R\to \R$ that is onto but not one-to-one? Explain your answer.
    \begin{solution}
      Yes, there is such function. Let the function $g:\R\to \R$ be defined by 
    \begin{align*}
      g(n) =
    \begin{cases}
      n, \; \mathrm{if}\; n\leq -\frac{\pi}{2}\\
      \tan(n), \; \mathrm{if }-\frac{\pi}{2}<n<\frac{\pi}{2}\\ 
      n, \; \mathrm{if}\; n\geq \frac{\pi}{2}.
    \end{cases}
    \end{align*}
    Clearly, $\mathrm{dom}(g)=\R$. Note that, the function $\varphi : \left(-\frac{\pi}{2},\frac{\pi}{2}\right)\to \R$ defined by $\varphi(n)=\tan(n)$ is by itself injective and surjective. However, by adding identity relations for the lower and upper bounds, namely,$\left(-\infty, \frac{\pi}{2}\right]$ and $\left[ \frac{\pi}{2},\infty \right)$, we make sure that $g$ is not injective, in other words, there are $a,b\in \R$ such that $f(a)=f(b)$ and $a\neq b$. 
    \end{solution}
    \end{problem}
    \begin{problem}{26}
      Give an example of a function $f:\N\to\N$ that is
    \begin{enumerate}[label=(\alph*)]
      \item one-to-one and onto 
    \begin{solution}
      Let $f:\N\to \N$ be defined by $f(n)=n$.
    \end{solution}
      \item one-to-one but not onto 
    \begin{solution}
      Let $f$ be defined by $f(n)=2n$. We show that it is one-to-one. Consider some $f(a)=f(b)$ for positive integers $a,b$. Then, $2a=2b$ and so $a=b$. However, note that $\{2n+1:n\in\N\}\not\subseteq \mathrm{range}(f)$. The function $f$ is not surjective.
    \end{solution}
      \item  onto but not one-to-one 
    \begin{solution}
      Let $f$ be defined by $f(1)=1$ and $f(n)=n-1$ if $n\geq 2$. Clearly, $f(1)=f(2)=1$ and so it is not injective. We prove that it is surjective. Consider any $b\in\N$, then $b+1\in \N$ and $f(b+1)=(b+1)-1=b$.
    \end{solution}
      \item neither one-to-one nor onto 
    \begin{solution}
      Let $f$ be defined by $f(n)=1$. Note that $f(a)=f(b)$ for any $a,b\in \N$ and $\mathrm{range}(f) = \left\{ 1 \right\}$. Hence, $f$ is neither onto nor one-to-one. 
    \end{solution}
    \end{enumerate}
    \end{problem}
\end{document}


