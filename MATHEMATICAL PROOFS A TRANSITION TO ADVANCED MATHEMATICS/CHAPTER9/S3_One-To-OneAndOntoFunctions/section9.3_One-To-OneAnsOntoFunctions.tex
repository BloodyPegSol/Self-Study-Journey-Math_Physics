\documentclass[12pt]{article}
\usepackage[margin=1in]{geometry}
\usepackage{amsmath, amsfonts,amsthm,amssymb,epigraph,etoolbox,mathtools,setspace,enumitem}  
\usepackage{tikz}
\usetikzlibrary{datavisualization} 
\usepackage[makeroom]{cancel} 
\usepackage[linguistics]{forest}
\usetikzlibrary{patterns}
\newcommand{\N}{\mathbb{N}}
\newcommand{\Z}{\mathbb{Z}}
\newcommand{\R}{\mathbb{R}}
\newcommand{\Q}{\mathbb{Q}}
\newcommand{\Mod}[1]{\ (\mathrm{mod}\ #1)}



\newlist{legal}{enumerate}{10}
\setlist[legal]{label=(\alph*)}

\DeclarePairedDelimiter\bra{\langle}{\rvert}
\DeclarePairedDelimiter\ket{\lvert}{\rangle}
\DeclarePairedDelimiterX\braket[2]{\langle}{\rangle}{#1\delimsize\vert #2}


\newenvironment{theorem}[2][Theorem]{\begin{trivlist} \item[\hskip \labelsep {\bfseries #1}\hskip \labelsep {\bfseries #2.}]}{\end{trivlist}}
\newenvironment{lemma}[2][Lemma]{\begin{trivlist} \item[\hskip \labelsep {\bfseries #1}\hskip \labelsep {\bfseries #2.}]}{\end{trivlist}}
\newenvironment{result}[2][Result]{\begin{trivlist} \item[\hskip \labelsep {\bfseries #1}\hskip \labelsep {\bfseries #2.}]}{\end{trivlist}}
\newenvironment{exercise}[2][Exercise]{\begin{trivlist} \item[\hskip \labelsep {\bfseries #1}\hskip \labelsep {\bfseries #2.}]}{\end{trivlist}}
\newenvironment{problem}[2][Problem]{\begin{trivlist} \item[\hskip \labelsep {\bfseries #1}\hskip \labelsep {\bfseries #2.}]}{\end{trivlist}}
\newenvironment{question}[2][Question]{\begin{trivlist} \item[\hskip \labelsep {\bfseries #1}\hskip \labelsep {\bfseries #2.}]}{\end{trivlist}}
\newenvironment{corollary}[2][Corollary]{\begin{trivlist} \item[\hskip \labelsep {\bfseries #1}\hskip \labelsep {\bfseries #2.}]}{\end{trivlist}}
\newenvironment{solution}[1][Solution]{\begin{trivlist} \item[\hskip \labelsep {\bfseries #1}]}{\end{trivlist}}

\setlength\epigraphwidth{8cm}
\setlength\epigraphrule{0pt}

\makeatletter
\patchcmd{\epigraph}{\@epitext{#1}}{\itshape\@epitext{#1}}{}{}
\makeatother

\begin{document}
  
 \title{Section 9.3: One-To-One and Onto Functions}
   \author{Juan Patricio Carrizales Torres}
     \date{Jul 23, 2022}
       \maketitle
  We have seen that a function from $A$ to $B$ is a relation that fulfills the following condition:      
    \begin{equation*}
      a=b \implies f(a)=f(b).
    \end{equation*}
  Furthermore, functions can posses to important properties. A function $f:A\to B$ is said to be \textbf{One-to-One} if every image is unique to its respective $x\in A$, namely,
    \begin{align*}
      f(a)=f(b) \implies a=b\\
      \equiv a\neq b \implies f(a)\neq f(b).
    \end{align*}
 Obviously, for this to be true, $B$ must contain at least the same number of elements as $A$, namely, $|A|\leq |B|$. On the other hand, the function $f$ is said to be \textbf{Onto} if every element in $B$ is the image of some element of $A$, namely,
    \begin{equation*}
      b\in B \implies \exists a\in A, f(a)=b.
    \end{equation*}
    Hence, $f(A)=B$. Clearly, $|B|\leq |A|$, otherwise, there would be not enough elements of $A$ to cover all elements of $B$. Then, if a function is both one-to-one and onto, then $|A|=|B|$.
    \begin{problem}{20}
      A function $f:\Z \to \Z$ is defined by $f(n)=2n+1$. Determine whether $f$ is injective, surjective. 
    \begin{solution}
      First we show that it is injective. Consider two $f(a)=f(b)$ for some $a,b\in \Z$. Then, $2a+1=2b+1$. Substracting 1 to both sides, we get $2a=2b$. Dividing by 2, we obtain $a=b$. However, it is not surjective. Consider any even integer $r$ and so there is no integer $n$ such that $f(n)=2n+1=r$.
    \end{solution}
    \end{problem}
\end{document}


