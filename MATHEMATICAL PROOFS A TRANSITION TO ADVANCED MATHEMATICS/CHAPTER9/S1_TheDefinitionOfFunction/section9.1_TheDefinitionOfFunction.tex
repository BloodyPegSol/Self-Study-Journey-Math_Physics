\documentclass[12pt]{article}
\usepackage[margin=1in]{geometry}
\usepackage{amsmath, amsfonts,amsthm,amssymb,epigraph,etoolbox,mathtools,setspace,enumitem}  
\usepackage{tikz}
\usetikzlibrary{datavisualization} 
\usepackage[makeroom]{cancel} 
\usepackage[linguistics]{forest}
\usetikzlibrary{patterns}
\newcommand{\N}{\mathbb{N}}
\newcommand{\Z}{\mathbb{Z}}
\newcommand{\R}{\mathbb{R}}
\newcommand{\Q}{\mathbb{Q}}
\newcommand{\Mod}[1]{\ (\mathrm{mod}\ #1)}
\newcommand{\dom}[1]{\mathrm{dom}\left(#1\right)}
\newcommand{\rng}[1]{\mathrm{range}\left(#1\right)}
\newcommand{\rel}[3]{#2\; #1\; #3}

\newlist{legal}{enumerate}{10}
\setlist[legal]{label=(\alph*)}

\DeclarePairedDelimiter\bra{\langle}{\rvert}
\DeclarePairedDelimiter\ket{\lvert}{\rangle}
\DeclarePairedDelimiterX\braket[2]{\langle}{\rangle}{#1\delimsize\vert #2}

\newenvironment{theorem}[2][Theorem]{\begin{trivlist} \item[\hskip \labelsep {\bfseries #1}\hskip \labelsep {\bfseries #2.}]}{\end{trivlist}}
\newenvironment{lemma}[2][Lemma]{\begin{trivlist} \item[\hskip \labelsep {\bfseries #1}\hskip \labelsep {\bfseries #2.}]}{\end{trivlist}}
\newenvironment{result}[2][Result]{\begin{trivlist} \item[\hskip \labelsep {\bfseries #1}\hskip \labelsep {\bfseries #2.}]}{\end{trivlist}}
\newenvironment{exercise}[2][Exercise]{\begin{trivlist} \item[\hskip \labelsep {\bfseries #1}\hskip \labelsep {\bfseries #2.}]}{\end{trivlist}}
\newenvironment{problem}[2][Problem]{\begin{trivlist} \item[\hskip \labelsep {\bfseries #1}\hskip \labelsep {\bfseries #2.}]}{\end{trivlist}}
\newenvironment{question}[2][Question]{\begin{trivlist} \item[\hskip \labelsep {\bfseries #1}\hskip \labelsep {\bfseries #2.}]}{\end{trivlist}}
\newenvironment{corollary}[2][Corollary]{\begin{trivlist} \item[\hskip \labelsep {\bfseries #1}\hskip \labelsep {\bfseries #2.}]}{\end{trivlist}}
\newenvironment{solution}[2][Solution]{\begin{trivlist} \item[\hskip \labelsep {\bfseries #1}\hskip \labelsep {\bfseries #2.}]}{\end{trivlist}}

\setlength\epigraphwidth{8cm}
\setlength\epigraphrule{0pt}

\makeatletter
\patchcmd{\epigraph}{\@epitext{#1}}{\itshape\@epitext{#1}}{}{}
\makeatother

\begin{document}

 \title{Section 9.1: The Definition of Function}
   \author{Juan Patricio Carrizales Torres}
     \date{July 7, 2022}
       \maketitle
       A very famous type of relation is the function. For some sets $A,B$, a function $f$ is a relation from $A$ to $B$, expressed as $f:A\to B$, such that for every $a\in A$, $(a,b)\in f$ for only one $b\in B$. Hence, $|A|=|f|$. Also, $\text{dom}(f) = A$ and $\text{codom}(f)=B$. 
       For a function $f: A \to B$, Consider some $(a,b)\in f$. Because every ordered pair in $f$ is adscribed to only one $a\in A$, it follows that $(a,b),(a,c)\in f$ implies $b=c$. Thus, $b=f(a)$ is considered as the \textbf{image} of $a$. In fact this is known as \textbf{mapping}. For instance, $f$ is said to map $a$ into $b$. Hence, the \textbf{range} of this relation $f$ can be expressed as
       \begin{align*}
	 \text{range}(f) &=\left\{b \in B:  (a,b)\in f, \; a\in A\right\}\\
	 &= \left\{ f(a): a\in A \right\}.
       \end{align*}
 Now, suppose that we have some subset $C$ of $A$. Then, 
 \begin{align*}
   f(C) &= \left\{ f(x):x\in C \right\}
 \end{align*}
 is known as the \textbf{image} of $C$. Obviously, if $C=A$, then $f(C) = \text{range}(f)$. Furthermore, for some subeset $D$ of $B$, its \textbf{inverse image} is denoted as
 \begin{align*}
   f^{-1} (D) &= \left\{a\in A: f(a)\in D\right\}.
 \end{align*}
 Due to the definition of a function, $f^{-1} (B) = A$. 

 \begin{problem}{3}
   Let $A$ be a nonempty set. If $R$ is a relation from $A$ to $A$ that is both an equivalence relation and a function, then what familiar function is $R$?
   \begin{solution}{3}
     It is some type of identity linear function, namely, $R=\left\{(x,x):x\in A\right\}$. Recall that for it to be a function, each $x\in A$ must be paired with only one $c\in A$. However, each $x$ must be paired with itself, since the reflexive property requires each element of $A$ be related to itself. Note that the property of symmetry follows directly and the transitivity follows vacuously. (Remember that this linear relation is the smallest equivalence relation for some nonempty set). 
   \end{solution}
 \end{problem}

 \begin{problem}{4}
   For the given subset $A_{i}$ of $\R$ and the relation $R_{i}(1\leq i\leq3)$ from $A_{i}$ to $\R$, determine whether $R_{i}$ is a function from $A_{i}$ to $\R$. 
   \begin{enumerate}[label=(\alph*)]
     \item $A_{1} = \R, R_{1}=\left\{ (x,y):x\in A_{1}, y=4x-3 \right\}$
       \begin{solution}{a}
	 It is true that $\text{dom}(R_{1}) = A_{1}$. Now, consider two $y_{1},y_{2}\in \R$ for some $x_{1}=x_{2}\in A_{1}$. Then, $4x_{1}-3=y_{1}=y_{2}=4x_{2}-3$. Hence, $R_{1}$ is a function.	  
       \end{solution}
     \item $A_{2} = [0,\infty), R_{2}=\left\{ (x,y):x\in A_{2}, (y+2)^{2}=x \right\}$
       \begin{solution}{b}
	 Consider two $y_{1},y_{2}\in \R$ such that $(y_{1}+2)^{2}=(y_{2}+2)^{2}$ (they have the same preimage, namely, the same $x\geq 0$). Then, $|y_{1}+2|=|y_{2}+2|$. Note that $y_{2}=-(y_{1}+4)$ fulfills the previous equality. For instance, $(2+2)^{2}=(2-6)^{2}=16$. Thus, $R_{2}$ is not  a function.
       \end{solution}
     \item $A_{3} = \R, R_{3}=\left\{ (x,y):x\in A_{3}, (x+y)^{2}=4 \right\}$
       \begin{solution}{c}
	 Consider two $y_{1},y_{2}\in \R$ such that $x_{1}=x_{2}=c$. Then, $(c+y_{1})^{2}=(c+y_{2})^{2}=4$ and so $|c+y_{1}|=|c+y_{2}|$. Note that for $y_{2} = -(y_{1}+2c)$ the previous equation is fulfuilled. For instance, $(2+0)^{2}=(2-4)^{2}=4$. Hence, $R_{3}$ is not a function. 
	 
       \end{solution}
   \end{enumerate}
 \end{problem}

 \begin{problem}{5}
   Let $A$ and $B$ be nonempty sets and let $R$ be a nonempty relation from $A$ to $B$. Show that there exists a subset $A'$ of $A$ and a subset $f$ of $\R$ such that $f$ is a function from $A'$ to $B$.     
   \begin{proof}
     Let $A$ and $B$ be two nonempty sets and let $R$ be a nonempty relation from $A$ to $B$. Since $R$ is nonempty, let $A'=\left\{a:(a,b)\in R \text{ for some }b\in B \right\}$. Now, for each $a'\in A'$ select only one $b$ such that $(a,b)\in R$, and let $B'\subset B$ be the set containing these elements. Then, let $f=\left\{(a,b):a\in A', b\in B'\right\}$. Thus, $\text{dom}(f) = A'$ and every $a$ is only related to only one $b$ through $f$. Therefore, $f:A'\to B$ is a function.
    \end{proof}
 \end{problem}

 \begin{problem}{8}
   Let $A=\left\{5,6 \right\}$, $B=\left\{ 5,7,8 \right\}$ and $S=\left\{n:n\geq 3 \text{ is an odd integer} \right\}$. A relation $R$ from $A\times B$ to $S$ is defined as $(a,b)\; R\; s$ if $s\mid (a+b)$. Is $R$ a function from $A\times B$ to $S$?
   \begin{solution}{8}
     Note that 
    \begin{align*}
      5+5 &= 10\\
      5+7 &= 12\\
      5+8 &= 13\\
      6+5 &= 11\\
      6+7 &= 13\\
      6+8 &= 14.
    \end{align*}
    Hence,
    \begin{align*}
    (5,5)\; R \;5\\
    (5,7)\; R \;3\\
    (5,8)\; R \;13\\
    (6,5)\; R \;11\\
    (6,7)\; R \;13\\
    (6,8)\; R \;7.
    \end{align*}
    Therefore, $A\times B = \text{dom}(R)$. Now, we have to see whether each element of $A\times B$ has only one image. Note that $11$ and $13$ are odd prime numbers whose only odd divisors greater or equal to three are themselves. On the other hand, $10,12,14$ are even numbers that just have one odd divisor greater or equal to three. Hence, $R$ is a funtion from $A\times B$ to $S$.
   \end{solution} 
    \end{problem}

    \begin{problem}{9}
	Determine which of the following five relations $R_{i}\; (i=1,2,\dots,5)$ are functions.
     \begin{enumerate}[label=(\alph*)]
       \item $R_{1}$ is defined on $\R$ by $x\; R_{1}\;y$ if $x^{2}+y^{2}=1$. 
	 \begin{solution}{(a)}
	   Note that if $x>1$, then $x^{2}> 1$ and $x^{2}+y^{2}> 1$ for all $y\in\R$ since $y^{2}\geq 0$. Therefore, $\R \neq \text{dom}(R_{1})$ and so $R_{1}$ is not a function.
         \end{solution}
       \item $R_{2}$ is defined on $\R$ by $x\; R_{2}\;y$ if $4x^{2}+3y^{2}=1$.
	 \begin{solution}{(b)}
	   The condition for this relation is similar to the one for the previous relation. Hence, it makes sense to think that it is not a function. We proceed to show that $\dom{R_2}\neq \R$. Let  $x>1/2$, then $4x^{2} > 4(1/4) =1$ and so for any real number $y$, $4x^{2} + 3y^{2} > 1$. Hence, $x\not\in \dom{R_{2}}$. 
         \end{solution}
       \item $R_{3}$ is defined from $\N$ to $\Q$ by $a\; R_{3}\;b$ if $3a+5b=1$.
	 \begin{solution}{(c)}
             Consider any positive integer $x$. Then, $\rel{R_{3}}{x}{b}$ if $b=(1-3x)/5$, which is a rational number since $1-3x\in\Z$. Note that each $b$ has only one value for each $x$. Hence, $\dom{R_{3}}=\N$ and every positive integer is related to only one rational number. Therefore, $R_{3}$ is a function.
         \end{solution}
       \item $R_{4}$ is defined on $\R$ by $x\; R_{4}\;y$ if $y=4-|x-2|$.
	 \begin{solution}{(d)}
	   For each real number $x$, $y=4-|x-2|\in \R$ has only one value. Hence, $R_{4}$ is a function.  
         \end{solution}
       \item $R_{5}$ is defined on $\R$ by $x\; R_{5}\;y$ if $|x+y|=1$.
	 \begin{solution}{(e)}
	   Consider any real values $x$ and $y$ such that $|x+y|=1$. Then, $|x+(-2x-y)| =1$. Therefore, $\rel{R_{5}}{x}{y}$ and $\rel{R_{5}}{x}{(-2x-y)}$, and so $R_{5}$ is not a function.
         \end{solution}
     \end{enumerate}
    \end{problem}

    \begin{problem}{10}
      A function $g:\Q\to \Q$ is defined by $g(r) = 4r+1$ for each $r\in \Q$.
      \begin{enumerate}[label=(\alph*)]
	\item Determine $g(\Z)$ and $g(E)$, where $E$ is the set of even integers.
	  \begin{solution}{(a)}
	    The sets are
    \begin{align*}
      g(\Z) &= \left\{g(r):r\in\Z \right\}\\
    &= \left\{4r+1:r\in\Z \right\}=\left\{\dots,-7,-3,1,5,9,\dots\right\}.
    \end{align*}
    and
    \begin{align*}
      g(E) &= \left\{g(2r):r\in\Z\right\}\\
      &= \left\{8r+1:r\in\Z\right\}=\left\{\dots,-15,-7,1,9,17,\dots\right\}.
    \end{align*}
    \end{solution}
	\item Determine $g^{-1}(\N)$ and $g^{-1}(D)$, where $D$ is the set of odd integers.
    \begin{solution}{(b)}
      The sets are
    \begin{align*}
      g^{-1}(\N) &= \left\{r\in\Q:g(r)\in\N\right\}\\
      &= \left\{r\in\Q:4r+1\in\N\right\}\\
      &= \left\{r\in\Q:4r=x, \; x\text{ is a nonegative integer}\right\}\\
      &= \left\{x/4: x\text{ is a nonegative integer}\right\} = \left\{0,1/4,1/2,3/4,1,5/4,\dots\right\}
    \end{align*}
    and
    \begin{align*}
      g^{-1}(D) &= \left\{r\in\Q:g(r)\in D\right\}\\
      &= \left\{r\in\Q: 4r+1=2k+1,\; k\in\Z\right\}\\
      &= \left\{k/2: k\in\Z\right\} = \left\{\dots,-1,-1/2,0,1/2,1,\dots\right\}.
    \end{align*}
    \end{solution}
    \end{enumerate}
    \end{problem}

    \begin{problem}{11}
      Let $C=\left\{ x\in\R:x\geq 1 \right\}$ and $D=\R^{+}$. For each function $f$ defined below, determine $f(C),f^{-1}(C),f^{-1}(D)$ and $f^{-1}({1})$.
      \begin{enumerate}[label=(\alph*)]
	\item $f:\R\to\R$ is defined by $f(x)=x^{2}$.
	\item $f:\R^{+}\to\R$ is defined by $f(x)=\ln{x}$.
	\item $f:\R\to\R$ is defined by $f(x)=e^{x}$.
	\item $f:\R\to\R$ is defined by $f(x)=\sin{x}$.
	\item $f:\R\to\R$ is defined by $f(x)=2x-x^{2}$.
    \end{enumerate}
    \end{problem}

    \begin{problem}{12}
      For a function $f:A\to B$ and subsets $C$ and $D$ of $A$ and $E$, and $F$ of $B$, prove the following.
      \begin{enumerate}[label=(\alph*)]
	\item $f(C\cup D) = f(C)\cup f(D)$
	\item $f(C\cap D) \subseteq f(C)\cap f(D)$
	\item $f(C)-f(D) \subseteq f(C-D)$
	\item $f^{-1}(E\cup F) = f^{-1}(E)\cup f^{-1}(F)$
	\item $f^{-1}(E\cap F) = f^{-1}(E)\cap f^{-1}(F)$
	\item $f^{-1}(E- F) = f^{-1}(E)- f^{-1}(F)$
	
    \end{enumerate}
    \end{problem}
 \end{document}


