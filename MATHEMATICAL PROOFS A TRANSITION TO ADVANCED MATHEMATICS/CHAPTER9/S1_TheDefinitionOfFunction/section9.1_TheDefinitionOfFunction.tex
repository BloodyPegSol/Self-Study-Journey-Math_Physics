\documentclass[12pt]{article}
\usepackage[margin=1in]{geometry}
\usepackage{amsmath, amsfonts,amsthm,amssymb,epigraph,etoolbox,mathtools,setspace,enumitem}  
\usepackage{tikz}
\usetikzlibrary{datavisualization} 
\usepackage[makeroom]{cancel} 
\usepackage[linguistics]{forest}
\usetikzlibrary{patterns}
\newcommand{\N}{\mathbb{N}}
\newcommand{\Z}{\mathbb{Z}}
\newcommand{\R}{\mathbb{R}}
\newcommand{\Q}{\mathbb{Q}}
\newcommand{\Mod}[1]{\ (\mathrm{mod}\ #1)}



\newlist{legal}{enumerate}{10}
\setlist[legal]{label=(\alph*)}

\DeclarePairedDelimiter\bra{\langle}{\rvert}
\DeclarePairedDelimiter\ket{\lvert}{\rangle}
\DeclarePairedDelimiterX\braket[2]{\langle}{\rangle}{#1\delimsize\vert #2}


\newenvironment{theorem}[2][Theorem]{\begin{trivlist} \item[\hskip \labelsep {\bfseries #1}\hskip \labelsep {\bfseries #2.}]}{\end{trivlist}}
\newenvironment{lemma}[2][Lemma]{\begin{trivlist} \item[\hskip \labelsep {\bfseries #1}\hskip \labelsep {\bfseries #2.}]}{\end{trivlist}}
\newenvironment{result}[2][Result]{\begin{trivlist} \item[\hskip \labelsep {\bfseries #1}\hskip \labelsep {\bfseries #2.}]}{\end{trivlist}}
\newenvironment{exercise}[2][Exercise]{\begin{trivlist} \item[\hskip \labelsep {\bfseries #1}\hskip \labelsep {\bfseries #2.}]}{\end{trivlist}}
\newenvironment{problem}[2][Problem]{\begin{trivlist} \item[\hskip \labelsep {\bfseries #1}\hskip \labelsep {\bfseries #2.}]}{\end{trivlist}}
\newenvironment{question}[2][Question]{\begin{trivlist} \item[\hskip \labelsep {\bfseries #1}\hskip \labelsep {\bfseries #2.}]}{\end{trivlist}}
\newenvironment{corollary}[2][Corollary]{\begin{trivlist} \item[\hskip \labelsep {\bfseries #1}\hskip \labelsep {\bfseries #2.}]}{\end{trivlist}}
\newenvironment{solution}[2][Solution]{\begin{trivlist} \item[\hskip \labelsep {\bfseries #1}\hskip \labelsep {\bfseries #2.}]}{\end{trivlist}}

\setlength\epigraphwidth{8cm}
\setlength\epigraphrule{0pt}

\makeatletter
\patchcmd{\epigraph}{\@epitext{#1}}{\itshape\@epitext{#1}}{}{}
\makeatother

\begin{document}
  
 \title{Section 9.1: The Definition of Function}
   \author{Juan Patricio Carrizales Torres}
     \date{July 7, 2022}
       \maketitle
       A very famous type of relation is the function. For some sets $A,B$, a function $f$ is a relation from $A$ to $B$, expressed as $f:A\to B$, such that for every $a\in A$, $(a,b)\in f$ for only one $b\in B$. Hence, $|A|=|f|$. Also, since $f$ is a relation, $\text{dom}(f) = A$ and $\text{codom}(f)=B$. 
       For a function $f: A \to B$, Consider some $(a,b)\in f$. Because every ordered pair in $f$ is adscribed to only one $a\in A$, it follows that $(a,b),(a,c)\in f$ implies $b=c$. Thus, $b=f(a)$ is considered as the \textbf{image} of $a$. In fact this is known as \textbf{mapping}. For instance, $f$ is said to map $a$ into $b$. Hence, the \textbf{range} of this relation $f$ can be expressed as
       \begin{align*}
	 \text{range}(f) &=\left\{b \in B:  (a,b)\in f, \; a\in A\right\}\\
	 &= \left\{ f(a): a\in A \right\}.
       \end{align*}
 Now, suppose that we have some subset $C$ of $A$. Then, 
 \begin{align*}
   f(C) &= \left\{ f(x):x\in C \right\}
 \end{align*}
 is known as the \textbf{image} of $C$. Obviously, if $C=A$, then $f(C) = \text{range}(f)$. Furthermore, for some subeset $D$ of $B$, its \textbf{inverse image} is denoted as
 \begin{align*}
   f^{-1} (D) &= \left\{a\in A: f(a)\in D\right\}.
 \end{align*}
 Due to the definition of a function, $f^{-1} (B) = A$. 
\end{document}


