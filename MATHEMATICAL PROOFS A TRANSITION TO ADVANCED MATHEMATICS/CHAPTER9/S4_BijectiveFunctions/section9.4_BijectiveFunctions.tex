\documentclass[12pt]{article}
\usepackage[margin=1in]{geometry}
\usepackage{amsmath, amsfonts,amsthm,amssymb,epigraph,etoolbox,mathtools,setspace,enumitem}  
\usepackage{tikz}
\usetikzlibrary{datavisualization} 
\usepackage[makeroom]{cancel} 
\usepackage[linguistics]{forest}
\usetikzlibrary{patterns}
\newcommand{\N}{\mathbb{N}}
\newcommand{\Z}{\mathbb{Z}}
\newcommand{\R}{\mathbb{R}}
\newcommand{\Q}{\mathbb{Q}}
\newcommand{\Mod}[1]{\ (\mathrm{mod}\ #1)}
\newcommand{\Lim}[1]{\mathrm{lim}(#1)}
\newcommand{\Abs}[1]{\left\vert #1 \right\vert}
\newcommand{\Dom}[1]{\mathrm{dom}(#1)}
\newcommand{\Range}[1]{\mathrm{range}(#1)}

\newlist{legal}{enumerate}{10}
\setlist[legal]{label=(\alph*)}
\setenumerate[legal]{label=(\alph*)}

\DeclarePairedDelimiter\bra{\langle}{\rvert}
\DeclarePairedDelimiter\ket{\lvert}{\rangle}
\DeclarePairedDelimiterX\braket[2]{\langle}{\rangle}{#1\delimsize\vert #2}


\newenvironment{theorem}[2][Theorem]{\begin{trivlist} \item[\hskip \labelsep {\bfseries #1}\hskip \labelsep {\bfseries #2.}]}{\end{trivlist}}
\newenvironment{lemma}[2][Lemma]{\begin{trivlist} \item[\hskip \labelsep {\bfseries #1}\hskip \labelsep {\bfseries #2.}]}{\end{trivlist}}
\newenvironment{result}[2][Result]{\begin{trivlist} \item[\hskip \labelsep {\bfseries #1}\hskip \labelsep {\bfseries #2.}]}{\end{trivlist}}
\newenvironment{exercise}[2][Exercise]{\begin{trivlist} \item[\hskip \labelsep {\bfseries #1}\hskip \labelsep {\bfseries #2.}]}{\end{trivlist}}
\newenvironment{problem}[2][Problem]{\begin{trivlist} \item[\hskip \labelsep {\bfseries #1}\hskip \labelsep {\bfseries #2.}]}{\end{trivlist}}
\newenvironment{question}[2][Question]{\begin{trivlist} \item[\hskip \labelsep {\bfseries #1}\hskip \labelsep {\bfseries #2.}]}{\end{trivlist}}
\newenvironment{corollary}[2][Corollary]{\begin{trivlist} \item[\hskip \labelsep {\bfseries #1}\hskip \labelsep {\bfseries #2.}]}{\end{trivlist}}
\newenvironment{solution}[1][Solution]{\begin{trivlist} \item[\hskip \labelsep {\bfseries #1}]}{\end{trivlist}}

\setlength\epigraphwidth{8cm}
\setlength\epigraphrule{0pt}

\makeatletter
\patchcmd{\epigraph}{\@epitext{#1}}{\itshape\@epitext{#1}}{}{}
\makeatother

\begin{document}
  
 \title{Section 9.4: Bijective Functions}
   \author{Juan Patricio Carrizales Torres}
     \date{Jul 28, 2022}
       \maketitle

       As it was mentioned in the previous section, for finite sets $A$ and $B$, $|A|\geq |B|$ is a necessary and sufficient condition for an onto function $f:A\to B$ to exist. The same can be said for $|A|\leq |B|$ and some one-to-one function $g:A\to B$. Since we are talking about positive integers, it must be true that $|A|=|B|$ is a necessary and sufficient condition for an onto and one-to-one function $\varphi:A\to B$ to exist, knwon as a bijective function. \\
       In fact, for finite sets $B$ and $C$ such that $|B|=|C|=n$, there are $n!$ distinct bijective functions from $B$ to $C$. Namely, every bijective function is a permutation of the elements of $|C|$ for $n$ spaces. Furthermore, for any function $f$ from $B$ to $C$, $f$ is onto if and only if $f$ is one-to-one. All this makes sense for finite sets, we must make sure to pair all elements of $C$ with the constriction of assigning one unique element to every element of $B$. However, this intuition does not work for analyzing the cases with infinite ones.\\
       Let $A,B$ be sets. So far, we defined the function $f:A\to B$ as a relation from $A$ to $B$ such that
    \begin{enumerate}
      \item $x\in A \implies \exists b\in B, (a,b)\in f$
      \item $(a,b),(a,c)\in f \implies b=c$
    \end{enumerate}
    If a relation satisfies (b), then it is called \textbf{well-defined}.\\
    Lastly, the identity function $i_{S}$ on ANY nonempty set $S$ defined by $i_{S}(n)=n$ for all $n\in S$ is bijective. 
    \begin{problem}{31}
      Let $f:\Z_{5} \to \Z_{5}$ be a function defined by $f([a]) = [2a+3]$.
    \begin{enumerate}
      \item Show that $f$ is well-defined.
    \begin{proof}
      Consider two $[a]=[b]$ such that $[a],[b]\in \Z_{5}$. Then, $a \equiv b \Mod{5}$ which implies that $a-b=5k$ for some $k\in\Z$. Then, $f([a]) = [2a+3]$ and $f([b])=[2b+3]$. Note that 
    \begin{align*}
      (2a+3)-(2b+3) &= 2(a-b) = 5(2k).
    \end{align*}
    Therefore, $(2a+3)\equiv (2b+3) \Mod{5}$ and so $f([a]) = f([b])$
    \end{proof}
      \item Determine wheter $f$ is bijective.
    \begin{proof}
      We know that $\Z_{5} = \left\{ [0],[1],[2],[3],[4] \right\}$. Note that $f([0]) = [3], f([1])=[5]=[0], f([2])=[7]=[2], f([3])=[4]$ and $f([4])=[11]=[1]$. Hence, all elements of $\Z_{5}$ are paired with a unique element of $\Z_{5}$. The function is bijective.
    \end{proof}
    \end{enumerate}
    \end{problem}
    \begin{problem}{33}
      Let $A=[0,1]$ denote the closed interval of real numbers between 0 and 1. Give an example of two different bijective functions $f_{1}$ and $f_{2}$ from $A$ to $A$, neither of which is the identity function.
    \begin{enumerate}
      \item $f:A\to A$ defined by $f(n)$
    \end{enumerate}
    \end{problem}
    \begin{problem}{34}
      Give a proof of Theorem 7 using mathematical induction.
    \begin{solution}
      If $A$ and $B$ are sets with $|A|=|B|=n$, then there are $n!$ bijective functions from $A$ to $B$. 
    \begin{proof}
      We proceed by induction. Let $A$ and $B$ be sets with $|A|=|B|=1$, then there is only $1=1!$ bijective function from $A$ to $B$, namely, the pairing of the only element of $A$ with the only element of $B$. In fact, this is the only function from $A$ to $B$ since $\Abs{B^{A}} = 1$. \\
      Suppose for sets $A_{1}$ and $B_{1}$ with $|A_{1}| = |B_{1}|=k$ that there are $k!$ bijective functions from $A$ to $B$. We prove for sets $A_{2}$ and $B_{2}$ with $|A_{2}|=|B_{2}|=k+1$ that there are $(k+1)!$ bijective functions.\\
      By our inductive hypothesis, we can only create $k!$ distinct bijective functions by fixing an element $(a_{k+1}, b_{k+1})$ in all of them since the remaining elements correspond to a bijective function from $\left\{ a_{1},a_{2},\dots,a_{k} \right\}$ to $\left\{ b_{1},b_{2},\dots,b_{k} \right\}$.  Note that we can do this with $(a_{k+1},b_{k}),(a_{k+1},b_{k-1}),\dots,$ $(a_{k+1},b_{2}),(a_{k+1},b_{1})$. Therefore, for each of the possible $k+1$ images of $a_{k+1}$, there are only $k!$ distinct bijective functions. By the Principle of Mathematical Induction,  there are $(k+1)k! = (k+1)!$ bijective functions from $A_{2}$ to $B_{2}$. 
    \end{proof}
    \end{solution}
    \end{problem}
    \begin{problem}{35}
      For two finite nonempty sets $A$ and $B$, let $R$ be a relation from $A$ to $B$ such that $\Range{R}=B$. Define the domination number $\gamma(R)$ of $R$ as the smallest cardinality of a subset $S\subseteq A$ such that for every element $y$ of $B$, there is an element $x\in S$ such that $x$ is related to $y$.
    \begin{enumerate}
      \item Let $A=\left\{ 1,2,3,4,5,6,7 \right\}$ and $B=\left\{ a,b,c,d,e,f,g \right\}$ and let 
    \begin{align*}
      R= \{ &(1,c),(1,e),(2,c),(2,f),(2,g),(3,b),(3,f),(4,a),\\
      &(4,c),(4,g),(5,a),(5,b),(5,c),(6,d),(6,e),(7,a),(7,g) \}.
    \end{align*}
    Determine $\gamma(R)$. 
    \begin{solution}
      Observe that $B$ has 7 elements and each element of $A$ is related to either 2 or 3 distinct elements of $B$. Therefore, wihtout looking at $R$, we can assure that $S\subseteq A$ has $|S|\geq 3$. For instance, $S=\left\{3,4,6\right\}$. Therefore, $\gamma(R)=3$.
    \end{solution}
      \item If $R$ is an equivalence relation defined on a finite nonempty set $A$ (and so $B=A$), then what is $\gamma(R)$?
    \begin{solution}
      Since $R$ is an equivalence relation on the finite set $A$, it follows that there are $n$ distinct equivalence classes. We know that the union of the distinct equivalence classes is $A$, they are pairwise disjoint and any element belonging to one of them is related to itself and all elements inside of the equivalence class. Hence, $\gamma(R) =n$.
    \end{solution}
      \item If $f$ is a bijective function from $A$ to $B$, then what is $\gamma(f)$?
    \begin{solution}
      Clearly, $\gamma(f) = |A|$ due to the onto and one-to-one properties of bijective functions.
    \end{solution}
    \end{enumerate}
    \end{problem}
    \begin{problem}{36}
      Let $A=\left\{ a,b,c,d,e,f \right\}$ and $B=\left\{u,v,w,x,y,z \right\}$. With each element $r\in A$, there is associated a list or subset $L(r) \subseteq B$. The goal is to define a ``list function'' $\varphi: A\to B$ with the property that $\varphi(r)\in L(r)$ for each $r\in A$.
    \begin{enumerate}
      \item For $L(a) = \left\{ w,x,y \right\}$, $L(b)=\left\{ u,z \right\}$, $L(c)=\left\{ u,v \right\}$, $L(d)=\left\{ u,w \right\}$, $L(e)=\left\{ u,x,y \right\}$, $L(f)=\left\{ v,y \right\}$, does there exist a bijective list function $\varphi: A\to B$ for these lists?
    \begin{solution}
      Let $\varphi = \left\{ (a,x), (b,z), (c,v), (d,w), (e,u), (f,y) \right\}$. Then $\varphi$ is a bijective list function from $A$ to $B$. 
    \end{solution}
      \item For $L(a) = \left\{ u,v,x,y \right\}$, $L(b)=\left\{ v,w,y \right\}$, $L(c)=\left\{ v,y \right\}$, $L(d)=\left\{ u,w,x,z \right\}$, $L(e)=\left\{ v,w \right\}$, $L(f)=\left\{ w,y \right\}$, does there exist a bijective list function $\varphi: A\to B$ for these lists?
    \begin{solution}
      Note that the only list that contains $z$ is $L(d)$. Hence, $(d,z)\in \varphi$. However, $u$ and $x$ are contained only in $L(d),L(a)$ and $a$ can only have one image. Hence, there is no onto (bijective) list function $\varphi: A\to B$. Also, note that $\varphi(b),\varphi(c),\varphi(e),\varphi(f)\in \left\{ v,w,y \right\}$.
    \end{solution}
    \end{enumerate}
    \end{problem}
       \end{document}

