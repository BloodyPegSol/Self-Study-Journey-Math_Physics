\documentclass[12pt]{article}
\usepackage[margin=1in]{geometry}
\usepackage{amsmath, amsfonts,amsthm,amssymb,epigraph,etoolbox,mathtools,setspace,enumitem}  
\usepackage{tikz}
\usetikzlibrary{datavisualization} 
\usepackage[makeroom]{cancel} 
\usepackage[linguistics]{forest}
\usetikzlibrary{patterns}
\newcommand{\N}{\mathbb{N}}
\newcommand{\Z}{\mathbb{Z}}
\newcommand{\R}{\mathbb{R}}
\newcommand{\Q}{\mathbb{Q}}
\newcommand{\Mod}[1]{\ (\mathrm{mod}\ #1)}
\newcommand{\Lim}[1]{\mathrm{lim}(#1)}
\newcommand{\Abs}[1]{\left\vert #1 \right\vert}
\newcommand{\Dom}[1]{\mathrm{dom}(#1)}
\newcommand{\Range}[1]{\mathrm{range}(#1)}

\newlist{legal}{enumerate}{10}
\setlist[legal]{label=(\alph*)}
\setenumerate[legal]{label=(\alph*)}

\DeclarePairedDelimiter\bra{\langle}{\rvert}
\DeclarePairedDelimiter\ket{\lvert}{\rangle}
\DeclarePairedDelimiterX\braket[2]{\langle}{\rangle}{#1\delimsize\vert #2}


\newenvironment{theorem}[2][Theorem]{\begin{trivlist} \item[\hskip \labelsep {\bfseries #1}\hskip \labelsep {\bfseries #2.}]}{\end{trivlist}}
\newenvironment{lemma}[2][Lemma]{\begin{trivlist} \item[\hskip \labelsep {\bfseries #1}\hskip \labelsep {\bfseries #2.}]}{\end{trivlist}}
\newenvironment{result}[2][Result]{\begin{trivlist} \item[\hskip \labelsep {\bfseries #1}\hskip \labelsep {\bfseries #2.}]}{\end{trivlist}}
\newenvironment{exercise}[2][Exercise]{\begin{trivlist} \item[\hskip \labelsep {\bfseries #1}\hskip \labelsep {\bfseries #2.}]}{\end{trivlist}}
\newenvironment{problem}[2][Problem]{\begin{trivlist} \item[\hskip \labelsep {\bfseries #1}\hskip \labelsep {\bfseries #2.}]}{\end{trivlist}}
\newenvironment{question}[2][Question]{\begin{trivlist} \item[\hskip \labelsep {\bfseries #1}\hskip \labelsep {\bfseries #2.}]}{\end{trivlist}}
\newenvironment{corollary}[2][Corollary]{\begin{trivlist} \item[\hskip \labelsep {\bfseries #1}\hskip \labelsep {\bfseries #2.}]}{\end{trivlist}}
\newenvironment{solution}[1][Solution]{\begin{trivlist} \item[\hskip \labelsep {\bfseries #1}]}{\end{trivlist}}

\setlength\epigraphwidth{8cm}
\setlength\epigraphrule{0pt}

\makeatletter
\patchcmd{\epigraph}{\@epitext{#1}}{\itshape\@epitext{#1}}{}{}
\makeatother

\begin{document}
  
 \title{Section 9.4: Bijective Functions}
   \author{Juan Patricio Carrizales Torres}
     \date{Jul 28, 2022}
       \maketitle

       As it was mentioned in the previous section, for finite sets $A$ and $B$, $|A|\geq |B|$ is a necessary and sufficient condition for an onto function $f:A\to B$ to exist. The same can be said for $|A|\leq |B|$ and some one-to-one function $g:A\to B$. Since we are talking about positive integers, it must be true that $|A|=|B|$ is a necessary and sufficient condition for an onto and one-to-one function $\varphi:A\to B$ to exist, knwon as a bijective function. \\
       In fact, for finite sets $B$ and $C$ such that $|B|=|C|=n$, there are $n!$ distinct bijective functions from $B$ to $C$. Namely, every bijective function is a permutation of the elements of $|C|$ for $n$ spaces. Furthermore, for any function $f$ from $B$ to $C$, $f$ is onto if and only if $f$ is one-to-one. All this makes sense for finite sets, we must make sure to pair all elements of $C$ with the constriction of assigning one unique element to every element of $B$. However, this intuition does not work for analyzing the cases with infinite ones.\\
       Let $A,B$ be sets. So far, we defined the function $f:A\to B$ as a relation from $A$ to $B$ such that
    \begin{enumerate}
      \item $x\in A \implies \exists b\in B, (a,b)\in f$
      \item $(a,b),(a,c)\in f \implies b=c$
    \end{enumerate}
    If a relation satisfies (b), then it is called \textbf{well-defined}.\\
    Lastly, the identity function $i_{S}$ on ANY nonempty set $S$ defined by $i_{S}(n)=n$ for all $n\in S$ is bijective. 
       \end{document}

