\documentclass[12pt]{article}
\usepackage[margin=1in]{geometry}
\usepackage{amsmath, amsfonts,amsthm,amssymb,epigraph,etoolbox,mathtools,setspace,enumitem}  
\usepackage{tikz}
\usetikzlibrary{datavisualization} 
\usepackage[makeroom]{cancel} 
\usepackage[linguistics]{forest}
\usetikzlibrary{patterns}
\newcommand{\N}{\mathbb{N}}
\newcommand{\Z}{\mathbb{Z}}
\newcommand{\R}{\mathbb{R}}
\newcommand{\Q}{\mathbb{Q}}
\newcommand{\Mod}[1]{\ (\mathrm{mod}\ #1)}
\newcommand{\Lim}[1]{\mathrm{lim}(#1)}
\newcommand{\Abs}[1]{\left\vert #1 \right\vert}
\newcommand{\Dom}[1]{\mathrm{dom}(#1)}
\newcommand{\Range}[1]{\mathrm{range}(#1)}

\newlist{legal}{enumerate}{10}
\setlist[legal]{label=(\alph*)}
\setenumerate[legal]{label=(\alph*)}

\DeclarePairedDelimiter\bra{\langle}{\rvert}
\DeclarePairedDelimiter\ket{\lvert}{\rangle}
\DeclarePairedDelimiterX\braket[2]{\langle}{\rangle}{#1\delimsize\vert #2}


\newenvironment{theorem}[2][Theorem]{\begin{trivlist} \item[\hskip \labelsep {\bfseries #1}\hskip \labelsep {\bfseries #2.}]}{\end{trivlist}}
\newenvironment{lemma}[2][Lemma]{\begin{trivlist} \item[\hskip \labelsep {\bfseries #1}\hskip \labelsep {\bfseries #2.}]}{\end{trivlist}}
\newenvironment{result}[2][Result]{\begin{trivlist} \item[\hskip \labelsep {\bfseries #1}\hskip \labelsep {\bfseries #2.}]}{\end{trivlist}}
\newenvironment{exercise}[2][Exercise]{\begin{trivlist} \item[\hskip \labelsep {\bfseries #1}\hskip \labelsep {\bfseries #2.}]}{\end{trivlist}}
\newenvironment{problem}[2][Problem]{\begin{trivlist} \item[\hskip \labelsep {\bfseries #1}\hskip \labelsep {\bfseries #2.}]}{\end{trivlist}}
\newenvironment{question}[2][Question]{\begin{trivlist} \item[\hskip \labelsep {\bfseries #1}\hskip \labelsep {\bfseries #2.}]}{\end{trivlist}}
\newenvironment{corollary}[2][Corollary]{\begin{trivlist} \item[\hskip \labelsep {\bfseries #1}\hskip \labelsep {\bfseries #2.}]}{\end{trivlist}}
\newenvironment{solution}[1][Solution]{\begin{trivlist} \item[\hskip \labelsep {\bfseries #1}]}{\end{trivlist}}

\setlength\epigraphwidth{8cm}
\setlength\epigraphrule{0pt}

\makeatletter
\patchcmd{\epigraph}{\@epitext{#1}}{\itshape\@epitext{#1}}{}{}
\makeatother

\begin{document}
  
 \title{Section 9.5: Composition of Functions}
   \author{Juan Patricio Carrizales Torres}
     \date{Aug 1, 2022}
       \maketitle
   
       We have previously defined operations on sets such as the integers modulo $n$. Some sets of functions are no exception. Let $A,B',B$ and $C$ be nonempty sets and consider the functions $f:A\to B'$ and $g:B\to C$. If $B' \subseteq B$, namely, if $\Range{f} \subseteq \Dom{g}$, then it is possible to create a new function from $A$ to $C$ called the composition of $f$ and $g$. This composition $g\circ f$ is defined by
    \begin{equation*}
   ( g\circ f)(x) = g(f(x)) \text{ for all } x\in A.
    \end{equation*}
    Furthermore, it has some useful properties. Consider two functions $f$ and $g$ such that their composition $g\circ f$ is defined, then 
    \begin{enumerate}
      \item If both $g$ and $f$ are injective (surjective), then the composition $g\circ f$ is injective (surjective).
    \end{enumerate}
    Clearly, one can further conclude that if $g$ and $f$ are bijective, then their composition $g\circ f$ is bijective. Keep in mind that in the beginning of the paragraph we assumed that their composition $g\circ f$ is defined. However, this is not a sufficient condition for $f\circ g$ to be defined. This depends on whether $\Range{g} \subseteq \Dom{f}$ is true or not. \\

    Also, for nonempty functions $f,g,h$, if the compositions $g\circ f$ and $h\circ g$ are defined, then $h\circ (g \circ f)$ and $(h\circ g)\circ f$ are defined. Furthermore, $h\circ(g\circ f) = (h\circ g)\circ f$ and so the composition of $f,g,h$ is \textbf{associative}. \\
    Lastly, let's prove the following theorem.
    \begin{theorem}{9.5.1}
      Let $g$ and $f$ be nonempty functions. If $\Range{f}\subseteq \Dom{g}$ then  $g\circ f$ is a function.
    \begin{proof}
      Assume that $\Range{f}\subseteq \Dom{g}$. Consider some $(x,y)\in f$. Then, $(y,z)\in g$ and so $(x,z)\in g\circ f$. Hence, for any $x\in \Dom{f}=\Dom{g\circ f}$, there is an image $g(f(x)) = (g\circ f)(x)$ defined. We now prove that $g\circ f$ is well-defined. Consider two $a,b\in \Dom{g\circ f} = \Dom{f}$ such that $a=b$. Then, $f(a)=f(b)\in \Dom{g}$ and so $g(f(a))=g(f(b))$. Hence, $(g\circ f)(a)=(g\circ f)(b)$.
    \end{proof}
    \end{theorem}
    \begin{problem}{38}
      Two functions $f:\R \to \R$ and $g:\R \to \R$ are defined by $f(x) = 3x^{2}+1$ and $g(x) = 5x-3$ for all $x\in \R$. Determine $(g\circ f)(1)$ and $(f\circ g)(1)$.
    \begin{solution}
      The composition functions $g\circ f:\R \to \R$ and $f\circ g: \R \to \R$ are defined by $g(f(x)) = 5\left( 3x^{2}+1 \right)-3 = 15x^{2}+2$ and $f(g(x)) = 3\left( 5x-3\right)^{2}+1 = 75x^{2}-90x+28$ for all $x\in \R$.\\
      Hence, $(g\circ f)(1) =17$ and $(f\circ g)(1) = 13$.
    \end{solution}
    \end{problem}
    \begin{problem}{39}
      Two functions $f:\Z_{10} \to \Z_{10}$ and $g:\Z_{10} \to \Z_{10}$ are defined by $f([a])=[3a]$ and $g([a])=[7a]$.
    \begin{enumerate}
      \item Determine $g\circ f$ and $f\circ g$.
    \begin{solution}
    The composition functions $g\circ f:\Z_{10}\to \Z_{10}$ and $f\circ g: \Z_{10} \to \Z_{10}$ are defined by $g(f([x])) = [21a]=[21][a]=[1][a]=[a]$ and $f(g([x])) = [21a]=[a]$ for every $[a]\in \Z_{10}$. Therefore, $g\circ f=f\circ g$.
    \end{solution}
      \item What can be concluded as a result of (a)?
    \begin{solution}
      Both $g\circ f$ and $f\circ g$ are identity functions on $\Z_{10}$.
    \end{solution}
    \end{enumerate}
    \end{problem}
    \begin{problem}{40}
      Let $A$ and $B$ be nonempty sets. Prove that if $f:A\to B$, then $f\circ i_{A} = f$ and $i_{B} \circ f = f$.
    \begin{proof}
      Note that $\Range{i_{A}} = A = \Dom{f}$ and $\Range{f} \subseteq \Dom{i_{B}} = B$. Hence, both functions $f\circ i_{A}:A\to B$ and $i_{B}\circ f:A\to B$ are defined by $(f\circ i_{A})(x) = f(i_{A}(x))=f(x)$ and $(i_{B}\circ f)(x) = i_{B}(f(x))= f(x)$ for every $x\in A$. Both have the same Dominion and rule as $f$. Hence, $f\circ i_{A} = i_{B} \circ f = f$.
    \end{proof}
    \end{problem}
    \begin{problem}{41}
      Let $A$ be a nonempty set and let $f:A \to A$ be a function. Prove that if $f\circ f = i_{A}$, then $f$ is bijective.
    \begin{proof}
      Assume that $f\circ f = i_{A}$ for a function $f:A\to A$. First, we show that $f$ is injective. Let $f(a)=f(b)$ for some $a,b\in A$. Then, $(f\circ f)(a) = f(f(a)) = i_{A}(a) = a$ and $(f\circ f)(b) = f(f(b)) = i_{A}(b) = b$. Since $f$ is a function and $f(a)=f(b)$, it follows that $f(f(a)) = a = b = f(f(b))$.\\
      We now show that $f$ is surjective. Consider any $c\in A$. Then, there is some $f(c)\in A$ and so $f(f(c)) = (f\circ f)(c) = c$.
    \end{proof}
    \end{problem}
    \begin{problem}{42}
      Prove or disprove the following:
    \begin{enumerate}
      \item If two functions $f:A\to B$ and $g:B\to C$ are both bijective, then $g\circ f:A\to C$ is bijective.
    \begin{proof}
      Since both $f$ and $g$ are bijective, it follows that $f$ and $g$ are both \textbf{injective} and \textbf{surjective}. By \textbf{Theorem 11}, $g\circ f: A\to C$ is injective and surjective, which, by definition, is a bijective function. 
    \end{proof}
      \item Let $f:A\to B$ and $g:B\to C$ be two functions. If $g$ is onto, then $g\circ f: A\to C$ is onto.
    \begin{solution}
      This is false. Let $A=\left\{ 1\right\}$, $B=\left\{ a,b,c \right\}$ and $C=\left\{ 1,2,3 \right\}$. Also, let $f=\left\{ (1,a) \right\}$ and $g=\left\{ (a,1),(b,2),(c,3) \right\}$. Hence, $g$ is onto and $g\circ f= \left\{ (1,1) \right\}$ is not onto. 
    \end{solution}
      \item Let $f:A\to B$ and $g:B\to C$ be two functions. If $g$ is one-to-one, then $g\circ f: A\to C$ is one-to-one.
    \begin{solution}
      This is false. Let $A=\left\{ 1,2 \right\}$, $B=\left\{ a \right\}$ and $C=\left\{ 1 \right\}$. Also, let $f=\left\{ (1,a), (2,a)\right\}$ and $g=\left\{ (a,1) \right\}$. Then, $g\circ f=\left\{ (1,1),(2,1) \right\}$ is not one-to-one.
    \end{solution}
      \item There exist functions $f:A\to B$ and $g:B\to C$ such that $f$ is not onto and $g\circ f: A\to C$ is onto.
    \begin{solution}
      Such functions exist. Let $A=\left\{1\right\}$, $B=\left\{a,b,c\right\}$ and $C=\left\{10 \right\}$. Also, let $f=\left\{ (1,a) \right\}$ and $g=\left\{ (a,10),(b,10),(c,10) \right\}$. Then, $g\circ f = \left\{ (1,10) \right\}$ is onto and $f$ is not onto.
    \end{solution}
      \item There exist functions $f:A\to B$ and $g:B\to C$ such that $f$ is not one-to-one and $g\circ f: A\to C$ is one-to-one.
    \begin{proof}
      We show that this is false. Namely, we prove that if $f$ is not one-to-one, then $g\circ f$ is not one-to-one. Since $f$ is not one-to-one, there are at least two distinct $a,b\in A$ such that $f(a)=f(b)$. Since $g$ is a function and $f(a)=f(b)\in B$, it follows that  $g(f(a)) = g(f(b))$. Therefore, there two distinct $a,b\in A$ such that $(g\circ f)(a) = (g\circ f)(b)$ and so $g\circ f$ is not one-to-one.
    \end{proof}
    \end{enumerate}
    \end{problem}
    \begin{problem}{43}
      For nonempty sets $A, B$ and $C$, let $f:A\to B$ and $g:B\to C$ be functions.
    \begin{enumerate}
      \item Prove:
    \begin{center}
      If $g\circ f$ is one-to-one, then $f$ is one-to-one. 
    \end{center}
    using as many of the following proof techniques as possible: direct proof, proof by contrapositive, proof by contradiction.
    \begin{solution}
      \begin{enumerate}[label=(\roman*)]
      \item Direct Proof
    \begin{proof}
      Assume that $g\circ f: A\to C$ is one-to-one. Consider some $f(a)=f(b)\in B$ for some $a,b\in A$. Then, $g(f(a)) = (g\circ f)(a) = g(f(b)) = (g\circ f)(b)$. Since $g\circ f$ is one-to-one, it follows that $a=b$. Hence, $f$ is one-to-one.  
    \end{proof}
      \item Proof by Contrapositive.
    \begin{proof}
      We show that if $f$ is not one-to-one, then $g\circ f$ is not one-to-one. Since $f$ is not one-to-one, it follows that there are two distinct $a,b\in A$ such that $f(a) = f(b)$. Then, $g(f(a)) = (g\circ f)(a) = g(f(b)) = (g\circ f)(b)$ and $a\neq b$. The function $g\circ f$ is not one-to-one.
    \end{proof}
      \item Proof by Contradiction.
    \begin{proof}
      Suppose that there are functions $f:A\to B$ and $g:B\to C$ such that $f$ is not one-to-one and $g\circ f$ is one-to-one. We can use the argument made in the \textbf{Proof by Contrapositive} to arrive at the conclusion that there are two distinct $a,b\in A$ such that $(g\circ f)(a) = (g\circ f)(b)$, which contradicts our assumption that $g\circ f$ is one-to-one.
    \end{proof}
    \end{enumerate}
    \end{solution}
  \item Disprove: If $g\circ f$ is one-to-one, then $g$ is one-to-one.
    \begin{solution}
      We disprove this statement by giving functions $f:A\to B$ and $g:B\to C$ such that $g$ is not one-to-one and $g\circ f: A\to C$ is one-to-one. Let $A=\left\{ 1 \right\}, B=\left\{ a,b \right\}$ and $C=\left\{ 10 \right\}$, and define $f:A\to B$ and $g:B\to C$ by $f=\left\{ (1,a) \right\}$ and $g=\left\{ (a,10),(b,10) \right\}$. Then, $g\circ f = \left\{ (1,10) \right\}$ is one-to-one and $g$ is not.
    \end{solution}
    \end{enumerate}
    \end{problem}
    \begin{problem}{47}
      For functions $f,g$ and $h$ with domain and codomain $\R$, prove or disprove the following.
    \begin{enumerate}
      \item $(g+h)\circ f = (g\circ f)+(h\circ f)$
    \begin{proof}
      Note that both functions have the same domain, namely, $\R$. We show that $((g+h)\circ f)(x) = ((g\circ f)+(h\circ f))(x)$ for every $x\in \R$. Observe that
    \begin{align*}
      \left( \left( g+h \right)\circ f \right)(x) &= (g+h)(f(x))\\
      &= g(f(x)) + h(f(x)) = (g\circ f)(x) + (h\circ f)(x)
    \end{align*}
    for any $x\in \R$. Both functions have the same domain and equal definition. 
    \end{proof}
      \item $f\circ (g+h) = (f\circ g) + (f\circ h)$.
    \begin{solution}
      This is not true for every functions $f,g,h$. Note that $f(g(x)+h(x)) = f(g(x)) +f(h(x))$ is not true for all cases, for instance 
    \begin{equation*}
      \Abs{g(x) + h(x)} = \Abs{g(x)} + \Abs{h(x)}
    \end{equation*}
    is not true for all real numbers $g(x),h(x) \in \R$. Hence, we can define functions $g:\R \to \R$, $h:\R\to \R$ and $f:\R\to \R$ by $g(x) = 3$, $h(x) = -2$ and $f(x) = \Abs{x}$ for all real numbers $x$ to create a counterexample.
    \end{solution}
    \end{enumerate}
    \end{problem}
    \begin{problem}{48}
      The composition $g\circ f: (0,1) \to \R$ of two functions $f$ and $g$ is given by $(g\circ f)(x) = \frac{4x-1}{2\sqrt{x-x^{2}}}$, where $f:(0,1)\to (-1,1)$ is defined by $f(x) = 2x-1$ for $x\in (0,1)$. Determine the function $g$.
    \begin{solution}
      Let $g: (-1,1) \to \R$ be defined by
    \begin{equation*}
      g(n) = \frac{2n+1}{\sqrt{1-n^{2}}}
    \end{equation*}
    for every $n\in (-1,1)$. Note that for all $n\in (-1,1)$, $0<1-n^{2}$ and so its denominator $\sqrt{1-n^{2}}$ is defined in the real numbers and is nonzero. Then, the function $g\circ f: (0,1) \to \R$ is defined by 
    \begin{align*}
      (g\circ f)(x) &= \frac{2(2x-1)+1}{\sqrt{1-(2x-1)^{2}}}\\
	&= \frac{4x-1}{\sqrt{1-4x^{2}+4x-1}}\\
	  &= \frac{4x-1}{\sqrt{4(x-x^{2})}} = \frac{4x-1}{2\sqrt{x-x^{2}}}
    \end{align*}
    for all real numbers $x\in (0,1)$.
    \end{solution}
    \end{problem}
    \end{document}


