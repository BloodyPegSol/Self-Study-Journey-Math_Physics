\documentclass[12pt]{article}
\usepackage[margin=1in]{geometry}
\usepackage{amsmath, amsfonts,amsthm,amssymb,epigraph,etoolbox,mathtools,setspace,enumitem}  
\usepackage{tikz}
\usetikzlibrary{datavisualization} 
\usepackage[makeroom]{cancel} 
\usepackage[linguistics]{forest}
\usetikzlibrary{patterns}
\newcommand{\N}{\mathbb{N}}
\newcommand{\Z}{\mathbb{Z}}
\newcommand{\R}{\mathbb{R}}
\newcommand{\Q}{\mathbb{Q}}
\newcommand{\Mod}[1]{\ (\mathrm{mod}\ #1)}
\newcommand{\Lim}[1]{\mathrm{lim}(#1)}
\newcommand{\Abs}[1]{\left\vert #1 \right\vert}
\newcommand{\Dom}[1]{\mathrm{dom}(#1)}
\newcommand{\Range}[1]{\mathrm{range}(#1)}

\newlist{legal}{enumerate}{10}
\setlist[legal]{label=(\alph*)}
\setenumerate[legal]{label=(\alph*)}

\DeclarePairedDelimiter\bra{\langle}{\rvert}
\DeclarePairedDelimiter\ket{\lvert}{\rangle}
\DeclarePairedDelimiterX\braket[2]{\langle}{\rangle}{#1\delimsize\vert #2}


\newenvironment{theorem}[2][Theorem]{\begin{trivlist} \item[\hskip \labelsep {\bfseries #1}\hskip \labelsep {\bfseries #2.}]}{\end{trivlist}}
\newenvironment{lemma}[2][Lemma]{\begin{trivlist} \item[\hskip \labelsep {\bfseries #1}\hskip \labelsep {\bfseries #2.}]}{\end{trivlist}}
\newenvironment{result}[2][Result]{\begin{trivlist} \item[\hskip \labelsep {\bfseries #1}\hskip \labelsep {\bfseries #2.}]}{\end{trivlist}}
\newenvironment{exercise}[2][Exercise]{\begin{trivlist} \item[\hskip \labelsep {\bfseries #1}\hskip \labelsep {\bfseries #2.}]}{\end{trivlist}}
\newenvironment{problem}[2][Problem]{\begin{trivlist} \item[\hskip \labelsep {\bfseries #1}\hskip \labelsep {\bfseries #2.}]}{\end{trivlist}}
\newenvironment{question}[2][Question]{\begin{trivlist} \item[\hskip \labelsep {\bfseries #1}\hskip \labelsep {\bfseries #2.}]}{\end{trivlist}}
\newenvironment{corollary}[2][Corollary]{\begin{trivlist} \item[\hskip \labelsep {\bfseries #1}\hskip \labelsep {\bfseries #2.}]}{\end{trivlist}}
\newenvironment{solution}[1][Solution]{\begin{trivlist} \item[\hskip \labelsep {\bfseries #1}]}{\end{trivlist}}

\setlength\epigraphwidth{8cm}
\setlength\epigraphrule{0pt}

\makeatletter
\patchcmd{\epigraph}{\@epitext{#1}}{\itshape\@epitext{#1}}{}{}
\makeatother

\begin{document}
  
 \title{Section 9.5: Composition of Functions}
   \author{Juan Patricio Carrizales Torres}
     \date{Aug 1, 2022}
       \maketitle
   
       We have previously defined operations on sets such as the integers modulo $n$. Some sets of functions are no exception. Let $A,B',B$ and $C$ be nonempty sets and consider the functions $f:A\to B'$ and $g:B\to C$. If $B' \subseteq B$, namely, if $\Range{f} \subseteq \Dom{g}$, then it is possible to create a new function from $A$ to $C$ called the composition of $f$ and $g$. This composition $g\circ f$ is defined by
    \begin{equation*}
   ( g\circ f)(x) = g(f(x)) \text{ for all } x\in A.
    \end{equation*}
    Furthermore, it has some useful properties. Consider two functions $f$ and $g$ such that their composition $g\circ f$ is defined, then 
    \begin{enumerate}
      \item If both $g$ and $f$ are injective (surjective), then the composition $g\circ f$ is injective (surjective).
    \end{enumerate}
    Clearly, one can further conclude that if $g$ and $f$ are bijective, then their composition $g\circ f$ is bijective. Keep in mind that in the beginning of the paragraph we assumed that their composition $g\circ f$ is defined. However, this is not a sufficient condition for $f\circ g$ to be defined. This depends on whether $\Range{g} \subseteq \Dom{f}$ is true or not. \\

    Also, for nonempty functions $f,g,h$, if the compositions $g\circ f$ and $h\circ g$ are defined, then $h\circ (g \circ f)$ and $(h\circ g)\circ f$ are defined. Furthermore, $h\circ(g\circ f) = (h\circ g)\circ f$ and so the composition of $f,g,h$ is \textbf{associative}. \\
    Lastly, let's prove the following theorem.
    \begin{theorem}{9.5.1}
      Let $g$ and $f$ be nonempty functions. If $\Range{f}\subseteq \Dom{g}$ then  $g\circ f$ is a function.
    \begin{proof}
      Assume that $\Range{f}\subseteq \Dom{g}$. Consider some $(x,y)\in f$. Then, $(y,z)\in g$ and so $(x,z)\in g\circ f$. Hence, for any $x\in \Dom{f}=\Dom{g\circ f}$, there is an image $g(f(x)) = (g\circ f)(x)$ defined. We now prove that $g\circ f$ is well-defined. Consider two $a,b\in \Dom{g\circ f} = \Dom{f}$ such that $a=b$. Then, $f(a)=f(b)\in \Dom{g}$ and so $g(f(a))=g(f(b))$. Hence, $(g\circ f)(a)=(g\circ f)(b)$.
    \end{proof}
    \end{theorem}
    \begin{problem}{38}
      Two functions $f:\R \to \R$ and $g:\R \to \R$ are defined by $f(x) = 3x^{2}+1$ and $g(x) = 5x-3$ for all $x\in \R$. Determine $(g\circ f)(1)$ and $(f\circ g)(1)$.
    \begin{solution}
      The composition functions $g\circ f:\R \to \R$ and $f\circ g: \R \to \R$ are defined by $g(f(x)) = 5\left( 3x^{2}+1 \right)-3 = 15x^{2}+2$ and $f(g(x)) = 3\left( 5x-3\right)^{2}+1 = 75x^{2}-90x+28$ for all $x\in \R$.\\
      Hence, $(g\circ f)(1) =17$ and $(f\circ g)(1) = 13$.
    \end{solution}
    \end{problem}
    \begin{problem}{39}
      Two functions $f:\Z_{10} \to \Z_{10}$ and $g:\Z_{10} \to \Z_{10}$ are defined by $f([a])=[3a]$ and $g([a])=[7a]$.
    \begin{enumerate}
      \item Determine $g\circ f$ and $f\circ g$.
    \begin{solution}
    The composition functions $g\circ f:\Z_{10}\to \Z_{10}$ and $f\circ g: \Z_{10} \to \Z_{10}$ are defined by $g(f([x])) = [21a]=[21][a]=[1][a]=[a]$ and $f(g([x])) = [21a]=[a]$ for every $[a]\in \Z_{10}$. Therefore, $g\circ f=f\circ g$.
    \end{solution}
      \item What can be concluded as a result of (a)?
    \begin{solution}
      Both $g\circ f$ and $f\circ g$ are identity functions on $\Z_{10}$.
    \end{solution}
    \end{enumerate}
    \end{problem}
    \begin{problem}{40}
      Let $A$ and $B$ be nonempty sets. Prove that if $f:A\to B$, then $f\circ i_{A} = f$ and $i_{B} \circ f = f$.
    \begin{proof}
      Note that $\Range{i_{A}} = A = \Dom{f}$ and $\Range{f} \subseteq \Dom{i_{B}} = B$. Hence, both functions $f\circ i_{A}:A\to B$ and $i_{B}\circ f:A\to B$ are defined by $(f\circ i_{A})(x) = f(i_{A}(x))=f(x)$ and $(i_{B}\circ f)(x) = i_{B}(f(x))= f(x)$ for every $x\in A$. Both have the same Dominion and rule as $f$. Hence, $f\circ i_{A} = i_{B} \circ f = f$.
    \end{proof}
    \end{problem}
       \end{document}


