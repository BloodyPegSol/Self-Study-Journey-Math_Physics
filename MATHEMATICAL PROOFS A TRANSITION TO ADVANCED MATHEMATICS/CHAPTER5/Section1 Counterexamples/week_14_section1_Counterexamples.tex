\documentclass[12pt]{article}
\usepackage[margin=1in]{geometry}
\usepackage{amsmath,amsthm,amssymb,epigraph,etoolbox,mathtools,setspace,enumitem}  
\usepackage{tikz}
\usetikzlibrary{datavisualization}
\usepackage[makeroom]{cancel} 
\usepackage[linguistics]{forest}
\usetikzlibrary{patterns}
\newcommand{\N}{\mathbb{N}}
\newcommand{\Z}{\mathbb{Z}}
\newcommand{\R}{\mathbb{R}}
\newcommand{\Q}{\mathbb{Q}}
\newcommand{\Mod}[1]{\ (\mathrm{mod}\ #1)}

\newenvironment{theorem}[2][Theorem]{\begin{trivlist}
		\item[\hskip \labelsep {\bfseries #1}\hskip \labelsep {\bfseries #2.}]}{\end{trivlist}}
\newenvironment{lemma}[2][Lemma]{\begin{trivlist}
		\item[\hskip \labelsep {\bfseries #1}\hskip \labelsep {\bfseries #2.}]}{\end{trivlist}}
\newenvironment{exercise}[2][Exercise]{\begin{trivlist}
		\item[\hskip \labelsep {\bfseries #1}\hskip \labelsep {\bfseries #2.}]}{\end{trivlist}}
\newenvironment{problem}[2][Problem]{\begin{trivlist}
		\item[\hskip \labelsep {\bfseries #1}\hskip \labelsep {\bfseries #2.}]}{\end{trivlist}}
\newenvironment{question}[2][Question]{\begin{trivlist}
		\item[\hskip \labelsep {\bfseries #1}\hskip \labelsep {\bfseries #2.}]}{\end{trivlist}}
\newenvironment{corollary}[2][Corollary]{\begin{trivlist}
		\item[\hskip \labelsep {\bfseries #1}\hskip \labelsep {\bfseries #2.}]}{\end{trivlist}}
\newenvironment{solution}[2][Solution]{\begin{trivlist}
		\item[\hskip \labelsep {\bfseries #1}\hskip \labelsep {\bfseries #2.}]}{\end{trivlist}}

\setlength\epigraphwidth{8cm}
\setlength\epigraphrule{0pt}

\makeatletter
\patchcmd{\epigraph}{\@epitext{#1}}{\itshape\@epitext{#1}}{}{}
\makeatother


\begin{document}
	
	\title{Week 13}
	\author{Juan Patricio Carrizales Torres \\
		Section 1: Counterexamples}
	\date{October 19, 2021}
	\maketitle

	A quantified statement of the type $\forall x \in S, R(x)$ can be \textbf{disproved} (proved to be false) by showing that $\sim (\forall x \in S, R(x)) \equiv \exists x \in S, \sim R(x)$ is true. If $\forall x \in S, R(x)$ is false, then there is some $x\in S$ for which the open sentence $R(x)$ is false, namely a \textbf{counterexample}. Therefore, the truth value of $\forall x \in S, R(x)$ not only depends on the open sentence $R(x)$ but also on the domain $S$.
	
	\begin{problem}{1}
		Disprove the statement: If $a$ and $b$ are any two real numbers, then $\log(ab) = \log(a) + \log(b)$.
		\begin{solution}{}
			Let $a\leq 0$ and $b>0$. Then $\log(ab)$ and $\log(a)$ are not defined (The domain over the open sentence influences on the truth value of the quantified statement).
		\end{solution}
	\end{problem}

	\begin{problem}{2}
		Disprove the statement: If $n\in \{0,1,2,3,4\}$, then $2^{n}+3^{n}+n(n-1)(n-2)$ is prime.
		\begin{solution}{}
			If $n=4$, then $2^{n}+3^{n}+n(n-1)(n-2) = 121$ is not a prime number. Therefore, $n=4$ is a counterexample
		\end{solution}
	\end{problem}

	\begin{problem}{3}
		Disprove the statement: If $n\in \{1,2,3,4,5\}$, then $3\mid (2n^{2}+1)$.
		\begin{solution}{}
			Since $3\nmid (2(3)^{2}+1)$, it follows that $n=3$ is a counterexample.
		\end{solution}
	\end{problem}

	\begin{problem}{4}
		Disprove the statement: Let $n\in \N$. If $\frac{n(n+1)}{2}$ is odd, then $\frac{(n+1)(n+2)}{2}$ is odd.
		\begin{solution}{}
			Let $n=2(2k+1)$ where $k\in \N$. Then 
			\begin{align*}
				\frac{n(n+1)}{2} &= \frac{2(2k+1)(2(2k+1)+1)}{2}\\
				&= (2k+1)(4k+3) = 8k^{2} + 10k +3 = 2(4k^{2}+5k+1)+1
			\end{align*}
		Since $4k^{2}+5k+1\in \N$, it follows that $\frac{n(n+1)}{2}$ is odd for this values of $n$. Then,
		\begin{align*}
			\frac{(n+1)(n+2)}{2} &= \frac{(2(2k+1)+1)(2(2k+1)+2)}{2} = \frac{2(2k+2)(2(2k+1)+1)}{2}\\
			&= (2k+2)(2(2k+1)+1) = 2(k+1)(2(2k+1)+1)
		\end{align*}
		The positive integer $2(k+1)(2(2k+1)+1)$ is even. Thus, all $n=2(2k+1)$ where $k\in \N$ are counterexamples.
		\end{solution}
	\end{problem}

	\begin{problem}{5}
		Disprove the statement: For every two positive integers $a$ and $b$, $(a+b)^{3}=a^{3}+2a^{2}b+2ab+2ab^{2}+b^{3}$.
		\begin{solution}{}
			Let $a,b\in \Z$ such that $a>0$ and $b>0$. Note that 
			\begin{align*}
				(a+b)^{3} &= (a^{2}+2ab+b^{2})(a+b)\\
				&= a^{3} +2a^{2}b+ab^{2}+a^{2}b+2ab^{2}+b^{3}\\
				&= a^{3} + 3a^{2}b + 3ab^{2} + b^{3}
			\end{align*}
		Then, let's check for which values of $a$ and $b$, $a^{3}+2a^{2}b+2ab+2ab^{2}+b^{3} = a^{3} + 3a^{2}b + 3ab^{2} + b^{3}$ holds.
		\begin{align*}
			a^{3}+2a^{2}b+2ab+2ab^{2}+b^{3} &= a^{3} + 3a^{2}b + 3ab^{2} + b^{3}\\
			2a^{2}b+2ab+2ab^{2} &= 3a^{2}b + 3ab^{2}\\
			2ab &= a^{2}b + ab^{2}\\
			ab(2) &= ab(a+b)
		\end{align*}
	Since $a>0$ and $b>0$, $ab > 0$ and so we can divide both sides by $ab$. Then, $2 = a+b$. Therefore, all those positive integers $a$ and $b$ such that $a+b\neq 2$ are counterexamples, namely, $a\neq 1$ or $b\neq 1$.
		\end{solution}
	\end{problem}

	\begin{problem}{6}
		Let $a,b \in \Z$. Disprove the statement: If $ab$ and $(a+b)^{2}$ are of opposite parity, then $a^{2}b^{2}$ and $a+ab+b$ are of opposite parity.
		\begin{solution}{}
			Let $a$ and $b$ be odd integers. Then $ab$ is odd (multiplication of two odd integers) and $a+b$ is even (sum of two odd integers); so $(a+b)^{2}$ is even. The hypothesis is true. Note that $(ab)^{2} = a^{2}b^{2}$ is odd (multiplication of two odd integers) and $(a+b) + ab$ is odd (sum of an even and odd integer). They are of the same parity. Therefore, all integers $a$ and $b$ such that both are odd will be counterexamples.
			
		\end{solution}
	\end{problem}
	
	\begin{problem}{7}
		For positive real numbers $a$ and $b$, it can be shown that $(a+b)\left(\frac{1}{a} + \frac{1}{b}\right)\geq 4$. If $a=b$, then this inequality is an equality. Consider the following statement:\\
		If $a$ and $b$ are positive real numbers such that $(a+b)\left(\frac{1}{a}+\frac{1}{b}\right) = 4$, then $a=b$. Is there a counterexample to this statement?
		\begin{solution}{}
			If we can show that the previous result is true, then there will be no counterexample. Let $a,b\in \R$ such that $a>0$, $b>0$ and $(a+b)\left(\frac{1}{a}+\frac{1}{b}\right) = 4$. Note that,
			\begin{align*}
				(a+b)\left(\frac{1}{a}+\frac{1}{b}\right) &= 4\\
				1+\frac{b}{a}+\frac{a}{b}+1 &= 4\\
				\frac{b}{a}+\frac{a}{b} &= 2\\
				a^{2}+b^{2} &= 2ab\\
				a^{2}-2ab+b^{2} &= 0\\
				(a-b)(a-b) &= 0
			\end{align*}
		Since $(a-b)(a-b) = 0$ and $(a-b)=(a-b)$, it follows by \textit{Theorem 4.13} that $a-b = 0$; so $a=b$. Therefore, the result is proven to be true and so there are no counterexamples. 
		\end{solution} 
	\end{problem}

	\begin{problem}{8}
		In Exercise 7, it is stated that $(a+b)(\frac{1}{a}+\frac{1}{b})\geq 4$ for every two positive real numbers $a$ and $b$. Does it therefore follows that $(c^{2}+d^{2})\left(\frac{1}{c^{2}}+\frac{1}{d^{2}}\right)\geq 4^{2}$ for every two positive real numbers $c$ and $d$? 
		\begin{solution}{}
			Since $c,d\in \R^{+}$, it follows that $c^{2},d^{2}\in \R^{+}$ and so it is true that $(c^{2}+d^{2})\left(\frac{1}{c^{2}}+\frac{1}{d^{2}}\right)\geq 4$. However let's check whether $(c^{2}+d^{2})\left(\frac{1}{c^{2}}+\frac{1}{d^{2}}\right)\geq 4^{2}$ holds. Note that,
			\begin{align*}
				(c^{2}+d^{2})\left(\frac{1}{c^{2}}+\frac{1}{d^{2}}\right) &\geq 4^{2}\\
				1+\frac{d^{2}}{c^{2}}+\frac{c^{2}}{d^{2}}+1 & \geq 16\\
				\frac{d^{2}}{c^{2}}+\frac{c^{2}}{d^{2}} & \geq 14\\
				d^{4} + c^{4} &\geq 14c^{2}d^{2}\\
				d^{4} -14c^{2}d^{2}+ c^{4} &\geq 0\\
				(d^{2}-c^{2})^{2}-12c^{2}d^{2} &\geq 0
			\end{align*}
		Thus all $c,d\in \R^{+}$ such that $(d^{2}-c^{2})^{2} < 12c^{2}d^{2}$ will be counterexamples. If $c=d$, then $c^{2} = d^{2}$; so $(d^{2}-c^{2})^{2} = 0$ and $12c^{2}d^{2}>0$. Then $(d^{2}-c^{2})^{2} < 12c^{2}d^{2}$. Therefore, all $c,d\in \R^{+}$ such that $c=d$ will be counterexamples. Something that we already knew from the previous problem, namely, $(a+b)\left(\frac{1}{a}+\frac{1}{b}\right) = 4$ if and only if $a=b$. 
		\end{solution}
	\end{problem}   

	\begin{problem}{9}
		Disprove the statement: For every positive integer $x$ and every integer $n\geq 2$, the equation $x^{n}+(x+1)^{n}=(x+2)^{n}$ has no solution.
		\begin{solution}{}
			A very famous counterexample is $3^{2}+4^{2} = 5^{2}$ ($x=3$ and $n=2$). By Fermat's Last Theorem, all posible counterexamples will only be found in the cases where $n=2$.
		\end{solution}
	\end{problem}
\end{document}