\documentclass[12pt]{article}
\usepackage[margin=1in]{geometry}
\usepackage{amsmath,amsthm,amssymb,epigraph,etoolbox,mathtools,setspace,enumitem}  
\usepackage{tikz}
\usetikzlibrary{datavisualization}
\usepackage[makeroom]{cancel} 
\usepackage[linguistics]{forest}
\usetikzlibrary{patterns}
\newcommand{\N}{\mathbb{N}}
\newcommand{\Z}{\mathbb{Z}}
\newcommand{\R}{\mathbb{R}}
\newcommand{\Q}{\mathbb{Q}}
\newcommand{\Mod}[1]{\ (\mathrm{mod}\ #1)}

\newenvironment{theorem}[2][Theorem]{\begin{trivlist}
		\item[\hskip \labelsep {\bfseries #1}\hskip \labelsep {\bfseries #2.}]}{\end{trivlist}}
\newenvironment{lemma}[2][Lemma]{\begin{trivlist}
		\item[\hskip \labelsep {\bfseries #1}\hskip \labelsep {\bfseries #2.}]}{\end{trivlist}}
\newenvironment{exercise}[2][Exercise]{\begin{trivlist}
		\item[\hskip \labelsep {\bfseries #1}\hskip \labelsep {\bfseries #2.}]}{\end{trivlist}}
\newenvironment{problem}[2][Problem]{\begin{trivlist}
		\item[\hskip \labelsep {\bfseries #1}\hskip \labelsep {\bfseries #2.}]}{\end{trivlist}}
\newenvironment{question}[2][Question]{\begin{trivlist}
		\item[\hskip \labelsep {\bfseries #1}\hskip \labelsep {\bfseries #2.}]}{\end{trivlist}}
\newenvironment{corollary}[2][Corollary]{\begin{trivlist}
		\item[\hskip \labelsep {\bfseries #1}\hskip \labelsep {\bfseries #2.}]}{\end{trivlist}}
\newenvironment{solution}[2][Solution]{\begin{trivlist}
		\item[\hskip \labelsep {\bfseries #1}\hskip \labelsep {\bfseries #2.}]}{\end{trivlist}}

\setlength\epigraphwidth{8cm}
\setlength\epigraphrule{0pt}

\makeatletter
\patchcmd{\epigraph}{\@epitext{#1}}{\itshape\@epitext{#1}}{}{}
\makeatother


\begin{document}
	
	\title{Week 13}
	\author{Juan Patricio Carrizales Torres \\
		Section 2: Proof by contradiction}
	\date{October 23, 2021}
	\maketitle

	Curiously, the open sentence $P(x)$ can be proven to be logically equivalent to $\sim P(x) \Rightarrow \bot$ by using some fundamental logical equivalences in propositional logic, namely, $P(x) \equiv P(x)\vee \bot \equiv \sim (\sim P(x))\vee \bot \equiv \sim P(x) \Rightarrow \bot$. Thus, if we prove, say by direct proof, $\sim P(x) \Rightarrow \bot$, then $P(x)$ is proven to be true. A common example of a contradiction is $C \wedge \sim C$ since a statement $C$ can only have one binary truth value. Let's give an example to better illustrate this type of proof.\\
	
	Let $R: \forall x\in S, P(x) \Rightarrow Q(x)$ be a quantified statement. Suppose we want to prove this by contradiciton and so we assume that $\sim R$ is true, namely, $\sim (\forall x \in S, P(x) \Rightarrow Q(x)) \equiv \exists x \in S, P(x)\wedge \sim Q(x)$. Then, we need to make some assumption or use some known fact $C$ to continue with our proof. However, we end up with the conclusion that $\sim C$ is true. Therefore, it is shown, by direct proof, that if $\sim R$ is true, then the contradiction $C\wedge \sim C$ must be true; so $R$ is proven to be true.
	
	\begin{problem}{10}
		Prove that there is no largest negative rational number.
		\begin{proof}
			Assume, to the contrary, that there is some $r\in \Q^{-}$ such that $r$ is the largest negative rational number, namely, for every $n\in \Q^{-}$, $n<r$. Since $r$ is a negative rational number, it follows that $r/2\in \Q^{-}$. Since $r<r/2<0$, we arrive at a contradiction.
		\end{proof}
	\end{problem}

	\begin{problem}{11}
		Prove that there is no smallest positive irrational number.
		\begin{proof}
			Assume, to the contrary, that there is a smallest positive irrational number $r$. Since $r$ is positive and irrational, it follows that $r/2$ is postitve and irrational. Because $0<r/2<r$, this leads to a contradiction.
		\end{proof}
	\end{problem}

	\begin{problem}{12}
		Prove that 200 cannot be written as a sum of an odd integer and two even integers.
		\begin{proof}
			Suppose, to the contrary, that 200 can be written as a sum of an odd integer $a$ and two even integers $b$ and $d$. Then, $a = 2m+1$, $b = 2n$ and $d=2l$ for some $m,n,l\in \Z$. Therefore, $a+b+d=2m+1+2n+2l = 2(m+n+l)+1$. Since $m+n+l \in \Z$, it follows that $a+b+d = 200$ is odd, which contradicts the fact that 200 is even.
		\end{proof}
	\end{problem}

	\begin{problem}{13}
		Use proof by contradiction to prove that if $a$ and $b$ are odd integers, then $4\nmid (a^{2}+b^{2})$.
		\begin{proof}
			Assume, to the contrary, that $a$ and $b$ are odd integers such that $4\mid (a^{2}+b^{2})$. Then, $a^{2}+b^{2}=4c$ for some $c\in \Z$. Since $a$ and $b$ are odd, it follows that $a^{2}$ and $b^{2}$ are odd. Thus, $a^{2} = 2m+1$ and $b^{2} = 2n+1$ for some integers $n$ and $m$. Therefore, $4c = 2(2c) = 2m+1+2n+1=2(m+n)+1$. Since $2c$ and $m+n$ are integers, we arrive at the contradiction that an even number is equal to an odd number.
		\end{proof}
	\end{problem} 

	\begin{problem}{14}
		Prove that if $a\geq 2$ and $b$ are integers, then $a\nmid b$ or $a\nmid (b+1)$.
		\begin{proof}
			Let $a$ and $b$ be integers such that $a\geq 2$ and assume, to the contrary, that $a\mid b$ and $a\mid (b+1)$. Then, $b = ac$ and $b+1=ad$ for some integers $c$ and $d$. Therefore, $b = ad-1=ac$ and so $ad-ac = a(d-c) =1$. Since $d-c\in \Z$, it follows that $a\mid 1$, which is a contradiction since $a\geq 2$.
		\end{proof}
	\end{problem}

	\begin{problem}{15}
		Prove that 1000 cannot be written as the sum of three integers, an even number of which are even.
		\begin{proof}
			Assume, to the contrary, that 1000 can be written as the sum of three integers $a$, $b$ and $c$, an even number of which are even. Then we consider two cases.\\
			\textit{Case 1.} None of $a$, $b$ and $c$ are even (zero of them are even). Then, $a = 2m+1$, $b=2n+1$ and $c= 2l+1$ where $m,n,l\in \Z$. Therefore, $a+b+c = 2m+1+2n+1+2l+1= 2(m+n+l)+3=2(m+n+l+1)+1=1000$. Since $m+n+l+1\in \Z$, the integer $a+b+c = 1000$ is odd, which contradicts the fact that $1000$ is even.\\ 
			\textit{Case 2.} 2 of the integers $a$, $b$ and $c$ are even. Without loss of generality, let  $a=2m$, $b=2n$ and $c=2l+1$ for integers $m$, $n$ and $l$. Therefore, $a+b+c = 2m+2n+2l+1 = 2(m+n+l)+1=1000$. Because $m+n+l \in \Z$, the integer $a+b+c=1000$ is odd, which contradicts the fact that $1000$ is even.
		\end{proof}
	\end{problem}

	\begin{problem}{16}
		Prove that the product of an irrational number and a nonzero rational number is irrational.
		\begin{proof}
			Assume, to the contrary, that there is an irrational number $r$ and a nonzero rational number $s$ such that $r\cdot s$ is rational. Then, $s=a/b$ where $a,b\in \Z$ such that $a\neq 0$ and $b\neq 0$. Thus, $r\cdot s = r\cdot (a/b) = c/d$ where $c,d\in \Z$ such that $d\neq 0$ and $c\neq 0$ (none of the factors is zero(rational number)). Since $a\neq 0$, we can multiply both sides by $b/a$. Thus $r = (cb)/(ad)$. Since $c\in \Z$ and $b\in \Z$, it follows that $cb\in \Z$. Because $a,d\in \Z$ and they are nonzero, it follows that $ad\in \Z$ and $ad\neq 0$, and so $r = (cb)/(cd)$ is a rational number, which contradicts our assumption that $r$ was irrational.
		\end{proof}
	\end{problem}

	\begin{problem}{17}
		Prove that when an irrational number is divided by a (nonzero) rational number, the resulting number is irrational.
		\begin{proof}
			Assume, to the contrary, that there are an irrational number $r$ and nonzero rational number $s$ such that $r/s$ is rational. Then, $s = a/b$ where $a,b\in \Z$ and $a,b\neq 0$. Therefore, $r/s = r(b/a) = c/d$ where $c,d\in \Z$ and $c,d\neq 0$. Thus, $r = (ca)/(bd)$. Since $ca,bd\in \Z$ and $bd\neq 0$, it follows that $r=(ca)/(bd)$ is a rational number, which is a contradiciton. 
		\end{proof}
	\end{problem}

	\begin{problem}{18}
		Let $a$ be an irrational number and $r$ a nonzero rational number. Prove that if $s$ is a real number, then either $ar+s$ or $ar-s$ is irrational.
		\begin{proof}
			Assume, to the contrary, that there are $a,s,r\in \R$ such that $a$ is irrational, $r$ is a nonzero rational number and both $ar+s$ and $ar-s$ are rational. Then, by the result proven in \textit{Problem 16}, the number $ar$ is an irrational number $q$. Therefore, $q+s = m/n$ and $q-s = k/l$ where $m,n,k,l \in \Z$ and $n,l\neq 0$. Thus, $q = m/n - s = k/l + s$. Note that,
			\begin{align*}
				\frac{m}{n}-\frac{k}{l} &= 2s\\
				\frac{ml-kn}{2ln} &= s 
			\end{align*}
			Since $(ml-kn),2ln\in \Z$ and $2ln \neq 0$, it follows that $s$ must be a rational number. However, this contradicts the proven \textit{Result 15}, which states that the sum of an irrational and rational number, both $q+s$ and $q+(-s)$, is irrational. 
		\end{proof}
	\end{problem}
 
	\begin{problem}{19}
		Prove that $\sqrt{3}$ is irrational. [Hint: First prove for an integer $a$ that $3\mid a^{2}$ if and only if $3\mid a$. Recall that every integer can be written as $3q$, $3q+1$ or $3q+2$ for some integer $q$.]
		\begin{lemma}{1}
			Let $a\in\Z$, then $3\mid a^{2}$ if and only if $3\mid a$.
			\begin{proof}
				Assume that $3\mid a$. Then $a = 3b$ for some $b\in \Z$. Therefore, $a^{2} = 9b^{2} = 3(3b^{2})$. Since $3b^{2}\in \Z$, it follows that $3\mid a^{2}$.\\
				For the converse, assume that $3\nmid a$. Then, either $a = 3q+1$ or $a=3q+2$ for some $q\in \Z$. We consider these two cases.\\
				\textit{Case 1.} $a = 3q+1$. Then $a^{2} = 9q^{2} +6q +1 = 3(3q^{2}+2q)+1$. Since $3q^{2}+2q\in \Z$, $3\nmid a^{2}$.\\
				\textit{Case 2.} $a = 3q+2$. Then $a^{2} = 9q^{2} +6q +4 = 3(3q^{2}+2q+1)+1$. Since $3q^{2}+2q+1\in \Z$, it follows that $3\nmid a^{2}$.\\
				Therefore, $3\nmid a^{2}$.
			\end{proof}
		\end{lemma}
		\textbf{Result} $\sqrt{3}$ is irrational
		\begin{proof}
			Assume, to the contrary, that $\sqrt{3}$ is rational. Then $\sqrt{3} = a/b$ where $a,b\in \Z$ and $b\neq 0$. We may further assume that $a/b$ has been reduced to its lowest terms. Therefore, $3 = a^{2} / b^{2}$ and so $a^{2} = 3b^{2}$. Since $b^{2}\in \Z$, it follows that $3\mid a^{2}$ and, by lemma, $3\mid a$; so $a=3c$ where $c\in \Z$. Thus,
			\begin{align*}
				a^{2} = 9c^{2} &= 3b^{2}\\
				3c^{2} &= b^{2}
			\end{align*}
		Since $c^{2}\in \Z$, it follows that $3\mid b^{2}$ and so, by lemma, $3\mid b$; so $b=3d$ where $d\in \Z$. Both $a = 3c$ and $b = 3d$ which contradicts our assumption that they were reduced to their lowest terms.
		\end{proof}
	\end{problem}

	\begin{problem}{20}
		Prove that $\sqrt{2}+\sqrt{3}$ is an irrational number.
		\begin{proof}
			Assume, to the contrary, that $\sqrt{2} + \sqrt{3}$ is a rational number. Then, $\sqrt{2} + \sqrt{3} = b$ where $b \in \Q$. Thus, $\sqrt{2} = b - \sqrt{3}$ and so $2 = (b-\sqrt{3})^{2} = b^{2}-2b\sqrt{3}+3$. Note that,
			\begin{align*}
				2 &= b^{2}-2b\sqrt{3}+3\\
				2b\sqrt{3} &= b^{2} +1\\
				\sqrt{3} &= \frac{b}{2} + \frac{1}{2b}
			\end{align*}
		Therefore, $\sqrt{3} = b/2+1/2b$ is a rational number (sum of two rational numbers). However, this contradicts the fact that $\sqrt{3}$ is irrational.
		\end{proof}  
	\end{problem}

	\begin{problem}{21}
		(a)Prove that $\sqrt{6}$ is an irrational number.
		\begin{proof}
			Note that $3\mid 6$ and $2\mid 6$. Thus, a similar proof to the ones used to prove that $\sqrt{3}$ and $\sqrt{2}$ are irraitonal can be used.\\
			
			Assume, to the contrary, that $\sqrt{6}$ is a rational number. Then, $\sqrt{6} = a/b$ where $a,b\in \Z$ and $b\neq 0$. We further assume that $a/b$ is reduced to the lowest terms. Thus, $6 = a^{2}/b^{2}$ and so $6b^{2} = 2(3b^{2}) = a^{2}$. Since $3b^{2}\in \Z$, it follows that $2\mid a^{2}$ and, by \textit{Theorem 3.12} (For integer $x$, $x^{2}$ is even iff $x$ is even), $2\mid a$. Therefore, $a = 2c$ for some integer $c$. Note that $a^{2} = (2c)^{2} = 2(2c^{2}) = 2(3b^{2})$ and so $2c^{2} = 3b^{2}$. Because $c^{2}\in \Z$, $2\mid 3b^{2}$. Therefore, by Theorem either $2\mid 3$ or $2\mid b^{2}$. Since $2\nmid 3$, it follows that $2\mid b^{2}$ and, by \textit{Theorem 3.12}, $2\mid b$. Thus, $2\mid a$ and $2\mid b$, and so they have a divisor in common, which contradicts the fact the $a/b$ was reduced to the lowest terms.
		\end{proof}
	
		(b) Prove that there are infinitely many positive integers $n$ such that $\sqrt{n}$ is irrational.
		\begin{proof}
			Assume, to the contrary, that there is a finite number of positive integers $n$ such that $\sqrt{n}$ is irrational. Then, there must be some $m\in \Z^{+}$ such that $\sqrt{m}$ is irrational and for any irrational number $\sqrt{n} < \sqrt{m}$, where $n\in \Z^{+}$. Let $c\in \Z^{+}$ such that $c\geq 2$. Then, $\sqrt{m} < c\sqrt{m}$. Since $c$ is a nonzero rational number and $\sqrt{m}$ is irrational, it follows by the result proven in \textit{Problem 16} that $c\sqrt{m}$ is irrational. Because $c\in \Z^{+}$, $c\sqrt{m} = \sqrt{c^{2}m}$. Thus, $c^{2}m\in \Z^{+}$, $\sqrt{c^{2}m}$ is irrational and $\sqrt{m} < \sqrt{c^{2}m}$, which contradicts our initial assumption. 
		\end{proof}
	\end{problem} 

	\begin{problem}{23}
		Prove that there is no integer $a$ such that $a\equiv 5 \Mod{14}$ and $a\equiv 3 \Mod{21}$.
		\begin{proof}
			Assume, to the contrary, that there is an integer $a$ such that $a\equiv 5 \Mod{14}$ and $a\equiv 3 \Mod{21}$. Then, $14\mid (a-5)$ and $21\mid (a-3)$, and so $a = 14m +5$ and $a= 21n +3$ where $m,n\in \Z$. Thus, $14m+5 = 21n+3$ and so $2 = 21n-14m = 7(3n-2m)$. Since $3n-2m\in \Z$, it follows that $7\mid 2$ which is a contradiction. 
		\end{proof}
	\end{problem}

	\begin{problem}{24}
		Prove that there exists no positive integer $x$ such that $2x<x^{2}<3x$.
		\begin{proof}
			Assume, to the contrary, that there is some positive integer $x$ such that $2x<x^{2}<3x$. Since $x\in \Z^{+}$, if follows that  $2 < x < 3$ (divide the original inequality by $x$). The number $x$ must be greater than 2 and lower than 3, namely, in between two consecutive integers and therefore can not be an integer. This contradicts our initial assumption about $x$. 
		\end{proof}
	\end{problem} 

	\begin{problem}{25}
		Prove that there do not exist three distinct positive integers $a$, $b$ and $c$ such that each integer divides the difference of the other two.
		\begin{proof}
			Assume, to the contrary, that there are three distinct positive integers $a$, $b$ and $c$ such that each divides the difference of the other two. Then, $a\neq b \neq c$, and without loss of generality it can be said that $b>a>c$. Therefore, $b\mid (a-c)$ and so $a-c = bm$ where $m\in \Z^{+}$ since $a-c>0$. However, note that $bm\geq b>b-c>a-c>0$, which leads to a contradiciton.
		\end{proof}
	\end{problem}

	\begin{problem}{26}
		Prove that the sum of the squares of two odd integers cannot be the square of an integer.
		\begin{lemma}{1}
			Let $k$ be a positive odd integer. Then $\sqrt{2k}$ is an irrational number.
			\begin{proof}
				Assume, to the contrary, that there is some positive odd integer $k$ such that $\sqrt{2k}$ is a rational number $m = a/b$ where $a,b\in \Z$ and $b\neq 0$. We may further assme that $a/b$ is reduced to the lowest terms. Then, $\sqrt{2k} = a/b$ and so $2k = a^{2}/b^{2}$. Therefore, $2kb^{2} = a^{2}$ and so $2\mid a^{2}$ and, by \textit{Theorem 3.12}, $2\mid a$. Thus, $a=2c$ for some integer $c$ and so $2kb^{2} = 2(2c^{2}) = (2c)^{2}$. Therefore, $kb^{2} = 2c^{2}$ and so $2\mid kb^{2}$. Thus, by theorem, either $2\mid k$ or $2\mid b^{2}$. Since $k$ is odd, it follows that $2\mid b^{2}$ and so $2\mid b$. Therefore, both $2\mid a$ and $2\mid b$, which means that they have a factor in common and contradicts our assumption.
			\end{proof}
		\end{lemma}
		\begin{proof}
			Assume, to the contrary, that there are two odd integers $a$ and $b$ such that $a^{2}+b^{2} = k^{2}$ where $k\in \Z$. Then, $a=2m+1$ and $b=2n+1$ where $n,m\in \Z$, and so
			\begin{align*}
			 a^{2}+b^{2} &= (2m+1)^{2}+(2n+1)^{2} \\
			 &= 4m^{2}+4m+1+4n^{2}+4n+1 \\
			 &= 2(2m^{2}+2m+2n^{2}+2n+1)\\
			 &= 2(2(m^{2}+m+n^{2}+n)+1) = k^{2}
		 	\end{align*}
	 Then by squaring both sides we get $\sqrt{(2(2(m^{2}+m+n^{2}+n)+1))} = |k|$. Since $m^{2}+m+n^{2}+n\in \Z$, it follows that $2(m^{2}+m+n^{2}+n)+1$ is odd. Let $2(m^{2}+m+n^{2}+n)+1=l$. Therefore, by \textit{lemma}, $\sqrt{2l}$ is an irrational number, which leads to a contradiciton.
		\end{proof}
	\end{problem}

	\begin{problem}{27}
		Prove that if $x$ and $y$ are positive real numbers, then $\sqrt{x+y}\neq \sqrt{x}+\sqrt{y}$.
		\begin{proof}
			Assume, to the contrary, that there exist two positive real numbers $x$ and $y$ such that $\sqrt{x+y}= \sqrt{x}+\sqrt{y}$. Squaring both sides we get $x+y = (\sqrt{x}+\sqrt{y})^{2}=x+2\sqrt{xy}+y$. Thus, $0 = 2\sqrt{xy}$ and therefore $xy= 0$, which leads to a contradiciton.
		\end{proof}
	\end{problem} 

	\begin{problem}{28}
		Prove that there do not exist positive integers $m$ and $n$ such that $m^{2}-n^{2}=1$.
		\begin{proof}
			Assume, to the contrary, that there exist two positive integers $m$ and $n$ such that $m^{2}-n^{2}=1$. Then, $m^{2}-n^{2} = (m+n)(m-n) = 1$.
 		Therefore, both $(m+n) = (m-n) = 1$ since $m+n$ and $m-n$ are integers. However, since $m,n\in \Z^{+}$, it follows that $m+n>1$ and this leads to a contradiciton.
		\end{proof}
	\end{problem}

\end{document}