\documentclass[12pt]{article}
\usepackage[margin=1in]{geometry}
\usepackage{amsmath,amsthm,amssymb,epigraph,etoolbox,mathtools,setspace,enumitem}  
\usepackage{tikz}
\usetikzlibrary{datavisualization}
\usepackage[makeroom]{cancel} 
\usepackage[linguistics]{forest}
\usetikzlibrary{patterns}
\newcommand{\N}{\mathbb{N}}
\newcommand{\Z}{\mathbb{Z}}
\newcommand{\R}{\mathbb{R}}
\newcommand{\Q}{\mathbb{Q}}
\newcommand{\Mod}[1]{\ (\mathrm{mod}\ #1)}

\DeclarePairedDelimiter\bra{\langle}{\rvert}
\DeclarePairedDelimiter\ket{\lvert}{\rangle}
\DeclarePairedDelimiterX\braket[2]{\langle}{\rangle}{#1\delimsize\vert #2}


\newenvironment{theorem}[2][Theorem]{\begin{trivlist}
		\item[\hskip \labelsep {\bfseries #1}\hskip \labelsep {\bfseries #2.}]}{\end{trivlist}}
\newenvironment{lemma}[2][Lemma]{\begin{trivlist}
		\item[\hskip \labelsep {\bfseries #1}\hskip \labelsep {\bfseries #2.}]}{\end{trivlist}}
\newenvironment{exercise}[2][Exercise]{\begin{trivlist}
		\item[\hskip \labelsep {\bfseries #1}\hskip \labelsep {\bfseries #2.}]}{\end{trivlist}}
\newenvironment{problem}[2][Problem]{\begin{trivlist}
		\item[\hskip \labelsep {\bfseries #1}\hskip \labelsep {\bfseries #2.}]}{\end{trivlist}}
\newenvironment{question}[2][Question]{\begin{trivlist}
		\item[\hskip \labelsep {\bfseries #1}\hskip \labelsep {\bfseries #2.}]}{\end{trivlist}}
\newenvironment{corollary}[2][Corollary]{\begin{trivlist}
		\item[\hskip \labelsep {\bfseries #1}\hskip \labelsep {\bfseries #2.}]}{\end{trivlist}}
\newenvironment{solution}[2][Solution]{\begin{trivlist}
		\item[\hskip \labelsep {\bfseries #1}\hskip \labelsep {\bfseries #2.}]}{\end{trivlist}}

\setlength\epigraphwidth{8cm}
\setlength\epigraphrule{0pt}

\makeatletter
\patchcmd{\epigraph}{\@epitext{#1}}{\itshape\@epitext{#1}}{}{}
\makeatother


\begin{document}
	
	\title{Week 15}
	\author{Juan Patricio Carrizales Torres \\
		Section 4: Existence Proofs}
	\date{November 11, 2021}
	\maketitle

	Let $R(x)$ be an open sentence over the set $S$. In order to disprove the existence statement $\exists x\in S, R(x)$, one must prove it's negations, namely, $\sim (\exists x\in S, R(x)) \equiv \forall x\in S, \sim R(x)$. Basically, one must show that the open sentence $R(x)$ is false for all $x\in S$.
	
	\begin{problem}{49}
		Disprove the statement: There exist odd integers $a$ and $b$ such that $4\mid (3a^{2}+7b^{2})$.
		\begin{solution}{}
			We show that for every odd integers $a$ and $b$, $4\nmid (3a^{2}+7b^{2})$. Let $a$ and $b$ be odd integers. Then $a=2m+1$ and $b=2n+1$ for some integers $m,n$. Therefore,
			\begin{align*}
				3a^{2}+7b^{2} &= 3(2m+1)^{2}+7(2n+1)^{2}\\
				&= 3(4m^{2}+4m+1)+7(4n^{2}+4n+1) =  4(3m^{2}+3m) +  4(7n^{2}+7n) + 10\\
				&= 4(3m^{2}+3m+7n^{2}+7n+2)+2
			\end{align*}
		Since $3m^{2}+3m+7n^{2}+7n+2 \in \Z$, it follows that $4\nmid (3a^{2}+7b^{2})$.
		\end{solution}
	\end{problem}

	\begin{problem}{50}
		Disprove the statement: There is a real number $x$ such that $x^{6}+x^{4}+1=2x^{2}$.
		\begin{solution}{}
			We show that for every $x\in \R$, $x^{6}+x^{4}+1\neq 2x^{2}$. Note that, $x^{6}+x^{4}-2x^{2}+1 = x^{6}+ (x^{2}-1)^{2} = 0$. Since $x^{6}\geq 0$ and $(x^{2}-1)^{2}\geq 0$, it follows that $x^{6}+ (x^{2}-1)^{2} = 0$ iff $x^{6} = (x^{2}-1)^{2} = 0$. Note that, $x^{6}=0$ iff $x=0$. However, $(x^{2}-1)^{2}=(-1)^{2}=1\neq 0$ for $x=0$. Thus, there is no real solution for $x^{6}+x^{4}+1=2x^{2}$.
		\end{solution}
	\end{problem}
 
	\begin{problem}{51}
		Disprove the statement: There is an integer $n$ such that $n^{4}+n^{3}+n^{2}+n$ is odd.
		\begin{solution}{}
			We show that for any integer $n$, the integer $n^{4}+n^{3}+n^{2}+n$ is even. Let $n\in \Z$. We consider two cases.\\
			\textit{Case 1.} $n$ is even. Then $n =2a$ for some integer $a$. Therefore, $n^{4}+n^{3}+n^{2}+n = n(n^{3}+n^{2}+n+1) = 2[a(n^{2}+n^{2}+n+1)]$. Since $a(n^{2}+n^{2}+n+1)$ is an integer, it follows that $n^{4}+n^{3}+n^{2}+n$ is even.\\
			\textit{Case 2.} $n$ is odd. According to \textit{Theorem 3.17}, $ab$ is odd if and only if $a$ and $b$ are odd. Therefore, $n^{4}=n^{2}\cdot n^{2}$, $n^{3} = n^{2}\cdot n$ and $n^{2}$ are odd. Let $n^{4}=2a+1$, $n^{3}=2b+1$, $n^{2} = 2c+1$ and $n = 2d+1$ for some integers $a,b,c,d$. Thus, $n^{4}+n^{3}+n^{2}+n = 2a+2b+2c+2d+4=2(a+b+c+d+2)$. Since $a+b+c+d+2\in \Z$, it follows that $n^{4}+n^{3}+n^{2}+n$ is even.
		\end{solution}
	\end{problem}

\end{document}