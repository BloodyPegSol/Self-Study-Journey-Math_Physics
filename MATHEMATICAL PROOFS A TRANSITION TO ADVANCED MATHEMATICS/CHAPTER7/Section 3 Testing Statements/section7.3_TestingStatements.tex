\documentclass[12pt]{article}
\usepackage[margin=1in]{geometry}
\usepackage{amsmath, amsfonts,amsthm,amssymb,epigraph,etoolbox,mathtools,setspace,enumitem}  
\usepackage{tikz}
\usetikzlibrary{datavisualization}
\usepackage[makeroom]{cancel} 
\usepackage[linguistics]{forest}
\usetikzlibrary{patterns}
\newcommand{\N}{\mathbb{N}}
\newcommand{\Z}{\mathbb{Z}}
\newcommand{\R}{\mathbb{R}}
\newcommand{\Q}{\mathbb{Q}}
\newcommand{\Mod}[1]{\ (\mathrm{mod}\ #1)}



\newlist{legal}{enumerate}{10}
\setlist[legal]{label*=\arabic*.}

\DeclarePairedDelimiter\bra{\langle}{\rvert}
\DeclarePairedDelimiter\ket{\lvert}{\rangle}
\DeclarePairedDelimiterX\braket[2]{\langle}{\rangle}{#1\delimsize\vert #2}


\newenvironment{theorem}[2][Theorem]{\begin{trivlist}
		\item[\hskip \labelsep {\bfseries #1}\hskip \labelsep {\bfseries #2.}]}{\end{trivlist}}
\newenvironment{lemma}[2][Lemma]{\begin{trivlist}
		\item[\hskip \labelsep {\bfseries #1}\hskip \labelsep {\bfseries #2.}]}{\end{trivlist}}
\newenvironment{result}[2][Result]{\begin{trivlist}
		\item[\hskip \labelsep {\bfseries #1}\hskip \labelsep {\bfseries #2.}]}{\end{trivlist}}
\newenvironment{exercise}[2][Exercise]{\begin{trivlist}
		\item[\hskip \labelsep {\bfseries #1}\hskip \labelsep {\bfseries #2.}]}{\end{trivlist}}
\newenvironment{problem}[2][Problem]{\begin{trivlist}
		\item[\hskip \labelsep {\bfseries #1}\hskip \labelsep {\bfseries #2.}]}{\end{trivlist}}
\newenvironment{question}[2][Question]{\begin{trivlist}
		\item[\hskip \labelsep {\bfseries #1}\hskip \labelsep {\bfseries #2.}]}{\end{trivlist}}
\newenvironment{corollary}[2][Corollary]{\begin{trivlist}
		\item[\hskip \labelsep {\bfseries #1}\hskip \labelsep {\bfseries #2.}]}{\end{trivlist}}
\newenvironment{solution}[2][Solution]{\begin{trivlist}
		\item[\hskip \labelsep {\bfseries #1}\hskip \labelsep {\bfseries #2.}]}{\end{trivlist}}

\setlength\epigraphwidth{8cm}
\setlength\epigraphrule{0pt}

\makeatletter
\patchcmd{\epigraph}{\@epitext{#1}}{\itshape\@epitext{#1}}{}{}
\makeatother


\begin{document} 
	
	\title{Section 7.3: Testing Statements (QUIZ)}
	\author{Juan Patricio Carrizales Torres}
	\date{April 025, 2022}
	\maketitle
	
	Prove or disprove each of the following statements.
	 
	\begin{problem}{1}
		If $n$ is a positive integer and $s$ is an irrational number, then $n/s$ is an irrational number.
		\begin{proof}
			Let $n\in \N$ and $s \in \R / \Q$. Since $n=\frac{n}{1}$, it follows that $n\in \Q$. Now, assume, to the contrary, that $n/s = q$ for some rational number $q\neq 0$ since $n>0$. Therefore, $n=s\cdot q$ is irrational, which leads to a contradiction. (Note that this implies that $s\in \R/\Q \iff s^{-1} \in \R/\Q$).
		\end{proof}
	\end{problem}
	
	\begin{problem}{2}
		For every integer $b$, there exists a positive integer $a$ such that $|a-|b||\leq 1$.
		\begin{proof}
			Let $b\in \Z$ and $a=|b|+1$. Thus, $a\geq 1>0$ and $|a-|b|| = |1|=1$.  
		\end{proof}
	\end{problem}

	\begin{problem}{3}
		If $x$ and $y$ are integers of the same parity, then $xy$ and $(x+y)^{2}$ are of the same parity.
		\begin{solution}{3}
			This statement is false. Let $x$ and $y$ be arbitrary odd integers. Therefore, $xy$ is odd (multiplication of odd integers; refer to \textbf{Lemma ODD} in pdf of Section7.2) and $x+y$ is even (sum of two odd integers), which implies that $(x+y)^{2}$ is even. Hence, $xy$ and $(x+y)^{2}$ are of opposite parity.
		\end{solution}
	\end{problem}
 
	\begin{problem}{4}
		Let $a,b\in \Z$. If $6\nmid ab$, then either (1) $2\nmid a$ and $3\nmid b$ or (2) $3\nmid a$ and $2\nmid b$. 
		\begin{solution}{4}
			This statement is false. Let $a=3$ and $b=9$. Then, $ab = 27$ and $6\nmid ab$. However, $3\mid a$ and $3\mid b$. Another example, let $a=2$ and $b=4$. Then, $ab=8$ and $6\nmid ab$. However, $2\mid a$ and $2\mid b$. \\
			
			This is so since $6=3\cdot 2$. Therefore, for $6\nmid ab$ to be true, it suffices that $a$ and $b$ are not divisible by either 2 or 3.
		\end{solution}
	\end{problem}

	\begin{problem}{5}
		For every positive integer $n$, $2^{2^{n}} \geq 4^{n!}$.
	\begin{solution}{5}
		This statement is false. Let $n=4$. Then $2^{2^{4}} = 2^{16}=4^{8}$ and $4^{4!} = 4^{24}$. Thus, $4^{8}<4^{24}$ and so $n=4$ represents a counterexample.
	\end{solution}
	\end{problem}

	\begin{problem}{6}
		If $A,B$ and $C$ are sets, then $(A-B)\cup (A-C)=A-(B\cup C)$.
		\begin{solution}{6}
			This statement is false. Let $A=\{1,2\}, B=\emptyset$ and $C=\{1\}$. Then, $(A-B)\cup (A-C) = \{1,2\}\cup \{2\} = A$ and $A-(B\cup C) = \{1,2\}-\{2\} = \{1\} \neq A$. Therefore, these specific sets $A,B$ and $C$ represent a counterexample.\\
			
			In general, let $C\subseteq A$ and $B=\emptyset$. Hence, 
			\begin{equation*}
				(A-B)\cup (A-C) = A\cup (A-C) = A
			\end{equation*}
			and 
			\begin{equation*}
				A-(B\cup C) = A-C = A\cap \overline{C} \subset A
			\end{equation*}
			since $C\subseteq A$. 
		\end{solution}
	\end{problem} 

	\begin{problem}{7}
		Let $n\in \N$. If $(n+1)(n+4)$ is odd, then $(n+1)(n+4)+3^{n}$ is odd.
		\begin{proof}
			Let $n\in \N$. Hence, $n$ is either odd or even. If $n$ is even, then $(n+4)$ is even (same parity) and so $(n+1)(n+4)$ is even. On the other hand, if $n$ is odd, then $(n+1)$ is even (same parity) and so $(n+1)(n+4)$ is even. Therefore, there is no positive integer such that $(n+1)(n+4)$ is odd and so the statement follows vacuously.\\
			
			Curiously, if someone tried to prove the implication directly it would lead to a false conclusion. $3^{n}$ is odd (multiplication of $n$ odd numbers \textbf{Theorem ODD}) and so $(n+1)(n+4)+3^{n}$ is even (sum of numbers with same parity). One must understand that proof techniques deal with the deduction process but not guarantee that the premises are true.
		\end{proof}
	\end{problem}

	\begin{problem}{8}
		\begin{enumerate}[label=(\alph*)]
		\item There exist distinct rational numbers $a$ and $b$ such that $(a-1)(b-1)=1$.
			\begin{proof}
				Consider some \textbf{nonzero} rational number $r$ such that $|r|\neq 1$. Let $a = 1+\frac{1}{r}$ and $b=1+r$. Then $a\neq b$ and 
				\begin{align*}
					(a-1)(b-1) &= \left(\left(1+ \frac{1}{r}\right)-1\right) \left(\left(1+r\right)-1\right)\\
					&= \frac{1}{r}\cdot r = 1
				\end{align*} 
			\end{proof}
		
		\item There exist distinct rational numbers $a$ and $b$ such that $\frac{1}{a} + \frac{1}{b}=1$.
			\begin{proof}
				 Let $a=\frac{3}{2}$ and $b=3$. Then, $a\neq b$ and 
				 \begin{equation*}
				 	a^{-1} + b^{-1} = \frac{2}{3} + \frac{1}{3} = 1.
				 \end{equation*}
			 Note that, $\frac{1}{a} + \frac{1}{b} = \frac{b+a}{ab} = 1$ implies $b+a = ab$. Therefore,
			 $0=ab-b-a$ and so $1=1+ab-b-a$. Thus,
			 \begin{align*}
			 	1 &= b(a-1)+1-a = b(a-1) - (a-1)\\
			 	&= (b-1)(a-1),
			 \end{align*}
		 	which is the statement (a). Hence, (a) $\iff$ (b) (They are logically equivalent).
			\end{proof}
		\end{enumerate}
	\end{problem}
	
	\begin{problem}{9}
		Let $a,b,c\in \Z$. If every two of $a, b$ and $c$ are of the same parity, then $a+b+c$ is even.
		\begin{solution}{9}
			This statement is false. Let $a,b$ and $c$ be odd. Then, every two of $a,b,c$ are of the same parity. Note that $a+b$ is even (sum of two odd numbers). However, $(a+b)+c$ is odd, since it is the sum of an even number with and odd one. 
		\end{solution}
	\end{problem}

	\begin{problem}{10}
		If $n$ is a nonnegative integer, then $5$ divides $2\cdot 4^{n} + 3\cdot9^{n}$.
		\begin{proof}
			Note that 
			\begin{align*}
				2\cdot 4^{n} + 3\cdot9^{n} &= 2\cdot 2^{2n} + 3\cdot 3^{2n}\\
				 &= 2^{2n+1}+3^{2n+1}.
			\end{align*}
		Let $n\geq 0$. We proceed by induction. Since $2^{1} + 3^{1} = 2+3 = 5$, it follows that the result is true for $n=0$. Assume that $5\mid\left(2^{2k+1} + 3^{2k+1}\right)$ for some $k\geq 0$. We show that $5\mid\left(2^{2k+3} + 3^{2k+3}\right)$. Note that $2^{2k+1} + 3^{2k+1} = 5c$ for some integer $c$ and so $2^{2k+1} = 5c-3^{2k+1}$. Therefore,
		\begin{align*}
			2^{2k+3} + 3^{2k+3} &= 2^{2}\cdot 2^{2k+1} + 3^{2}\cdot 3^{2k+1}\\
			&= 2^{2}\left(5c-3^{2k+1}\right) + 3^{2}\cdot 3^{2k+1}\\
			&= 2^{2}\cdot 5c -2^{2}\cdot 3^{2k+1}+3^{2}\cdot 3^{2k+1}\\
			&= 2^{2}\cdot 5c + (3^{2}-2^{2})\cdot 3^{2k+1} \\
			&= 2^{2}\cdot 5c + 5\cdot 3^{2k+1} = 5\left(2^{2}c +3^{2k+1}\right).
		\end{align*} 
	Since $2^{2}c +3^{2k+1}$ is an integer, it follows that $5\mid \left(2^{2k+3} + 3^{2k+3}\right)$. By the Principle of Mathematical Induction, if $n\geq 0$, then
	\begin{equation*}
		5\mid \left(2^{2n+1}+3^{2n+1}\right).
	\end{equation*}

	An interesting observation is that both $2$ and $3$ are raised to the same odd power. Note that
\begin{align*}
		2^{1} = 2 &\quad 3^{1} = 3\\ 
		2^{3} = 8 &\quad 3^{3} = 27\\
		2^{5} = 32 &\quad 3^{5} = 243\\
		2^{7} = 128 &\quad 3^{7} = 2187\\
		2^{9} = 512 &\quad 3^{9} = 19,683\\
		2^{11} = 2048 &\quad 3^{11} = 177,147\\
\end{align*}
Some type of pattern seems to hold for the last digits for both integers, namely, $8$ and $2$ alternate in $2$ raised to an odd power, and $7$ and $3$ alternate in $3$ raised to an odd power $n\geq 1$. If the power is the same for both, then either the last digits are $8$ for $2^{n}$ and $7$ for $3^{n}$, or $2$ for  $2^{n}$ and $3$ for $3^{n}$. Note that their sum give a number that ends in $5$, which is the last digit of $2^{n} + 3^{n}$.
		\end{proof}
	\end{problem}
\end{document}