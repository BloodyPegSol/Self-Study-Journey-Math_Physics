\documentclass[12pt]{article}
\usepackage[margin=1in]{geometry}
\usepackage{amsmath, amsfonts,amsthm,amssymb,epigraph,etoolbox,mathtools,setspace,enumitem}  
\usepackage{tikz}
\usetikzlibrary{datavisualization}
\usepackage[makeroom]{cancel} 
\usepackage[linguistics]{forest}
\usetikzlibrary{patterns}
\newcommand{\N}{\mathbb{N}}
\newcommand{\Z}{\mathbb{Z}}
\newcommand{\R}{\mathbb{R}}
\newcommand{\Q}{\mathbb{Q}}
\newcommand{\Mod}[1]{\ (\mathrm{mod}\ #1)}



\newlist{legal}{enumerate}{10}
\setlist[legal]{label*=\arabic*.}

\DeclarePairedDelimiter\bra{\langle}{\rvert}
\DeclarePairedDelimiter\ket{\lvert}{\rangle}
\DeclarePairedDelimiterX\braket[2]{\langle}{\rangle}{#1\delimsize\vert #2}


\newenvironment{theorem}[2][Theorem]{\begin{trivlist}
		\item[\hskip \labelsep {\bfseries #1}\hskip \labelsep {\bfseries #2.}]}{\end{trivlist}}
\newenvironment{lemma}[2][Lemma]{\begin{trivlist}
		\item[\hskip \labelsep {\bfseries #1}\hskip \labelsep {\bfseries #2.}]}{\end{trivlist}}
\newenvironment{result}[2][Result]{\begin{trivlist}
		\item[\hskip \labelsep {\bfseries #1}\hskip \labelsep {\bfseries #2.}]}{\end{trivlist}}
\newenvironment{exercise}[2][Exercise]{\begin{trivlist}
		\item[\hskip \labelsep {\bfseries #1}\hskip \labelsep {\bfseries #2.}]}{\end{trivlist}}
\newenvironment{problem}[2][Problem]{\begin{trivlist}
		\item[\hskip \labelsep {\bfseries #1}\hskip \labelsep {\bfseries #2.}]}{\end{trivlist}}
\newenvironment{question}[2][Question]{\begin{trivlist}
		\item[\hskip \labelsep {\bfseries #1}\hskip \labelsep {\bfseries #2.}]}{\end{trivlist}}
\newenvironment{corollary}[2][Corollary]{\begin{trivlist}
		\item[\hskip \labelsep {\bfseries #1}\hskip \labelsep {\bfseries #2.}]}{\end{trivlist}}
\newenvironment{solution}[2][Solution]{\begin{trivlist}
		\item[\hskip \labelsep {\bfseries #1}\hskip \labelsep {\bfseries #2.}]}{\end{trivlist}}

\setlength\epigraphwidth{8cm}
\setlength\epigraphrule{0pt}

\makeatletter
\patchcmd{\epigraph}{\@epitext{#1}}{\itshape\@epitext{#1}}{}{}
\makeatother


\begin{document}
	
	\title{Week 1}
	\author{Juan Patricio Carrizales Torres \\
		Section 7.1: Conjectures in Mathematics}
	\date{March 24, 2022}
	\maketitle
	
	\begin{problem}{1}
		Consider the following sequence of equalities:
		\begin{align*}
			1&=0+1\\
			2&+3+4=1+8\\
			5&+6+7+8+9=8+27\\
			10&+11+12+13+14+15+16=27+64
		\end{align*}
	\begin{enumerate}[label=(\alph*)]
		\item What is the next equality in this sequence?
		\begin{solution}{(a)}
		The next equality seems to be
		\begin{equation*}
			17+18+19+20+21+22+23+24+25 = 64+125
		\end{equation*}
		\end{solution}
	
		\item What conjecture is suggested by these equalities?
		\begin{solution}{(b)}
			For any nonnegative integer $n$, we have that 
			\begin{equation*}
			\sum_{i=1}^{2n+1}(n^{2}+i) = n^{3} + (n+1)^{3}
			\end{equation*}
		\end{solution}
		
		\item Prove the conjecture in (b) by induction?
		\begin{proof}
			We proceed by induction. Since $0^{2}+(0^{2}+1) = 1 = 0^{3} + (0+1)^{3}$, it follows that the conjecture is true for $n=0$. Assume that 
			\begin{equation*}
			 \sum_{i=1}^{2k+1}(k^{2}+i) = k^{3} + (k+1)^{3}
			\end{equation*}
			for some nonnegative integer $k$. Hence,
			\begin{equation*}
				(k+1)^{3} =  \sum_{i=1}^{2k+1}(k^{2}+i) - k^{3}.
			\end{equation*}
			 We show that
			\begin{equation*}
				\sum_{i=1}^{2(k+1)+1}\left[(k+1)^{2}+i\right] = (k+1)^{3} + (k+2)^{3}
			\end{equation*}
		Note that 
		\begin{align*}
			\sum_{i=1}^{2(k+1)+1}\left[(k+1)^{2}+i\right] &= 	\sum_{i=1}^{2k+3}\left[k^{2}+2k+1+i\right]\\
			&= \sum_{i=1}^{2k+3}(k^{2}+i) + (2k+3)(2k+1)\\
			&= \sum_{i=1}^{2k+1}(k^{2}+i) + [k^{2}+(2k+2)] + [k^{2} + (2k+3)] + (2k+3)(2k+1)\\
			&= \sum_{i=1}^{2k+1}(k^{2}+i) + k^{2} + 2k + 2 + k^{2} + 2k + 3 + 4k^{2} +8k+3\\
			&= \sum_{i=1}^{2k+1}(k^{2}+i) + 6k^{2} + 12k + 8\\
			&= \sum_{i=1}^{2k+1}(k^{2}+i) - k^{3} + [k^{3} + 6k^{2} + 12k + 8]\\ 
			&= \sum_{i=1}^{2k+1}(k^{2}+i) - k^{3} + (k+2)^{3}
			= (k+1)^{3} + (k+2)^{3},
		\end{align*}
	according to the inductive hypothesis. By the Principle of Mathematical Induction, it is true that
	\begin{equation*}
		\sum_{i=1}^{2n+1}(n^{2}+i) = n^{3} + (n+1)^{3}.
	\end{equation*}
for any nonnegative integer $n$.
		\end{proof}
	\end{enumerate}
	\end{problem}

	\begin{problem}{2}
		Consider the following statements:
		\begin{align*}
			(1+2)^{2} -1^{2} &= 2^{3}\\
			(1+2+3)^{2}- (1+2)^{2} &= 3^{3}\\
			(1+2+3+4)^{2}- (1+2+3)^{2} &= 4^{3}
		\end{align*}
		\begin{enumerate}[label=(\alph*)]
			\item Based on the three statements given above, what is the next statement suggested by these?
			\begin{solution}{a}
				The next statement suggested is
				\begin{equation*}
					(1+2+3+4+5)^{2} - (1+2+3+4)^{2} = 5^{3}
				\end{equation*}
			\end{solution}
			\item What conjecture is suggested by these statements?
			\begin{solution}{b}
				For any integer $n\in \N$, 
				\begin{equation*}
					[1+2+\ldots+(n+1)]^{2} - (1+2+\ldots+n)^{2} = (n+1)^{3}
				\end{equation*}
			\end{solution}
			\item Verify the conjecture in (b).
			\begin{proof}
				Let $n\in \N$. Note that
				\begin{align*}
					[1+2+\ldots+(n+1)]^{2} - (1+2+\ldots+n)^{2} &= \left[(n+1)\left(\frac{n+2}{2}\right)\right]^{2} - \left[n\left(\frac{n+1}{2}\right)\right]^{2}\\
					&= \left(\frac{n^{2}+3n+2}{2}\right)^{2} - \left(\frac{n^{2}+n}{2}\right)^{2}\\
					&= \frac{(n^{4}+9n^{2}+4+6n^{3}+4n^{2}+12n) - (n^{4}+2n^{3}+n^{2})}{4}\\
					&= \frac{4n^{3}+12n^{2}+12n+4}{4} = n^{3} + 3n^{2} + 3n +1\\
					&= (n+1)^{3}.
				\end{align*}
			(Consider that $1 = 1\frac{1+1}{2}$).
			\end{proof}
		\end{enumerate}
	\end{problem}

	\begin{problem}{3}
		A sequence $\{a_{n}\}$ of real numbers is defined recursively by $a_{1} = 2$ and for $n\geq 2$,
		\begin{equation*}
			a_{n} = \frac{2+1\cdot a_{1}^{2} + 2\cdot a_{2}^{2} + \ldots + (n-1)a_{n-1}^{2}}{n}.
		\end{equation*}
		\begin{enumerate}[label = (\alph*)]
			\item Determine $a_{2}, a_{3}$ and $a_{4}$.
			\begin{solution}{a}
				\begin{align*}
					a_{2} &= \frac{2+1\cdot2^{2}}{2} = 3\\
					a_{3} &= \frac{6+2\cdot 3^{2}}{3} = 8\\
					a_{4} &= \frac{24+3\cdot 8^{2}}{4} = 54
				\end{align*}
			\end{solution}
			\item Clearly, $a_{n}$ is a rational number for each $n\in \N$. Based on the information in (a), however, what conjecture does this suggest?
			\begin{solution}{b}
				For every $n\in \N$, $a_{n}$ is a positive integer.\\
				 This conjecture implies for every $n\in \N$ that 
				\begin{equation*}  
					 \left(2+1\cdot a_{1}^{2} + 2\cdot a_{2}^{2} + \ldots + (n-1)a_{n-1}^{2}\right) = n\cdot a
				\end{equation*}
				 for some $a\in \N$.
			\end{solution}
		\end{enumerate}
	\end{problem}
	
	\begin{problem}{5}
		By an ordered partition of an integer $n\geq 2$ is meant a sequence of positive integers whose sum is $n$. For example, the ordered partitions of 3 are $3,1+2,2+1,1+1+1$.
		\begin{enumerate}[label=(\alph*)]
			\item Determine the ordered partitions of 4.
			\begin{solution}{a}
				Let $(a,b,c,\ldots,n)$ represent an n-tuple for the integers of some ordered partition of 4. Then 4 has the following partitions:
				\begin{align*}
					(4)\\
					(1,3), (3,1)\\
					(2,2)\\
					(2,1,1), (1,2,1), (2,1,1)\\
					(1,1,1,1)
				\end{align*}
				
			\end{solution}
			\item Make a conjecture concerning the number of ordered partitions of an integer $n\geq 2$.
			\begin{solution}{b}
				Let $M(n)$ be number of ordered partitions of some integer $n\geq 2$. Note that $M(2) = 2, M(3) = 4$ and $M(4) = 8$ are powers of 2 and so one may conjecture the following:\\
				 Let some integer $n\geq 2$. Then, the number of ordered partitions of $n$ is $2^{n-1}$.		
			\end{solution}
		\end{enumerate}
	\end{problem}

	\begin{problem}{6}
		Two recursively defined sequences $\{a_{n}\}$ and $\{b_{n}\}$ of positive integers have the same recurrence relation, namely $a_{n} = 2a_{n-1}+a_{n-2}$ and $b_{n} = 2b_{n-1}+b_{n-2}$ for $n\geq 3$. The initial values for $\{a_{n}\}$ are $a_{1} = 1$ and $a_{2} = 3$, while the initial values for $\{b_{n}\}$ are $b_{1} = 1$ and $b_{2} = 2$.
		\begin{enumerate}[label=(\alph*)]
			\item Determine $a_{3}$ and $a_{4}$.
			\begin{solution}{a}
				\begin{align*}
					a_{3} &= 2\cdot 3 +1=7\\
					a_{4} &= 2\cdot 7 + 3 = 17.
				\end{align*}
			\end{solution}
			
			\item Determine whether the following is true or false:\\
			\textbf{Conjecture:} $a_{n} = 2^{n-2}\cdot n + 1$ for every integer $n\geq 2$.
			\begin{proof}
				We proceed by strong induction. Since $a_{2} = 3 = 2^{0}\cdot 2 + 1$, it follows that the conjecture is true for $n=2$. Suppose for $2\leq i  \leq k$ that $a_{i} = 2^{i-2}\cdot i +1$. We prove that $a_{k+1} \neq 2^{k-1}\cdot (k+1) +1$. Note that $ a_{3} = 7 = 2^{1}\cdot 3 +1$ and $a_{4} = 17 = 2^{2}\cdot 4 + 1$ and so $k\geq 4$. Hence, $k+1\geq 5$ and
				\begin{align*}
					a_{k+1} &= 2a_{k}+a_{k-1} = 2\left(2^{k-2}\cdot k +1\right) + \left(2^{k-3} \cdot (k-1) +1\right)\\
					&= 2^{k-1}\cdot k + 2^{k-3}\cdot (k-1) +2+1\\
					&= k(2^{k-1}+2^{k-3}) -2^{k-3}+2+1\\
					&\neq 2^{k-1}\cdot k + 2^{k+1}+1 = 2^{k-1}\cdot (k+1) +1
				\end{align*}
				since $2^{k-3}> 0$ and so $2^{k-1}+2^{k-3} \neq 2^{k-1}$. By the Strong Principle of Mathematical Induction, this conjecture is false ($(p\implies q, \neg p) \implies \neg q$). Note that $a_{k+1}$ represents a counterexample to our conjecture.
			\end{proof}
			
			\item Determine $b_{3}$ and $b_{4}$.
			\begin{solution}{c}
				\begin{align*}
					b_{3} &= 2\cdot 2 +1 = 5\\
					b_{4} &= 2\cdot 5 +2 = 12.
				\end{align*}
			\end{solution}
		
			\item Determine whether the following is true or false:\\
			\textbf{Conjecture:} $b_{n} = \frac{(1+\sqrt{2})^{n} - (1-\sqrt{2})^{n}}{2\sqrt{2}}$ for every integer $n\geq 2$.\\
			\begin{proof}
				We proceed by strong induction. Since 
				\begin{align*}
				 \frac{(1+\sqrt{2})^{1} - (1-\sqrt{2})^{1}}{2\sqrt{2}} &= \frac{2\sqrt{2}}{2\sqrt{2}} = 1 = b_{1}.
				\end{align*}
			it follows for $n=1$ that the conjecture is true. Suppose for $2\leq i \leq k$ that 
			\begin{equation*}
				b_{i} = \frac{(1+\sqrt{2})^{i} - (1-\sqrt{2})^{i}}{2\sqrt{2}}.
			\end{equation*}
			We show that 
			\begin{equation*}
				b_{k+1} = \frac{(1+\sqrt{2})^{k+1} - (1-\sqrt{2})^{k+1}}{2\sqrt{2}}.
			\end{equation*}
		Because 
		\begin{align*}
			\frac{(1+\sqrt{2})^{2} - (1-\sqrt{2})^{2}}{2\sqrt{2}} &= \frac{4\sqrt{2}}{2\sqrt{2}} = 2 = b_{2},
		\end{align*}
	it follows that the conjecture is true for $k\geq 2$ and so $k+1\geq 3$. By definition of $\{a_{n}\}$, we have
	\begin{align*}
		b_{k+1} &= 2b_{k} + b_{k-1} = 2\left(\frac{(1+\sqrt{2})^{k} - (1-\sqrt{2})^{k}}{2\sqrt{2}}\right) + \frac{(1+\sqrt{2})^{k-1} - (1-\sqrt{2})^{k-1}}{2\sqrt{2}}\\
		&= \frac{2\left(1+\sqrt{2}\right)^{k} + \left(1+\sqrt{2}\right)^{k-1} - \left[ 2\left(1-\sqrt{2}\right)^{k} +  \left(1-\sqrt{2}\right)^{k-1}\right]}{2\sqrt{2}}\\
		&= \frac{2\left(1+\sqrt{2}\right)^{k-1}\left(1+\sqrt{2}\right) + \left(1+\sqrt{2}\right)^{k-1} - \left[ 2\left(1-\sqrt{2}\right)^{k-1}\left(1-\sqrt{2}\right) + \left(1-\sqrt{2}\right)^{k-1}\right]}{2\sqrt{2}}\\
		&= \frac{
			\left(1+\sqrt{2}\right)^{k-1}\left(2+2\sqrt{2}+1\right) - \left[ \left(1-\sqrt{2}\right)^{k-1}\left(2-2\sqrt{2}+1\right)\right]}{2\sqrt{2}}\\
		&= \frac{
			\left(1+\sqrt{2}\right)^{k-1}\left(\sqrt{2}+1\right)^{2} -  \left(1-\sqrt{2}\right)^{k-1}\left(1-\sqrt{2}\right)^{2}}{2\sqrt{2}} = \frac{
			\left(1+\sqrt{2}\right)^{k+1} -   \left(1-\sqrt{2}\right)^{k+1}}{2\sqrt{2}}.
	\end{align*}
	By the Strong Principle of Mathematical Induction, 
	\begin{equation*}
		b_{n} = \frac{\left(1+\sqrt{2}\right)^{n} - \left(1-\sqrt{2}\right)^{n}}{2\sqrt{2}}
	\end{equation*}
	for any $n\in \N$.
			\end{proof}
		\end{enumerate}
	\end{problem}

	\begin{problem}{7}
		We know that $1+2+3 = 1\cdot 2\cdot 3$; that is, there exist three positive integers whose sum equals their product. Prove or disprove (a) and (b).
		\begin{enumerate}[label=(\alph*)]
			\item There exist four positive integers whose sum equals their product.
			\begin{solution}{a}
				\[1+1+2+4 = 1\cdot1\cdot2\cdot4\]
			\end{solution}
			\item There exist five positive integers whose sum equals their product.
			\begin{solution}{b}
				\[1+1+1+2+5 = 1\cdot1\cdot1\cdot2\cdot5\]
			\end{solution}
		 
			\item What conjecture does this suggest to you?
			\begin{solution}{c}
				Let some positive integer $n\geq 2$. Then $\{2,n,1,1,1,\ldots\}$ is the set of $n$ positive integers whose sum is equal to their product.
				\begin{proof}
					Consider some positive integer $n\geq 2$  and let $\{2,n,1,1,1,\ldots\}$ be some set of $n$ positive integers. Then 
					\begin{align*}
						2+n+1+1+1+\ldots &= 2+n+(n-2)\\
						&= 2n = 2n\cdot1\cdot1\cdot1\cdots.
					\end{align*}
				\end{proof}
				This clearly implies that for $n\geq 2$ there exist $n$ positive integers whose sum equals their product.
			\end{solution}
		\end{enumerate}
	\end{problem}
\end{document}