\documentclass[12pt]{article}
\usepackage[margin=1in]{geometry}
\usepackage{amsmath,amsthm,amssymb,epigraph,etoolbox}

\newcommand{\N}{\mathbb{N}}
\newcommand{\Z}{\mathbb{Z}}

\newenvironment{theorem}[2][Theorem]{\begin{trivlist}
		\item[\hskip \labelsep {\bfseries #1}\hskip \labelsep {\bfseries #2.}]}{\end{trivlist}}
\newenvironment{lemma}[2][Lemma]{\begin{trivlist}
		\item[\hskip \labelsep {\bfseries #1}\hskip \labelsep {\bfseries #2.}]}{\end{trivlist}}
\newenvironment{exercise}[2][Exercise]{\begin{trivlist}
		\item[\hskip \labelsep {\bfseries #1}\hskip \labelsep {\bfseries #2.}]}{\end{trivlist}}
\newenvironment{problem}[2][Problem]{\begin{trivlist}
		\item[\hskip \labelsep {\bfseries #1}\hskip \labelsep {\bfseries #2.}]}{\end{trivlist}}
\newenvironment{question}[2][Question]{\begin{trivlist}
		\item[\hskip \labelsep {\bfseries #1}\hskip \labelsep {\bfseries #2.}]}{\end{trivlist}}
\newenvironment{corollary}[2][Corollary]{\begin{trivlist}
		\item[\hskip \labelsep {\bfseries #1}\hskip \labelsep {\bfseries #2.}]}{\end{trivlist}}
	
\setlength\epigraphwidth{8cm}
\setlength\epigraphrule{0pt}

\makeatletter
\patchcmd{\epigraph}{\@epitext{#1}}{\itshape\@epitext{#1}}{}{}
\makeatother

\begin{document}
	
\title{Week 1}
\author{Juan Patricio Carrizales Torres \\
Section 4: The Implication}

\maketitle
\epigraph{``At times, the desire becomes almost overpowering, in its intensity. It is not mere curiosity, that prompts me; but more as though some unexplained influence were at work``}{--- \textup{William Hope Hodgson}, The House on the Borderland}

\begin{problem}{19}
	Consider the statements \emph{P: 17 is even.} and \emph{Q: 19 is prime}. Write each of the following statements in words and indicate whether it is true or false.\\

(A) $\sim P$ \textbf{TRUE}
\begin{proof}
The real integer 17 is odd, thus the statement $P$ is false. The negation of this statement will have the opposite truth value, \emph{true}.
\end{proof}
(B) $P \vee Q$ \textbf{TRUE}
\begin{proof}
The real integer 19 is prime, thus $Q$ is true. Because this statement is the disjunction of $P$ and $Q$, it is only needed for one of them to be true for the statement to be true. 
\end{proof}
(C) $P \wedge Q$ \textbf{FALSE}
\begin{proof}
	We already know that the statement $Q$ is true and $P$ is false. This statement is a conjunction of both $Q$ and $P$, and since one of them is false, the statement is false.
\end{proof}
(D) $P \implies Q$ \textbf{TRUE}
\begin{proof}
	Acording to this statement $P$ implies $Q$, so the only way for it to be false is when $P$ is true and $Q$ is false. We know that $P$ is false, which means that the statement is true, because it tells nothing about what would happen when $P$ is false. 
\end{proof}
\end{problem}
\begin{problem}{20}
For statements $P$ and $Q$, construct a truth table for $(P\implies Q)\implies (\sim P)$ 
\begin{tabular}{c c c c c}
	$P$ & $Q$ & $P \implies Q$ & $\sim P$ & $(P \implies Q) \implies (\sim P)$
\end{tabular}
\end{problem}

\end{document}