\documentclass[12pt]{article}
\usepackage[margin=1in]{geometry}
\usepackage{amsmath, amsfonts,amsthm,amssymb,epigraph,etoolbox,mathtools,setspace,enumitem}  
\usepackage{tikz}
\usetikzlibrary{datavisualization} 
\usepackage[makeroom]{cancel} 
\usepackage[linguistics]{forest}
\usetikzlibrary{patterns}
\newcommand{\N}{\mathbb{N}}
\newcommand{\Z}{\mathbb{Z}}
\newcommand{\R}{\mathbb{R}}
\newcommand{\Q}{\mathbb{Q}}
\newcommand{\Mod}[1]{\ (\mathrm{mod}\ #1)}



\newlist{legal}{enumerate}{10}
\setlist[legal]{label=(\alph*)}

\DeclarePairedDelimiter\bra{\langle}{\rvert}
\DeclarePairedDelimiter\ket{\lvert}{\rangle}
\DeclarePairedDelimiterX\braket[2]{\langle}{\rangle}{#1\delimsize\vert #2}


\newenvironment{theorem}[2][Theorem]{\begin{trivlist} \item[\hskip \labelsep {\bfseries #1}\hskip \labelsep {\bfseries #2.}]}{\end{trivlist}}
\newenvironment{lemma}[2][Lemma]{\begin{trivlist} \item[\hskip \labelsep {\bfseries #1}\hskip \labelsep {\bfseries #2.}]}{\end{trivlist}}
\newenvironment{result}[2][Result]{\begin{trivlist} \item[\hskip \labelsep {\bfseries #1}\hskip \labelsep {\bfseries #2.}]}{\end{trivlist}}
\newenvironment{exercise}[2][Exercise]{\begin{trivlist} \item[\hskip \labelsep {\bfseries #1}\hskip \labelsep {\bfseries #2.}]}{\end{trivlist}}
\newenvironment{problem}[2][Problem]{\begin{trivlist} \item[\hskip \labelsep {\bfseries #1}\hskip \labelsep {\bfseries #2.}]}{\end{trivlist}}
\newenvironment{question}[2][Question]{\begin{trivlist} \item[\hskip \labelsep {\bfseries #1}\hskip \labelsep {\bfseries #2.}]}{\end{trivlist}}
\newenvironment{corollary}[2][Corollary]{\begin{trivlist} \item[\hskip \labelsep {\bfseries #1}\hskip \labelsep {\bfseries #2.}]}{\end{trivlist}}
\newenvironment{solution}[2][Solution]{\begin{trivlist} \item[\hskip \labelsep {\bfseries #1}\hskip \labelsep {\bfseries #2.}]}{\end{trivlist}}

\setlength\epigraphwidth{8cm}
\setlength\epigraphrule{0pt}

\makeatletter
\patchcmd{\epigraph}{\@epitext{#1}}{\itshape\@epitext{#1}}{}{}
\makeatother

\begin{document}
  
 \title{Section 8.6: The Integers Modulo n}
  \author{Juan Patricio Carrizales Torres}
   \date{July 4, 2022}
     \maketitle
	
     We know that for any positive integer $n\in \N$, the relation $R$ defined on $\Z$ by $a\; R\; b$ if $a\equiv b \Mod n$ is an equivalence relation that results in the distinct equivalence classes $[0],[1],\dots,[n-1]$. Then, we can define some class that contains these equivalences classes, namely, $\Z_{n} = \left\{ [0],[1],\dots,[n-1] \right\}$, where $\Z_{n}$ is known as \textbf{integers modulo n}. Although, some may refer to it as the set of \textbf{residue classes}.
     Furthermore, one can define some type of addition and multiplication on $\Z_{n}$ as follows:
     \begin{equation*}
       [a]+[b] = [a+b] \quad [a]\cdot[b] = [ab],
     \end{equation*}
	for any $[a],[b]\in \Z_{n}$. Since the elements of $\Z_{n}$ are equivalence classes (partitions of $\Z$), it follows that both $a+b\in [c]$ and $ab\in [d]$ for some $[c],[d]\in\Z_{n}$, which implies that $[a+b]=[c]$ and $[ab]=[d]$. Hence, this addition and multiplication are \textit{operations} in $\Z_{n}$, which means that both the sum and product of two equivalence classes are also equivalence classes. In fact, these operations are \textit{well-defined} and so the sum and product of two equivalence classes do not depend on the representative integers. More precisely, if $[a]=[b]$ and $[c]=[d]$, then $[a+c] = [b+d]$ and $[ac]=[bd]$.
	This operations have the familiar properties of addition and product on $\Z$, namely, 
	\begin{enumerate}[label=(\alph*)]
	  \item Commutative Property\\
	  $[a]+[b]=[b]+[a]$ and $[a]\cdot[b] = [b]\cdot[a]$ for all $a,b\in \Z$\\

	\item Associative Property\\
	  $([a]+[b])+[c] = [a] + ([b]+[c])$ and $([a]\cdot[b])\cdot[c] = [a]\cdot([b]\cdot[c])$ for all $a,b,c\in \Z$\\

	\item Distributive Property\\
	  $[a]\cdot([b]+[c]) = [a]\cdot[b]+[a]\cdot[c]$ for all $a,b,c\in \Z$.
       \end{enumerate}
   \begin{problem}{57}
     Let $S=\Z$ and $T=\left\{ 4k:k\in\Z \right\}$. Thus $T$ is a nonempty subset of $S$.
     \begin{enumerate}[label=(\alph*)]
       \item Prove that $T$ is closed under addition and multiplication.
	 \begin{proof}
	   Let $a,b\in T$. Then, $a=4m$ and $b=4n$ for some $n,m\in \Z$. Then, $a+b = 4m+4n=4(n+m)$ and $ab = 16nm= 4(4nm)$. Since both $n+m$ and $4nm$ are integers, it follows that $a+b,ab\in T$. Hence, $T$ is closed under addition and multiplication.
	 \end{proof}
	\item If $a\in S-T$ and $b\in T$, is $ab\in T$?
	  \begin{solution}{(b)}
	    Yes. Since multiplying the integer divisible by four $b=4m$ by the integer $a$, one gets the integer divisible by four $ab=4(ma)$ which is an element of $T$.
	  \end{solution}
	\item If $a\in S-T$ and $b\in T$, is $a+b\in T$?
	  \begin{solution}{(c)}
	    No. Since $a\in S-T$, it follows that $a=4k+m$ where $k\in\Z$ and $m\in{1,2,3}$. Hence, $a+b=4l+m$, where $l\in \Z$, is not divisible by $4$ and so it is not an element of $T$.
	  \end{solution}
	\item If $a,b\in S-T$, is it possible that $ab \in T$?
	  \begin{solution}{(d)}
	    Yes. Let $a=4n+2$ and $b=4m+2$ for integers $n,m$. Hence, $a,b\in S-T$. However, $ab = 16mn + 8n + 8m +4 = 4(4mn+2m+2n+1)$ which is divisible by 4. Thus, $ab\in T$. 
	  \end{solution}
	 \item If $a,b\in S-T$, is it possible that $a+b\in T$?
	   \begin{solution}{(e)}
	     Yes. Let $a=4n+2$ and $b=4m+2$ for integers $n,m$. Hence, $a,b\in S-T$. However, $a+b = 4n+4m+4 = 4(m+n+1)$ which is divisible by 4. Thus, $a+b\in T$.
	   \end{solution}
We can conclude that $S-T$ is not closed under addition and multiplication.
     \end{enumerate}
   \end{problem}

   \begin{problem}{58}
     Prove that the multiplication in $\Z_{n}$, $n\geq 2$, defined by $[a][b]=[ab]$ is well-defined.
     \begin{proof}
       Consider the equivalence classes $[a]=[b]$ and $[c]=[d]$ in $\Z_{n}$ where $a,b,c,d\in \Z$. Then, $a\equiv b \Mod n$ and $c\equiv d \Mod n$. By \textbf{theorem 4.11}, $ac \equiv bd \Mod n$ and so $[ac]=[bd]$.
     \end{proof}
   \end{problem}
    
\end{document}


