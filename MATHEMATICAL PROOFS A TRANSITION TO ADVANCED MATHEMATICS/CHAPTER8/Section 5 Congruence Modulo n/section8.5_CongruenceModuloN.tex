\documentclass[12pt]{article}
\usepackage[margin=1in]{geometry}
\usepackage{amsmath, amsfonts,amsthm,amssymb,epigraph,etoolbox,mathtools,setspace,enumitem}  
\usepackage{tikz}
\usetikzlibrary{datavisualization} 
\usepackage[makeroom]{cancel} 
\usepackage[linguistics]{forest}
\usetikzlibrary{patterns}
\newcommand{\N}{\mathbb{N}}
\newcommand{\Z}{\mathbb{Z}}
\newcommand{\R}{\mathbb{R}}
\newcommand{\Q}{\mathbb{Q}}
\newcommand{\Mod}[1]{\ (\mathrm{mod}\ #1)}



\newlist{legal}{enumerate}{10}
\setlist[legal]{label=(\alph*)}

\DeclarePairedDelimiter\bra{\langle}{\rvert}
\DeclarePairedDelimiter\ket{\lvert}{\rangle}
\DeclarePairedDelimiterX\braket[2]{\langle}{\rangle}{#1\delimsize\vert #2}


\newenvironment{theorem}[2][Theorem]{\begin{trivlist} \item[\hskip \labelsep {\bfseries #1}\hskip \labelsep {\bfseries #2.}]}{\end{trivlist}}
\newenvironment{lemma}[2][Lemma]{\begin{trivlist} \item[\hskip \labelsep {\bfseries #1}\hskip \labelsep {\bfseries #2.}]}{\end{trivlist}}
\newenvironment{result}[2][Result]{\begin{trivlist} \item[\hskip \labelsep {\bfseries #1}\hskip \labelsep {\bfseries #2.}]}{\end{trivlist}}
\newenvironment{exercise}[2][Exercise]{\begin{trivlist} \item[\hskip \labelsep {\bfseries #1}\hskip \labelsep {\bfseries #2.}]}{\end{trivlist}}
\newenvironment{problem}[2][Problem]{\begin{trivlist} \item[\hskip \labelsep {\bfseries #1}\hskip \labelsep {\bfseries #2.}]}{\end{trivlist}}
\newenvironment{question}[2][Question]{\begin{trivlist} \item[\hskip \labelsep {\bfseries #1}\hskip \labelsep {\bfseries #2.}]}{\end{trivlist}}
\newenvironment{corollary}[2][Corollary]{\begin{trivlist} \item[\hskip \labelsep {\bfseries #1}\hskip \labelsep {\bfseries #2.}]}{\end{trivlist}}
\newenvironment{solution}[2][Solution]{\begin{trivlist} \item[\hskip \labelsep {\bfseries #1}\hskip \labelsep {\bfseries #2.}]}{\end{trivlist}}

\setlength\epigraphwidth{8cm}
\setlength\epigraphrule{0pt}

\makeatletter
\patchcmd{\epigraph}{\@epitext{#1}}{\itshape\@epitext{#1}}{}{}
\makeatother

\begin{document}
  
 \title{Section 8.5: Congruence Modulo n}
 \author{Juan Patricio Carrizales Torres}
\date{May 30, 2022}
\maketitle

This chapter discusses the previously seen topic of \textbf{Congruence Modulo n}, but now with the lens of
\textbf{Equivalence relations}. Basically, the author proved that every relation on $\Z$
defined by the congruence modulo of some $n\geq 2$ is an equivalence relation with $n$ equivalence
classes. This follows from the \textbf{Division Algorithm}, namely in the case for $n\geq 2$, any integer $m$ can be expressed uniquely 
as $m=kn+r$, where $k\in \Z$ and $0\leq r < n$.\\

Another interesting idea is the logical equivalence between coditions that define equivalence relations.
For example, let $R_{1}$ and $R_{2}$ be relations on some nonempty set defined by $a\; R_{1} \;b$ if $P(a,b)$ and $a\; R_{2} \;b$ if $Q(a,b)$. The fact that 
$P(a,b) \iff Q(a,b)$ for some other condition $Q(n)$, implies that $R_{1} = R_{2}$. 
Hence, one can show that two relations have the same distinct equivalence classes by just showing that there is a 
biconditional relation between the conditions that define them.

\begin{problem}{47}
  The relation $R$ on $\Z$ defined by $a \; R \; b$ if $a^{2} \equiv b^{2} \Mod{4}$ is known to be
  an equivalence relation. Determine the distinct equivalence classes.
  \begin{solution}{47}
    Let's first consider $[0]$. We know that
    \begin{align*}
      [0] &= \left(x\in\Z:x\; R\;0 \right)\\
      &= \left(x\in\Z: x^{2} = 4k,\; k\in \Z\right)\\
      &= \left(x\in\Z: 4\mid x^{2}\right) = \left(x\in\Z: 2\mid x^{2}\right)\\
      &= \left(x\in\Z:2\mid x\right). 
    \end{align*}
    Hence, $[0]$ is the set of all even integers. Now we are left with the odd ones, so let's check what are the elements of $[1]$. We know that
    \begin{align*}
       [1] &= \left(x\in\Z:x\; R\;1 \right)\\
       &= \left(x\in\Z: 4\mid (x^{2}-1)\right)\\
    \end{align*}
    We know that $x^{2}$ is either even or odd. If it is even, then $x^{2}-1$ is odd (sum of an even and odd integer) which contradicts the assumption that it is a multiple of 4. Hence, we may assume that $x^{2}$ is odd. Recall that $x^{2}$ is odd if and only if $x$ is odd and so $x=2k+1$ for some $k\in\Z$. Hence, 
    \begin{align*}
      x^{2}-1 &= (2k+1)^{2}-1 \\
      &= 4k^{2} + 4k +1 -1 = 4(k^{2}+k).
    \end{align*}
    Since $k^{2}+k$ is an integer, it follows that $4\mid(x^{2}-1)$. Hence, $x$ being odd is a necessary and sufficient condition for $4\mid(x^{2}-1)$ to be true, and so $[1]$ is the set of odd integers.
  \end{solution}
\end{problem}

\begin{problem}{48}
  The relation $R$ defined on $\Z$ by $x \; R \; y$ if $x^{3} \equiv y^{3} \Mod 4$ is known to be an equivalence relation . Determine the distinct equivalence classes.
  \begin{solution}{48}
   Let's first consider the equivalence class [0]. Then
   \begin{align*}
     [0] &= \{x\in \Z:x\; R \;0\}\\
     &= \{x\in \Z:4\mid x^{3}\}.
   \end{align*}
   Consider some $x\in [0]$. We know that either $x$ is odd or even. If it is odd, then $x^{3}$ is odd which contradicts our assumption that $4\mid x^{3}$. Hence, $x=2k$ for some $k\in \Z$ and so $x^{3} = 8k^{3} = 4\left( 2k^{3} \right)$. Since $2k^{3} \in \Z$, it follows that $x$ being even is a necessary and sufficient condition for $4\mid x^{3}$ to be true. Thus, [0] is the set of even integers.\\

   Now, we are left with the odd integers. Consider the equivalence class $[1]$. Then
   \begin{align*}
     [1] &= \left\{x\in \Z: x \; R \; 1 \right\}\\
     &= \left\{ x\in\Z: 4\mid\left(x^{3}-1 \right) \right\}.
   \end{align*}
   Let $x\in[1]$. Then $x$ must be odd because $[0]$ contains all even integers. Thus, $x=2k+1$ for some $k\in\Z$ and so $x^{3} = 8k^{3} + 6k + 12k^{2} + 1$. Then, $x^{3} -1 = 8k^{3} + 6k + 12k^{2}$. Note that $4\mid (3(2k))$ if and only if $2\mid k$. Hence, $4\mid \left( x^{3} -1 \right)$ if and only if $x=2k+1$ for some even integer $k$. \\

   Now, we are left with the set of odd integers $2k+1$ where $k$ is an odd integer. Consider the equivalence class $[3]$. Then,
   \begin{align*}
     [3] &= \left\{x\in \Z: x \; R \; 3 \right\}\\
     &= \left\{ x\in\Z: 4\mid\left(x^{3}-27 \right) \right\}.
   \end{align*}
   Let $x\in[3]$. Then $x=2k+1$ for some odd integer $k=2b+1$, where $b\in\Z$. Thus, $x^{3}-27= (8k^{3}+12k^{2}+6k+1)-27 = 8k^{3}+12k^{2}+12b-20 = 4(2k^{3}+3k^{2}+3b-5)$. Because $2k^{3}+3k^{2}+3b-5$ is an integer, it follows that $4\mid\left(x^{3}-27 \right)$ if and only if $x=2k+1$, where $k$ is an odd integer. Therefore, the distinct equivalence classes are as follows:
   \begin{align*}
     [0] &= \left\{ x\in \Z: x \text{ is even}\right\}\\
     [1] &= \left\{ x\in \Z: x=2k+1, \text{ where }k\text{ is even} \right\}\\
     [3] &= \left\{ x\in\Z: x=2k+1, \text{ where }k\text{ is odd} \right\}.
   \end{align*}
  \end{solution}
\end{problem}

\begin{problem}{49}
  A relation $R$ is defined on $\Z$ by $a\; R\; b$ if $5a \equiv 2b \Mod 3$. Prove that $R$ is an equivalence relation. Determine the distinct equivalence classes.
  \begin{solution}{49}
    We first show that $R$ is an equivalence relation. 
    \begin{proof}
      We first show that $R$ is reflexive. Consider some integer $x$. Then, $5x-2x = 3x$. Hence, $5x \equiv 2x \Mod 3$. Consider some integers $x,y$ such that $5x \equiv 2y \Mod 3$. Then, $5x-2y =3k$ for some integer $k$. Note that
      \begin{align*}
	5y-2x &= (3k-5x+7y)+(3k+2y-7x) \\
	&= 6k-12x+9y = 3(2k-4x+3y).
      \end{align*}
     Since $2k-4x+3y \in \Z$, it follows that $5y \equiv 2x \Mod 3$ and so $R$ is symmetric. Now, consider some integers $x,y,z$ such that $5x \equiv 2y \Mod 3$ and $5y \equiv 2z \Mod 3$, namely, $x\; R \;y$ and $y\; R \;z$. Then $5x-2y = 3a$ and $5y-2z = 3b$ for some integers $a,b$. Note that
     \begin{align*}
       5x-2z &= (5x-2y)+(5y-2z)-3y\\
       &= 3a+3b-3y = 3(a+b-y).
     \end{align*}
     Because $a+b-y$ is an integer, it follows that $5x \equiv 2z \Mod 3$ ($x\; R \;z$) and so $R$ is transitive.
    \end{proof}

    Now, all we have left to do is determine the equivalence classes. We initially consider [0]. Then
    \begin{align*}
      [0] &= \left\{ x\in \Z:3\mid 5x \right\}\\
      &= \left\{ x\in \Z: 3\mid x \right\}
    \end{align*}
    since 5 is prime. Now, we are left with the set of every integer $x$ such that either $x \equiv 1 \Mod 3$ or $x \equiv 2 \Mod 3$. Thus, let's check $[1]$. Then
    \begin{align*}
      [1] &= \left\{x\in \Z:3\mid(5x-2)\right\}.
    \end{align*}
    Consider some $x\in [1]$. If $x \equiv 2\Mod 3$, then $x=3a+2$, where $a\in\Z$, and so $5(3a+2)-2=15a+8=15k+6+2$ which contradicts our assumption that $3\mid(5x-2)$. Hence, $x \equiv 1\Mod 3$ and so $x=3b+1$ for some integer $b$. Therefore, $5(3b+1)-2=15a+3 = 3(5a+1)$ which implies that $3\mid (5x-2)$. Thus, $x\equiv 1\Mod 3$ is a necessary and sufficient condition for $3\mid(5x-2)$ to be true and so
     \begin{align*}
      [1] &= \left\{x\in \Z:x\equiv 1\Mod 3\right\}.
    \end{align*}
    Since we are left with the set of every integer $x$ such that $x\equiv 2\Mod 3$, it follows that it makes sense to consider $[2]$. We know that
    \begin{align*}
      [2] &= \left\{ x\in \Z:3\mid(5x-4) \right\}.
    \end{align*}
    Note that $x\in [2]$ implies that $x\equiv 2\Mod 3$ due to the partition nature of equivalence classes. If $x\equiv 2\Mod 3$, then $x=3k+2$ for some $k\in\Z$ and so $5x-4 = 5(3k+2)-4= 15k+10-4= 15k+6$, which implies that $3\mid (5x-4)$ and $x\in [2]$. Hence, 
    \begin{align*}
      [2] &= \left\{ x\in \Z:x\equiv 2 \Mod3 \right\}.
    \end{align*}
    Therefore, the equivalence classes are as follows:
    \begin{align*}
      [0] &= \left\{ x\in \Z:x\equiv 0\Mod 3 \right\}\\
      [1] &= \left\{x\in \Z:x\equiv 1\Mod 3\right\}\\
      [2] &= \left\{ x\in \Z:x\equiv 2 \Mod3 \right\}.
    \end{align*}
  \end{solution}
\end{problem}

\end{document}


