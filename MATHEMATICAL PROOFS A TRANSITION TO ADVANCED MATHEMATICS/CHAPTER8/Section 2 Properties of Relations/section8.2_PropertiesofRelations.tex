\documentclass[12pt]{article}
\usepackage[margin=1in]{geometry}
\usepackage{amsmath, amsfonts,amsthm,amssymb,epigraph,etoolbox,mathtools,setspace,enumitem}  
\usepackage{tikz}
\usetikzlibrary{datavisualization}
\usepackage[makeroom]{cancel} 
\usepackage[linguistics]{forest}
\usetikzlibrary{patterns}
\newcommand{\N}{\mathbb{N}}
\newcommand{\Z}{\mathbb{Z}}
\newcommand{\R}{\mathbb{R}}
\newcommand{\Q}{\mathbb{Q}}
\newcommand{\Mod}[1]{\ (\mathrm{mod}\ #1)}



\newlist{legal}{enumerate}{10}
\setlist[legal]{label*=\arabic*.}

\DeclarePairedDelimiter\bra{\langle}{\rvert}
\DeclarePairedDelimiter\ket{\lvert}{\rangle}
\DeclarePairedDelimiterX\braket[2]{\langle}{\rangle}{#1\delimsize\vert #2}


\newenvironment{theorem}[2][Theorem]{\begin{trivlist}
		\item[\hskip \labelsep {\bfseries #1}\hskip \labelsep {\bfseries #2.}]}{\end{trivlist}}
\newenvironment{lemma}[2][Lemma]{\begin{trivlist}
		\item[\hskip \labelsep {\bfseries #1}\hskip \labelsep {\bfseries #2.}]}{\end{trivlist}}
\newenvironment{result}[2][Result]{\begin{trivlist}
		\item[\hskip \labelsep {\bfseries #1}\hskip \labelsep {\bfseries #2.}]}{\end{trivlist}}
\newenvironment{exercise}[2][Exercise]{\begin{trivlist}
		\item[\hskip \labelsep {\bfseries #1}\hskip \labelsep {\bfseries #2.}]}{\end{trivlist}}
\newenvironment{problem}[2][Problem]{\begin{trivlist}
		\item[\hskip \labelsep {\bfseries #1}\hskip \labelsep {\bfseries #2.}]}{\end{trivlist}}
\newenvironment{question}[2][Question]{\begin{trivlist}
		\item[\hskip \labelsep {\bfseries #1}\hskip \labelsep {\bfseries #2.}]}{\end{trivlist}}
\newenvironment{corollary}[2][Corollary]{\begin{trivlist}
		\item[\hskip \labelsep {\bfseries #1}\hskip \labelsep {\bfseries #2.}]}{\end{trivlist}}
\newenvironment{solution}[2][Solution]{\begin{trivlist}
		\item[\hskip \labelsep {\bfseries #1}\hskip \labelsep {\bfseries #2.}]}{\end{trivlist}}

\setlength\epigraphwidth{8cm}
\setlength\epigraphrule{0pt}

\makeatletter
\patchcmd{\epigraph}{\@epitext{#1}}{\itshape\@epitext{#1}}{}{}
\makeatother


\begin{document} 
	
	\title{Section 8.2: Properties of relations}
	\author{Juan Patricio Carrizales Torres}
	\date{May 13, 2022}
	\maketitle

	This chapter mentioned three properties of interested for some relation $R$ on a single set $A$. Since most of these properties involve implications with universal quantifiers, the easiest way to check wether a relation has certain property is by looking for specific examples for which the implication in question is false.
	\begin{enumerate}[label=(\alph*)]
		\item \textbf{Reflexive Property:} if $x\in A$, then $(x,x)\in R$. ($x$ is related to itself)
		\item \textbf{Symmetric Property:} $\forall x,y\in A$, if $x\; R \; y$, then $y\; R \; x$ ($x$ is related to $y$ and viceversa). Note that for the relation $R$ to not be symmetric, it must be true that $x \; R \; y$ and $y \; \cancel R \; x$. For this to happen, it is necessary that $x\neq y$. 
		\item \textbf{Transitive Property:} $\forall x,y,z\in A$, if $x\; R \; y$ and  $y \; R \; z$, then $x\; R \;z$. Note that for the relation $R$ to not be symmetric, it must be true that $x \; R \; y$, $y \; R \; z$ and $x \; \cancel R \; z$. For this to happen, it is necessary that $x\neq y$ and $z\neq y$.
	\end{enumerate} 
 
\end{document}