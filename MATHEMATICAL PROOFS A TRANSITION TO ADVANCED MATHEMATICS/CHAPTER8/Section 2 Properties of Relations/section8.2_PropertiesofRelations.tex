\documentclass[12pt]{article}
\usepackage[margin=1in]{geometry}
\usepackage{amsmath, amsfonts,amsthm,amssymb,epigraph,etoolbox,mathtools,setspace,enumitem}  
\usepackage{tikz}
\usetikzlibrary{datavisualization} 
\usepackage[makeroom]{cancel} 
\usepackage[linguistics]{forest}
\usetikzlibrary{patterns}
\newcommand{\N}{\mathbb{N}}
\newcommand{\Z}{\mathbb{Z}}
\newcommand{\R}{\mathbb{R}}
\newcommand{\Q}{\mathbb{Q}}
\newcommand{\Mod}[1]{\ (\mathrm{mod}\ #1)}

 

\newlist{legal}{enumerate}{10}
\setlist[legal]{label*=\arabic*.}

\DeclarePairedDelimiter\bra{\langle}{\rvert}
\DeclarePairedDelimiter\ket{\lvert}{\rangle}
\DeclarePairedDelimiterX\braket[2]{\langle}{\rangle}{#1\delimsize\vert #2}


\newenvironment{theorem}[2][Theorem]{\begin{trivlist}
		\item[\hskip \labelsep {\bfseries #1}\hskip \labelsep {\bfseries #2.}]}{\end{trivlist}}
\newenvironment{lemma}[2][Lemma]{\begin{trivlist}
		\item[\hskip \labelsep {\bfseries #1}\hskip \labelsep {\bfseries #2.}]}{\end{trivlist}}
\newenvironment{result}[2][Result]{\begin{trivlist}
		\item[\hskip \labelsep {\bfseries #1}\hskip \labelsep {\bfseries #2.}]}{\end{trivlist}}
\newenvironment{exercise}[2][Exercise]{\begin{trivlist}
		\item[\hskip \labelsep {\bfseries #1}\hskip \labelsep {\bfseries #2.}]}{\end{trivlist}}
\newenvironment{problem}[2][Problem]{\begin{trivlist}
		\item[\hskip \labelsep {\bfseries #1}\hskip \labelsep {\bfseries #2.}]}{\end{trivlist}}
\newenvironment{question}[2][Question]{\begin{trivlist}
		\item[\hskip \labelsep {\bfseries #1}\hskip \labelsep {\bfseries #2.}]}{\end{trivlist}}
\newenvironment{corollary}[2][Corollary]{\begin{trivlist}
		\item[\hskip \labelsep {\bfseries #1}\hskip \labelsep {\bfseries #2.}]}{\end{trivlist}}
\newenvironment{solution}[2][Solution]{\begin{trivlist}
		\item[\hskip \labelsep {\bfseries #1}\hskip \labelsep {\bfseries #2.}]}{\end{trivlist}}

\setlength\epigraphwidth{8cm}
\setlength\epigraphrule{0pt}

\makeatletter
\patchcmd{\epigraph}{\@epitext{#1}}{\itshape\@epitext{#1}}{}{}
\makeatother


\begin{document} 
	
	\title{Section 8.2: Properties of relations}
	\author{Juan Patricio Carrizales Torres}
	\date{May 13, 2022}
	\maketitle

	This chapter mentioned three properties of interested for some relation $R$ on a single set $A$. Since most of these properties involve implications with universal quantifiers, the easiest way to check wether a relation has certain property is by looking for specific examples for which the implication in question is false.
	\begin{enumerate}[label=(\alph*)]
		\item \textbf{Reflexive Property:} if $x\in A$, then $(x,x)\in R$. ($x$ is related to itself)
		\item \textbf{Symmetric Property:} $\forall x,y\in A$, if $x\; R \; y$, then $y\; R \; x$ ($x$ is related to $y$ and viceversa). Note that for the relation $R$ to not be symmetric, it must be true that $x \; R \; y$ and $y \; \cancel R \; x$. For this to happen, it is necessary that $x\neq y$. 
		\item \textbf{Transitive Property:} $\forall x,y,z\in A$, if $x\; R \; y$ and  $y \; R \; z$, then $x\; R \;z$. Note that for the relation $R$ to not be symmetric, it must be true that $x \; R \; y$, $y \; R \; z$ and $x \; \cancel R \; z$. For this to happen, it is necessary that $x\neq y$ and $z\neq y$.
	\end{enumerate} 
 
 	\begin{problem}{11}
 		Let $A=\{a,b,c,d\}$ and let 	
 		\begin{equation*}
 			R=\{(a,a),(a,b),(a,c),(a,d),(b,b),(b,c),(b,d),(c,c),(c,d),(d,d)\}
 		\end{equation*}
 		
 		be a relation on $A$. Which of the properties reflexive, symmetric and transitive does the relation $R$ possess? Justify your answers.
 		\begin{solution}{11}
 			The relation is reflexive since $\{(a,a),(b,b),(c,c),(d,d)\}\subset R$. Also, it is transitive since $(x,y),(y,z)\in R \implies (x,z)\in R$ for any $x,y,z\in A$ is fulfilled. However, the relation is not symmetric since $(a,b)\in R$ and $(b,a)\not\in R$.
 		\end{solution}
 	\end{problem}
 
 	\begin{problem}{13}
 		Let $S=\{a,b,c\}$. Then $R=\{(a,b)\}$ is a relation on $S$. Which of the properties reflexive, symmetric and transitive does the relation $R$ possess? Justify your answers.
 		\begin{solution}{13}
 			The relation $S$ is transitive since the implication $(x,y),(y,z)\in R \implies (x,z)\in R$ for any $x,y,z\in S$ is fulfilled vacuously. However, it is neither reflexive because $(a,a)\not\in R$ nor symmetrice since $(a,b)\in R$ but $(b,a)\not\in R$.
 		\end{solution}
 	\end{problem}
 
 	\begin{problem}{14}
 		Let $A=\{a,b,c,d\}$. Give an example (with justification) of a relation $R$ on $A$ that has none of the following properties: reflexive, symmetric, transitive.
 		\begin{solution}{14}
 			Let $R=\{(a,b),(b,c)\}$. The relation $R$ is not reflexive since $(a,a)\not\in R$, it is not symmetric because $(a,b) \in R$ and $(b,a)\not\in R$ and it is not transitive since $(a,b),(b,c)\in R$ but $(a,c)\not\in R$.
 		\end{solution}
 	\end{problem}
 
 	\begin{problem}{15}
 		A relation $R$ is defined on $\Z$ by $a \; R \; b$ if $|a-b|\leq 2$. Which of the properties reflexive, symmetric and transitive does the relation $R$ possess? Justify your answers.
 		\begin{solution}{15}
 			The relation $R$ is reflexive since $|a-a| = 0\leq 2$ for any $a\in \Z$ and so $a \; R \; a$. It is symmetric since for any $a,b\in \Z$, if $|a-b| \leq 2$, then $|b-a|=|a-b|\leq 2$. However, it is not transitive since $|3-1|=2$ and $|1-0|=1$ but $|3-0| = 3>2$.
 		\end{solution}
 	\end{problem} 
 
 	\begin{problem}{16}
 		Let $A=\{a,b,c,d\}$. How many relations defined on $A$ are reflexive, symmetric and transitive and contain the ordered pairs $(a,b),(b,c),(c,d)$?
 		\begin{solution}{16}
 			In order for a relation $R$ on $A$ to be reflexive it must be true that \\$\{(a,a),(b,b),(c,c),(d,d)\}\subseteq R$. Since $(a,b),(b,c),(c,d)\in R$, it follows that $(b,a),(c,b),(d,c)\in R$ so that $R$ is symmetric. Because, so far
 			\begin{equation*}
 				\{(a,a),(b,b),(c,c),(d,d),(a,b),(b,c),(c,d),(b,a),(c,b),(d,c)\}\subseteq R
 			\end{equation*}
 		, it follows that $(a,c),(c,a), (b,d)\in R$ for $R$ to be transitive. Since $(b,d)\in R$, it follows that $(d,b)\in R$ so that the symmetric property is mantained. However, $(d,b),(b,a)\in R$ and so $(d,a)\in R$ so that it is transitive. This implies $(a,d)\in R$ since $R$ must be symmetric.
 		Hence, 
 		\begin{align*}
 			R &= \{(a,a),(b,b),(c,c),(d,d),(a,b),(b,c),(c,d),(b,a),(c,b),(d,c),(a,c),(c,a), (b,d), (d,b), (d,a), (a,d)\}\\
 			&= A\times A
 		\end{align*}
 		Since $R\subseteq A\times A$, it follows that there is only one possible relation $R$ on $A$ that fulfills the conditions.
 		\end{solution}
 	\end{problem}
 
 	\begin{problem}{18}
 		Let $A=\{1,2,3,4\}$. Give an example of a relation on $A$ that is:
 		\begin{enumerate}[label=(\alph*)]
 			\item reflexive and symmetric but not transitive.
 			\begin{solution}{(a)}
 				$R=\{(1,1),(2,2),(3,3),(4,4),(2,3),(3,2),(3,1),(1,3)\}$
 			\end{solution}
 			\item reflexive and transitive but not symmetric.
 			\begin{solution}{(b)}
 				$R=\{(a,a),(b,b),(c,c),(d,d),(b,c)\}$
 			\end{solution}
 			\item symmetric and transitive but not reflexive.
 			\begin{solution}{(c)}
 				$R=\emptyset$ (the symmetric and transitive logical implications are vacuously true)
 			\end{solution}
 			\item reflexive but neither symmetric nor transitive.
 			\begin{solution}{(d)}
 				$R=\{(a,a),(b,b),(c,c),(d,d),(a,b),(b,c)\}$
 			\end{solution}
 			\item symmetric but neither reflexive nor transitive.
 			\begin{solution}{(e)}
 				$R=\{(a,b),(b,a)\}$
 			\end{solution}
 			\item transitive but neither reflexive nor symmetric.
 			\begin{solution}{(f)}
 				$R=\{(a,b)\}$ (The transitive implication follows vacuously)
 			\end{solution}
 		
 		All of these are counterexamples to the statement that one property implies the other for any relation $R$ on some nonempty set $A$.
 		\end{enumerate}
 	\end{problem}
 
 	\begin{problem}{19}
 		A relation $R$ is defined on $\Z$ by $x\; R \; y$ if $x\cdot y \geq 0$. Prove or disprove the following:
 		\begin{enumerate}[label=(\alph*)]
 			\item $R$ is reflexive.
 			\begin{proof}
 				Consider some $x\in \Z$, then $x^{2} \geq 0$ and so $x \; R \; x$. The relation $R$ is reflexive.
 			\end{proof}
 			\item $R$ is symmetric.
 			\begin{proof}
 				Consider some $x,y\in \Z$. Assume that $x\; R \; y$ which implies that $x\cdot y \geq 0$. Since multiplication on real numbers is commutative, it follows that $y\cdot x = x\cdot y \geq 0$ and so $y \; R \; x$. The relation $R$ is symmetric.
 			\end{proof}
 			\item $R$ is transitive.
 			\begin{solution}{c}
 				The relation $R$ on $\Z$ is not transitive. Note that $-3 \; R \; 0$ and $0 \; R \; 1$, but $-3 \cdot 1 =-3 < 0$ and so $-3 \; \cancel R \; 1$. 
 			\end{solution}
 		\end{enumerate}
 	\end{problem}
 
 	\begin{problem}{20}
 		Determine the maximum number of elements in a relation $R$ on a $3-element$ set such that $R$ has none of the properties reflexive, symmetric and transitive.
 		\begin{solution}{20}
 			Let $R$ be a relation on a 3-element set $B$ that has none of the properties reflexive, symmetric and transitive. Let's check the maximum number of elements $R$ can contain. Since $R\subseteq B\times B$, it follows that $|R|\leq 9$. However, since $R$ is not reflexive, it follows that $(b,b)\not\in R$ for some $b\in B$ and so $|R|\leq 8$. \\
 			 
 			Because $R$ is not symmetric, it follows that $(b,a)\in R$ and $(a,b)\not\in R$ for some different $a,b\in B$ and so $|R|\leq 7$. Also, since $R$ is not transitive, it follows that $(a,b),(b,c)\in R$ and $(a,c)\not\in R$ for some $a,b,c\in B$ such that $a\neq b$ and $b\neq c$. Thus, either $c\neq a$ or $c=a$, however note that we already got rid of those two such ordered pairs and so the maximum number of elements in $R$ is $7$.
 		\end{solution}
 	\end{problem}
 
 	\begin{problem}{22}
 		Let $S$ be the set of all polynomials of degree at most 3. An element $s(x)$ of $S$ can then be expressed as $s(x) = ax^{3} + bx^{2} +cx +d$, where $a,b,c,d\in \R$. A relation $R$ is defined on $S$ by $p(x) \; R \; q(x)$ if $p(x)$ and $q(x)$ have a real root in common. (For example, $p(x) = (x-1)^{2}$ and $q(x) = x^{2}-1$ have the root 1 in common so that $p \; R \; q$.) Determine which of the properties reflexive, symmetric, and transitive are possessed by $R$.
 		\begin{enumerate}
 			\item The relation $R$ is reflexive.
 			\begin{solution}{(a)}
 				The relation $R$ on $S$ is not reflexive. Consider $p(x) = x^{2}+1$. Therefore, $p(x)\in S$ but $p(x) \; \cancel R \; p(x)$ since $p(x)$ has no real root.
 			\end{solution}
 			\item The relation $R$ is symmetric.
 			\begin{proof}
 				Consider some $p(x),q(x)\in S$. Assume that $p(x) \; R \; q(x)$ and so $p(x)$ and $q(x)$ share some real root $c$. Therefore, $q(x)$ and $p(x)$ share the real root $c$ which implies that $q(x)\; R \; p(x)$.
 			\end{proof}
 			\item The relation $R$ is transitive.
 			\begin{solution}{(c)}
 				The relation $R$ is not transitive.
 				Let $p(x)=x^{2}-1$, $q(x)=(x-1)^{2}$ and $r(x) = (x+1)^{2}$. Hence, $p(x),q(x),r(x)\in S$. Note that $p(x)$ has real roots $-1$ and $1$, $q(x)$ has only the real root $1$ and $r(x)$ only has the real root $-1$. Then, $r(x) \; R \; p(x)$ and $p(x) \; R \; q(x)$. However, $r(x)$ and $q(x)$ do not have some real root in common and so $r(x)\; \cancel R \; q(x)$.
 			\end{solution}
 		\end{enumerate}
 	\end{problem}
 
 	\begin{problem}{23}
 		A relation $R$ is defined on $\N$ by $a\; R \; b$ if either $a\mid b$ or $b\mid a$. Determine which of the properties reflexive, symmetric and transitive are possessed by $R$.
 		\begin{solution}{23}
 			The reflexive property follows instantly, every positive integer is divisible by itself. The symmetric property follows immeaditly too since, by the condition of the relation, if $a\; R \;b$ it is assured that $b\; R \;a$.\\
 			However, this relations is not transitive (this has to do with the disjunction). Consider the positive integers $4,3$ and $1$. Then, $4\; R \; 1$ and $1\; R\; 3$ (recall that $(a,b)\in \R \iff$ either $a\mid b$ or $b\mid a$). However, $3\nmid 4$ and $4\nmid 3$ and so $4\; \cancel R \;3$. 
 		\end{solution}
 	\end{problem}
\end{document}