\documentclass[12pt]{article}
\usepackage[margin=1in]{geometry}
\usepackage{amsmath, amsfonts,amsthm,amssymb,epigraph,etoolbox,mathtools,setspace,enumitem}  
\usepackage{tikz}
\usetikzlibrary{datavisualization} 
\usepackage[makeroom]{cancel} 
\usepackage[linguistics]{forest}
\usetikzlibrary{patterns}
\newcommand{\N}{\mathbb{N}}
\newcommand{\Z}{\mathbb{Z}}
\newcommand{\R}{\mathbb{R}}
\newcommand{\Q}{\mathbb{Q}}
\newcommand{\Mod}[1]{\ (\mathrm{mod}\ #1)}



\newlist{legal}{enumerate}{10}
\setlist[legal]{label=(\alph*)}

\DeclarePairedDelimiter\bra{\langle}{\rvert}
\DeclarePairedDelimiter\ket{\lvert}{\rangle}
\DeclarePairedDelimiterX\braket[2]{\langle}{\rangle}{#1\delimsize\vert #2}


\newenvironment{theorem}[2][Theorem]{\begin{trivlist}
		\item[\hskip \labelsep {\bfseries #1}\hskip \labelsep {\bfseries #2.}]}{\end{trivlist}}
\newenvironment{lemma}[2][Lemma]{\begin{trivlist}
		\item[\hskip \labelsep {\bfseries #1}\hskip \labelsep {\bfseries #2.}]}{\end{trivlist}}
\newenvironment{result}[2][Result]{\begin{trivlist}
		\item[\hskip \labelsep {\bfseries #1}\hskip \labelsep {\bfseries #2.}]}{\end{trivlist}}
\newenvironment{exercise}[2][Exercise]{\begin{trivlist}
		\item[\hskip \labelsep {\bfseries #1}\hskip \labelsep {\bfseries #2.}]}{\end{trivlist}}
\newenvironment{problem}[2][Problem]{\begin{trivlist}
		\item[\hskip \labelsep {\bfseries #1}\hskip \labelsep {\bfseries #2.}]}{\end{trivlist}}
\newenvironment{question}[2][Question]{\begin{trivlist}
		\item[\hskip \labelsep {\bfseries #1}\hskip \labelsep {\bfseries #2.}]}{\end{trivlist}}
\newenvironment{corollary}[2][Corollary]{\begin{trivlist}
		\item[\hskip \labelsep {\bfseries #1}\hskip \labelsep {\bfseries #2.}]}{\end{trivlist}}
\newenvironment{solution}[2][Solution]{\begin{trivlist}
		\item[\hskip \labelsep {\bfseries #1}\hskip \labelsep {\bfseries #2.}]}{\end{trivlist}}

\setlength\epigraphwidth{8cm}
\setlength\epigraphrule{0pt}

\makeatletter
\patchcmd{\epigraph}{\@epitext{#1}}{\itshape\@epitext{#1}}{}{}
\makeatother


\begin{document} 
	
	\title{Section 8.3: Equivalence Relations}
	\author{Juan Patricio Carrizales Torres}
	\date{May 30, 2022}
	\maketitle
	
	This chapter reviews some properties that we realized and proved in the problems of \textbf{Section 8.3}. However, there's something worth noting. Let $R$ be some relation on some nonempty set $A$. I previously showed that the union of the equivalence classes by $R$ is $A$ and they all are pairwise disjoint. Nevertheless, I didn't ponder on it much to realize what this meant, namely, that the set of these distinct equivalence classes is a partition of $A$!!!!\\
	 This was proven by the authors by just showing that each $x\in A$ belongs to exaclty one equivalence class by $R$. 
	 
	 \begin{problem}{36}
	 	Give an example of an equivalence relation $R$ on the set $A=\{v,w,x,y,z\}$ such that there are exactly three distinct equivalence classes. What are the equivalence classes for your example?
	 	\begin{solution}{36}
	 		Consider the parition $P = \{\{v\},\{w\},\{x,y,z\}\}$ of $A$. By \textbf{Theorem 4}, the relation $R$ definded by $a\; R \; b$ if $a,b\in  X$ for some $X\in P$ is an equivalence relation. Hence, the distinct equivalance classe are
	 		\begin{align*}
	 			a_{1} &= \{x,y,z\}\\
	 			a_{2} &= \{w\}\\
	 			a_{3} &= \{v\}
	 		\end{align*}
	 	\end{solution}
	 \end{problem} 
 
 	\begin{problem}{37}
 		A relation $R$ is defined on $\N$ by $a\; R \; b$ if $a^{2} +b^{2}$ is even. Prove that $R$ is an equivalence relation. Determine the distinct equivalence classes.
 		\begin{proof}
 			We first prove that $R$ is an equivalence relation. Consider some positive integer $c$. Then, $c^{2} + c^{2} = 2c^{2}$. Since $c^{2}$ is an integer, it follows that $2c^{2}$ is even and so $c\; R \; c$. Hence, $R$ is reflexive. Let $a,b\in \N$. By the commutative property of sums on real numbers, it follows that if $a^{2} + b^{2}$ is even, then $b^{2} + a^{2}$ is equal to the same even number. Therefore, $a\; R \; b$ implies $b\; R \; a$ and so $R$ is symmetric. Consider $x,y,z\in \Z$ such that $x\; R \;y$ and $y\; R \; z$. Hence, $x^{2} + y^{2} = 2m$ and $y^{2} + z^{2} = 2n$ for $m,n\in \Z$. Thus, $x^{2} = 2m-y^{2}$ and $z^{2} = 2n-y^{2}$. Therefore,
 			\begin{align*}
 				x^{2} + z^{2} &= (2m-y^{2}) + (2n-y^{2}) \\
 				&= 2m+2n-2y^{2} = 2(m+n-y^{2}).
 			\end{align*}
 		Because $m+n-y^{2} \in \Z$, it follows that $x^{2}+z^{2}$ is even and so $x\; R \;z$, which implies that $R$ is transitive.\\
 		
 		Once $R$ is shown to be an equivalence relation, we now determine the dsitinct equivalence classes. Let $x$ be an even positive integer. Then $x^{2}$ is even. Consider some $y\in \N$. Note that $y^{2}+x^{2}$ is even if and only if $y^{2}$ is even. We also know that $y^{2}$ is even if and only if $y$ is even. Therefore,
 		\begin{equation*}
 			[x] = \{n\in \N: n \text{ is even}\}.
 		\end{equation*} 
 		Consider positive integers $y$ and $z$. If $y$ is and odd positive integer, then $z^{2}+y^{2}$ is odd if and only if $z^{2}$ is odd. Hence, $z$ must be odd.
 		\begin{equation*}
 			[y] = \{n\in \N: n\text{ is odd}\}.
 		\end{equation*}
 		Since the set of even and odd positive integers is a partition of $\N$, it follows that there only two distinct equivalence classes.
 		\end{proof} 
 	\end{problem}
	 
	 \begin{problem}{38}
	 	Let $R$ be a relation defined on the set $\N$ by $a\; R \;b$ if either $a\mid 2b$ or $b\mid 2a$. Prove or disprove: $R$ is and equivalence relation.
	 	\begin{solution}{38}
	 		The relation $R$ on $\N$ is not an equivalence relation. Consider the positive integers $2$, $3$ and $5$. Since $2\mid (2\cdot 3)$ and $2\mid (2\cdot 5)$, it follows that $3\; R \; 2$ and $2\; R \; 5$. However, $3 \nmid (2\cdot 5)$ and $5\nmid (2\cdot 3)$. Hence, $3\; \cancel R \; 5$ and so $R$ is not transitive. This implies that $R$ is not an equivalence relation.
	 	\end{solution}
	 \end{problem}
 
 	\begin{problem}{39}
 		Let $S$ be a nonempty subset of $\Z$ and let $R$ be a relation defined on $S$ by $x\; R \; y$ if $3\mid (x+2y)$.
 		\begin{enumerate}[label=(\alph*)]
 			\item Prove that $R$ is an equivalence relation.
 			\begin{proof}
 				Let $S$ be some nonempty subset of $\Z$ and $R$ some relation on $S$ defined by $x\; R \; y$ if $3\mid (x+2y)$. For some integer $x\in S$, $x+2x = 3x$ and so $3\mid 3x$. Hence, $x\; R \; x$ is reflexive.\\
 				Let $x,y\in S$ such that $x\; R \; y$. Hence, $x+2y = 3c$ for some integer $c$. Then, $x = 3c-2y$ and so
 			\begin{align*}
 				 y+2x &= y+2(3c-2y) \\
 				 &= y+6c-4y \\
 				 &= 3(2c-y).
 			 \end{align*}
 		 Since $2c-y\in \Z$, it follows that $3\mid (y+2x)$ and so $y\; R \; x$ ($R$ is symmetric). \\
 		 Consider some $x,y,z\in S$ sucht that $x\; R \; y$ and $y\; R \; z$. Therefore, $x+2y = 3a$ and $y+2z= 3b$ for $a,b\in \Z$. Then, $x= 3a-2y$ and $2z=3b-y$. Note that
 		 \begin{align*}
 		 	x+2z &= 3a-2y+3b-y\\
 		 	&= 3(a-y+b).
 		 \end{align*}
 	 Since $a-y+b\in \Z$, it follows that $3\mid (x+2z)$ and so $x\; R \; z$ ($R$ is transitive).
 			\end{proof}
 			\item If $S=\{-7,-6,-2,0,1,4,5,7\}$, then what are the distinct equivalence classes in this case?
 			\begin{solution}{(b)}
 				The distinct equivalence classes are:
 				\begin{align*}
 					A_{1} &= \{-6,0\} = [-6] = [0]\\
 					A_{2} &= \{5,-7\} = [-7] = [5]\\
 				A_{3} &= \{-2,1,4,7\} = [-2] = [1] = [4] = [7]\\
 				\end{align*}
 			\end{solution}
 		\end{enumerate}
 	\end{problem}
 
 	\begin{problem}{40}
 		A relation $R$ is defined on $\Z$ by $x\; R \;y$ if $3x-7y$ is even. Prove that $R$ is an equivalence relation. Determine the distinct equivalence classes.
 		\begin{solution}{40}
 			First, we show that $R$ is an equivalence relation.
 			\begin{proof}
 				We show that $R$ is reflexive. Consider some integer $x$. Then, $3x-7x = -4(x) = 2(-2x)$, where $-2x\in \Z$ and so it is even. Hence, $x\;R \;x$. \\
 				We prove that $R$ is symmetric. Consider two integers $x$ and $y$ such that $x\; R \; y$. Hence, $3x-7y = 2c$ for some integer $c$. Then, $3y-7x = 2c +10y-10x = 2(c+5y-5x)$. Since $c+5y-5x\in \Z$, it follows that $3y-7x$ is even and so $y\; R \;x$.\\
 				Now, consider three integers $x,y,z$ such that $x\; R \;y$ and $y\; R \; z$. Thus, $3x-7y = 2a$ and $3y-7z = 2b$ for some $a,b\in \Z$. Note that $(3x-7y) + (3y-7z) = 2a+2b$ and so $3x-7z = 2a+2b +4y = 2(a+b+y)$. Since $a+b+y\in \Z$, it follows that $3x-7y$ is even and so $x\; R \;z$.
 			\end{proof}
 		Now that it has been proven that $R$ is an equivalence relation. We proceed to determine its equivalence classes. We first determine the equivalence class for some even integer, say 0. Then
 		\begin{align*}
 			[0] &= \{x\in \Z:x\; R \; 0\}\\
 			&= \{x\in \Z: 3x-70 \text{ is even}\}\\
 			&= \{x\in \Z: 3x \text{ is even}\}\\
 			&= \{x\in \Z: x \text{ is even}\}.
 		\end{align*}
 	Now, consider some odd integer, say $1$. Then
 	\begin{align*}
 		[1] &= \{x\in \Z : x\; R \;1\}\\
 		&= \{x\in \Z : 3x-7 \text{ is even}\}\\
 		&= \{x\in \Z: 3x \text{ is odd}\}\\
 		&= \{x\in \Z: x \text{ is odd}\}.
 	\end{align*}
 Therefore, there are two distinct equivalence classes, namley, the set of even integers and the set of odd ones.
 		\end{solution} 
 	\end{problem} 
 
 	\begin{problem}{41}
 		\begin{enumerate}[label=(\alph*)]
 			\item Prove that the intersection of two equivalence relations on a nonempty set is an equivalence relation.
 			\begin{proof}
 				Let $R_{1}$ and $R_{2}$ be two equivalence relations on some nonempty set $A$. Let their intersection be the set $K$. Since both $R_{1}$ and $R_{2}$ are reflexive, it follows that if $x\in A$, then $(x,x)\in R_{1},R_{2}$, and so $(x,x)\in K$. Hence, $K$ is reflexive. Consider some $a,b\in A$ such that $a\; K \;b$ (Recall that $a\; K \;b$ is the same as saying $(a,b)\in K$). Then, $a\; R_{1} \;b$ and $a\; R_{2} \; b$. Since both are symmetric, $b\; [R_{1},R_{2}]\; a$ ($b$ is related to $a$ by both $R_{1}$ and $R_{2}$) and so $b\; K \;a$, which implies that $K$ is symmetric.\\
 				Now consider some $a,b,c\in A$ such that $a\; K \;b$ and $b\; K \;c$. Therefore, $a \; [R_{1},R_{2}]\; b$ and $b \; [R_{1},R_{2}]\; c$. Since both relations are transitive, it follows that $a\; [R_{1},R_{2}]\;c$. Therefore, $a\; K \; c$, which implies that $K$ is transitive. Thus, $K$, namely, the intersection of two equivalence relations on a nonempty set, is an equivalence relation.
 			\end{proof}
 		\vspace{0.5 cm}
 		
 		\begin{lemma}{8.4.1}
 			Let  $a,b$ be integers. $a \equiv b \Mod{2}$ and $a\equiv b \Mod{3}$ is a necessary and sufficiente condition for $a\equiv b \Mod{6}$
 			\begin{proof}
 				Assume that $a\equiv b \Mod{6}$, then $a=6(c) +b = 2(3c) + b = 3(2c)+b$ for some integer $c$. Since $2c,3c\in \Z$, it follows that $a \equiv b \Mod{3}$ and $a\equiv b \Mod{2}$. \\
 				Suppose that $a \equiv b \Mod{3}$ and $a\equiv b \Mod{2}$. Hence, $a=3x+b=2y+b$ and so $3x=2y$ for some $x,y\in \Z$. Hence, $3x$ is even and so $2 \mid x$.  Hence $3x = 3\cdot 2(c)$ for some $c\in \Z$. Therefore, $a=6(c) + b$, which implies that $a\equiv b \Mod{6}$.
 			\end{proof}
 		\end{lemma}
 			\item Consider the equivalence relations $R_{2}$ and $R_{3}$ defined on $\Z$ by $a\; R_{2} \;b$ if $a\equiv b \Mod 2$ and $a\; R_{3}\; b$ if $a\equiv b \Mod{3}$. By (a), $R_{1} = R_{2}\cap R_{3}$ is an equivalence relation on $\Z$. Determine the distinct equivalence classes in $R_{1}$.
 			\begin{solution}{b}
 				Note that $R_{1}$ is defined by $a\; R_{1}\; b$ if $a\equiv b \Mod 2$ and $a\equiv b \Mod{3}$. Since both 2 and 3 are prime, by the prevoius Lemma, $a\equiv b \Mod 2$ and $a\equiv b \Mod{3} \iff a\equiv b\Mod{6}$. Thus, 
 				\begin{align*}
 					[a] &= \{x\in \Z:x\; R\;a\}\\
 					&= \{x\in \Z: x\equiv a\Mod{6}\}\\
 					&= \{x\in \Z: x = 6m + a, m\in \Z\}.
 				\end{align*}
 			Recall that any integer can be expressed as $6c+b$ for exaclty one $(c,b)$, where $c\in \Z$ and $b\in \{0,1,2,3,4,5\}$ by the \textbf{Divition Algorithm}. Hence,
 			\begin{align*}
 				[0] &= \{6x+0:x\in \Z\}\\
 				[1] &= \{6x+1:x\in \Z\}\\
 				[2] &= \{6x+2:x\in \Z\}\\
 				[3] &= \{6x+3:x\in \Z\}\\
 				[4] &= \{6x+4:x\in \Z\}\\
 				[5] &= \{6x+5:x\in \Z\}\\
 			\end{align*}
 			\end{solution}
  		\end{enumerate}
 	\end{problem}
\end{document}