\documentclass[12pt]{article}
\usepackage[margin=1in]{geometry}
\usepackage{amsmath, amsfonts,amsthm,amssymb,epigraph,etoolbox,mathtools,setspace,enumitem}  
\usepackage{tikz}
\usetikzlibrary{datavisualization}
\usepackage[makeroom]{cancel} 
\usepackage[linguistics]{forest}
\usetikzlibrary{patterns}
\newcommand{\N}{\mathbb{N}}
\newcommand{\Z}{\mathbb{Z}}
\newcommand{\R}{\mathbb{R}}
\newcommand{\Q}{\mathbb{Q}}
\newcommand{\Mod}[1]{\ (\mathrm{mod}\ #1)}



\newlist{legal}{enumerate}{10}
\setlist[legal]{label*=\arabic*.}

\DeclarePairedDelimiter\bra{\langle}{\rvert}
\DeclarePairedDelimiter\ket{\lvert}{\rangle}
\DeclarePairedDelimiterX\braket[2]{\langle}{\rangle}{#1\delimsize\vert #2}


\newenvironment{theorem}[2][Theorem]{\begin{trivlist}
		\item[\hskip \labelsep {\bfseries #1}\hskip \labelsep {\bfseries #2.}]}{\end{trivlist}}
\newenvironment{lemma}[2][Lemma]{\begin{trivlist}
		\item[\hskip \labelsep {\bfseries #1}\hskip \labelsep {\bfseries #2.}]}{\end{trivlist}}
\newenvironment{result}[2][Result]{\begin{trivlist}
		\item[\hskip \labelsep {\bfseries #1}\hskip \labelsep {\bfseries #2.}]}{\end{trivlist}}
\newenvironment{exercise}[2][Exercise]{\begin{trivlist}
		\item[\hskip \labelsep {\bfseries #1}\hskip \labelsep {\bfseries #2.}]}{\end{trivlist}}
\newenvironment{problem}[2][Problem]{\begin{trivlist}
		\item[\hskip \labelsep {\bfseries #1}\hskip \labelsep {\bfseries #2.}]}{\end{trivlist}}
\newenvironment{question}[2][Question]{\begin{trivlist}
		\item[\hskip \labelsep {\bfseries #1}\hskip \labelsep {\bfseries #2.}]}{\end{trivlist}}
\newenvironment{corollary}[2][Corollary]{\begin{trivlist}
		\item[\hskip \labelsep {\bfseries #1}\hskip \labelsep {\bfseries #2.}]}{\end{trivlist}}
\newenvironment{solution}[2][Solution]{\begin{trivlist}
		\item[\hskip \labelsep {\bfseries #1}\hskip \labelsep {\bfseries #2.}]}{\end{trivlist}}

\setlength\epigraphwidth{8cm}
\setlength\epigraphrule{0pt}

\makeatletter
\patchcmd{\epigraph}{\@epitext{#1}}{\itshape\@epitext{#1}}{}{}
\makeatother


\begin{document} 
	
	\title{Section 8.1: Equivalence Relations}
	\author{Juan Patricio Carrizales Torres}
	\date{May 9, 2022}
	\maketitle
	
	Consider two sets $A$ and $B$. A \textbf{relation $R$ from} $A$ \textbf{to} $B$ is a subset of the cartesian product $A\times B$. Therefore, $(a,b)\in R$ means that $a\in A$ is related to $b\in B$ by $R$, namely, $a \, R \, b$. On the other hand, if $(a,b)\not\in R$, then $a\in A$ is not related to $b\in B$ by $R$, namely, $a \, \cancel R \, b$.\\
	
	 Also, if $|A\times B| = n$ for some positive integer $n$, then there are $2^{n}$ possible relations from $A$ to $B$. Note that $R$ can be seen as a function with some domain and range. Hence
	\begin{equation*}
		\text{dom}(R) = \{a\in A: (a,b)\in R \text{ for some } b\in B\}
	\end{equation*} 
	and
	\begin{equation*}
		\text{range}(R) = \{b\in B: (a,b)\in R \text{ for some } a\in B\}.
	\end{equation*}
	The relation $R$ has too some type of inverse called the \textbf{inverse relation} $R^{-1}$, which is defined as
	\begin{equation*}
		R^{-1} =\{(b,a):(a,b)\in R\}.
	\end{equation*}
	However, functions are just a subset of relations. All functions are relations, but not all relations are functions.
	Finally, a \textbf{relation on a set} $A$ is just a relation from $A$ to $A$. Therefore, it is a subset of $A\times A$.
	
	\begin{problem}{1}
		Let $A=\{a,b,c\}$ and $B=\{r,s,t,u\}$. Furthermore, let $R=\{(a,s),(a,t),(b,t)\}$ be a relation from $A$ to $B$. Determine dom$(R)$ and range$(R)$.
		\begin{solution}{1}
			The domain and range of the relation $R$ are as follows:
			\begin{align*}
				\text{dom}(R) &= \{a,b\} \quad \text{and}\\
				\text{range}(R) &= \{s,t\}
			\end{align*}
		\end{solution}
	\end{problem}

	\begin{problem}{2}
		Let $A$ be a nonempty set and $B\subseteq \mathcal{P}(A)$. Define a relation $R$ from $A$ to $B$ by $x \, R \, Y$ if $x\in Y$. Give an example of two sets $A$ and $B$ that illustrate this. What is $R$ for these two sets?
		\begin{solution}{2}
			Let $A=\{a,b,3,4\}$ and $B=\{\{a,b\},\{3,4\},\emptyset\}$, and so $B\subseteq \mathcal{P}(A)$. Hence,
			\begin{align*}
			 R&=\{(x,Y):x\in Y\}\\
			 &=\{(a,\{a,b\}),(b,\{a,b\}), (3,\{3,4\}), (4,\{3,4\}) \}
		 	\end{align*}
		\end{solution}
	\end{problem}

	\begin{problem}{3}
		Let $A=\{0,1\}$. Determine all the relations on $A$.
		\begin{solution}{3}
			Since $|A\times A|=4$, it follows that there are $|\mathcal{P}(A\times A)|=2^{4} = 16$ possible relations on $A$.  These are the following
			\begin{align*}
				R_{1} &= \emptyset\\
				R_{2} &= \{(0,0)\}\\
				R_{3} &= \{(0,1)\}\\
				R_{4} &= \{(0,0),(0,1)\}\\
				R_{5} &= \{(1,0)\}\\
				R_{6} &= \{(1,1)\}\\
				R_{7} &= \{(1,0),(1,1)\}\\
				R_{8} &= \{(0,0),(1,0)\}\\
				R_{9} &= \{(0,0),(1,1)\}\\
				R_{10} &= \{(0,1),(1,0)\}\\
				R_{11} &= \{(0,1),(1,1)\}\\
				R_{12} &= \{(0,0),(0,1), (1,0)\}\\
				R_{13} &= \{(0,0),(0,1),(1,1)\}\\
				R_{14} &= \{(1,0),(1,1),(0,0)\}\\
				R_{15} &= \{(1,0),(1,1),(0,1)\}\\
				R^{16} &= A\times A
			\end{align*}
		\end{solution}
	\end{problem}

	\begin{problem}{6}
		A relation $R$ is defined on $\N$ by $a \; R \; b$ if $a/b \in \N$. For $c,d\in \N$, under what conditions is $c \; R^{-1} \; d?$ 
		\begin{solution}{6}
			Note that $d/c\in \N \iff d\; R\;c$. Also, we know that $d \; R \; c \iff c \; R^{-1} \; d$. Hence, $d/c \in \N \iff c \; R^{-1} \; d$. 
		\end{solution}
	\end{problem} 

	From this colloraly we can derived some interesting and useful theorem:
	\begin{theorem}{inverse}
		Let $R$ be a relation with condition $P()$. Then, the condition for $R^{-1}$ is also $P()$.
		\begin{proof}
			We know that $P(x,y)\iff (x,y)\in R$ and $(x,y)\in R\iff (y,x) \in R^{-1}$. Therefore, $P(x,y) \iff (y,x)\in R^{-1}$.
		\end{proof}
	\end{theorem}

	\begin{problem}{7}
		For the relation $R=\{(x,y):x+4y \text{ is odd}\}$ defined on $\N$, what is $R^{-1}$?
		\begin{solution}{7}
			Note that $4y = 2(2y)$ is even for all $y\in \N$. Therefore, $x$ must be an odd integer in order for $x+4y$ to be odd. Using \textbf{theorem inverse}, 
			\begin{align*}
				R^{-1} &= \{(y,x):x+4y \text{ is odd}\} \\
				&= \{(y,x):x \text{ is odd}\}\\
				&= \N \times \mathbb{I},
			\end{align*}
		where $\mathbb{I}$ is the set of odd positive integers.
		\end{solution}
	\end{problem}
 
	\begin{problem}{8}
		For the relation $R=\{(x,y):x\leq y\}$ defined on $\N$, what is $R^{-1}$?
		\begin{solution}{8}
			Using \textbf{theorem inverse}.
			\begin{align*}
				R^{-1} &= \{(y,x):x\leq y\}\\
			\end{align*}
		We can even change variables:
		\begin{align*}
			R^{-1} &= \{(a,b):b\leq a\}
		\end{align*}
		\end{solution}
	\end{problem}

	\begin{problem}{9}
		Let $A$ and $B$ be sets such that $|A|=|B|=4$.
		\begin{enumerate}[label=(\alph*)]
			\item Prove or disprove: If $R$ is a relation from $A$ to $B$ where $|R| = 9$ and $R=R^{-1}$, then $A=B$.
			\begin{solution}{9}
				This statement is false. Note that $|R|=9=3!$. Let's give a counterexmaple. Let $A=\{2,4,6,9\}$ and $B=\{2,4,6,11\}$. Consider some relation $R$ from $A$ to $B$ such that 
				\begin{align*}
				 R &= \{(a,b):a\text{ and }b \text{ are even}\}\\
				 &=\{(2,2),(2,4),(2,6),(4,2),(4,4),(4,6),(6,2),(6,4),(6,6)\}
				\end{align*}
				and so
				\begin{align*}
				R^{-1} &= \{(2,2),(4,2),(6,2),(2,4),(4,4),(6,4),(2,6),(4,6),(6,6)\}.
				\end{align*}
				Hence, $|R|=9$ and $R=R^{-1}$. However, $A\neq B$.
			\end{solution}
			\item Show that by making a small change in the statement in (a), a different a different response to the resulting statement can be obtained.
			\begin{solution}{b}
				\begin{lemma}{9.b.l}
					Let $A$ and $B$ be two nonempty sets. Then, $A\times B = B\times A$ is a necessary and sufficient condition for $A=B$. 
					\begin{proof}
						Let $A=B$. Then $A\times B = A\times A = B\times A$. \\
						For the converse, assume that $A\neq B$. Then, either $A\not\subseteq B$ or $B\not\subseteq A$. Without loss of generality, let $A\not\subseteq B$ and so there is some $x\in A$ such that $x\not\in B$. Then, there is some $(x,b) \in A\times B$. Since $x\not\in B$, $(x,b) \not\in B\times A$. Hence, $A\times B \neq B\times A$.
					\end{proof}
				\end{lemma}
				\begin{theorem}{9.b}
					If $R$ is a relation from $A$ to $B$ where $|R| = 16$ and $R=R^{-1}$, then $A=B$.
					\begin{proof}
						Consider some sets $A$ and $B$ with $|A|=|B|=4$. Sine $|R| = 16$ and $|A\times B| = 4\cdot 4 = 16$, it follows, by definition of $R\subseteq A\times B$, that $R= A\times B$. Hence, $R^{-1}= \{(b,a):(a,b)\in R=A\times B\}=B\times A$. However, we assumed that $R = R^{-1}$, which implies that $A\times B = B\times A$. By \textbf{Lemma 9.b.l}, $A=B$.
					\end{proof}
				\end{theorem}
			
				However, we are interested in the minimum amount of $|R|$ to imply $A=B$. Since $3!=9$, adding another element would need for a fourth element of $A$ and $B$ to be equal.
			\end{solution}
		\end{enumerate}  
	\end{problem}

	\begin{problem}{10}
	Let $A$ be a set with $|A|=4$. What is the maximum number of elements that a relation $R$ on $A$ can contain so that $R\cap R^{-1}=\emptyset$.
	\begin{solution}{10}
		In order for $R\cap R^{-1} = \emptyset$, it must be true that $(x,y)\in R\implies (y,x)\not\in R$, namely, $x$ is related to $y$ but not viceversa for $x,y \in A$. We can see this as some type of combination where the there is no repetition of elements and the order does not matter. Hence, we are interested in how many subsets of 2 elements can we build up from the set $A$ with $|A|=4$. Let's consider some arbitrary set $A=\{a,b,c,d\}$, then the possible combinations are the following
		\begin{align*}
			(a,b), (a,c), (a,d)\\
			(b,c), (b,d)\\
			(c,d)
		\end{align*}
		Hence, the maximum number of elements that a relation $R$ on $A$ can contain so that $R\cap R^{-1}=\emptyset$ is 6.
	\end{solution} 
	\end{problem}
\end{document}