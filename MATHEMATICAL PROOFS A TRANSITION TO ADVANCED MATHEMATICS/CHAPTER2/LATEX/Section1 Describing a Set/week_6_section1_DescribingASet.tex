\documentclass[12pt]{article}
\usepackage[margin=1in]{geometry}
\usepackage{amsmath,amsthm,amssymb,epigraph,etoolbox,mathtools,setspace,enumitem}  

\newcommand{\N}{\mathbb{N}}
\newcommand{\Z}{\mathbb{Z}}

\newenvironment{theorem}[2][Theorem]{\begin{trivlist}
		\item[\hskip \labelsep {\bfseries #1}\hskip \labelsep {\bfseries #2.}]}{\end{trivlist}}
\newenvironment{lemma}[2][Lemma]{\begin{trivlist}
		\item[\hskip \labelsep {\bfseries #1}\hskip \labelsep {\bfseries #2.}]}{\end{trivlist}}
\newenvironment{exercise}[2][Exercise]{\begin{trivlist}
		\item[\hskip \labelsep {\bfseries #1}\hskip \labelsep {\bfseries #2.}]}{\end{trivlist}}
\newenvironment{problem}[2][Problem]{\begin{trivlist}
		\item[\hskip \labelsep {\bfseries #1}\hskip \labelsep {\bfseries #2.}]}{\end{trivlist}}
\newenvironment{question}[2][Question]{\begin{trivlist}
		\item[\hskip \labelsep {\bfseries #1}\hskip \labelsep {\bfseries #2.}]}{\end{trivlist}}
\newenvironment{corollary}[2][Corollary]{\begin{trivlist}
		\item[\hskip \labelsep {\bfseries #1}\hskip \labelsep {\bfseries #2.}]}{\end{trivlist}}
\newenvironment{solution}[2][Solution]{\begin{trivlist}
		\item[\hskip \labelsep {\bfseries #1}\hskip \labelsep {\bfseries #2.}]}{\end{trivlist}}

\setlength\epigraphwidth{8cm}
\setlength\epigraphrule{0pt}

\makeatletter
\patchcmd{\epigraph}{\@epitext{#1}}{\itshape\@epitext{#1}}{}{}
\makeatother


\begin{document}
	
	\title{Week 6}
	\author{Juan Patricio Carrizales Torres \\
		Section 1: Describing a Set}
	\date{August 27, 2021}
	\maketitle
	 
	\epigraph{``Now an earthquake that is commanded by two gods has double the chance of fulfilment than when it is willed by one, and an incalculably greater chance than when two gods are pulling different ways; as, to take the case of older and greater gods, when the sun and the moon pull in the same direction we have the biggest tides.``}{--- \textup{Lord Dunsany}, Chu-Bu and Sheemish}
	
	\begin{problem}{1}
		Which of the following are sets?
		\begin{enumerate}[label=\alph*]
			\item 1,2,3
			\item $\{1,2\},3$
			\item $\{\{1\},2\},3$
			\item $\{1,\{2\},3\}$
			\item $\{1,2,a,b\}$
		\end{enumerate}
		\begin{solution}{}
		Only (d) and (e) are a collection of elements between braces. They are therefore sets. The other examples are just elements not contained in a set.
		\end{solution}
	\end{problem}

	\begin{problem}{2}
		Let $S=\{-2,-1,0,1,2,3\}$. Describe each of the following sets as $\{x\in S:p(x)\}$, where $p(x)$ is some condition on $x$.\\
		
		(a) $A=\{1,2,3\}$
		\begin{solution}{a}
			$A=\{x\in S: x>0\}$
		\end{solution}
		(b) $B=\{0,1,2,3\}$
		\begin{solution}{b}
			$B=\{x\in S: x\geq 0\}$
		\end{solution}
		(c) $C=\{-2,-1\}$
		\begin{solution}{c}
			$C=\{x\in S: x < 0\}$
		\end{solution} 
		(d) $D=\{-2,2,3\}$
		\begin{solution}{d}
			$D=\{x\in S: |x|\geq 2\}$
		\end{solution}
	\end{problem}

	\begin{problem}{3}
		Deterime the cardinality of each of the following sets:\\
		
		(a) $A=\{1,2,3,4,5\}$
		\begin{solution}{a}
			$|A|=5$
		\end{solution}
	
		(b) $B=\{0,2,4,\ldots, 20\}$
		\begin{solution}{b}
			Between 1 and 20 there are $20/2$ multiples of 2 (even integers). Then we just add 1 considering the number 0, which is even.
			\begin{equation*}
				|B|=\frac{20}{2} + 1 = 11
			\end{equation*}
		\end{solution}
		(c) $C=\{25,26,27,\ldots,75\}$
		\begin{solution}{c}
			$|C|=(75-25)+1=51$
		\end{solution}
		(d) $D=\{\{1,2\},\{1,2,3,4\}\}$
		\begin{solution}{d}
			$|D|=2$
		\end{solution}
		(e) $E=\{\emptyset\}$
		\begin{solution}{e}
			$|E|=1$
		\end{solution}
		(f) $F=\{2,\{2,3,4\}\}$
		\begin{solution}{f}
			$|F|=2$
		\end{solution}
	\end{problem}

	\begin{problem}{4}
		Write each of the following sets by listing its elements within braces.\\
		
		(a) $A=\{n\in \Z:-4<n\leq 4\}$
		\begin{solution}{a}
			$A=\{-3,-2,-1,\ldots,4\}$
		\end{solution}
		(b)  $B=\{n\in\Z:n^{2}<5\}$
		\begin{solution}{b}
			$B=\{-2,-1,0,1,2\}$
		\end{solution}
		(c) $C=\{n\in \mathbb{N}:n^{3}<100\}$
		\begin{solution}{c}
			$C=\{1,2,3,4\}$
		\end{solution}
		(d) $D=\{x\in \mathbb{R}:x^{2}-x=0\}$
		\begin{solution}{d}
			$D=\{0,1\}$
		\end{solution}
		(e) $E=\{x\in \mathbb{R}:x^{2}+1=0\}$
		\begin{solution}{e}
			$E=\emptyset$
		\end{solution}
	\end{problem}

	\begin{problem}{5}
		Write each of the following sets in the form $\{x \in \Z:p(x)\}$, where $p(x)$ is a property concerning $x$.\\
		
		(a) $A=\{-1,-2,-3,\ldots\}$
		\begin{solution}{a}
			$A=\{x\in \Z:x<0\}$
		\end{solution} 
		(b) $B=\{-3,-2,\ldots,3\}$
		\begin{solution}{b}
			$B=\{x\in \Z: -4<x\leq 3\}$
		\end{solution}
		(c) $C=\{-2,-1,1,2\}$
		\begin{solution}{c}
			$C=\{x\in \Z:0<|x|\leq 2\}$
		\end{solution}
	\end{problem}

	\begin{problem}{6}
		The set $E=\{2x:x\in \Z\}$ can be described by listing its elements, namely $E=\{\ldots,-4,-2,0,2,4,\ldots\}$. List the elements of the following sets in a similar manners.\\
		
		(a) $A=\{2x+1:x\in \Z\}$
		\begin{solution}{a} 
			The set of odd integers.\\
			$A=\{\ldots,-5,-3,-1,1,3,5,\ldots\}$
		\end{solution}
		(b) $B=\{4n:n\in \Z\}$
		\begin{solution}{b}
			The set of integers that are multiples of 4.\\
			$B=\{\ldots,-8,-4,0,4,8,\ldots\}$
		\end{solution}
		(c) $C=\{3q+1:q\in \Z\}$
		\begin{solution}{c}
			$C=\{\ldots,-5,-2,1,4,7,\ldots\}$
		\end{solution}
	\end{problem}

	\begin{problem}{7}
		The set $E=\{\ldots,-4,-2,0,2,4,\ldots\}$ of even integers can be described by means of a defining condition by $E=\{y=2x:x\in \Z\} = \{2x:x\in \Z\}$. Describe the following sets in a similar manner.\\
		
		(a) $A=\{\ldots,-4,-1,2,5,8,\ldots\}$
		\begin{solution}{a}
			$A=\{3n+2:n\in \Z\}$
		\end{solution} 
		(b) $B=\{\ldots,-10,-5,0,5,10,\ldots\}$
		\begin{solution}{b}
			The set of integers that are multiples of 5.\\
			$B=\{5n:n\in \Z\}$
		\end{solution}
		(c) $C=\{1,8,27,64,125,\ldots\}$
		\begin{solution}{c}
			$C=\{x^{3}:x\in \Z \text{ and } x>0\} = \{x^{3}:x\in \Z^{+}\}=\{x^{3}:x\in \mathbb{N}\}$.
		
		\end{solution}
	\end{problem}

	\begin{problem}{8}
		Let $A=\{n \in \Z:2\leq |n|<4\}$, $B=\{x\in \mathbb{Q}:2<x\leq 4\}$, $C=\{x \in \mathbb{R}:x^{2}-(2+\sqrt{2})x+2\sqrt{2}=0\}$ and $D=\{x \in \mathbb{Q}: x^{2}-(2+\sqrt{2})x+2\sqrt{2}=0\}$.\\
		
		(a) Describe the set $A$ by listing its elements.
		\begin{solution}{a}
			$A=\{-3,-2,2,3\}$
		\end{solution}
		(b) Give an example of three elements that belong to $B$ but do not belong to $A$.
		\begin{solution}{b}
			The rational numbers $19/6$, $5/2$ and $4$ belong to $B$ and do not belong to $A$.
		\end{solution}
		(c) Describe the set $C$ by listing its elements.
		\begin{solution}{c}
			One can factor the equation in the condition of the description of $C$ in order to get its solutions.
			\begin{align*}
				x^{2}-(2+\sqrt{2})x+2\sqrt{2}& =  (x-2)(x-\sqrt{2}) = 0\\
						x &\in \{2,\sqrt{2}\}
			\end{align*}
			Both 2 and $\sqrt{2}$ are real numbers. Therefore, the set $C=\{2,\sqrt{2}\}$.
		\end{solution}
		(d) Describe the set $D$ in another manner.
		\begin{solution}{d}
			$D=\{x\in \mathbb{Q}:x=2\}=\{2\}$
		\end{solution}
		(e) Determine the cardinality of each of the sets $A,C$ and $D$.
		\begin{solution}{e} 
			$|A|=4$\\$|C|=2$\\$|D|=1$
		\end{solution}
	\end{problem}

	\begin{problem}{9}
		For $A=\{2,3,5,7,8,10,13\}$, let
		\begin{equation*}
			B=\{x \in A:x=y+z, \text{ where }y,z\in A\} \text{ and } C=\{r\in B: r+s\in B \text{ for some } s\in B\}.
		\end{equation*}
	Determine $C$.
	\begin{solution}{}
		$B=\{5,7,8,10,13\}$\\
		$C=\{5,8\}$
	\end{solution}
	\end{problem}
\end{document}