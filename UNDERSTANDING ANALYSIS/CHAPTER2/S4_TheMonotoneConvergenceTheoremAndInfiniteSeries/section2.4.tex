\documentclass[12pt]{article}
\usepackage[margin=1in]{geometry}
\usepackage{amsmath, amsfonts,amsthm,amssymb,epigraph,etoolbox,mathtools,setspace,enumitem}  
\usepackage{tikz}
\usetikzlibrary{datavisualization} 
\usepackage[makeroom]{cancel} 
\usepackage[linguistics]{forest}
\usetikzlibrary{patterns}
\newcommand{\N}{\mathbb{N}}
\newcommand{\Z}{\mathbb{Z}}
\newcommand{\R}{\mathbb{R}}
\newcommand{\Q}{\mathbb{Q}}
\newcommand{\Mod}[1]{\ (\mathrm{mod}\ #1)}
\newcommand{\Lim}[1]{\mathrm{lim}(#1)}
\newcommand{\Abs}[1]{\left\vert #1 \right\vert}
\newcommand{\Dom}[1]{\mathrm{dom}\left(#1\right)}
\newcommand{\Range}[1]{\mathrm{range}(#1)}

\newlist{legal}{enumerate}{10}
\setlist[legal]{label=(\alph*)}
\setenumerate[legal]{label=(\alph*)}

\DeclarePairedDelimiter\bra{\langle}{\rvert}
\DeclarePairedDelimiter\ket{\lvert}{\rangle}
\DeclarePairedDelimiterX\braket[2]{\langle}{\rangle}{#1\delimsize\vert #2}


\newenvironment{theorem}[2][Theorem]{\begin{trivlist} \item[\hskip \labelsep {\bfseries #1}\hskip \labelsep {\bfseries #2.}]}{\end{trivlist}}
\newenvironment{lemma}[2][Lemma]{\begin{trivlist} \item[\hskip \labelsep {\bfseries #1}\hskip \labelsep {\bfseries #2.}]}{\end{trivlist}}
\newenvironment{result}[2][Result]{\begin{trivlist} \item[\hskip \labelsep {\bfseries #1}\hskip \labelsep {\bfseries #2.}]}{\end{trivlist}}
\newenvironment{exercise}[2][Exercise]{\begin{trivlist} \item[\hskip \labelsep {\bfseries #1}\hskip \labelsep {\bfseries #2.}]}{\end{trivlist}}
\newenvironment{problem}[2][Problem]{\begin{trivlist} \item[\hskip \labelsep {\bfseries #1}\hskip \labelsep {\bfseries #2.}]}{\end{trivlist}}
\newenvironment{question}[2][Question]{\begin{trivlist} \item[\hskip \labelsep {\bfseries #1}\hskip \labelsep {\bfseries #2.}]}{\end{trivlist}}
\newenvironment{corollary}[2][Corollary]{\begin{trivlist} \item[\hskip \labelsep {\bfseries #1}\hskip \labelsep {\bfseries #2.}]}{\end{trivlist}}
\newenvironment{solution}[1][Solution]{\begin{trivlist} \item[\hskip \labelsep {\bfseries #1}]}{\end{trivlist}}

\setlength\epigraphwidth{8cm}
\setlength\epigraphrule{0pt}

\makeatletter
\patchcmd{\epigraph}{\@epitext{#1}}{\itshape\@epitext{#1}}{}{}
\makeatother

\begin{document}
  
\title{Section 2.4: The Monotone Convergence Theorem and Infinite Series}
   \author{Juan Patricio Carrizales Torres}
     \date{Sep 2, 2022}
       \maketitle

       In this chapter we are introduced to the Monotone Convergence Theorem, which is very useful in cheecking the convergence of sequences of partial sums. Let $(a_{n})$ be a sequence. This theorem states that if $(a_{n})$ is monotone (either increasing or decreasing), namely $a_{n} \leq a_{n+1}$ or $a_{n}\geq a_{n+1}$ for all $n\in \N$ respectively, and it is bounded, then it converges to some limit. Its usfulness comes in two ``flavors''. First, the fact that partial sums of positive real numbers are elements of an increasing sequence. Second, it suffices to show that a sequence is increasing and bounded to conclude that converges without the necessity to come up with a particular limit. 
       We are interested in the convergence of partial sums, since an infinite series
\begin{align*}
  \sum_{n\in\N} a_{n}
\end{align*}
is said to converge (equal) some number $N$ if the sequence of its partial sums $(s_{n}) = (a_{1}+a_{2}+\dots+a_{n})$ converges to $N$. 
One way to show that an increasing sequence of partial sums  is bounded is by proving that every element is lower or equal to other element from a bounded sequence. On the other hand, a sequence of partial sums $(s_{n})$ is not bounded if for every element $k$ of some unbounded sequence $(p_{n})$ there is an element in $(s_{n})$ that is greater or equal to $p_{k}$. For instance, one can extract another sequence $(m_{n})$ from $(s_{n})$ such that $m_{k} \geq p_{k}$ for all $k\in\N$.\\
Let's state this in a clear and clean way. Let $(a_{n})$ and $(b_{n})$ be sequences. If for every $n\in\N$ there is some positive integer $k$ such that $a_{n}\leq b_{k}$, then $(a_{n})\leq (b_{n})$. Now, let $(s_{n})$ and $(p_{n})$ be bounded and unbounded sequences, respectively. Then, the increasing sequence $(a_{n})$ is bounded if $(a_{n})\leq (s_{n})$. On the other hand, $(a_{n})$ is unbounded if $(p_{n}) \leq (a_{n})$.\\

For example, the  $\mathbf{Cauchy \; Condensation \; Test}$ uses the infinite series
\begin{align*}
  \sum_{n\in\N} 2^{n}b_{2^{n}}.
\end{align*}
to check the converge or divergence of the infinite series of some decreasing sequence $(b_{n})$ of nonengative real numbers since $(s_{2^{n}b_{2^{n}}}) \leq (s_{b_{n}})$ and viceversa.
\begin{problem}{2.4.1}
\begin{enumerate}
  \item Prove that the sequence defined by $x_{1} = 3$ and 
\begin{equation*}
  x_{n+1} = \frac{1}{4-x_{n}}
\end{equation*}
converges.
\begin{proof}
  We proceed by induction. We show that $3\geq x_{n} > x_{n+1} >0$ for all $n\in \N$. Note that $x_{1} = 3$ and $x_{2} = 1/(4-3) = 1$. Hence, $3\geq x_{1} > x_{2} > 0$. Now, assume for some $k\in \N$ that $3\geq x_{k} > x_{k+1} > 0$. We prove that $3\geq x_{k+1} > x_{k+2}>0$. Note that
\begin{align*}
  3&\geq x_{k} > x_{k+1} >0 \implies\\
  1&\leq 4-x_{k} < 4-x_{k+1} < 4 \implies\\
  1&\geq \frac{1}{4-x_{k}} > \frac{1}{4-x_{k+1}} > \frac{1}{4}.
\end{align*}
Therefore, $3\geq x_{k+1} > x_{k+2} >0$. By the Principle of Mathmatical Induction, $3\geq x_{k}>x_{k+1}>0$ for all $k\in \N$. Thus, $x_{n}$ is decreasing and bounded. It converges to some $a$, and according to the given argument, it seems that $a=\frac{1}{4}$.
\end{proof}
\item Now that we know $\lim x_{n}$ exists, explain why $\lim x_{n+1}$ must also exist and equal the same value.
\begin{solution}
  Recall that when dealing with convergence of sequences, we are mostly intersted in the ``tail'', namely, how infinitely but finite many of them behave. Note that $x_{n+1} = \left(x_{n}:n\geq 2 \right)$ is the same sequence as $x_{n}$ minus the first term. We keep infinitely many of them (tail). Hence, for any $\epsilon$ such that for any $n\geq N$ we have $|x_{n} -a|<\epsilon$, there is still some $K\geq N$ such that $|x_{n+1}-a|<\epsilon$ for all $n\geq K$. Thus, $\lim x_{n} = \lim x_{n+1}$.
\end{solution}
\item Take the limit of each side of the recursive equation in part (a) to explicitly xompute $\lim x_{n}$. 
\end{enumerate}
\end{problem}
\end{document}


