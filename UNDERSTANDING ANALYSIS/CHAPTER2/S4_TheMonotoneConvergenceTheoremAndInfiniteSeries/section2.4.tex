\documentclass[12pt]{article}
\usepackage[margin=1in]{geometry}
\usepackage{amsmath, amsfonts,amsthm,amssymb,epigraph,etoolbox,mathtools,setspace,enumitem}  
\usepackage{tikz}
\usetikzlibrary{datavisualization} 
\usepackage[makeroom]{cancel} 
\usepackage[linguistics]{forest}
\usetikzlibrary{patterns}
\newcommand{\N}{\mathbb{N}}
\newcommand{\Z}{\mathbb{Z}}
\newcommand{\R}{\mathbb{R}}
\newcommand{\Q}{\mathbb{Q}}
\newcommand{\Mod}[1]{\ (\mathrm{mod}\ #1)}
\newcommand{\Lim}[1]{\mathrm{lim}(#1)}
\newcommand{\Abs}[1]{\left\vert #1 \right\vert}
\newcommand{\Dom}[1]{\mathrm{dom}\left(#1\right)}
\newcommand{\Range}[1]{\mathrm{range}(#1)}

\newlist{legal}{enumerate}{10}
\setlist[legal]{label=(\alph*)}
\setenumerate[legal]{label=(\alph*)}

\DeclarePairedDelimiter\bra{\langle}{\rvert}
\DeclarePairedDelimiter\ket{\lvert}{\rangle}
\DeclarePairedDelimiterX\braket[2]{\langle}{\rangle}{#1\delimsize\vert #2}


\newenvironment{theorem}[2][Theorem]{\begin{trivlist} \item[\hskip \labelsep {\bfseries #1}\hskip \labelsep {\bfseries #2.}]}{\end{trivlist}}
\newenvironment{lemma}[2][Lemma]{\begin{trivlist} \item[\hskip \labelsep {\bfseries #1}\hskip \labelsep {\bfseries #2.}]}{\end{trivlist}}
\newenvironment{result}[2][Result]{\begin{trivlist} \item[\hskip \labelsep {\bfseries #1}\hskip \labelsep {\bfseries #2.}]}{\end{trivlist}}
\newenvironment{exercise}[2][Exercise]{\begin{trivlist} \item[\hskip \labelsep {\bfseries #1}\hskip \labelsep {\bfseries #2.}]}{\end{trivlist}}
\newenvironment{problem}[2][Problem]{\begin{trivlist} \item[\hskip \labelsep {\bfseries #1}\hskip \labelsep {\bfseries #2.}]}{\end{trivlist}}
\newenvironment{question}[2][Question]{\begin{trivlist} \item[\hskip \labelsep {\bfseries #1}\hskip \labelsep {\bfseries #2.}]}{\end{trivlist}}
\newenvironment{corollary}[2][Corollary]{\begin{trivlist} \item[\hskip \labelsep {\bfseries #1}\hskip \labelsep {\bfseries #2.}]}{\end{trivlist}}
\newenvironment{solution}[1][Solution]{\begin{trivlist} \item[\hskip \labelsep {\bfseries #1}]}{\end{trivlist}}

\setlength\epigraphwidth{8cm}
\setlength\epigraphrule{0pt}

\makeatletter
\patchcmd{\epigraph}{\@epitext{#1}}{\itshape\@epitext{#1}}{}{}
\makeatother

\begin{document}
  
\title{Section 2.4: The Monotone Convergence Theorem and Infinite Series}
   \author{Juan Patricio Carrizales Torres}
     \date{Sep 2, 2022}
       \maketitle

       In this chapter we are introduced to the Monotone Convergence Theorem, which is very useful in cheecking the convergence of sequences of partial sums. Let $(a_{n})$ be a sequence. This theorem states that if $(a_{n})$ is monotone (either increasing or decreasing), namely $a_{n} \leq a_{n+1}$ or $a_{n}\geq a_{n+1}$ for all $n\in \N$ respectively, and it is bounded, then it converges to some limit. Its usfulness comes in two ``flavors''. First, the fact that partial sums of positive real numbers are elements of an increasing sequence. Second, it suffices to show that a sequence is increasing and bounded to conclude that converges without the necessity to come up with a particular limit. 
       We are interested in the convergence of partial sums, since an infinite series
\begin{align*}
  \sum_{n\in\N} a_{n}
\end{align*}
is said to converge (equal) some number $N$ if the sequence of its partial sums $(s_{n}) = (a_{1}+a_{2}+\dots+a_{n})$ converges to $N$. 
One way to show that an increasing sequence of partial sums  is bounded is by proving that every element is lower or equal to other element from a bounded sequence. On the other hand, a sequence of partial sums $(s_{n})$ is not bounded if for every element $k$ of some unbounded sequence $(p_{n})$ there is an element in $(s_{n})$ that is greater or equal to $p_{k}$. For instance, one can extract another sequence $(m_{n})$ from $(s_{n})$ such that $m_{k} \geq p_{k}$ for all $k\in\N$.\\
Let's state this in a clear and clean way. Let $(a_{n})$ and $(b_{n})$ be sequences. If for every $n\in\N$ there is some positive integer $k$ such that $a_{n}\leq b_{k}$, then $(a_{n})\leq (b_{n})$. Now, let $(s_{n})$ and $(p_{n})$ be bounded and unbounded sequences, respectively. Then, the increasing sequence $(a_{n})$ is bounded if $(a_{n})\leq (s_{n})$. On the other hand, $(a_{n})$ is unbounded if $(p_{n}) \leq (a_{n})$.\\

For example, the  $\mathbf{Cauchy \; Condensation \; Test}$ uses the infinite series
\begin{align*}
  \sum_{n\in\N} 2^{n}b_{2^{n}}.
\end{align*}
to check the converge or divergence of the infinite series of some decreasing sequence $(b_{n})$ of nonengative real numbers since $(s_{2^{n}b_{2^{n}}}) \leq (s_{b_{n}})$ and viceversa.
\begin{problem}{2.4.1}
\begin{enumerate}
  \item Prove that the sequence defined by $x_{1} = 3$ and 
\begin{equation*}
  x_{n+1} = \frac{1}{4-x_{n}}
\end{equation*}
converges.
\begin{proof}
  We proceed by induction. We show that $3\geq x_{n} > x_{n+1} >0$ for all $n\in \N$. Note that $x_{1} = 3$ and $x_{2} = 1/(4-3) = 1$. Hence, $3\geq x_{1} > x_{2} > 0$. Now, assume for some $k\in \N$ that $3\geq x_{k} > x_{k+1} > 0$. We prove that $3\geq x_{k+1} > x_{k+2}>0$. Note that
\begin{align*}
  3&\geq x_{k} > x_{k+1} >0 \implies\\
  1&\leq 4-x_{k} < 4-x_{k+1} < 4 \implies\\
  1&\geq \frac{1}{4-x_{k}} > \frac{1}{4-x_{k+1}} > \frac{1}{4}.
\end{align*}
Therefore, $3\geq x_{k+1} > x_{k+2} >0$. By the Principle of Mathmatical Induction, $3\geq x_{k}>x_{k+1}>0$ for all $k\in \N$. Thus, $x_{n}$ is decreasing and bounded. It converges to some $a$, and according to the given argument, it seems that $a=\frac{1}{4}$.
\end{proof}
\item Now that we know $\lim x_{n}$ exists, explain why $\lim x_{n+1}$ must also exist and equal the same value.
\begin{solution}
  Recall that when dealing with convergence of sequences, we are mostly intersted in the ``tail'', namely, how infinitely but finite many of them behave. Note that $x_{n+1} = \left(x_{n}:n\geq 2 \right)$ is the same sequence as $x_{n}$ minus the first term. We keep infinitely many of them (tail). Hence, for any $\epsilon$ such that for any $n\geq N$ we have $|x_{n} -a|<\epsilon$, there is still some $K\geq N$ such that $|x_{n+1}-a|<\epsilon$ for all $n\geq K$. Thus, $\lim x_{n} = \lim x_{n+1}=a$.
\end{solution}
\item Take the limit of each side of the recursive equation in part (a) to explicitly xompute $\lim x_{n}$. 
\begin{solution}
  The sequence $(x_{n+1})$ is recursively defined by 
\begin{align*}
  x_{n+1} &= \frac{1}{4-x_{n}}
\end{align*}
for all $n\in \N$. Taking the limit in both sides we get that
\begin{align*}
  \lim x_{n+1} &= \lim \frac{1}{4-x_{n}} \implies \\
  a &= \frac{1}{4-a},
\end{align*}
by the Algebraic Limit Theorem. Hence, $(x_{n+1})\to a$ and so $(x_{n+1})=\left( \frac{1}{4-x_{n}} \right) \to \frac{1}{4-a} = a$. Then, $a^{2}-4a+1=0$. Using the quadratic formula, we get $a=2\pm \sqrt{3}$. Since $3\geq x_{n}+1$ for all $n\in \N$, it follows that $3\geq a$ and so $a=2-\sqrt{3}$. Note that $a\approx 0.267949$ which is very near to our initial guess of $\frac{1}{4}$.
\end{solution}
\end{enumerate}
\end{problem}
\begin{problem}{2.4.2}
\begin{enumerate}
  \item Consider the recursively defined sequence $y_{1} = 1$,
\begin{equation*}
  y_{n+1} = 3-y_{n},
\end{equation*}
and set $y=\lim y_{n}$. Because $(y_{n})$ and $(y_{n+})$ have the same limit, taking the limit across the the recursive equation gives $y=3-y$. Solving for $y$, we conclude $\lim y_{n} = 3/2$.\\
What is wrong with this argument?
\begin{solution}
  Computing the first 5 terms of $(y_{n})$ we realize that $(y_{n}) = \left( 2,1,2,1,2,\dots \right)$ is a sequence that alternates between 2 values and so it does not converges. The argument is good if the assumption is true, however the stament $y=\lim y_{n}$ is false.
\end{solution}
\item This time set $y_{1} = 1$ and $y_{n+1} = 3-\frac{1}{y_{n}}$. Can the strategy in (a) be applied to compute the limit of this sequence?
\begin{solution}
  In order to be sure that the strategy in (a) can be applied to this example we must show that the recursive sequence converges. One way is using the Monotone Convergence Theorem.  
\begin{proof}
  We procced by induction. We show that $3>y_{n+1}>y_{n}\geq 1$ for all $n\in \N$. Observe that $y_{1} = 1$ and $y_{2} = 3-1=2$. Hence, $3>y_{2}>y_{1}\geq1$. Suppose for some $k\in \N$ that $3>y_{k+1}>y_{k}\geq1$. We prove that $3>y_{k+2}>y_{k+1}>1$. Note that $1/3 < 1/y_{k+1} \leq 1/y_{k} < 1$ and so $3-1/3 > 3-1/y_{k+1} > 3-1/y_{k} \geq3-1$. Therefore, 
\begin{equation*}
  3>\frac{8}{3} > y_{k+2} > y_{k+1} \geq 2 > 1.
\end{equation*}
Thus, by the Principle of Mathematical Induction, $3>y_{n+1}>y_{n}\geq 1$ for all $n\in \N$. Then, $(y_{n})$ is an increasing and bounded sequence, and so, by the Monotone Convergence sequence, $\lim y_{n} = y$ for some $y\in \R$.
\end{proof}
Then, it is valid to apply the argument of (a) in this case. Hence,
\begin{align*}
  \lim y_{n+1} &= \lim \left( 3-\frac{1}{y_{n}} \right)\implies\\
  y &= 3-\frac{1}{y}
\end{align*}
since $y_{n} > 0$ for all $n\in \N$. Thus, $y^{2}-3y+1=0$ and so, using the cuadratic equation, $y= \frac{3\pm\sqrt{5}}{2}$. Since $y_{n}\geq1$ for all $n\in \N$, then $y\geq1$ and so $y=\frac{3+\sqrt{5}}{2}\approx 2.6180$. This technique is possible thanks to the \textbf{Algebraic Theorem of Limits}.
\end{solution}
\end{enumerate}
\end{problem}
\begin{problem}{2.4.3}
\begin{enumerate}
  \item Show that 
\begin{align*}
  \sqrt{2}, \sqrt{2+\sqrt{2}}, \sqrt{2+\sqrt{2+\sqrt{2}}},\dots
\end{align*}
converges and find the limit.
\begin{proof}
  Note that this is can be expressed as a  recursive sequence defined by $y_{1} = \sqrt{2}$ and $y_{n+1} = \sqrt{2+y_{n}}$. We first show that it is bounded and an increasing sequence. Hence, we proceed by induction and prove that $1<y_{n} < y_{n+1}<2$ for all $n\in \N$. Note that $1<2<2+\sqrt{2}<4$ and, by taking the root, $1<\sqrt{2}<\sqrt{2+\sqrt{2}} < 2$. Hence, $1<y_{1}<y_{2}<2$.\\
  Now, assume that $1<y_{k}<y_{k+1}<2$ for some $k\in \N$. We show that $1<y_{k+1}<y_{k+2}<2$. \\
  Observe that $1<2+y_{k}<2+y_{k+1}<4$ and, by taking the square root, 
\begin{align*}
  1<y_{k+1} < \sqrt{2+y_{k+1}} = y_{k+2} < 2.
\end{align*}
By the Principle of Mathematical Induction, $1<y_{n}<y_{n+1}<2$ for all $n\in \N$ and so $(y_{n})$ is a bounded and increasing sequence. Thus, it converges to some limit, by the  Bounded Monotone Convergence Theorem.\\

Now, we proceed to find its limit. Note that $\lim (y_{n+1}) = \lim (y_{n}) = y$ and so
\begin{align*}
  \lim{y_{n+1}} &= \sqrt{2+y} \implies\\
  y^{2} -y -2 &= 0 
\end{align*}
Thus, $(y-2)^{2}=0$ and so $y=2$.
\end{proof}
\item Does the sequence  
\begin{align*}
  \sqrt{2},\sqrt{2\sqrt{2}},\sqrt{2\sqrt{2\sqrt{2}}},\dots
\end{align*}
converge? If so, find the limit.
\begin{proof}
  Note that sequence can be expressed as a recursive sequence defined by $y_{1} = \sqrt{2}$ and $y_{n+1} = \sqrt{2y_{n}}$. We show that it is an increasing bounded sequence. Therefore, we proceed by induction and show that $1<y_{n}<y_{n+1}<2$ for all $n\in \N$. First, note that $1<2<2\sqrt{2}<4$ and, by taking the square root, $1<\sqrt{2}<\sqrt{2\sqrt{2}}<2$. Hence, $1<y_{1}<y_{2}<2$.\\
  Now, suppose that $1<y_{k}<y_{k+1}<2$ for some $k\in\N$. We show that $1<y_{k+1}<y_{k+2}<2$.\\
  Observe that $1<2y_{k}<2y_{k+1}<4$ and, by taking the square root, $1<y_{k+1}<y_{k+2}<2$. By the Principle of Mathematical Induction, $1<y_{n}<y_{n+1}<2$ for all $n\in \N$. Hence, $(y_{n})$ is an increasing bounded sequence and, by the Monotone Bounded Sequence, $(y_{n})$ converges to some $y$.\\
  We have that $\lim y_{n+1} = \lim \sqrt{2y_{n}}$ and so $y^{2}-2y = y(y-2)=0$. Thus, either $y=0$ or $y=2$. However, since $1<y_{n}$ for all $n\in\N$, it follows that $1<y$, and so $y=2$. 
\end{proof}
\end{enumerate}
\end{problem}
\begin{lemma}{1}
  Let $(a_{n})$ be an increasing sequence ($a_{n}\leq a_{n+1}$ for all $n\in  \N$) that converges to $a$. Then, $a\geq a_{n}$ for all $n\in \N$.
\begin{proof}
  Assume, to the contrary, that there is some $N\in\N$ such that $a_{N}>a$. Thus, $\epsilon = a_{N}-a>0$ and so $|a_{n}-a|\geq\epsilon$ for all $n\geq N$ (recall $(a_{n})$ is increasing). This contradicts the fact that $a$ is the limit of $(a_{n})$.
\end{proof}
\end{lemma}
\begin{problem}{2.4.4}
\begin{enumerate}
  \item In Section 1.4 we used the Axiom of Completeness (AoC) to prove the Archimedean Property of $\R$ (Theorem 1.4.2). Show that the Monotone Convergence Theorem can also be used to prove the Archimedean Property without making any use of AoC
\begin{proof}
  First, let $(a_{n})$ be a sequence defined by $a_{n}=n$. Note that $a_{n}=n<n+1=a_{n+1}$ for all $n\in \N$ and so it is an increasing sequence. Just like the proof from \textbf{Section 1.4}, we proceed by contradiction. Suppose, to the contrary, that there is some real number $y$ such that $a_{n}<y$ for all $n\in \N$. Thus, $0<a_{n}<y$ for all $n\in \N$ and it is a bounded increasing sequence. By the Bounded Monotone Convergence Sequence, there is some $\lim a_{n} = a$. Since $(a_{n})$ is an increasing sequence, it follows that $a_{n} \leq  a$ for all $n\in\N$.

  Now, consider some $m\in \N$ and so there is some $N\in \N$ such that $|a_{n} -a|=|n-a| < m$ for all $n\geq \N$. Thus, $a-n<m$ and so $a<m+n$. Because $\N$ is closed under addition, $m+n\in \N$ and so $a<a_{m+n}$ for all $n\geq N$, which contradicts the fact that $a$ is the limit of an increasing sequence.
\end{proof}
\item Use the Monotone Convergence Theorem to supply a proof for the Nested Interval Property (\textbf{Theorem 1.4.1}) that doesn't make use of AoC.\\
\begin{proof}
  Let $A=\left\{ a_{n}:n\in\N \right\}$ and every $b_{n}$ serves as an upper bound for $A$. Also, note that $a_{n}\leq a_{n+1}$ for all $n\in \N$ since each $I_{n+1} = [a_{n+1},b_{n+1}] \subseteq [a_{n},b_{n}] = I_{n}$ for all $n\in \N$ (nested). Hence, $(a_{n})$ is an increasing and bounded sequence, and so, by the Bounded Monotone Convergence Theorem, it converges to some $x$. Thus, $a_{n} \leq x$ for all $n\in\N$ since $x$ is the limit of an increasing sequence.

  As we have previously seen, $\lim a_{n} = x$ must the lowest upper bound of $A$ (l.u.b for all elements in the sequence), otherwise, it leads to a contradiction. Therefore, $x\leq b_{n}$ for all $n\in\N$. Now, consider some $I_{n}$ and so $a_{n}\leq x \leq b_{n}$. Hence, $x\in I_{n}$ and $x\in \bigcap_{n\in\N} I_{n} \neq \emptyset$.
\end{proof}
  These two results suggest that we could have used the Monotone Convergence Theorem in place of AoC as our starting axiom for building theory of  the real numbers.
\end{enumerate}
\end{problem}
\begin{problem}{2.4.5 (Calculating Square Roots)}
  Let $x_{1}=2$, and define
\begin{align*}
  x_{n+1} = \frac{1}{2} \left( x_{n}+\frac{2}{x_{n}} \right).
\end{align*}
\begin{enumerate}
  \item Show that $x_{n}^{2}$ is always greater than or equal to 2, and then use this to prove that $x_{n}-x_{n+1} \geq 0$. Conclude that $\lim x_{n} = \sqrt{2}$.
\begin{proof}
  We proceed by induction to prove $2\leq x_{n}^{2}$ for all $n\in\N$. Note that $2\leq 4 = x_{1}^{2}$. Now, suppose that $2\leq x_{k}^{2}$ for some $k\in \N$. We show that $2\leq x_{k+1}^{2}$. Note that
\begin{align*}
  x_{k+1}^{2} &= \frac{1}{4}\left( x_{k}+\frac{2}{x_{k}} \right)^{2} = \frac{1}{4}\left( x_{k}^{2}+4+\frac{4}{x_{k}^{2}} \right) = \frac{1}{4}\left( x_{k}^{2}-4+\frac{4}{x_{k}^{2}} \right) + 2\\  &=\frac{1}{4}\left( x_{k}-\frac{2}{x_{k}} \right)^{2} + 2 \geq 2.
\end{align*}
By the Principle of Mathematical Induction, $x_{n}\geq 2$ for every $n\in \N$.\\
Furthermore, $x_{1} = 2 > 0$ and if $x_{k} > 0$, then $x_{k+1} = \frac{1}{2}\left( x_{k}+\frac{2}{x_{k}} \right)>0$. This implies that $x_{n} >0$ for all $n\in \N$. Observe that  
\begin{align*}
  x_{n}-x_{n+1} &= x_{n} - \left( \frac{1}{2}x_{n}+\frac{1}{x_{n}} \right)\\
  &= \frac{1}{2}x_{n} -\frac{1}{x_{n}} = \frac{x_{n}^{2}-2}{2x_{n}}\\
  &\geq \frac{2-2}{2x_{n}} = 0
\end{align*}
Hence, $2\geq x_{n}\geq x_{n+1}>0$ and so $x_{n}$ is a bounded decreasing sequence that converges to some limit $a$. Since $a>0$, we have
\begin{align*}
  \lim x_{n+1} &= \frac{1}{2}\left( \lim x_{n} + \lim \frac{2}{x_{n}} \right) \implies\\
  a &= \frac{1}{2}a +\frac{1}{a}.
\end{align*}
Substracting $a/2$ and multiplying by $2a$, we get $a^{2}=2$ and so $a=\sqrt{2}$.
\end{proof}
  \item Modify the sequence $(x_{n})$ so that it converges to $\sqrt{c}$.
\begin{solution}
  Let $(x_{n})$ be a sequence defined by $x_{1}=c$ and
\begin{align*}
  x_{n+1} = \frac{1}{2}\left( x_{n}+\frac{c}{x_{n}} \right).
\end{align*}
Using the same argument and procedure in $(a)$, it can be shown that $(x_{n})\to \sqrt{c}$.
\end{solution}
\end{enumerate}
\end{problem}
\begin{problem}{2.4.6 (Arithmetic-Geometric Mean)}
\begin{enumerate}
  \item Explain why $\sqrt{xy} \leq (x+y)/2$ for any two positive real numbers $x$ and $y$. (The geometric mean is always less than the arithmetic mean.)
\begin{proof}
  Let $\sqrt{x}=a$ and $\sqrt{y}=b$. Then, $x=a^{2}$ and $y=b^{2}$. Because, $0\leq (a-b)^{2}$, it follows that $2ab\leq a^{2}+b^{2}$ and so $ab = \sqrt{xy} \leq (a^{2}+b^{2})/2 = (x+y)/2$.
\end{proof}
  \item Now let $0\leq x_{1} \leq y_{1}$ and define
\begin{align*}
  x_{n+1} = \sqrt{x_{n}y_{n}}& &\text{and}& &y_{n+1} = \frac{x_{n}+y_{n}}{2}.
\end{align*}
Show $\lim x_{n}$ and $\lim y_{n}$ both exist and are equal.
\begin{proof}
  Since $0\leq x_{1} \leq y_{1}$  and, by the arithmetic-geometric mean, $0\leq \sqrt{x_{n}y_{n}}\leq (x_{n}+y_{n})/2$ for any $n\in \N$, it follows that $0\leq x_{n} \leq y_{n}$ for every $n\in \N$. Hence, $0\leq x_{n} + y_{n} \leq 2y_{n}$ and so $0\leq (x_{n}+y_{n})/2 \leq y_{n}$ for any $n\in \N$. This implies that $(y_{n})$ is a decreasing bounded sequence. \\

  Furthermore, $0\leq \sqrt{x_{n}}\leq \sqrt{y_{n}}$ and so $0\leq x_{n} \leq \sqrt{y_{n}x_{n}}$. Then, $x_{n}$ is an increasing sequence. Note that $0\leq x_{n}\leq y_{n}$ and so each $x_{n}$ can be ``enclosed'' in the limits of the bounded decreasing sequence $(y_{n})$. By the Bounded Monotone Theorem, both $(x_{n})$ and $(y_{n})$ converge to some limit $x$ and $y$, respectively. Therefore, 
\begin{align*}
  \lim y_{n+1} &= \lim \left( \frac{x_{n}+y_{n}}{2} \right)\\
  &= \frac{\lim x_{n} + \lim y_{n}}{2} \implies\\
  y &= \frac{x+y}{2}.
\end{align*}
This implies that $y=x$. 
\end{proof}
\end{enumerate}
\end{problem}
\begin{problem}{2.4.7 (Limit Superior)}
  Let $(a_{n})$ be a bounded sequence.
\begin{enumerate}
  \item Prove that the sequence defined by $y_{n}=\sup\{a_{k}:k\geq n\}$ converges.
\begin{proof}
  Consider some $y_{n}=\sup\{a_{k}:k\geq n\}$. Note that $y_{n+1}=\sup\{a_{k}:k\geq n+1\}$ is the supremum of the set $\{a_{k}:k\geq n+1\}/\{a_{n}\}$. Thus, $y_{n+1}\leq y_{n}$ which implies that $(y_{n})$ is a decreasing sequence. Furthermore, since $|a_{n}|\leq M$ for some $M\in \R$ (bounded), it follows that each $y_{n}\leq M$ (supremum) and $-M\leq a_{n} \leq y_{n}$. \\
  By the Bounded Monotone Sequence, $(y_{n})$ converges.
\end{proof}
  \item The \textit{limit superior} of $(a_{n})$, or $\lim\sup a_{n}$, is defined by
\begin{align*}
  \lim\sup a_{n} &= \lim y_{n},
\end{align*}
where $y_{n}$ is the sequence from part $(a)$ of this excercise. Provide a reasonable definition for $\lim\inf a_{n}$ and briefly explain why it always exists for any bounded sequence.
\begin{solution}
  The \textit{limit inferior} of $(a_{n})$ can be defined by  
\begin{align*}
  \lim\inf a_{n} = \lim y_{n}
\end{align*}
where $(y_{n})$ is a sequence defined by $y_{n} = \inf\{a_{k}:k\geq n\}$. Note that $y_{n}$ is an increasing sequence. Furthermore, if $(a_{n})$ is bounded by $M$, then each $y_{n}\geq -M$ (infimum) and $M\geq a_{n} \geq y_{n}$.
\end{solution}
\item Prove that $\lim\inf a_{n} \leq \lim\sup a_{n}$ for every bounded sequence, and give an example of a sequence for which the inequality is strict.
\begin{proof}
  Let $x_{n} = \inf\left\{ a_{k}:k\geq n \right\}$ and $y_{n} = \sup\left\{ a_{k}:k\geq n \right\}$. Note that $x_{n} \leq y_{n}$ for all $n\in \N$ and so $\lim x_{n} = \lim\inf a_{n} \leq \lim y_{n}= \lim\sup a_{n}$.
\end{proof}
\item Show that $\lim\inf a_{n} = \lim\sup a_{n}$ if and only if $\lim a_{n}$ exists. In this case, all three share the same value. 
\begin{proof}
  Suppose that $\lim\inf a_{n} = \lim\sup a_{n}$ and let $x_{n} = \inf a_{n}$ and $y_{n} = \sup a_{n}$. By definition, $x_{n} \leq a_{n} \leq y_{n}$ for all $n\in \N$. By the \textbf{Squeeze Theorem} proved in the previous section, the sequence $(a_{n})$ converges and $\lim a_{n} = \lim\inf a_{n} = \lim\sup a_{n}$.\\

  For the converse, assume that $\lim a_{n} = a$ exists. Consider some $\epsilon >0$, then there is some $N\in \N$ such that  $a_{n} \in (a-\epsilon, a+\epsilon)$ for all $n\geq N$. Consider some $k\geq N$, then $y_{k}$ and $x_{k}$ are the gratest lower and lowest upper bounds of $\left\{ a_{n}:n\geq k \right\}$, respectively. This implies that $a-\epsilon < y_{k} \leq a_{k}$ and $a_{k} \leq x_{k} < a+\epsilon$. Hence, $y_{k},x_{k}\in \left( a-\epsilon, a+\epsilon \right)$ for all $k\geq N$ and so $\lim\inf a_{n} = \lim\sup a_{n} = \lim a_{n} = a$.
\end{proof}
\end{enumerate}
\end{problem}
\begin{problem}{2.4.10 (Infinite Products)}
  A close relative of infinite series is the \textit{infinite product}
\begin{align*}
  \prod_{n=1}^{\infty} b_{n} = b_{1}b_{2}b_{3}\dots
\end{align*}
which is understood in terms of its sequence of \textit{partial products}
\begin{align*}
  p_{m} = \prod_{n=1}^{m}b_{n} = b_{1}b_{2}b_{3}\dotsb_{m}.
\end{align*}
Consider the special class of infinite products of the form
\begin{align*}
  \prod_{n=1}^{\infty}(1+a_{n}) = (1+a_{1})(1+a_{2})(1+a_{3})\dots, \: \text{ where }a_{n}\geq 0.
\end{align*}
\begin{enumerate}
  \item Find an explicit formula for the sequence of partial products in the case where $a_{n} = 1/n$ and decide whether the sequence converges. Write out the first few terms in the sequence of partial products in the case where $a_{n} = 1/n^{2}$ and make a conjecture about the convergence of this sequence.
\begin{solution}
  We consider the case where $a_{n} = 1/n$. Then, let's define the sequence $(x_{n})$ of partial products by  
\begin{align*}
  x_{n} &= \prod_{m=1}^{n} (1+1/m) = \prod_{m=1}^{n}\left( \frac{m+1}{m} \right)\\
  &= \frac{2}{1}\cdot \frac{3}{2}\cdot \frac{4}{3}\cdot \dots\cdot \frac{n}{n-1}\cdot \frac{n+1}{n} = n+1
\end{align*}
Clearly, this sequence diverges. Now, let's consider the case when $a_{n} = \frac{1}{n^{2}} = \left( \frac{1}{n} \right)^{2}$. The first 6 terms are $(2,2.5,2.7777,2.951,3.069, 3.154,\dots)$. It seems that this sequence slowly diverges, however, calculating the first 6000 terms with a program, it seems that the sequence approaches 3.675443. \\

Another way to prove the divergence  when $a_{n} = 1/n$. 

Let the sequence $(x_{n})$ be defined by $a_{1} = 2$ and
\begin{align*}
  a_{n+1} = a_{n}\left( 1+\frac{1}{n+1} \right) = a_{n}\left( \frac{n+2}{n+1} \right).
\end{align*}
Clearly, this is an increasing sequence and $1<x_{n}$ for all $n\in \N$. However, it is not bounded above by any real number. Consider any real number $y$ such that $x_{n} <y$ for some $n\in \N$. By the Archimedean principle, there is some positive integer $b$ such that $y<x_{n}b$. Now, consider the positive integer $k=(n+1)b$ and observe that
\begin{align*}
  x_{k} &= x_{n}\left( \frac{n+2}{n+1} \right)\left( \frac{n+3}{n+2} \right)\left( \frac{n+4}{n+3} \right)\dots\left( \frac{k}{k-1} \right)\left( \frac{k+1}{k} \right)\\
  &= x_{n}\left( \frac{k+1}{n+1} \right) = x_{n}\left( b+\frac{1}{n+1} \right) > x_{n} b > y. 
\end{align*}
Hence, $(x_{n})$ diverges and so $\prod_{n=1}^{\infty}(1+a_{n})$ diverges.
\end{solution}
  \item Show, in general, that the sequence of partial products converges if and only if $\sum_{n=1}^{\infty}a_{n}$ converges. (The inequality $1+x\leq 3^{x}$ for positive $x$ will be useful in one direction.)
\begin{proof}
  First, assume that the sequence of partial products $\left(\prod_{i=1}^{n} (1+a_{i})\right)$ converges. Note that $\left( \sum_{i\leq n} a_{i} \right)=(y_{n})$ is an increasing sequence since $0\leq a_{n}$. Furthermore, observe that 
\begin{align*}
  \prod_{i=1}^{n}(1+a_{i}) &= (1+a_{1})(1+a_{2})(1+a_{3})\dots(1+a_{n})\\
    &= \sum_{i=1}^{n}a_{i} + b +1 > \sum_{i=1}^{n}a_{i}
\end{align*}
for som $0\leq b$. Hence, $(y_{n})$ is bounded above and so it converges.\\

For the converse, suppose that $\sum_{n\in \N} a_{n}$ converges. Hence, the sequence of partial sums $\left( \sum_{i\leq n}a_{i} \right)$ converges and so it is bounded by some $M$. We know that $1+x\leq 3^{x}$ is true for every real number $x\geq 0$. Hence, for any positive integer $n$,
\begin{align*}
  \prod_{i=1}^{n}(1+a_{i}) &\leq \prod_{i=1}^{n} 3^{a_{i}}\\
  &= 3^{\sum_{i=1}^{n}a_{i}}\leq 3^{M}
\end{align*}
since $a_{n}\geq 0$ for every $n\in \N$. Furthermore, $\left( \prod_{i=1}^{n}(1+a_{i}) \right)$ is an increasing sequence and so, by the Monotone Bounded Theorem, it converges.
\end{proof}
\end{enumerate}
\end{problem}
\end{document}


