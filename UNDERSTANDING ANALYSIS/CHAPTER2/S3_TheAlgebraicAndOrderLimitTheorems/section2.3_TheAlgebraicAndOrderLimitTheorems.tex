\documentclass[12pt]{article}
\usepackage[margin=1in]{geometry}
\usepackage{amsmath, amsfonts,amsthm,amssymb,epigraph,etoolbox,mathtools,setspace,enumitem}  
\usepackage{tikz}
\usetikzlibrary{datavisualization} 
\usepackage[makeroom]{cancel} 
\usepackage[linguistics]{forest}
\usetikzlibrary{patterns}
\newcommand{\N}{\mathbb{N}}
\newcommand{\Z}{\mathbb{Z}}
\newcommand{\R}{\mathbb{R}}
\newcommand{\Q}{\mathbb{Q}}
\newcommand{\Mod}[1]{\ (\mathrm{mod}\ #1)}
\newcommand{\Lim}[1]{\mathrm{lim}\left( #1\right)}
\newcommand{\Abs}[1]{\left\vert #1 \right\vert}

\newlist{legal}{enumerate}{10}
\setlist[legal]{label=(\alph*)}
\setenumerate[legal]{label=(\alph*)}

\DeclarePairedDelimiter\bra{\langle}{\rvert}
\DeclarePairedDelimiter\ket{\lvert}{\rangle}
\DeclarePairedDelimiterX\braket[2]{\langle}{\rangle}{#1\delimsize\vert #2}


\newenvironment{theorem}[2][Theorem]{\begin{trivlist} \item[\hskip \labelsep {\bfseries #1}\hskip \labelsep {\bfseries #2.}]}{\end{trivlist}}
\newenvironment{lemma}[2][Lemma]{\begin{trivlist} \item[\hskip \labelsep {\bfseries #1}\hskip \labelsep {\bfseries #2.}]}{\end{trivlist}}
\newenvironment{result}[2][Result]{\begin{trivlist} \item[\hskip \labelsep {\bfseries #1}\hskip \labelsep {\bfseries #2.}]}{\end{trivlist}}
\newenvironment{exercise}[2][Exercise]{\begin{trivlist} \item[\hskip \labelsep {\bfseries #1}\hskip \labelsep {\bfseries #2.}]}{\end{trivlist}}
\newenvironment{problem}[2][Problem]{\begin{trivlist} \item[\hskip \labelsep {\bfseries #1}\hskip \labelsep {\bfseries #2.}]}{\end{trivlist}}
\newenvironment{question}[2][Question]{\begin{trivlist} \item[\hskip \labelsep {\bfseries #1}\hskip \labelsep {\bfseries #2.}]}{\end{trivlist}}
\newenvironment{corollary}[2][Corollary]{\begin{trivlist} \item[\hskip \labelsep {\bfseries #1}\hskip \labelsep {\bfseries #2.}]}{\end{trivlist}}
\newenvironment{solution}[1][Solution]{\begin{trivlist} \item[\hskip \labelsep {\bfseries #1}]}{\end{trivlist}}

\setlength\epigraphwidth{8cm}
\setlength\epigraphrule{0pt}

\makeatletter
\patchcmd{\epigraph}{\@epitext{#1}}{\itshape\@epitext{#1}}{}{}
\makeatother

\begin{document}
  
\title{Section 2.3: The Algebraic and Order Limit Theorems}
   \author{Juan Patricio Carrizales Torres}
     \date{Jul 25, 2022}
       \maketitle 
       Now that we have a more formal and clearer definition of convergence for sequences, we can check some properties that come with this meaning. Namely, algebraic and order properties. Before that, we must mention an important theorem, which says that for any convergent sequence $(a_{n})$, there is some real number $M$ such that $(a_{n})$ is inside $[-M,M]$. Namely, $|a_{n}|\leq M$ for all $n\in \N$.
       Let $(a_{n})\to a$ and $(b_{n})\to b$. The algebraic property states the following: 
       \begin{enumerate}[label=(\roman*)]
	 \item $\Lim{ca_{n}} = ca$
	 \item $\Lim{a_{n}+b_{n}} = a + b$
	 \item $\Lim{a_{n}b_{n}} = ab$
	 \item $\Lim{a_{n}/b_{n}} = a/b$, if $b\neq0$.
    \end{enumerate}
    THe interesting thing about the arguments given by the author to prove them is that they use the fact that one can make $|a_{n} -a|$ as small as one wants, namely, for any positive real number $\varepsilon$ as small as one can imagine, there is some $a_{n}$ such that $|a_{n}-a|<\varepsilon$. Also, these properties helps us interact with combinations of sequences and their limits on a more ``familiar'' way. On the other hand, the order property states the following:
       \begin{enumerate}[label=(\roman*)]
	 \item If $a_{n}\geq 0$ for all $n\in \N$, then $a\geq0$.
	 \item If $a_{n}\leq b_{n}$ for all $n\in\N$, then $a\leq b$.
	 \item If there exists $c\in \R$ for which $c\leq b$ for all $n\in \N$, then $c\leq b$. Similarly, if $a_{n}\geq c$ for all $n\in \N$, then $a\geq c$. 
    \end{enumerate}
    Noteworthy, if one changes the initial assumption from \textit{for all} for \textit{infinitely many}, the theorem holds. In other words, the property of a finite amount of elements in a sequence is not sufficient to predict the general property of its limit. The first $10^{10^{100}}$ elements can be positive but the rest be negative, which means that the sequence \textit{eventually} aquires the property of negativity. 
    \begin{problem}{2.3.1}
      Let $x_{n}\geq 0$ for all $n\in \N$.
      \begin{enumerate}[label=(\alph*)]
	\item If $(x_{n})\to 0$, show that $(\sqrt{x_{n}})\to0$.
    \begin{proof}
      Observe that $|x_{n}-0|=|x_{n}|$. Now, consider some positive real number $\epsilon$. Since $(x_{n})\to 0$, there is some $N\in \N$ such that $|x_{n}|=x_{n} < \epsilon^{2}$ for all $n\geq N$ ($0\leq x_{n}$). Then, $\sqrt{x_{n}} < \sqrt{\epsilon^{2}}=\epsilon$. Because $\sqrt{x_{n}}\geq 0$, it follows that $\sqrt{x_{n}}=\Abs{\sqrt{x_{n}}} = \Abs{\sqrt{x_{n}}-0}<\epsilon$ for all $n\geq N$. 
    \end{proof}
	\item If $(x_{n})\to x$, show that $(\sqrt{x_{n}})\to\sqrt{x}$.
    \begin{proof}
      Consider some positive real number $\epsilon$. Since $(x_{n}) \to 0$, it follows that there is some $N\in \N$ such that $|x_{n}-x| < \epsilon^{2}$ for all $n\geq N$. Thus, $\sqrt{\epsilon^{2}} > \sqrt{|x_{n}-x|} \geq |\sqrt{x_{n}}-\sqrt{x}|$, and so  $|\sqrt{x_{n}}-\sqrt{x}|<\epsilon$ for all $n\leq N$. Therefore, $(\sqrt{x_{n}})\to \sqrt{x}$.
    \end{proof}
    \end{enumerate}
    \end{problem}
    \begin{problem}{2.3.2}
      Using only Definition 2.2.3 (definition of convergence), prove that if $(x_{n})\to 2$, then 
    \begin{enumerate}
      \item $\left( \frac{2x_{n}-1}{3} \right)\to 1$
    \begin{proof}
      First, note that $\Abs{\frac{2x_{n}-1}{3}-1} = \Abs{\frac{2x_{n}-4}{3}}$. Now, consider some positive real number $\epsilon$. Then, there is some $N\in \N$ such that $\Abs{x_{n}-2}<\frac{3}{2} \epsilon$ for all $n\geq N$. Observe that
    \begin{align*}
    \frac{2}{3}\Abs{x_{n}-2} &= \Abs{\frac{2}{3}}\Abs{x_{n}-2}\\
  &= \Abs{\frac{2}{3}x_{n}-\frac{4}{3}} = \Abs{\frac{2x_{n}-4}{3}}\\
  &< \epsilon.
    \end{align*}
    Therefore, $\Abs{\frac{2x_{n}-1}{3}-1}<\epsilon$ for all $n\geq \N$ and so $\left( \frac{2x_{n}-1}{3} \right)\to 1$. 
    \end{proof}
      \item $\left( \frac{1}{x_{n}} \right)\to 1/2$
    \begin{proof}
      Since $(x_{n})\to 2$, it follows that there is an infinity of nonzero $x_{n}$ that eventually get nearer and nearer to $2$. Then, let's consider all $x_{n} \neq 0$ in the sequence $\left( \frac{1}{x_{n}} \right)$. \\
      Note that $\Abs{\frac{1}{x_{n}}-1/2} = \Abs{\frac{2-x_{n}}{2x_{n}}} = \frac{\Abs{2-x_{n}}}{2\Abs{x_{n}}}$. Now, let $\epsilon$ be any positive real number. Then, there is some $K_{1}\in \N$ such that $|x_{n}-2| < 1$, which implies that $|x_{n}| > 1$. Furthermore, there is a positive integer $K_{2}$ such that $|x_{n}-2| < 2\epsilon$ for all $n\geq K_{2}$. Thus, 
    \begin{align*}
      \Abs{2-x_{n}}\frac{1}{2|x_{n}|} < \Abs{2-x_{n}}\frac{1}{2} < 2\epsilon\;\frac{1}{2} 
    \end{align*}
    for any $n\geq M$, where $M=\mathrm{max}(K_{1},K_{2})$. Then, $\Abs{\frac{1}{x_{n}}-1/2}<\epsilon$ for all $n\geq M$. Hence, $\left( \frac{1}{x_{n}} \right)\to 1/2$.
    \end{proof}
    \end{enumerate}
    \end{problem}
    \begin{problem}{2.3.3 (Squeeze Theorem)}
      Show that if $x_{n} \leq y_{n} \leq z_{n}$ for all $n\in \N$, and if $\lim x_{n} = \lim z_{n} = l$, then $\lim y_{n} = l$ as well.
    \begin{proof}
      Since $(x_{n})\to l$ and $(z_{n}) \to l$, it follows that for some positive real numbr $\epsilon$, there is some positive integer $K$, such that $|x_{n}-l|<\epsilon$ and $|z_{n}-l|<\epsilon$ for all $n\geq K$. Hence, $-\epsilon< x_{n}-l < \epsilon$  and $-\epsilon < z_{n} -l <\epsilon$, and so $l-\epsilon < x_{n},z_{n} < l+\epsilon$. Since $x_{n} \leq y_{n} \leq z_{n}$, it follows that $l-\epsilon < x_{n} \leq y_{n} \leq z_{n} < l+\epsilon$. Thus, $|y_{n}-l| < \epsilon$ for each $n\geq K$, namely, $(y_{n}) \to l$.  
    \end{proof}
    \end{problem}
    \begin{problem}{2.3.4} 
      Let $(a_{n})\to 0$, and use the Algebraic Limit Theorem to compute each of the following limits (assuming the fractions are always defined):
    \begin{enumerate}
      \item $\Lim{\frac{1+2a_{n}}{1+3a_{n}-4a_{n}^{2}}}$
    \begin{solution}
      In the denominator and numerator, there is a $1$ being added. This is constant for all $n\in \N$. Hence, one can define the sequence $(1)$ (all elements are 1), which $(1) \to 1$. Since $(2a_{n}) \to 2\cdot 0 = 0$, $(3a_{n}) \to 3\cdot 0 = 0$ and $(4a_{n}a_{n}) \to 4\cdot 0\cdot 0 = 0$, it follows that $\left( 1+2a_{n} \right)\to 1+0 = 1$ and $\left( 1+3a_{n}-4a_{n}^{4} \right)\to 1+ 0 -0= 1$. Thus, $\left( \frac{1}{1+3a_{n}-4a_{n}^{4}} \right)\to \frac{1}{1}=1$ (recall that all fractions are defined and so $1+3a_{n}-4a_{n}^{4}\neq 0$ for all $n\in \N$) and 
    \begin{align*}
      \Lim{\frac{1+2a_{n}}{1+3a_{n}-4a_{n}^{2}}} &= 1*1 = 1
    \end{align*}
    \end{solution}
      \item $\Lim{\frac{(a_{n}+2)^{2}-4}{a_{n}}}$
    \begin{solution}
      Note that  
    \begin{align*}
      \frac{(a_{n}+2)^{2}-4}{a_{n}} &= \frac{a_{n}^{2}+4a_{n}}{a_{n}}\\
      &= a_{n} + 4
    \end{align*}
    for all $n\in \N$. Thus, $(a_{n}+4) \to 0+4 = 4$.
    \end{solution}
      \item $\Lim{\frac{\frac{2}{a_{n}}+3}{\frac{1}{a_{n}}+5}}$
    \begin{solution}
      Note that
    \begin{align*}
      \frac{\frac{2}{a_{n}}+3}{\frac{1}{a_{n}}+5} &= \frac{\frac{2+3a_{n}}{a_{n}}}{\frac{1+5a_{n}}{a_{n}}}\\
      &= \frac{2+3a_{n}}{1+5a_{n}}
    \end{align*}
    for all $n\in \N$. Also, $(2+3a_{n})\to 2+3\cdot0 = 2$ and $(1+5a_{n})\to 1+5\cdot0 = 1$. Thus, 
    \begin{equation*}
      \Lim{\frac{\frac{2}{a_{n}}+3}{\frac{1}{a_{n}}+5}} = \Lim{\frac{2+3a_{n}}{1+5a_{n}}} = 1/1=1.
    \end{equation*}
    \end{solution}
    \end{enumerate}
    \end{problem}
    \begin{problem}{2.3.5}
      Let $(x_{n})$ and $(y_{n})$ be given, and define $(z_{n})$ to be the ``shuffled'' sequence $(x_{1},y_{1},x_{2},y_{2},x_{3},y_{3},\dots,x_{n},y_{n},\dots)$. Prove that $z_{n}$ is convergent if and only if $(x_{n})$ and $(y_{n})$ are both convergent with $\Lim{x_{n}} = \Lim{y_{n}}$.
    \begin{proof}
      Assume that $(x_{n})$ and $(y_{n})$ are both convergent with $\Lim{x_{n}} = \Lim{y_{n}}=M$, where $M \in \R$. We show that $(z_{n}) \to M$. Consider some positive real number $\epsilon$. By definition of convergence, there is some $K\in \N$ such that $|x_{n}-M|,|y_{n}-M|<\epsilon$ for all $n\geq K$. Thus, $|z_{n} - M| < \epsilon$ for all $n\geq K$.\\
      For the converse, assume that $(z_{n})$ is convergent. Specifically, $(z_{n}) \to M$ for some real number $M$. Consider some positive real number $\epsilon$. Then, there is some $K\in \N$ such that $|z_{n} -M| < \epsilon$ for all $n\geq K$. By definition of $z_{n}$, it follows that $|x_{n} -M|,|y_{n}-M| < \epsilon$ for all $n\geq K$. Thus, $(x_{n})\to M$ and $(y_{n})\to M$.
    \end{proof}
    \end{problem}
    \begin{problem}{2.3.6}
      Consider the sequence given $b_{n} = n-\sqrt{n^{2}+2n}$. Taking $(1/n)\to 0$ as given, and using both the Algebraic Limit Theorem and the result in Excercise 2.3.1, show $\Lim{b_{n}}$ exists and find the value of the limit.
    \begin{proof}
      Note that
    \begin{align*}
      b_{n} &= n-\sqrt{n^{2}+2n}\\
      &= \left( n-\sqrt{n^{2}+2n} \right)\frac{n+\sqrt{n^{2}+2n}}{n+\sqrt{n^{2}+2n}}\\
      &= \frac{\left( n^{2}-\left(n^{2}+2n\right) \right)}{n+\sqrt{n^{2}+2n}}\\
      &= \frac{-2n}{n+\sqrt{n^{2}\left( 1+\frac{2}{n} \right)}}\\
      &= \frac{-2n}{n\left( 1+\sqrt{1+\frac{2}{n}} \right)} = \frac{-2}{1+\sqrt{1+(2/n)}}
    \end{align*}
    for each $n\in \N$. Observe that $\left( 1+(2/n) \right)\to 1+2\cdot 0 = 1$. Thus, by \textbf{Excercise 2.3.1}, $\left( \sqrt{1+(2/n)} \right) \to \sqrt{1} = 1$ and so $\left( \frac{1}{1+\sqrt{1+(2/n)}} \right) \to 1/(1+1) =1/2$. Therefore, $(b_{n}) = \left(\frac{-2}{1+\sqrt{1+(2/n)}}\right)\to  -2(1/2) = -1$ and so the limit exists.
    \end{proof}
    \end{problem}
    \begin{problem}{2.3.7}
      Give an example of each of the following, or state that such a request is impossible by referencing the proper theorem(s):
    \begin{enumerate}
      \item sequences $(x_{n})$ and $(y_{n})$, which both diverge, but whose sum $(x_{n} + y_{n})$ converges;
    \begin{solution}
      Such sequences can be constructed. Let $(x_{n})$ and $(y_{n})$ be defined by $x_{n} = n$ and $y_{n} = -n$ for all $n\in \N$. Note that both $(x_{n})$ and $(y_{n})$ keep approaching objects that are not real numbers, namely, $\infty$ and $-\infty$ respectively. They diverge. However, $(z_{n}) = (x_{n}+y_{n}) = (n-n) = (0)\to 0$ converges.
    \end{solution}
      \item sequences $(x_{n})$ and $(y_{n})$, where $(x_{n})$ converges, $(y_{n})$ diverges, and $(x_{n} + y_{n})$ converges;
    \begin{proof}
      We prove that this leads to a contradiction. Let $(x_{n})$ be a convergent sequence and $(y_{n})$ be a divergent one. Then $(-x_{n})$ is convergent. Suppose that $(x_{n} + y_{n})$ converges and so $((x_{n}+y_{n})+(-x_{n}))=(y_{n})$ is the sum of two convergent sequences. Hence, $(y_{n})$ is a convergent sequence. This contradicts our initial assumption. 
    \end{proof}
      \item a convergent sequence $(b_{n})$ with $b_{n} \neq 0$ for all $n$ such that $(1/b_{n})$ diverges;
    \begin{solution}
      Let $(b_{n}) = (1/n)$. Then $(b_{n})\to 0$ and $b_{n} \neq 0$ for all $n\in \N$. However, $(1/b_{n}) = (n)$ diverges to $\infty$.
    \end{solution}
      \item an unbounded sequence $(a_{n})$ and a convergent sequence $(b_{n})$ with $(a_{n}-b_{n})$ bounded;
    \begin{proof}
      We show that this leads to a contradiction. Let $(a_{n})$ and $(b_{n})$ be unbounded and convergent sequences, respectively. Since $(b_{n})$ is convergent, it follows that it is bounded. Also, assume that $(a_{n}-b_{n})$ is bounded. Then, there are real numbers $M,C\geq 0$ such that $|b_{n}|\leq M$ and $|a_{n}-b_{n}|\leq C$ for any $n\in \N$. Thus,
    \begin{align*}
      M+C &\geq |a_{n}-b_{n}| + |b_{n}| \\
      &\geq |(a_{n}-b_{n})+b_{n}| = |a_{n}|
    \end{align*}
    for every positive integer $n$. Hence, $(a_{n})$ is a bounded sequence. This contradicts our initial assumption.
    \end{proof}
      \item two sequences $(a_{n})$ and $(b_{n})$, where $(a_{n}b_{n})$ and $(a_{n})$ converge but $(b_{n})$ does not.
    \begin{solution}
      Let $(a_{n}) = (0,0,0,0,0,0,0,0,\dots)$ and $(b_{n}) = (1,2,3,4,5,6,7,\dots)$. Then $(a_{n})\to 0$ and $(b_{n})$ does not converge to a real number. However, $(z_{n}) = (a_{n}b_{n}) = (0\cdot 1, 0\cdot 2, 0\cdot 3, \dots) = (0,0,0,0,0,0,0) \to 0$.
    \end{solution}
    \end{enumerate}
    \end{problem}
    \begin{problem}{2.3.8}
      Let $(x_{n})\to x$ and let $p(x)$ be a polynomial.
    \begin{enumerate}
      \item Show $p(x_{n}) \to p(x)$.
    \begin{proof}
      Note that a finite polynomial $p(x)$ of degree $k$ can be expressed as $\sum_{j=0}^{k} a_{j}x^{j}$, where $a_{j}$ are the real coefficients. Hence, each $p(x_{n})$ is a member of the sequence of real numbers $ \left( \sum_{j=0}^{k} a_{j}x_{n}^{j} \right)$. Since $(x_{n})\to x$, it follows, by the algebraic properties of limits of sequences, that $(a_{j}x_{n}^{j}) \to (a_{j}x^{j})$ for all $j$ and so $\left( \sum_{j=0}^{k} a_{j}x_{n}^{j} \right)\to \left( \sum_{j=0}^{k} a_{j}x^{j} \right) = p(x)$. Thus, $p(x_{n}) \to p(x)$.
    \end{proof}
      \item Find an example of a function $f(x)$ and a convergent sequence $(x_{n}) \to x$ where the sequence $f(x_{n})$ converges, but not to $f(x)$.
    \begin{solution}
      Let $f:\R\to \R$ be a function be defined by $f(x) = 0$ and $f(a)=1$ for the rest of $a\in \R$. Then, $f(x_{n})=1$ for all but finitely many $n\in \N$ and so $f(x_{n}) \to 1 \neq 0 = f(x)$.
    \end{solution}
    \end{enumerate}
    \end{problem}
    \begin{problem}{2.3.9}
    \begin{enumerate}
      \item Let $(a_{n})$ be a bounded (not necessarily convergent) sequence, and assume $\Lim{b_{n}} = 0$. Show that $\Lim{a_{n}b_{n}} = 0$. Why are we not allowed to use the Algebraic Limit Theorem to prove this?
    \begin{proof}
      Since $(a_{n})$ is bounded, we have that $|a_{n}|\leq M$ for some nonegative real number $M$. If $M=0$, then $a_{n} = 0$ for all $n\in \N$ and so $(a_{n}) \to 0$. This implies that $(a_{n}b_{n}) \to 0$ since $(b_{n})\to 0$. Therefore, we assume that $M$ is nonzero. Now, consider some positive real number $\epsilon$. Then, $\epsilon/M$ is positive and so there is some positive integer $N$ such that $|b_{n} -0| = |b_{n}| < \epsilon/M$ for all $n\geq N$. Note that $0\leq |b_{n}|$ for all $n\in \N$ and $0<M$, and so
    \begin{align*}
      |a_{n}||b_{n}| \leq M|b_{n}| < (\epsilon/M) M = \epsilon.
    \end{align*}
    for all $n\geq N$. Therefore, $|a_{n}||b_{n}| = |a_{n}b_{n}| = |a_{n}b_{n} - 0| < \epsilon$ for all $n\geq N$. Hence, $(a_{n}b_{n}) \to 0$.\\

  We were not allowed to use the Algebraic Limit Theorem since both $(a_{n})$ and $(b_{n})$ must converge, but in this case $(a_{n})$ is bounded but not necessarily convergent. 
    \end{proof}
      \item Can we conclude anything about the convergence of $(a_{n}b_{n})$ if we assume that $(b_{n})$ converges to some nonzero limit $b$?
    \begin{proof}

    \end{proof}
      \item Use (a) to prove Theorem 2.3.3, part (iii), for the case when $a=0$.
    \end{enumerate}
    \end{problem}
\end{document}


