\documentclass[12pt]{article}
\usepackage[margin=1in]{geometry}
\usepackage{amsmath, amsfonts,amsthm,amssymb,epigraph,etoolbox,mathtools,setspace,enumitem}  
\usepackage{tikz}
\usetikzlibrary{datavisualization} 
\usepackage[makeroom]{cancel} 
\usepackage[linguistics]{forest}
\usetikzlibrary{patterns}
\newcommand{\N}{\mathbb{N}}
\newcommand{\Z}{\mathbb{Z}}
\newcommand{\R}{\mathbb{R}}
\newcommand{\Q}{\mathbb{Q}}
\newcommand{\Mod}[1]{\ (\mathrm{mod}\ #1)}
\newcommand{\Abs}[1]{\left\vert #1 \right\vert}


\newlist{legal}{enumerate}{10}
\setlist[legal]{label=(\alph*)}

\DeclarePairedDelimiter\bra{\langle}{\rvert}
\DeclarePairedDelimiter\ket{\lvert}{\rangle}
\DeclarePairedDelimiterX\braket[2]{\langle}{\rangle}{#1\delimsize\vert #2}


\newenvironment{theorem}[2][Theorem]{\begin{trivlist} \item[\hskip \labelsep {\bfseries #1}\hskip \labelsep {\bfseries #2.}]}{\end{trivlist}}
\newenvironment{lemma}[2][Lemma]{\begin{trivlist} \item[\hskip \labelsep {\bfseries #1}\hskip \labelsep {\bfseries #2.}]}{\end{trivlist}}
\newenvironment{result}[2][Result]{\begin{trivlist} \item[\hskip \labelsep {\bfseries #1}\hskip \labelsep {\bfseries #2.}]}{\end{trivlist}}
\newenvironment{exercise}[2][Exercise]{\begin{trivlist} \item[\hskip \labelsep {\bfseries #1}\hskip \labelsep {\bfseries #2.}]}{\end{trivlist}}
\newenvironment{problem}[2][Problem]{\begin{trivlist} \item[\hskip \labelsep {\bfseries #1}\hskip \labelsep {\bfseries #2.}]}{\end{trivlist}}
\newenvironment{question}[2][Question]{\begin{trivlist} \item[\hskip \labelsep {\bfseries #1}\hskip \labelsep {\bfseries #2.}]}{\end{trivlist}}
\newenvironment{corollary}[2][Corollary]{\begin{trivlist} \item[\hskip \labelsep {\bfseries #1}\hskip \labelsep {\bfseries #2.}]}{\end{trivlist}}
\newenvironment{solution}[1][Solution]{\begin{trivlist} \item[\hskip \labelsep {\bfseries #1}]}{\end{trivlist}}

\setlength\epigraphwidth{8cm}
\setlength\epigraphrule{0pt}

\makeatletter
\patchcmd{\epigraph}{\@epitext{#1}}{\itshape\@epitext{#1}}{}{}
\makeatother

\begin{document}
  
\title{Section 2.2: The Limit of a Sequence}
   \author{Juan Patricio Carrizales Torres}
     \date{July 12, 2022}
       \maketitle
     
       The section started with some examples as arguments for the idea that our intuition regarding the properties and manipulations (i.e. reordering the terms, splitting it into finite sums) of finite sums are ambiguous in the field of infinite sums. In fact saying ''This infinite sum \textit{equals} \dots'' for any inifnite sum is kind of problematic. That's why, one must first check the concepts of sequences and convergence, which are very related to the infinite sums. 

       A \textit{sequence} in real analysis is defined as a function from $\N$ to $\R$, which orders some real numbers. Recall that a function is a relation and a sequence is a subset of $\N \times \R$. The contents of a  sequence define an ordered list of ordered real numbers. For instance, the list of ordered real numbers $(0,1,2,3,\dots)$ is defined by the sequence with rule $f(n) = n-1$. \\

       The ordered list of real numbers $(a_{n})$ defined by a sequence is said to converge to $a$ if for any real number $\varepsilon > 0$, there exists some $N\in \N$ such that for all $n\geq N$, $|a_{n}-a|< \varepsilon$. This states that as we go higher in the sequence at some point, the distance between the next elements and the limit $a$ gets eventually smaller and smaller approaching 0. The elements eventually get arbitrarily close to the limit $a$. Arbitrarily in the sense that we are choosing any positive real number $\varepsilon$ and eventually since there is some positive integer $N$ after which the sequence gets closer to the limit with a distance lower than $\varepsilon$.\\
       Also, one can define convergence in a topological manner, namely, an ordered list of real number $(a_{n})$ defind by some sequence is said to converge to $a$ if for every $\varepsilon$-neighbourhood $V_{\varepsilon}(a)$, there is some $N\in \N$ such that $a_{n} \in V_{\varepsilon}(a)$ for every $n\geq \N$. A neighbourhood $V_{\varepsilon}(a)$ is defined as
    \begin{align*}
      V_{\varepsilon}(a) &= \left\{x\in \R: |x-a|<\varepsilon \right\}\\
      &= (a-\varepsilon,a+\varepsilon).
    \end{align*}
    for some $a\in \R$ and positive real number $\varepsilon$. This definition says that every $\varepsilon$-neighbourhood $V_{\varepsilon}(a)$ contains all but finitely many elements of the sequence $(a_{n})$.
    \begin{problem}{2.2.1}
	What happens if we reverse the order of the quantifiers in Definition 2.2.3?
	
	\textit{Definition:} A sequence $(x_{n})$ \textit{verconges} to $x$ if
	\textit{there exists} some $\varepsilon > 0$ such that \textit{for all} $N\in \N$ it is true that $n\geq \N$ implies $|x_{n}-x|<\varepsilon$. 
	Give an example of a vercongent sequence. Is there an example of a vercongent sequnce that is divergent? Can a sequence verconge to two different values? What exactly is being described in this strange definition?
    \begin{solution}
      First, note that ``\textit{for all} $N\in\N$ it is true that $n\geq \N$ implies $|x_{n}-x|<\varepsilon$" is equivalent to "\textit{for all} $n\in\N$, $|x_{n}-x|<\varepsilon$". Now, a very simple example of a vercongent sequence is $(1,0,1,0,1,0,1,\dots)$, namely, the number 1 (0) is assigned to the odd (even) places. Now consider any real number $a$ and let $m=\text{max}(\left\{|0-a|,|1-a|\right\})$. Thus, $0\leq |x_{n}-a|\leq m< m+1$ for all $n\in\N$ and so $m+1$ is the $\varepsilon$ that we were looking for. In fact, this sequence is divergent and verconges to any real number. The interesting part of this definition is that it suggests that if some sequence $(x_{n})$ verconges to some real value $a$, then it can be ``enclosed'' inside some $\varepsilon$-neighborhood $V_{\varepsilon}(a)$. Actually, one could further conjecture that if one can show that $m=\text{sup}(\left\{|x_n-a|:n\in\N\right\}$ exists for some real number $a$ and sequence $(a_{n})$, then the sequence verconges to any real value by mere adjustement of $\varepsilon$.  
    \end{solution}
    \end{problem}

    \begin{problem}{2.2.2}
      Verify, using the dfinition of convergence of a sequence, that the following sequence converge to the proposed limit.
      \begin{enumerate}[label=(\alph*)]
	\item $\lim \frac{2n+1}{5n+4}=\frac{2}{5}$.
    \begin{solution}
     Consider some real number $\varepsilon>0$. Note that
    \begin{align*}
      \left\vert\frac{2n+1}{5n+4}-\frac{2}{5}\right\vert &= \left\vert\frac{(10n+5)-(10n+8)}{5(5n+4)}\right\vert\\
      &= \left\vert\frac{-3}{25n+20}\right\vert = \frac{3}{25n+20},
    \end{align*}
    for every positive integer $n$. Consider some $N\in \N$ such that $\frac{3}{\varepsilon}<N$ and so $\frac{1}{N} < \frac{\varepsilon}{3}$. Thus, $\frac{3}{25N+20}<\frac{3}{N} <\varepsilon$ and so $\frac{3}{25n+20}\leq \frac{3}{25N+20} <\varepsilon$ for any $n\geq N$.
    \end{solution}
	\item $\lim \frac{2n^{2}}{n^{3}+3}=0$.
    \begin{solution}
      Consider some real number $\varepsilon>0$. Note that $\Abs{\frac{2n^{2}}{n^{3}+3} - 0} = \frac{2n^{2}}{n^{3}+3}$ for any postive integer $n$. Then, $\frac{2n^{2}}{n^{3}+3} < \frac{2n^{2}}{n^{3}} = \frac{2}{n}$. Then, choose som $N\in \N$ such that $N>\frac{2}{\varepsilon}$, which further implies that $\frac{1}{N}<\frac{\varepsilon}{2}$ and so 
    \begin{align*}
      \frac{2n^{2}}{n^{3}+3} < \frac{2}{n}\leq \frac{2}{N}<\varepsilon
    \end{align*}
    for every $n\geq N$.
    \end{solution}
	\item $\lim \frac{\sin(n^{2})}{\sqrt[3]{n}}=0$.
	  \begin{solution}
	    Consider some positive real number $\varepsilon$. Note that $\Abs{\frac{\sin(n^{2})}{\sqrt[3]{n}}-0} = \frac{\Abs{\sin(n^{2})}}{\sqrt[3]{n}}\leq \frac{1}{\sqrt[3]{n}}$. Consider some $N\in \N$ such that $N>\frac{1}{\varepsilon^{3}}$. Now, let $n\geq N$. Hence, $n>\frac{1}{\varepsilon^{3}}$ and so $\sqrt[3]{n}>\frac{1}{\varepsilon}$. Therefore, $\frac{\Abs{\sin(n^{2})}}{\sqrt[3]{n}}\leq \frac{1}{\sqrt[3]{n}}< \varepsilon$. Thus, $|x_{n}-0|<\varepsilon$ for all $n\geq N$.
    \end{solution}
    \end{enumerate}
    \end{problem}
    \begin{problem}{2.2.4}
      Give an example of each or state that the request is impossible. For any that are imposible, give a compelling argument for why that is the case.
      \begin{enumerate}[label=(\alph*)]
	\item A squence with an infinite number of ones that does not converge to one.
    \begin{solution}
      This is possible. Define a sequence $(a_{n})$ by $a_{n}=1$ if $n$ is odd and $a_{n} =0$ if $n$ is even. The sequence constantly alternates between $1$ and $0$.
    \end{solution}
  \item A sequence with an infinite number of ones that converges to a limit not equal to one.
    \begin{solution}
      There is no such sequence. Since $a_{n}=1$ for infinitely many $n\in \N$, it follows for any $a\neq 1$ that $|a_{n}-a|=\varepsilon$ for some positive real number $\varepsilon$ and infinitely many $n\in \N$. Therefore, for some positive real number $r<\varepsilon$, for any $N\in \N$, there is some $n\geq N$ such that $r<|a_{n}-a|=\varepsilon$.    
    \end{solution}
  \item A divergent sequence such that for every $n\in \N$ it is possible to find $n$ consecutive ones somewhere in the sequence.
    \begin{solution}
      Let $S=(a_{n})$ be a sequence defined by $a_{1}=1$ and $a_{n}=a_{n-1}+n$ for $n\geq 2$. Then,
    \begin{equation*}
      S=(1,3,6,10,\dots).
    \end{equation*}
    Now, define the sequence $M=(b_{n})$ by $b_{n} = 0$ for $n\in S$ and $b_{k}=1$ for $k\in \N-S$. Then, 
    \begin{equation*}
      M=(0,1,0,1,1,0,1,1,1,0,\dots).
    \end{equation*}
        We first show that one can find $k$ consecutive ones in the sequence. Suppose that $k=1$, then $b_{1}=0, b_{2}=1,b_{3}=0$  and so there is 1 consecutive one. We may further assume that $k\geq 2$. Hence, $a_{k}=a_{k-1}+k$ and $a_{k+1}=a_{k}+k+1$. Note that $a_{k+1}-a_{k} = k+1$ and so in between $a_{k+1}$ and $a_{k}$ there are $k$ consecutive positive integers such that they are not elements of $S$. Thus, these $k$ spaces are ones. Also, $M$ is divergent, since the sequence constantly alternates between $1$ and $0$.
    \end{solution}
    \end{enumerate}
    \end{problem}
    \begin{problem}{2.2.5}
      Let $\left[ \left[ x \right] \right]$ be the greatest integer less than or equal to $x$. For example, $\left[ \left[ \pi \right] \right]=3$ and $\left[ \left[ 3 \right] \right]=3$. For each sequence, find $\lim a_{n}$ and verify it with the definition of convergence.
      \begin{enumerate}[label=(\alph*)]
      \item $a_{n}=\left[ \left[ 5/n \right] \right]$
    \begin{solution}
      Note that $0<5/n$ for every $n\in\N$ and for $k\geq 6$, we have that $0<5/k<1$. This implies that $\left[ \left[ 5/k \right]\right]=0$ for every positive integer $k\geq 6$. Therefore, for any positive real number $\varepsilon$, we have that $|a_{n}-0| = 0 < \varepsilon$ for every $n\geq 6$. Thus, $\lim a_{n}=0$.
    \end{solution}
      \item $a_{n} = \left[ \left[ \left( 12+4n \right)/3n \right] \right]$
    \begin{solution}
      Note that $\left( 12+4n \right)/3n = 12/3n + 4/3$ and so it makes sense that $a_{n}$ approaches $4/3$ as $n$ increases. For instance, $1<a + 4/3<2$ for $0<a < 2/3$, where $a=12/3n$. Hence, for every positive integer $n>6$, $[[12/3n+4/3]] = 1$. Thus, for any positive real number $\varepsilon$, for every positive integer $n\geq 6$, we have that $|a_{n} - 1| = 0 < \varepsilon$. Therefore, $\lim a_{n} = 1$.    
    \end{solution}
    \end{enumerate}
    The last example gives some insight for a possible conjecture, namely, $\lim a_{n} + \lim b_{n} = \lim(a_{n}+b_{n})$, for convergent sequences $(a_{n})$ and $(b_{n})$.
    \end{problem}
    \begin{problem}{2.2.6}
      Prove Theorem 2.2.7. To get started, assume $(a_{n})\to a$ and also that $(a_{n})\to b$. Now argue $a=b$.
    \begin{solution}
      The limit of a sequence, when it exists, must be unique.
    \begin{proof} 
      Let's try to prove it directly. Let $(a_{n})$ be some sequence such that $(a_{n})\to a$ and $(a_{n})\to b$. Consider some positive real number $\varepsilon$. Then, $\varepsilon/2$ is a positive real number and so there are $M,N\in \N$ such that $|a_{n_{1}}-a|,|a_{n_{2}}-b| < \varepsilon/2$ for all $n_{1}\geq M$ and $n_{2}\geq N$. Therefore, there is some $k\geq M,N$ such that $|a_{k}-a|,|a_{k}-b|< \varepsilon/2$ and so $|a_{k}-a|+|a_{k}-b| < \varepsilon$. \\
      Note that
    \begin{align*}
      |(a_{k}-a)-(a_{k}-b)| &= |(a_{k}-a)+(-(a_{k}-b))|\\
      &\leq |a_{k}-a|+|-(a_{k}-b)| = |a_{k}-a|+|a_{k}-b|.
    \end{align*}
    Since $|(a_{k}-a)-(a_{k}-b)| = |b-a|$, it follows that $|b-a|<\varepsilon$, for any positive real number $\varepsilon$. Therefore, $|b-a|=0$ and so $b=a$. 
    \end{proof}
    \end{solution}
    \end{problem}
    \begin{problem}{2.2.7}
      Here are two useful definitions:
      \begin{enumerate}
	\item A sequence $(a_{n})$ is \textit{eventually} in a set $A\subseteq \R$ if there exists an $N\in \N$ such that $a_{n}\in A$ for all $n\geq N$.
	\item A sequence $(a_{n})$ is \textit{frequently} in a set $A\subseteq \R$ if, for every $N\in \N$, there exists an $n\geq \N$ such that $a_{n}\in A$.
	  \begin{enumerate}[label=(\alph*)]
	    \item Is the sequence $(-1)^{n}$ eventually or frequently in the set $\left\{ 1 \right\}$?
    \begin{solution}
      It is frequently in $\left\{ 1 \right\}$. Consider some $N\in \N$. If $N$ is even, then $(-1)^{N} = 1\in \left\{ 1 \right\}$. On the other hand, if it is odd, then $(-1)^{N+1} = 1\in \left\{ 1 \right\}$.
    \end{solution}
	    \item Which definition is stronger? Does frequently imply eventually or does eventually imply frequently?
    \begin{solution}
      Eventually implies frequently. Consider some sequence $(a_{n})$ such that it is eventually in a set $A\subseteq \R$. Then, there is some $N\in \N$ such that $a_{n}\in A$ for all $n\geq N$. Let $k\in \N$. If $k\leq N$, then $a_{N}\in A$. On the other hand, if $k> N$, then $a_{k}\in A$. 
    \end{solution}
	    \item Give an alternative rephrasing of Defintion 2.2.3 using either frequently or eventually. Which is the term we want?
    \begin{solution}
      A sequence $(a_{n})$ converges to a real numbr $a$ if, for every positive real number $\varepsilon$, $(a_{n})$ is eventually in $A=(a-\varepsilon,a+\varepsilon)$. 
    \end{solution}
	     \item Suppose an infinite number of terms of a sequence $(x_{n})$ are equal to 2. Is $(x_{n})$ necessarily eventually in the interval $(1.9,2.1)$? Is it frequently in $(1.9,2.1)$?
    \begin{solution}
      It is not necessarily eventually in $(1.9,2.1)$. For instance, consider the sequence $(x_{n})=(1,2,1,2,1,2,1,2,1,2,\dots)$. On the other hand, it is necessarily frequently in $(1.9,2.1)$, since for any $N\in \N$, there is some $n\geq N$ such that $x_{n} = 2$. Otherwise, there would be a finite quantity of $2$s, namely, some quantity lower or equal to $N$.
    \end{solution}
    \end{enumerate}
    \end{enumerate}
    \end{problem}
       \end{document}


