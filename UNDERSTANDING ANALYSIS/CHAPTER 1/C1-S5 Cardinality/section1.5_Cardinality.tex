\documentclass[12pt]{article}
\usepackage[margin=1in]{geometry}
\usepackage{amsmath, amsfonts,amsthm,amssymb,epigraph,etoolbox,mathtools,setspace,enumitem,tkz-euclide}  
\usepackage{tikz}
\usetikzlibrary{datavisualization}
\usepackage[makeroom]{cancel} 
\usepackage[linguistics]{forest}
\usetikzlibrary{patterns}
\newcommand{\N}{\mathbb{N}}
\newcommand{\Z}{\mathbb{Z}}
\newcommand{\R}{\mathbb{R}}
\newcommand{\Q}{\mathbb{Q}}
\newcommand{\Mod}[1]{\ (\mathrm{mod}\ #1)}  



\newlist{legal}{enumerate}{10}
\setlist[legal]{label*=\arabic*.}

\DeclarePairedDelimiter\bra{\langle}{\rvert}
\DeclarePairedDelimiter\ket{\lvert}{\rangle}
\DeclarePairedDelimiterX\braket[2]{\langle}{\rangle}{#1\delimsize\vert #2}


\newenvironment{theorem}[2][Theorem]{\begin{trivlist}
		\item[\hskip \labelsep {\bfseries #1}\hskip \labelsep {\bfseries #2.}]}{\end{trivlist}}
\newenvironment{lemma}[2][Lemma]{\begin{trivlist}
		\item[\hskip \labelsep {\bfseries #1}\hskip \labelsep {\bfseries #2.}]}{\end{trivlist}}
\newenvironment{result}[2][Result]{\begin{trivlist}
		\item[\hskip \labelsep {\bfseries #1}\hskip \labelsep {\bfseries #2.}]}{\end{trivlist}}
\newenvironment{exercise}[2][Exercise]{\begin{trivlist}
		\item[\hskip \labelsep {\bfseries #1}\hskip \labelsep {\bfseries #2.}]}{\end{trivlist}}
\newenvironment{problem}[2][Problem]{\begin{trivlist}
		\item[\hskip \labelsep {\bfseries #1}\hskip \labelsep {\bfseries #2.}]}{\end{trivlist}}
\newenvironment{question}[2][Question]{\begin{trivlist}
		\item[\hskip \labelsep {\bfseries #1}\hskip \labelsep {\bfseries #2.}]}{\end{trivlist}}
\newenvironment{corollary}[2][Corollary]{\begin{trivlist}
		\item[\hskip \labelsep {\bfseries #1}\hskip \labelsep {\bfseries #2.}]}{\end{trivlist}}
\newenvironment{solution}[2][Solution]{\begin{trivlist}
		\item[\hskip \labelsep {\bfseries #1}\hskip \labelsep {\bfseries #2.}]}{\end{trivlist}}

\setlength\epigraphwidth{8cm}
\setlength\epigraphrule{0pt}

\makeatletter
\patchcmd{\epigraph}{\@epitext{#1}}{\itshape\@epitext{#1}}{}{}
\makeatother 


\begin{document} 
	
	\title{Section 1.5: Cardinality}
	\author{Juan Patricio Carrizales Torres}
	\date{May 31, 2022}
	\maketitle
	
	This chapter introduces to the concept of cardinality and the classification of infinite sets exclusively as either countable or uncountable ones. \\
	The term \textbf{Cardinality} accounts, hand-wavy speaking, for the "quantity" of elements of some set. This informal definition is intuitive for finite ones, however lacks clarity for infinite sets. Nevertheless, it gives some proper foundation to understand the human questioning that catalized it, namely, how can we compare the "size" of two infinite sets (Check Galileo's Paradox).\\
	 
	The equivalence relation $\sim$ is defined by $A\sim B$ for some sets $A$ and $B$, if they have the same cardinality, namely, the same "size". By definition, two sets $A$ and $B$ have the same cardinality if there exists some $f: A\to B$ that is \textit{one-to-one} and \textit{onto}. In fact, this function $f: A\to B$ represents the \textbf{1-1 correspondence} between $A$ and $B$, and so $A\sim B$. Basically, every element of $B$ is assigned to one unique element of $A$ and every element of $A$ is paired with one unique element of $B$. \\
	
	This is quite interesting since one can demonstarte that the the set of positive even integers has the same cardinality as $\N$!!! The parts are not necesarily "smaller" than the whole. \\  
	
	 Just like in physical sciences, one can use some "standard" or "norm" to compare sizes between two sets, namely, $A\sim C \wedge C\sim B \implies A\sim B$ (transitive property). A very useful "norm" is the set $\N$. In fact, for any infinite set $A$, if $A\sim \N$, then $A$ is \textbf{countable}. Three interesting theorems state the following:
	 \begin{enumerate}[label=(\alph*)]
	 	\item $A\subseteq C \wedge C\sim \N \implies A\sim \N $ or $A$ is finite.  
	 	\item For some sequence of $n$ countable sets $A_{1}, A_{2}, A_{3},\ldots, A_{n}$, the union $\bigcup_{i=1}^{n} A_{i}$ is countable.
	 	\item For some sequence of countable sets $\{B_{n}:n\in \N\}$, the union $\bigcup_{n\in \N} B_{n}$ is countable.
	 \end{enumerate}
 	Using the \textbf{Nested Interval Property}, one can show that $\R$ is not countable. That's why, one concludes that the uncountability of $\R$ is another consequence of the \textbf{Axiom of Completeness}. Since $\Q$ is countable and $\R = \Q\cup \mathbb{I}$ is uncountable, it follows that $\mathbb{I}$ is uncountable. Otherwise, it will lead to a contradiction (The union of two countable sets is countable). This is quite interesting!!! The set of irrational numbers is of greater "size" than the set of rational ones.  
	
	\begin{problem}{1.5.1}
		Finish the following proof for \textbf{Theorem 1.5.7}.\\
		Assume $B$ is a countable set. Thus, there exists $f:\N \to B$, which is 1-1 and onto. Let $A\subseteq B$ be an infinite subset of $B$. We must show that $A$ is countable.\\
		Let $n_{1} = min\{n\in \N:f(n)\in A\}$. As a start to a definition of $g:\N\to A$, set $g(1) = f(n_{1})$. Show how to inductively continue this process to produce a 1-1 function $g$ from $\N$ onto $A$.
		\begin{solution}{1.5.1}
			 By the \textbf{Well-ordering Principle}, we can define $n_{1} = \min\{n\in \N : f(n)\in A\}$.  Now, let $n_{2} = \min\left(\{n\in \N : f(n)\in A\}/\{n_{1}\}\right)$. In fact, let 
			 \begin{equation*}
			 n_{k} = \min\left(\{n\in \N : f(n)\in A\}/\{n_{1},n_{2},n_{3},\ldots,n_{k-1}\}\right)
			 \end{equation*}
		 	for any  integer $k\geq 2$. We first prove that $\{n_{k}:k\in \N\} = \{n\in \N : f(n)\in A\}$.
		 
		 	\begin{proof}
		 	Assume, to the contrary, that $\{n_{k}:k\in \N\} \neq \{n\in \N : f(n)\in A\}$. Then, either
		 	\begin{align*}
		 	 &\{n_{k}:k\in \N\} \not\subseteq \{n\in \N : f(n)\in A\} &\text{or}&   
		 	  &\{n_{k}:k\in \N\} \not\supseteq \{n\in \N : f(n)\in A\} 
		 	 \end{align*}
		 	 Consider the earlier. Then, there exists some integer $x$ such that  $n_{x}\not\in  \{n\in \N : f(n)\in A\}$. However, this contradicts the fact that $n_{x} = \min\left(\{n\in \N : f(n)\in A\}/\{n_{1},n_{2},n_{3},\ldots,n_{x-1}\}\right)$. Hence, we may assume that $\{n_{k}:k\in \N\} \not\supseteq \{n\in \N : f(n)\in A\}$. Then, there is some $x\in \{n\in \N : f(n)\in A\}$ such that $x\not\in \{n_{k}:k\in \N\}$. This implies that there is no $k\in \N$ such that $x = \min\left(\{n\in \N : f(n)\in A\}/\{n_{1},n_{2},n_{3},\ldots,n_{k-1}\}\right)$. Hence, $x>a$ for all $a\in \{n\in \N : f(n)\in A\}$. This means that $x\not\in \{n\in \N : f(n)\in A\}$, which is a contradiction. 
			 \end{proof}
	 
		 	 Let the function $g:\N \to A$ be defined by
		 	\begin{equation*}
		 		g(x) = f(n_{x})
		 	\end{equation*}
	 		for every $x\in \N$. Then, $g:\N \to A$ is a 1-1 function. 
	 		 
	 		\begin{proof}
	 			Since $\{n_{k}:k\in \N\} = \{n\in \N : f(n)\in A\}$ and $f:\N \to B$ is biijective, it follows for every $a\in A$ that $a = f(n_{l}) = g(l)$ for some positive integer $l$ (onto), and for every $k\in \N$, $f(n_{k}) = g(k)$ is equal to a unique element of $A$ (one-to-one).
	 		\end{proof}
		\end{solution}
	\end{problem}

	\begin{problem}{1.5.2}
		Review the proof of \textbf{Theorem 1.5.6} part (ii) showing that $\R$ is uncountable, and then find the flaw in the following erroneous proof that $\Q$ is uncountable:\\
		Assume, for contradiction, that $\Q$ is countable. Thus we can write $\Q = \{r_{1},r_{2},r_{3},\ldots\}$ and, as before, construct a nested sequence of closed intervals with $r_{n} \in I_{n}$. Our construction implies $\bigcap_{n=1}^{\infty} I_{n} = \emptyset$ while NIP implies $\bigcap_{n=1}^{\infty} I_{n} \neq \emptyset$. This contradiction implies $\Q$ must therefore be uncountable.  
		\begin{solution}{1.5.2}
			 It is true that $\bigcap_{n=1}^{\infty} I_{n} = \emptyset$, however, it is also true that \textbf{NIP} is a consequence of \textbf{AoC}. In this closed intervals, we are just considering elements of $\Q$ and we know it is not complete (It lacks the cut property).
		\end{solution}
	\end{problem}

	\begin{problem}{1.5.3}
		Use the following outline to supply proofs for the statements in \textbf{Theorem 1.5.8}.
		\begin{enumerate}[label=(\alph*)]
			\item First, prove statement (i) for two countable sets, $A_{1}$ and $A_{2}$. Example \textbf{1.5.3} (ii) may be a useful reference. Some technicalities can be avoided by first replacing $A_{2}$ with the set $B_{2} =A_{2}\backslash A_{1} = \{x\in A_{2}:x\not\in A_{1}\}$. The point of this is that the union $A_{1}\cup B_{2}$ is equal to $A_{1}\cup A_{2}$ and the sets $A_{1}$ and $B_{2}$ are disjoint. (What happens if $B_{2}$ is finite?). Then explain how the more general statement in (i) follows.
			\begin{solution}{(a)}
				We first show for any two countable sets $A_{1}$ and $A_{2}$ that their union is countable.
				\begin{proof}
					Since $A_{1}$ and $A_{2}$ are countable, it follows that there are biijective functions $f:\N \to A_{1}$ and $g:\N \to A_{2}$. Let $B_{2} = A_{2}\backslash A_{1}$ and so $A_{1}\cap B_{2} = \emptyset$. Also, consider the following sets $S_{1} = \{n\in \N: f(n)\in A_{1}\}$ and $S_{2} = \{n\in \N: g(n) \in B_{2}\}$ since $B_{2} \subseteq A_{2}$. If $B_{2} = \emptyset$, then $A_{1}\cup A_{2} = A_{1}$ and so $f:\N \to A_{1}\cup A_{2}$ is a biijective function and $A_{1}\cup A_{2}$ is countable. Hence, we may assume that $B_{2}\neq \emptyset$. By the \textbf{Well Ordering Principle}, $s_{2} = \min (S_{2})$ exists. Note that $S_{1} = \N$ and $1\leq s_{2}$.
					
					Therefore, there are two possible cases. If $B_{2}$ is finite with $k$ elements, then define the function $h:\N \to A_{1}\cup B_{2}$ by
					\begin{align*}
						h(n) \begin{cases}
							g(s_{2}+(n-1)), \text{ if }n\leq k\\
							f(n-k), \text{ if }n>k.
						\end{cases}
					\end{align*}
				Therefore, $A_{1}\cup B_{2} = A_{1}\cup A_{2}$ is countable.\\
				
				If $B_{2}$ is infinite, then define some function $h:\N \to A_{1}\cup B_{2}$ by
					\begin{equation*}
						h(n) = \begin{cases}
							f(n/2), \text{ if }n \text{ is even}.\\
							g(s_{2} + (n-1)/2), \text{ if }n \text{ is odd}.
						\end{cases}
					\end{equation*}
				Thus, $A_{1}\cup B_{2} = A_{1}\cup A_{2}$ is countable. 
				\end{proof}
			Maybe this proof is more complicated than it has to be, however, it's just to understand the "joints and bolts" of what's happening. We now show for $n\geq 2$ countable sets $A_{1},A_{2},A_{3},\ldots, A_{n}$ that
			\begin{equation*}
				\bigcup_{i=1}^{n} A_{i} 
			\end{equation*}
			is countable.
			\begin{proof}
				We proceed by induction. Since the union of any two countable sets is countable, then the statement is true for $n=2$. Suppose for $k\geq 2$ countable sets $A_{1},A_{2},A_{3},\ldots,A_{k}$ that $\bigcup_{i=1}^{k} A_{i}$ is countable. We show for $k+1$ countable sets that $B_{1},B_{2},B_{3},\ldots,B_{k+1}$ that $\bigcup_{i=1}^{k+1} B_{i}$ is countable. Note that
				\begin{align*}
					\bigcup_{i=1}^{k+1} B_{i} &= \left(\bigcup_{i=1}^{k} B_{i}\right)\cup B_{k+1}.
				\end{align*}
			Since $\bigcup_{i=1}^{k} B_{i}$ is a countable set according to our inductive hypothesis, it follows that we have a union of two countable sets. Therefore, $\bigcup_{i=1}^{k+1} B_{i}$ is countable.
			\end{proof}
			\end{solution}
			\item Explain why induction \textit{cannot} be used to prove part (ii) of \textbf{Theorem 1.5.8} from part (i).
			\begin{solution}{(c)}
				Because with induction we just can prove, in this case, that it is true for a finite quantity of countable sets (it is true for any positive integer $n\geq 2$). This always leads to finite cases. 
			\end{solution}
			\item Show how arranging $\N$ into the two-dimensional array
			\begin{center}
			\begin{tabular}{c c c c c c}
				1 & 3 & 6 & 10 & 15 & $\cdots$\\
				2 & 5 & 9 & 14 & $\cdots$&\\
				4 & 8 & 13 & $\cdots$ & & \\
				7 & 12 & $\cdots$ & & & \\
				11 & $\cdots$ & & & &\\
				$\vdots$ & & & & &
			\end{tabular}
			\end{center}
			leads to a proof of \textbf{Theorem 1.5.8} (ii).
			\begin{proof}
				Consider some sequence of countables sets $\{A_{n};n\in \N\}$. Since the sequence is countable and every set in it is countable, we can arrange the elements of every $S\in \{A_{n};n\in \N\}$ into the two-dimensional array
				\begin{center}
					\begin{tabular}{c c c c c c}
						$S_{11}$ & $S_{12}$ & $S_{13}$ & $S_{14}$ & $S_{15}$ & $\cdots$\\
						$S_{21}$ & $S_{22}$ & $S_{23}$ & $S_{24}$ & $\cdots$&\\
						$S_{31}$ & $S_{32}$ & $S_{33}$ & $\cdots$ & & \\
						$S_{41}$ & $S_{42}$ & $\cdots$ & & & \\
						$S_{51}$ & $\cdots$ & & & &\\
						$\vdots$ & & & & &,
					\end{tabular}
				\end{center}
			where $S_{ij}$ represents the $j$th element of the $i$th set. However, we know that we can arrange all elements of $\N$ in a same fashion, namely, 
				\begin{center}
				\begin{tabular}{c c c c c c}
					1 & 3 & 6 & 10 & 15 & $\cdots$\\
					2 & 5 & 9 & 14 & $\cdots$&\\
					4 & 8 & 13 & $\cdots$ & & \\
					7 & 12 & $\cdots$ & & & \\
					11 & $\cdots$ & & & &\\
					$\vdots$ & & & & &.
				\end{tabular}
			\end{center}
		Hence, the union of the sets in  $\{A_{n};n\in \N\}$ will be at most infinitely countable. In the case that they share elements in common, we just don't count the copies, that's why we say \textbf{AT MOST} infinitely countable.
			\end{proof}
		\end{enumerate}
	\end{problem}

\begin{problem}{1.5.4}
	\begin{enumerate}[label=(\alph*)]
		\item Show $(a,b) \sim \R$ for any interval $(a,b)$.
		\begin{proof}

		  \begin{figure}
                \centering
                \begin{tikzpicture}
%               \draw[very thin,color=gray] (-0.1,-1.1) grid (3.9,3.9);
                \draw[<->] (1.5,0) -- (7.6,0) node[right] {$\R$};
                \draw[<->,scale=0.7] (6.5,-5) -- (6.5,5) node[above] {$\R$};

                \draw [scale=0.7,dashed,domain=-6:6] plot (3,\x);
                \tkzDefPoint (2.1,0){A};
                \tkzLabelPoint[below left](A){$3$};
                \node at (A)[circle,fill,inner sep=1.5pt]{};

                \draw [domain=3.5:9.5,range=-0.5:0.5,smooth,scale=0.7]plot (\x,{(\x-6.5)/(abs(\x-6.5) - 3.5)});
                        \tkzDefPoint (4.55,0){C};
                \tkzLabelPoint[above right](C){$6.5$};
                \node at (C)[circle,fill,inner sep=1.5pt]{};

                \draw [scale=0.7,dashed,domain=-6:6] plot (10,\x);
                \tkzDefPoint (7,0){B};
                \tkzLabelPoint[below right](B){$10$};
                \node at (B)[circle,fill,inner sep=1.5pt]{};
                \end{tikzpicture}
                \label{fig:open}
                \caption{$(3,10)\sim \R$ USING $f(x)=(x-6.5)/(\vert x-6.5\vert - 3.5)$}
                \end{figure}

			Consider some open interval $(a,b)\subset \R$. Then, we can define a function $f:(a,b)\to \R$ that is \textit{one-to-one} and \textit{onto} by 
			\begin{equation*}
				f(x) = \frac{(x-\varphi)}{\vert x-\varphi\vert - \Delta},
			\end{equation*}
		where $\varphi = \frac{a+b}{2}$ and $\Delta = \frac{b-a}{2}$ represent the middle point and half the difference of $(a,b)$, respectively. For example, the function $f(x)=(x-6.5)/(\vert x-6.5\vert - 3.5)$ takes the interval $(3,10)$ onto $\R$ in a $1-1$ fashion (Figure \ref{fig:open}).
		

		Hence, $(a,b)\sim \R$ for any $a,b\in \R$. This can be proven using a little calculus, but I don't provide it since I lack theoretical knowledge in this field.
		\end{proof}
	
		\item Show that an unbounded interval like $(a,\infty) = \{x:x>a\}$ has the same cardinality as $\R$ as well.
		\begin{proof}
			Consider some open interval $(a,\infty)$ for some $a\in \R$. Then, we can define a function $g:(a,\infty) \to \R$ that is \textit{one-to-one} and \textit{onto} by
			\begin{equation*}
				g(x)=\log (x-a).
			\end{equation*}
			For example, the function $f(x)=\log(x-5)$ takes the interval $(5,\infty)$ onto $\R$ in a $1-1$ fashion (Figure \ref{fig:inf}).

		\begin{figure}
                \centering
                \begin{tikzpicture}
%               \draw[very thin,color=gray] (-0.1,-1.1) grid (3.9,3.9);
		  \draw[<->] (1.5,0) -- (7.6,0) node[right] {$\R$};
                \draw[<->,scale=0.7] (6.5,-5) -- (6.5,4) node[above] {$\R$};
        
                \draw [scale=0.7,dashed,domain=-6:6] plot (5,\x);
		\tkzDefPoint (3.5,0){A};
                \tkzLabelPoint[below left](A){$5$}; 
                \node at (A)[circle,fill,inner sep=1.5pt]{}; 
                
		\draw [domain=5.01:10.5,range=-0.5:0.5,smooth,scale=0.7]plot (\x,{ln(\x-5)});
                \end{tikzpicture}
                \label{fig:inf}
		\caption{$(5,\infty)\sim \R$ USING $f(x)=\log(x-5)$}
                \end{figure}
		\end{proof}
	      \item Using open intervals makes it more convenient to produce the required 1-1, onto functions, but it is not really necessary. Show that $[0,1)~(0,1)$ by exhibiting a 1-1 onto function between the two sets.
		\begin{proof}
		  We know that $\R$ is the union of an uncountable set $\mathbb{I}$ and a countable set $\Q$. Let $r = \left\{r_{n}:n\in \N \right\}$ be the sequence of all rational numbers in $(0,1)$ and let $r_{0} = 0$. Then, define some function $\varphi$ by
		  \begin{align*}
		    \varphi(x) = \begin{cases}
		      x, \; x\in\mathbb{I}.\\
		      r_{n}, \; x = r_{n-1} \text{ for all } n\in\N.
		    \end{cases}
		  \end{align*}
		  Hence, $\varphi: [0,1) \to (0,1)$ is a 1-1 onto function and so $[0,1)~(0,1)$.
		\end{proof}
	 \end{enumerate}
      
\end{problem}

\begin{problem}{1.5.5}
  \begin{enumerate}[label=(\alph*)]
    \item Why is $A\sim A$ for every set $A$?
      \begin{solution}{(a)}
	Just consider the function $g(x)=x$ for every $x\in A$ (some type of identity function) which clearly is bijective. 
      \end{solution}
   \item Given sets $A$ and $B$, explain why $A\sim B$ is equivalent to asserting $B\sim A$.
     \begin{solution}{(b)}
       If there is a bijective function $\varphi: A\to B$, then there is an inverse function $\varphi^{-1}: B\to A$ that is also bijective.
     \end{solution}
   \item For three sets $A,B,$ and $C$, show that $A\sim B$ and $B\sim C$ implies $A\sim C$. These three properties are what is meant by saying that $\sim$ is an \textit{equivalence relation}.
     \begin{solution}{(c)}
       If there are bijective functions $\varphi: A\to B$ and $\gamma: B\to C$, there is a composite function $\varphi \circ \gamma:A\to C$ that is bijective.
     \end{solution}
  \end{enumerate}
\end{problem}

\begin{problem}{1.5.6}
  \begin{enumerate}[label=(\alph*)]
    \item Give an example of a countable collection of disjoint open intervals.
      \begin{solution}{(a)}
	Consider the collection $S=\left\{(n,n+1):n\in\N \right\}$. Then, $S$ is disjoint and countable.
	\end{solution}
    \item Give an example of an uncountable collection of disjoint open intervals, or argue that no such collection exists.
      \begin{solution}{(b)}
	No such collection exists. Although, I don't have a proof, I can give some hand-wavy argument of why. This has to do with the density property of rational numbers in $\R$. For every open interval, there will be a rational point in it and since all are disjoint, one can pair each interval with a distinct rational number. Since $\Q$ is countable, it follows that any collection of open intervals is countable.
      \end{solution}
  \end{enumerate}
\end{problem}

\begin{problem}{1.5.7}
  Consider the open interval $(0,1)$, and let $S$ be the set of points in the open unit square; that is, $S=\left\{ (x,y):0<x,y<1 \right\}$.
  \begin{enumerate}
    \item Find a 1-1 function that maps $(0,1)$ into, but not necessarily onto, $S$. 
      \begin{solution}{(a)}
	Let the function $\varphi: (0,1) \to S$ be defined by
	\begin{align*}
	  \varphi(x) = (x,0).
	\end{align*}
	This is clearly 1-1 but not onto. 
      \end{solution}
    \item Use the fact that every real number has a decimal expansion to produce a 1-1 function that maps $S$ into $(0,1)$. Discuss whether the formulated function is onto. (Keep in mind that any terminating decimal expanison such as $.235$ represents the same real number as $.234999\dots$).
      \begin{solution}{(b)}
	Consider some $(x,y)\in S$. Since $x,y\in(0,1)$, it follows that both have unique decimal expansions that can be expressed as $x=0.x_{1}x_{2}x_{3}\dots$ and $y=0.y_{1}y_{2}y_{3}\dots$. Then, let the function $\varphi:S\to (0,1)$ be defined by
	\begin{align*}
    \varphi((x,y))  = 
      0.x_{1}y_{1}x_{2}y_{2}x_{3}y_{3}\dots,
	\end{align*}
	which clearly is 1-1 since each decimal expansion represents one unique real number. This function seems to be onto. Consider some real number $z=0.z_{1}z_{2}z_{3}z_{4}z_{5}z_{6}\dots$ contained in $(0,1)$, then one can obtain two real numbers $x=0.z_{1}z_{3}z_{5}\dots$ and $y=0.z_{2}z_{4}z_{6}\dots$ such that $\varphi\left((x,y)\right) = z$.  However, one may come up with some counterexample, namely $z=0.101010101010\dots$, where the even decimal places are 0, which implies that $(x,y) \not\in S$. So $\varphi$ is not onto. 
      \end{solution}
  \end{enumerate}
\end{problem}

\begin{problem}{1.5.9}
  A real number $x\in\R$ is called \textit{algebraic} if there exist integers $a_{0},a_{1},a_{2},\dots,a_{n}\in\Z$, not all zero, such that
  \begin{equation*}
    a_{n}x^{n}+a_{n-1}x^{n-1}+\dots+a_{1}x+a_{0}=0.
  \end{equation*}
  Said another way, a real number is algebraic if it is the root of a polynomial with integer coefficients. Real numbers that are not algebraic are called \textit{transcendental} numbers. Reread the last paragraph of Section 1.1. The final question posed here is closely related to the question of whether or not transcendental numbers exist.
  \begin{enumerate}[label=(\alph*)]
    \item Show that $\sqrt{2}, \sqrt[3]{2},$ and $\sqrt{3}+\sqrt{2}$ are algebraic.
      \begin{solution}{(a)}
	The number $\sqrt{2}$ is a root of $x^{2}-2$, $\sqrt[3]{2}$ is a root of $x^{3}-2$ and $\sqrt{3}+\sqrt{2}$ is a root of $-x^{5}+10x^{3}-x$.
      \end{solution}
    \item Fix $n\in \N$, and let $A_{n}$ be the algebraic numbers obtained as roots of polynomials with integer coefficients that have degree $n$. Using the fact that every polynomial has a finite number of roots, show that $A_{n}$ is countable.
      \begin{solution}{(b)}
	First let's show that for any positive integer $n\geq 2$, $\N\sim \N^{n}$.
	\begin{proof}
	  We proceed by induction. We know that $\N\sim \N\times \N$. Thus, our  statement is true for $n=2$. Assume that $\N\sim \N^{k}$ for some $k\geq 2$. We show that $\N\sim \N^{k+1}$. By our inductive hypothesis, there is some bijective function $f:\N^{k}\to \N$. Consider the set, $A=\left\{(f(a),b):a\in\N^{k},b\in\N\right\}$, which is equal to $\N\times\N$ since $\N\sim \N^{k}$. Hence, $A\sim \N$. Now, if we pair each $\left( (a_{1},a_{2},\dots,a_{k}),b \right)$ with $\left( a_{1},a_{2},\dots,a_{k},b\right)$ it follows that $\N^{k+1}\sim A$ and so $\N^{k+1}\sim \N$.
	\end{proof}
	Note that the set $P$ of all polynomials of degree $n$ with integer coefficients can be shown to be correspondent with the set $\Z^{n+1}$ if we pair each polynomial $a_{n}x^{n}+a_{n-1}x^{n-1}+\dots+a_{1}x+a_{0}$ with $(a_{n},a_{n-1},\dots,a_{1},a_{0})$. Furthermore, each element $(a_{n},a_{n-1},\dots,a_{1},a_{0})$ can be paired with $(\varphi(a_{n}),\varphi(a_{n-1}),\dots,\varphi(a_{1}),\varphi(a_{0}))$ for the bijective funciton $\varphi:\Z\to\N$. Hence, $P\sim\Z^{n+1}\sim \N^{n+1}\sim \N$ and so $P\sim \N$. Therefore, the set $A_{n}$ contains the roots of countably many polynomials of degree $n$ with integer coefficients ($A_{n}$ is the union of countably many sets with finite elements), and so $A_{n}$ is countable. 
      \end{solution}
    \item Now, argue that the set of all algebraic numbers is countable. What may we conclude about the set of transcendental numbers?
      \begin{solution}{(c)}
	Consider the collection $\left\{ P_{n}:n\in\N \right\}$, where each $P_{n}$ contains all algebraic numbers as roots for polynomials of degree $n$ with integer coefficients. Then, $A=\bigcup_{n\in\N}P_{n}$ is the set of all algebraic numbers. Note that $A$ is the union of countably many countable sets, which implies that $A$ is countable. Since each real number is either algebraic or transcendental, it follows that $\R=A\cup T$, where $T$ is the set of transcendentals. Therefore, $T$ must be uncountable.
      \end{solution}
  \end{enumerate}
  
\end{problem}

\begin{problem}{1.5.10}
  \begin{enumerate}[label=(\alph*)]
    \item Let $C\subseteq [0,1]$ be uncountable. Show that there exists $a\in(0,1)$ such that $C\cap [a,1]$ is uncountable.
      \begin{solution}{(a)}
	Choose any $a\in (0,1)$ such that $a < \sup(C)$. Since $(a,\sup(C))$ has inifinitely many uncountable elements, it follows that $C\cap [a,1]$ is uncountable. 
      \end{solution}

    \item Now let $A$ be the set of all $a\in(0,1)$ such that $C\cap[a,1]$ is countable, and set $a=\sup A$. Is $C\cap[a,1]$ an uncountable set?
      \begin{solution}{(b)}
	It is not since $\sup A = \sup C$. Therefore, either $C\cap\left[a, 1 \right]=\left\{ \sup C \right\}$ or $C\cap\left[a,1\right]=\emptyset$ depending on wether $\sup C \in C$ or not. 
      \end{solution}

    \item Does the statement in (a) remain true if ''uncountable'' is replaced by ''infinite''?
      \begin{solution}{(c)}
	Yes, since a set being uncountable already implies that it is infinite.
      \end{solution}
  \end{enumerate}
\end{problem}

\begin{problem}{1.5.11 (Schr\"oder-Bernstein Theorem).}
  Assume there exists a 1-1 function $f:X\to Y$ and another 1-1 function $g:Y\to X$. Follow the steps to show that there exists a 1-1, onto function $h:X\to Y$ and hence $X\sim Y$. The stratgey is to partition $X$ and $Y$ into components
  \begin{equation*}
    X=A\cup A' \quad \text{ and } \quad Y=B\cup B'
  \end{equation*}
  with $A\cap A' = \emptyset$ and $B\cap B'=\emptyset$, in such a way that $f$ maps $A$ onto $B$, and $g$ maps $B'$ onto $A'$.
  \begin{enumerate}
    \item Explain how achieving this would lead to a proof that $X\sim Y$.
      \begin{solution}{(a)}
	By achieving this, one would show that $f:A\to B$ and $g:B'\to A'$ are bijective functions. Then, one can define the function $\varphi: X\to Y$ by
	\begin{align*}
	  \varphi(x) = \begin{cases}
	    f(x), \; x\in A\\
	    g^{-1}(x), \; x\in A'
	  \end{cases}
	\end{align*}
	which is 1-1 and onto. This suggests that if $A\sim B$ and $C\sim D$, then $(A\cup C) \sim (B\cup D)$.
      \end{solution}
    \item Set $A_{1}=X\backslash g(Y) = \left\{ x\in X:x\not\in g(Y) \right\}$ (what happens if $A_{1}=\emptyset$?) and inductively define a sequence of sets by letting $A_{n+1} = g(f(A_{n}))$. Show that $\left\{A_{n}:n\in \N\right\}$ is a pairwise disjoint collection of subsets of $X$, while $\left\{ f(A_{n}):n\in\N \right\}$ is a similar collection in $Y$.
     \begin{solution}{(b)}
       First we prove the following lemma:
       \begin{lemma}{1.5.11.b}
	 Let $f:A\to B$ be a 1-1 function and $A_{1},A_{2}$ be two subsets of $A$. Then, $f(A_{1})\cap f(A_{2})\neq \emptyset$ if and only if $A_{1}\cap A_{2} \neq \emptyset$. 
	 \begin{proof}
	   Let $A_{1}\cap A_{2} \neq \emptyset$. Then, there is some $x\in A_{1}\cap A_{2}$ and so $f(x) \in A_{1}\cap A_{2}$. For the converse, assume that $f(A_{1})\cap f(A_{2})\neq \emptyset$. Hence, there are $f(a)\in f(A_{1})$ and $f(b)\in f(A_{2})$ such that $f(a)=f(b)$ for $a\in A_{1}$ and $b\in A_{2}$. Due to the injective nature of $f$, it follows that $a=b\in A_{1}\cap A_{2}$. 
	 \end{proof}
       \end{lemma}

       We now proceed to show the required result:
      \begin{proof}
	Suppose, to the contrary, that there are distinct positive integers $k,p$ such that $A_{k}\cap A_{p}\neq \emptyset$. Note that $A_{1}\cap A_{n}=\emptyset$ for any integer $n\geq 2$ since $A_{1}=X\backslash g(Y)$ and $A_{n}\subseteq g(Y)$. Then, we may further assume that $k,p\geq 2$. Hence, $g\left( f\left( A_{k-1} \right) \right)\cap g\left(f \left( A_{p-1} \right) \right) \neq \emptyset$. Note that $g\circ f$ is injective and so, by \textbf{Lemma 1.5.11.b}, $A_{k-1}\cap A_{p-1}\neq \emptyset$. We can use the same argument recursively in a finite fashion, until we get to the statement that $A_{1}\cap A_{n}\neq \emptyset$ for some $n\geq 2$, which is a contradiction. Therefore, the collection $\left\{ A_{n}:n\in\N \right\}$ is pairwise disjoint.
	Also, by \textbf{Lemma 1.5.11.b},  the collection $\left\{ f(A_{n}):n\in\N \right\}$ is pairwise disjoint.
      \end{proof}
     \end{solution}
    \item Let $A=\bigcup_{n=1}^{\infty} A_{n}$ and $B=\bigcup_{n=1}^{\infty}f(A_{n})$. Show that $f$ maps $A$ onto $B$.
      \begin{proof}
	Consider some $b\in B$. Since $\bigcup_{n=1}^{\infty}f(A_{n})$ is a union of pairwise disjoint sets, it follows that $b\in f(A_{k})$ for one unique $k\in \N$. Thus, there is some $a\in A_{n}$ such that $f(a)=b$. Hence, $f:A\to B$ is onto. 
      \end{proof}
    \item Let $A'=X\backslash A$ and $B' = Y\backslash B$. SHow $g$ maps $B'$ onto $A'$.
      \begin{proof}
	Note that $X/g(Y) \subseteq A$ and so $A'$ is a partition of $g(Y)$. Assume, to the contrary, that there is some $a\in A'$ such that for all $b\in B'$, $g(b)\neq a$. Hence, there is some  $c\in B$, where $c\in f(A_{n})$ for some positive integer $n$. However, this implies that $g(c)\in A_{n+1}$, which contradicts the fact that $A'=X\backslash A$. Thus, $g:B'\to A'$ is onto.  
      \end{proof}
  \end{enumerate}
\end{problem}
\end{document}
