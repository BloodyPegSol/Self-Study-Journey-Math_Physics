\documentclass[12pt]{article}
\usepackage[margin=1in]{geometry}
\usepackage{amsmath, amsfonts,amsthm,amssymb,epigraph,etoolbox,mathtools,setspace,enumitem}  
\usepackage{tikz}
\usetikzlibrary{datavisualization}
\usepackage[makeroom]{cancel} 
\usepackage[linguistics]{forest}
\usetikzlibrary{patterns}
\newcommand{\N}{\mathbb{N}}
\newcommand{\Z}{\mathbb{Z}}
\newcommand{\R}{\mathbb{R}}
\newcommand{\Q}{\mathbb{Q}}
\newcommand{\Mod}[1]{\ (\mathrm{mod}\ #1)}



\newlist{legal}{enumerate}{10}
\setlist[legal]{label*=\arabic*.}

\DeclarePairedDelimiter\bra{\langle}{\rvert}
\DeclarePairedDelimiter\ket{\lvert}{\rangle}
\DeclarePairedDelimiterX\braket[2]{\langle}{\rangle}{#1\delimsize\vert #2}


\newenvironment{theorem}[2][Theorem]{\begin{trivlist}
		\item[\hskip \labelsep {\bfseries #1}\hskip \labelsep {\bfseries #2.}]}{\end{trivlist}}
\newenvironment{lemma}[2][Lemma]{\begin{trivlist}
		\item[\hskip \labelsep {\bfseries #1}\hskip \labelsep {\bfseries #2.}]}{\end{trivlist}}
\newenvironment{result}[2][Result]{\begin{trivlist}
		\item[\hskip \labelsep {\bfseries #1}\hskip \labelsep {\bfseries #2.}]}{\end{trivlist}}
\newenvironment{exercise}[2][Exercise]{\begin{trivlist}
		\item[\hskip \labelsep {\bfseries #1}\hskip \labelsep {\bfseries #2.}]}{\end{trivlist}}
\newenvironment{problem}[2][Problem]{\begin{trivlist}
		\item[\hskip \labelsep {\bfseries #1}\hskip \labelsep {\bfseries #2.}]}{\end{trivlist}}
\newenvironment{question}[2][Question]{\begin{trivlist}
		\item[\hskip \labelsep {\bfseries #1}\hskip \labelsep {\bfseries #2.}]}{\end{trivlist}}
\newenvironment{corollary}[2][Corollary]{\begin{trivlist}
		\item[\hskip \labelsep {\bfseries #1}\hskip \labelsep {\bfseries #2.}]}{\end{trivlist}}
\newenvironment{solution}[2][Solution]{\begin{trivlist}
		\item[\hskip \labelsep {\bfseries #1}\hskip \labelsep {\bfseries #2.}]}{\end{trivlist}}

\setlength\epigraphwidth{8cm}
\setlength\epigraphrule{0pt}

\makeatletter
\patchcmd{\epigraph}{\@epitext{#1}}{\itshape\@epitext{#1}}{}{}
\makeatother


\begin{document}
	
	\title{Section 1.3: Axiom of Completeness}
	\author{Juan Patricio Carrizales Torres}
	\date{May 09, 2022}
	\maketitle
	
	I find it interesting that the author wants to follow and historical approach in the book about Real Analysis. The completness of the Real Numbers is stated as an axiom and the set $\R$ is defined as an ordered field. Naturally, these properties can be proven from more fundamental principles but this may be misleading and terse for a first exposure to Real Analysis. Also, once seen the most important theorems of the 1800s, one can fully appreciate the construction of $\R$ from $\Q$. \\
	
	\begin{problem}{1.3.1}
		\begin{enumerate}[label=(\alph*)]
			\item Write a formal definition in the style of Definition $1.3.2$ for the $infimum$ or $greatest \, lower \, bound$ of a set. 
			\begin{solution}{a}
				A real number $s$ is the $greatest \, lower \, bound$ of a set $A\subseteq \R$ if the following criteria are met:
				\begin{enumerate}[label=\arabic*)]
					\item $s$ is a lower bound of $A$;
					\item if $b$ is a lower bound of $A$, then $b\leq s$.
				\end{enumerate}
			\end{solution}
			\item Now, state and prove a version of Lemma $1.3.8$ for greatest lower bounds. 
			\begin{solution}{b}
				Assume $s$ is some lower bound of $A\subseteq \R$. Then, $\inf(A) = s$ if and only if for any $\varepsilon > 0$, it is true that $a< s+\varepsilon$ for some $a\in A$.
				\begin{proof}
					Assume that $\inf(A) = s$. Then, $s$ is the $greatest \; lower \; bound$ of $A$ and so $s+\varepsilon > s$, where $\varepsilon>0$, is not a lower bound of $A$. Hence, $s+\varepsilon > a$ for some $a\in A$.
					 
					For the converse, let $s$ be a lower bound for $A$ and suposse for any $\varepsilon>0$ that $a< s+\varepsilon$ for some $a \in A$. Since $s+\varepsilon > s$, it follows that any real number greater than $s$ will not be an upper bound. Hence, any lower bound $b$ of $A$ will be lower or equal to $s$, and so $s = \inf A$.
				\end{proof}
			\end{solution}
		\end{enumerate}
	\end{problem}

		\begin{problem}{1.3.2}
				Give an example of each of the following, or state that the request is impossible.
			\begin{enumerate}[label=(\alph*)]
				\item A set $B$ with $\inf B \geq \sup B$.
				\begin{solution}{(a)}
					This is not possible, since $\inf B \geq \sup B$ implies that $\inf B$ is an upper bound of $A$ and so $\inf A \geq a$ for all $a\in A$. This contradicts the fact that $\inf A \leq a$ for all $a\in A$.
				\end{solution}
				\item A finite set that contains its infimum but not its supremum.
				\begin{solution}{(b)}
					This is not possible, since real numbers are ordered, a finite set will contain a greatest element $x$. Then $x\geq a$ for all $a\in A$ and at the same time $x\in A$, which implies that $x=\sup A$.
				\end{solution}
				\item A bounded subset of $\Q$ that contains its supremum but not its infimum. 
				\begin{solution}{(c)}
					This is possible. Let $A=\{x\in \Q: 2<x\leq 4\}$. Then $A\subseteq \Q$. Note that $\inf A=2\not\in A$ and $\sup A = 4\in A$. 
				\end{solution}
			\end{enumerate}
		\end{problem}
		
		\begin{problem}{1.3.3}
			\begin{enumerate}[label=(\alph*)]
			\item Let $A$ be nonempty and bounded below, and define $B=\{b\in \R:b \text{ is a lower bound for }A\}$. Show that $\sup B = \inf A$.
			\begin{proof}
				Since $B$ is the set of all lower bounds for $A$, it follows that $a\geq b$ for every $a\in A$ and $b\in B$, and so $B$ is bounded above. By the Axiom of Completeness, there exists some $\sup B = s$. Note that all elements of $A$ are upper bounds of $B$ and so $s\leq a$ for every $a\in A$ since $s=\sup B$. Hence, $s$ is a lower bound for $A$. Also, $s\geq b$ for all $b\in B$. Therefore, $s=\sup B = \inf A$.
			\end{proof}
			\item Use (a) to explain why there is no need to assert that greatest lower bounds exist as part of the Axiom of Completeness.
			\begin{solution}{b}
				Note that for any set $A$ that is bounded below, one can obtain the $supremum$ of the set of all lower bounds of $A$, which we previously showed that is the $infimum$ of $A$. Hence, one can derive the theorem of $greatest \; lower \; bound$ from the axiom of completeness.
			\end{solution}
			
			\end{enumerate}
		\end{problem}
	
		\begin{problem}{1.3.4}
			Let $A_{1},A_{2},A_{3},\ldots$ be a collection of nonempty sets, each of which is bounded above.
			\begin{enumerate}[label=(\alph*)]
				\item Find a formula for $\sup(A_{1}\cup A_{2})$. Extend this to $\sup\left(\bigcup_{k=1}^{n}A_{k}\right)$.
				\begin{solution}{(a)}
				First we show the following result:
				\begin{lemma}{1}
				Let $A_{1}$ and $A_{2}$ be nonempty sets of real numbers such that both are bounded above. Then $\sup (A_{1}\cup A_{2}) = \sup\{\sup A_{1}, \sup A_{2}\}$.
				\begin{proof}
				 Since $A_{1}$ and $A_{2}$ are bounded above, it follows that $\sup A_{1}$ and $\sup A_{2}$ exist. Since real numbers are ordered, it follows that $\max\{\sup A_{1},\sup A_{2}\} = s$ exists. By definition, $s$ is greater than or equal to all elements of $A_{1}$ and $A_{2}$ and so it is considered and upper bound for $A_{1}\cup A_{2}$, this also implies that $A_{1}\cup A_{2}$ is bounded above. Note that if $b$ is an upper bound for $A_{1}\cup A_{2}$, then $b$ is an upper bound for $A_{1}$ and $A_{2}$ and so $b\geq s$. Therefore, $\max\{\sup A_{1},\sup A_{2}\} = \sup \left\{\sup A_{1},\sup A_{2}\right\} = \sup (A_{1}\cup A_{2})$.
				 \end{proof}
				 \end{lemma}
					
					We can expand this to a finite quantity of set, namely:
					
					\begin{theorem}{1}
						Let $A_{1},A_{2},A_{3},\ldots, A_{n}$ be collection of $n\geq 2$ nonempty sets, each of which is bounded above. Then, $\sup\left(\bigcup_{i=1}^{n} A_{k}\right) = \sup\{\sup A_{1},\sup A_{2}, \sup A_{3}, \ldots, \sup A_{n}\}$.
						\begin{proof}
							We proceed by induction. Since, by \textbf{Lemma 1}, $\sup (A_{1}\cup A_{2}) = \sup\{\sup A_{1}, \sup A_{2}\}$ is true, it follows that the result is true for $n=2$. Assume for a collection of $k\geq 2$ nonempty sets $B_{1},B_{2},B_{3},\ldots,B_{k}$, each of them being bounded above, that
							\begin{equation*}
								\sup\left(\bigcup_{i=1}^{k} B_{i}\right) = \sup\{\sup B_{1},\sup B_{2}, \sup B_{3}, \ldots,\sup B_{k}\}.
							\end{equation*} 
						We show for a collection of $k+1$ nonempty sets $C_{1},C_{2},C_{3},\ldots,C_{k+1}$, each of them being bounded above, that
						\begin{equation*}
							\sup\left(\bigcup_{i=1}^{k+1} C_{i}\right) = \sup\{\sup C_{1},\sup C_{2}, \sup C_{3}, \ldots,\sup C_{k+1}\}.
						\end{equation*} 
						Note that
						\begin{align*}
							\sup\left(\bigcup_{i=1}^{k+1} C_{i}\right) &= \sup\left(\left(\bigcup_{i=1}^{k} C_{i}\right) \cup C_{k+1}\right)\\
							&= \sup\left\{\sup\left(\bigcup_{i=1}^{k} C_{i}\right),\sup C_{k+1}\right\} \; (\textbf{Lemma 1})\\
							&= \sup\left\{\sup\{\sup C_{1},\sup C_{2}, \sup C_{3}, \ldots,\sup C_{k}\},\sup C_{k+1}\right\} \; (\textbf{Inductive Hypothesis}).\\
							&= \sup\left\{\sup C_{1},\sup C_{2}, \sup C_{3}, \ldots,\sup C_{k+1}\right\}
						\end{align*}
					Since we are taking the $supremum$ of a finite set of real numbers, namely, the $maximum$ for this specific case.
						\end{proof}
					\end{theorem}
				\end{solution}
				\item Consider $\sup\left(\bigcup_{k=1}^{\infty} A_{k}\right)$. Does the formula in $(a)$ extend to the infinite case?
				\begin{solution}{(b)}
					In order for our result to extend for a general infinite case, the $supremum$ of the $supremums$ of the sets must exist. However, the set of $supremums$ of the sets can be an inifinite set of real numbers not bounded above. This could be a counterexample. 
				\end{solution}
			\end{enumerate}
			
		\end{problem}
	
		\begin{problem}{1.3.5}
			As in \textbf{Example 1.3.7}, let $A\subseteq \R$ be nonempty and bounded above, and let $c\in \R$. This time define the set $cA = \{ca:a\in A\}$.
			\begin{enumerate}[label=(\alph*)]
				\item If $c\geq 0$, show that $\sup(cA) = c\sup A$.
				\begin{proof}
					Consider some real number $c\geq 0$. If $c=0$, then $cA = \{0\}$ and so $\sup (cA) = 0 = c\sup A$. Hence, we may assume that $c>0$.\\
					Since $A$ is bounded above, it follows that $\sup A$ exists. Now, consider the set $cA = \{ca:a\in A\}$. Note that $a\leq \sup A$ for every $a\in A$ and so $ca \leq c\sup A$ for every $a\in A$. Therefore, $cA$ is bounded above by $c\sup A$. \\
					
					Consider some upper bound $b$ for $cA$, then $ca \leq b$ for all $a\in A$. Then, $a\leq b/c$ (recall that $c>0$). Therefore, $b/c$ is an upper bound for $A$ and so $\sup A \leq b/c$. This implies that $c\sup A \leq b$. However, recall that $c\sup A$ is an upper bound for $cA$.
					Thus, $\sup (cA) = c\sup A$.
				\end{proof}
				\item Postulate a similar type of statement for $\sup(cA)$ for the case $c<0$.
				\begin{solution}{b}
					For this, we need to further assume that $A$ is boundedd below. To illustrate this take for example the set $A=(-\infty, 0]$, which is bounded above and let $c=-1$. However, $cA = [0,\infty)$ is not bounded above and so $\sup (cA)$ does not exist. \\
					
					If $c<0$, then $\sup(cA) = c\inf
					A$.
					\begin{proof}
						Let $c<0$. Since $A$ is bounded belowe, it follows that  $\inf A$ exists and so $a\geq \inf A$ for every $a\in A$. Note that $ca\leq c\inf A$ since $c<0$, which implies that $c\inf A$ is an upper bound for $cA$.\\
						Now, consider some upper bound $b$ for $cA$. Then, $b\geq ca$ for every $a\in A$. Thus, $b/c \leq a$ since $1/c < 0$, and so $b/c$ is a lower bound for $A$. Thus, $b/c \leq \inf A$ and so $b \geq c\inf A$. Thus, $c\inf A = \sup (cA)$. 
					\end{proof}
				\end{solution}
			\end{enumerate}
		\end{problem}

\end{document}