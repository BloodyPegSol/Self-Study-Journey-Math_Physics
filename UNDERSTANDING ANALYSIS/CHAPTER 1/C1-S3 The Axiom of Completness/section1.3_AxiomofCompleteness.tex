\documentclass[12pt]{article}
\usepackage[margin=1in]{geometry}
\usepackage{amsmath, amsfonts,amsthm,amssymb,epigraph,etoolbox,mathtools,setspace,enumitem}  
\usepackage{tikz}
\usetikzlibrary{datavisualization}
\usepackage[makeroom]{cancel} 
\usepackage[linguistics]{forest}
\usetikzlibrary{patterns}
\newcommand{\N}{\mathbb{N}}
\newcommand{\Z}{\mathbb{Z}}
\newcommand{\R}{\mathbb{R}}
\newcommand{\Q}{\mathbb{Q}}
\newcommand{\Mod}[1]{\ (\mathrm{mod}\ #1)}



\newlist{legal}{enumerate}{10}
\setlist[legal]{label*=\arabic*.}

\DeclarePairedDelimiter\bra{\langle}{\rvert}
\DeclarePairedDelimiter\ket{\lvert}{\rangle}
\DeclarePairedDelimiterX\braket[2]{\langle}{\rangle}{#1\delimsize\vert #2}


\newenvironment{theorem}[2][Theorem]{\begin{trivlist}
		\item[\hskip \labelsep {\bfseries #1}\hskip \labelsep {\bfseries #2.}]}{\end{trivlist}}
\newenvironment{lemma}[2][Lemma]{\begin{trivlist}
		\item[\hskip \labelsep {\bfseries #1}\hskip \labelsep {\bfseries #2.}]}{\end{trivlist}}
\newenvironment{result}[2][Result]{\begin{trivlist}
		\item[\hskip \labelsep {\bfseries #1}\hskip \labelsep {\bfseries #2.}]}{\end{trivlist}}
\newenvironment{exercise}[2][Exercise]{\begin{trivlist}
		\item[\hskip \labelsep {\bfseries #1}\hskip \labelsep {\bfseries #2.}]}{\end{trivlist}}
\newenvironment{problem}[2][Problem]{\begin{trivlist}
		\item[\hskip \labelsep {\bfseries #1}\hskip \labelsep {\bfseries #2.}]}{\end{trivlist}}
\newenvironment{question}[2][Question]{\begin{trivlist}
		\item[\hskip \labelsep {\bfseries #1}\hskip \labelsep {\bfseries #2.}]}{\end{trivlist}}
\newenvironment{corollary}[2][Corollary]{\begin{trivlist}
		\item[\hskip \labelsep {\bfseries #1}\hskip \labelsep {\bfseries #2.}]}{\end{trivlist}}
\newenvironment{solution}[2][Solution]{\begin{trivlist}
		\item[\hskip \labelsep {\bfseries #1}\hskip \labelsep {\bfseries #2.}]}{\end{trivlist}}

\setlength\epigraphwidth{8cm}
\setlength\epigraphrule{0pt}

\makeatletter
\patchcmd{\epigraph}{\@epitext{#1}}{\itshape\@epitext{#1}}{}{}
\makeatother


\begin{document}
	
	\title{Section 1.3: Axiom of Completeness}
	\author{Juan Patricio Carrizales Torres}
	\date{May 09, 2022}
	\maketitle
	
	I find it interesting that the author wants to follow and historical approach in the book about Real Analysis. The completness of the Real Numbers is stated as an axiom and the set $\R$ is defined as an ordered field. Naturally, these properties can be proven from more fundamental principles but this may be misleading and terse for a first exposure to Real Analysis. Also, once seen the most important theorems of the 1800s, one can fully appreciate the construction of $\R$ from $\Q$. \\
	
	\begin{problem}{1.3.1}
		\begin{enumerate}[label=(\alph*)]
			\item Write a formal definition in the style of Definition $1.3.2$ for the $infimum$ or $greatest \, lower \, bound$ of a set. 
			\begin{solution}{a}
				A real number $s$ is the $greatest \, lower \, bound$ of a set $A\subseteq \R$ if the following criteria are met:
				\begin{enumerate}[label=\arabic*)]
					\item $s$ is a lower bound of $A$;
					\item if $b$ is a lower bound of $A$, then $b\leq s$.
				\end{enumerate}
			\end{solution}
			\item Now, state and prove a version of Lemma $1.3.8$ for greatest lower bounds. 
			\begin{solution}{b}
				Assume $s$ is some lower bound of $A\subseteq \R$. Then, $\sup(A) = s$ if and only if for any $\varepsilon > 0$, it is true that $a> s-\varepsilon$ for some $a\in A$.
			\end{solution}
		\end{enumerate}
	\end{problem}
	

\end{document}