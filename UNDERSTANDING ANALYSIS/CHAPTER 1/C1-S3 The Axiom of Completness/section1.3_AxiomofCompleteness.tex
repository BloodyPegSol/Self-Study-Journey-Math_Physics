\documentclass[12pt]{article}
\usepackage[margin=1in]{geometry}
\usepackage{amsmath, amsfonts,amsthm,amssymb,epigraph,etoolbox,mathtools,setspace,enumitem}  
\usepackage{tikz}
\usetikzlibrary{datavisualization}
\usepackage[makeroom]{cancel} 
\usepackage[linguistics]{forest}
\usetikzlibrary{patterns}
\newcommand{\N}{\mathbb{N}}
\newcommand{\Z}{\mathbb{Z}}
\newcommand{\R}{\mathbb{R}}
\newcommand{\Q}{\mathbb{Q}}
\newcommand{\Mod}[1]{\ (\mathrm{mod}\ #1)}



\newlist{legal}{enumerate}{10}
\setlist[legal]{label*=\arabic*.}

\DeclarePairedDelimiter\bra{\langle}{\rvert}
\DeclarePairedDelimiter\ket{\lvert}{\rangle}
\DeclarePairedDelimiterX\braket[2]{\langle}{\rangle}{#1\delimsize\vert #2}


\newenvironment{theorem}[2][Theorem]{\begin{trivlist}
		\item[\hskip \labelsep {\bfseries #1}\hskip \labelsep {\bfseries #2.}]}{\end{trivlist}}
\newenvironment{lemma}[2][Lemma]{\begin{trivlist}
		\item[\hskip \labelsep {\bfseries #1}\hskip \labelsep {\bfseries #2.}]}{\end{trivlist}}
\newenvironment{result}[2][Result]{\begin{trivlist}
		\item[\hskip \labelsep {\bfseries #1}\hskip \labelsep {\bfseries #2.}]}{\end{trivlist}}
\newenvironment{exercise}[2][Exercise]{\begin{trivlist}
		\item[\hskip \labelsep {\bfseries #1}\hskip \labelsep {\bfseries #2.}]}{\end{trivlist}}
\newenvironment{problem}[2][Problem]{\begin{trivlist}
		\item[\hskip \labelsep {\bfseries #1}\hskip \labelsep {\bfseries #2.}]}{\end{trivlist}}
\newenvironment{question}[2][Question]{\begin{trivlist}
		\item[\hskip \labelsep {\bfseries #1}\hskip \labelsep {\bfseries #2.}]}{\end{trivlist}}
\newenvironment{corollary}[2][Corollary]{\begin{trivlist}
		\item[\hskip \labelsep {\bfseries #1}\hskip \labelsep {\bfseries #2.}]}{\end{trivlist}}
\newenvironment{solution}[2][Solution]{\begin{trivlist}
		\item[\hskip \labelsep {\bfseries #1}\hskip \labelsep {\bfseries #2.}]}{\end{trivlist}}

\setlength\epigraphwidth{8cm}
\setlength\epigraphrule{0pt}

\makeatletter
\patchcmd{\epigraph}{\@epitext{#1}}{\itshape\@epitext{#1}}{}{}
\makeatother


\begin{document}
	
	\title{Section 1.3: Axiom of Completeness}
	\author{Juan Patricio Carrizales Torres}
	\date{May 09, 2022}
	\maketitle
	
	I find it interesting that the author wants to follow and historical approach in the book about Real Analysis. The completness of the Real Numbers is stated as an axiom and the set $\R$ is defined as an ordered field. Naturally, these properties can be proven from more fundamental principles but this may be misleading and terse for a first exposure to Real Analysis. Also, once seen the most important theorems of the 1800s, one can fully appreciate the construction of $\R$ from $\Q$. \\
	
	\begin{problem}{1.3.1}
		\begin{enumerate}[label=(\alph*)]
			\item Write a formal definition in the style of Definition $1.3.2$ for the $infimum$ or $greatest \, lower \, bound$ of a set. 
			\begin{solution}{a}
				A real number $s$ is the $greatest \, lower \, bound$ of a set $A\subseteq \R$ if the following criteria are met:
				\begin{enumerate}[label=\arabic*)]
					\item $s$ is a lower bound of $A$;
					\item if $b$ is a lower bound of $A$, then $b\leq s$.
				\end{enumerate}
			\end{solution}
			\item Now, state and prove a version of Lemma $1.3.8$ for greatest lower bounds. 
			\begin{solution}{b}
				Assume $s$ is some lower bound of $A\subseteq \R$. Then, $\inf(A) = s$ if and only if for any $\varepsilon > 0$, it is true that $a< s+\varepsilon$ for some $a\in A$.
				\begin{proof}
					Assume that $\inf(A) = s$. Then, $s$ is the $greatest \; lower \; bound$ of $A$ and so $s+\varepsilon > s$, where $\varepsilon>0$, is not a lower bound of $A$. Hence, $s+\varepsilon > a$ for some $a\in A$.
					 
					For the converse, let $s$ be a lower bound for $A$ and suposse for any $\varepsilon>0$ that $a< s+\varepsilon$ for some $a \in A$. Since $s+\varepsilon > s$, it follows that any real number greater than $s$ will not be an upper bound. Hence, any lower bound $b$ of $A$ will be lower or equal to $s$, and so $s = \inf A$.
				\end{proof}
			\end{solution}
		\end{enumerate}
	\end{problem}

		\begin{problem}{1.3.2}
				Give an example of each of the following, or state that the request is impossible.
			\begin{enumerate}[label=(\alph*)]
				\item A set $B$ with $\inf B \geq \sup B$.
				\begin{solution}{(a)}
					This is not possible, since $\inf B \geq \sup B$ implies that $\inf B$ is an upper bound of $A$ and so $\inf A \geq a$ for all $a\in A$. This contradicts the fact that $\inf A \leq a$ for all $a\in A$.
				\end{solution}
				\item A finite set that contains its infimum but not its supremum.
				\begin{solution}{(b)}
					This is not possible, since real numbers are ordered, a finite set will contain a greatest element $x$. Then $x\geq a$ for all $a\in A$ and at the same time $x\in A$, which implies that $x=\sup A$.
				\end{solution}
				\item A bounded subset of $\Q$ that contains its supremum but not its infimum. 
				\begin{solution}{(c)}
					This is possible. Let $A=\{x\in \Q: 2<x\leq 4\}$. Then $A\subseteq \Q$. Note that $\inf A=2\not\in A$ and $\sup A = 4\in A$. 
				\end{solution}
			\end{enumerate}
		\end{problem}
		
		\begin{problem}{1.3.3}
			\begin{enumerate}[label=(\alph*)]
			\item Let $A$ be nonempty and bounded below, and define $B=\{b\in \R:b \text{ is a lower bound for }A\}$. Show that $\sup B = \inf A$.
			\begin{proof}
				Since $B$ is the set of all lower bounds for $A$, it follows that $a\geq b$ for every $a\in A$ and $b\in B$, and so $B$ is bounded above. By the Axiom of Completeness, there exists some $\sup B = s$. Note that all elements of $A$ are upper bounds of $B$ and so $s\leq a$ for every $a\in A$ since $s=\sup B$. Hence, $s$ is a lower bound for $A$. Also, $s\geq b$ for all $b\in B$. Therefore, $s=\sup B = \inf A$.
			\end{proof}
			\item Use (a) to explain why there is no need to assert that greatest lower bounds exist as part of the Axiom of Completeness.
			\begin{solution}{b}
				Note that for any set $A$ that is bounded below, one can obtain the $supremum$ of the set of all lower bounds of $A$, which we previously showed that is the $infimum$ of $A$. Hence, one can derive the theorem of $greatest \; lower \; bound$ from the axiom of completeness.
			\end{solution}
			
			\end{enumerate}
		\end{problem}

\end{document}